\documentclass[12pt]{article}

\usepackage[margin=2.0cm]{geometry}
\usepackage{amsmath,amssymb,amsthm}
\usepackage{multicol}

\DeclareMathOperator{\tr}{tr}

\begin{document}

\begin{multicols}{2}

\section{Definitions}

\subsection{Vector spaces}

\begin{itemize}
    \item Vector space, subspace
    \item Algebraic sum, external direct sum
    \item Vector span, spanning family, linearly independent set, basis
\end{itemize}


\subsection{Bases}

\begin{itemize}
    \item Coordinates with respect to a basis.
\end{itemize}


\subsection{Linear maps}

\begin{itemize}
    \item Linear map.
    \item Isomorphism.
    \item The matrix representing a linear map.
    \item Inner direct sum.
    \item Projection.
    \item Quotient space.
\end{itemize}


\subsection{Determinants}

\begin{itemize}
    \item Permutation, inversion, sign of a permutation,
        transposition, $n$-cycle.
    \item Determinant (using 4 axioms).
    \item The adjoint matrix.
    \item Orientation, isomorphism preserving orientation.
    \item Wronskian matrix.
\end{itemize}


\subsection{Matrices}

\begin{itemize}
    \item Column rank, row rank.
\end{itemize}


\subsection{The dual space}

\begin{itemize}
    \item Linear functional, the dual space of a vector space.
    \item Annihilator.
    \item Dual map.
    \item Polar dual to a convex set in $\mathbb{R}^2$ containing
        $0$.
\end{itemize}


\subsection{Inner product}

\begin{itemize}
    \item Inner product, norm, distance.
    \item Orthogonal basis, orthonormal basis.
    \item Orthogonal subspace.
    \item Gram matrix.
    \item Approximate solution of the least squares method to the
        overdetermined system $Ax = y$ where the columns of $A$ are
        linearly independent.
\end{itemize}


\subsection{Eigenvalues}

\begin{itemize}
    \item Eigenvalue, eigenvector.
    \item Similar matrices.
    \item Characteristic polynomial.
    \item Eigenspace.
    \item Geometric and algebraic multiplicity of eigenvalues.
    \item Diagonalizable matrix.
\end{itemize}


\subsection{The Jordan normal form}

\begin{itemize}
    \item Minimal polynomial.
    \item $T$-invariant subspace.
    \item Nilpotent map, Jordan block, Jordan form.
\end{itemize}


\subsection{Self-adjoint maps}

\begin{itemize}
    \item Hermitian adjoint.
    \item Self-adjoint map.
    \item Legendre polynomials.
    \item Unitary map, unitary matrix.
    \item Symmetric bilinear form.
    \item Quadrics.
    \item Normal map.
\end{itemize}



\section{Proofs}

\subsection{Vector spaces}

\begin{itemize}
    \item The span of a family of vectors is a subspace.
    \item The three definitions of a basis are equivalent.
\end{itemize}


\subsection{Bases}

\begin{itemize}
    \item The Steinitz substitution lemma.
    \item (Corollary) The cardinality of a linearly independent set
        cannot exceed the cardinality of a basis.
    \item (Corollary) Every linearly independent set can be
        completed to a basis.
    \item The identities involving the matrix of basis change.
    \item A matrix is invertible if and only if its columns form a
        basis.
\end{itemize}


\subsection{Linear maps}

\begin{itemize}
    \item The kernel and image of a linear map are subspaces of the
        domain and codomain, respectively.
    \item Triviality of the kernel is equivalent to injectivity.
    \item Every finitely dimensional vector space is isomorphic to
        $\mathbb{F}^n$.
    \item The theorem on the kernel and the image.
    \item The inner direct sum of vector spaces is isomorphic to
        the external direct sum of vector spaces.
    \item A projection is equivalent to an inner direct sum.
\end{itemize}


\subsection{Determinants}

\begin{itemize}
    \item The sign of a permutation is multiplicative.
    \item The determinant exists and is unique, satisfying the four
        determinant axioms.
    \item Functions satisfying only the first three of the
        determinant axioms are equivalent to the determinant
        function up to scaling.
    \item Cauchy's Theorem, $\det {(AB)} = \det{A} \det{B}$, i.e.
        the determinant is multiplicative.
    \item (Corollary) A matrix is invertible if and only if its
        determinant is nonzero.
    \item Laplace's Expansion Theorem.
    \item Cramer's rule.
\end{itemize}

\subsection{Matrices}

\begin{itemize}
    \item The column and row rank of a matrix are equal.
    \item The algorithm for computing the inverse matrix.
\end{itemize}


\subsection{The dual space}

\begin{itemize}
    \item The dimension of a vector space is equal to the dimension
        of its dual space.
    \item Lemma on the (first three) properties of the annihilator.
    \item The annihilator of a sum is the intersection of the
        annihilators,
        $(U_1 + U_2)^\perp = U_1^\perp \cap U_2^\perp$.
    \item Calculation of the dual basis.
\end{itemize}

\subsection{Inner product}

\begin{itemize}
    \item Pythegorean lemma.
    \item Cauchy-Schwarz inequality.
    \item The properties of the norm.
    \item Existance and formula of the orthogonal projection.
    \item A finitely dimensional inner product space can be
        decomposed into a direct sum of a subspace and its
        orthogonal subspace.
\end{itemize}


\subsection{Eigenvalues}

\begin{itemize}
    \item Similar matrices have the same eigenvalues.
    \item $\det M$ and $\tr M$ appear in the characteristic
        polynomial of $M$.
    \item The remark on the 3 equivalent definitions of an
        eigenvalue.
    \item The lemma and theorem on equivalent conditions for
        diagonalizability.
    \item The closed form for the terms of the Fibonacci sequence.
\end{itemize}

\subsection{The Jordan normal form}

\begin{itemize}
    \item The corollary of the Primary Decomposition Theorem giving
        a condition for diagonalizability.
\end{itemize}


\subsection{Self-adjoint maps}

\begin{itemize}
    \item The existence and uniqueness of the Hermitian adjoint in
        finitely dimensional inner product spaces.
    \item Self-adjoint maps have real eigenvalues and orthogonal
        eigenspaces.
    \item Principal Axis Theorem.
    \item A Hermitian matrix is positive definite if and only if
        all its eigenvalues are positive.
    \item Sylvester's Theorem.
    \item Normal maps are diagonalizable.
    \item Spectral Theorem.
\end{itemize}


\section{Statements of theorems}

\subsection{Vector spaces}

None.


\subsection{Bases}

None.


\subsection{Linear maps}

None.


\subsection{Determinants}

None.


\subsection{Matrices}

None.


\subsection{The dual space}

None.


\subsection{Inner product}

\begin{itemize}
    \item The theorem on the volume of an m-parallelipiped in
        $\mathbb{R}^n$.
\end{itemize}


\subsection{Eigenvalues}

None.


\subsection{The Jordan normal form}

\begin{itemize}
    \item The Cayley-Hamilton theorem.
    \item The proposition and theorem on the properties of the
        minimal polynomial.
    \item The Primary Decomposition theorem.
    \item The Jordan Normal Form theorem.
\end{itemize}


\subsection{Self-adjoint maps}

None.


\end{multicols}
\end{document}
