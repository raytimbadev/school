\documentclass{article}

\usepackage{amsmath, amssymb, amsthm}

\newtheorem{proposition}{Proposition}

\DeclareMathOperator{\diag}{diag}

\title{Assignment \#9\\Algebra 2 (MATH 251)}
\author{Jacob Thomas Errington (260636023)}
\date{23 March 2015}

\begin{document}

\maketitle

\section{Linearly independent families of functions}

\begin{proposition}
    Let $V$ be the space of infinitely differentiable functions of $[0, 1]$.
    The family of functions $\{e^{\lambda x}\}_{\lambda \in \mathbb{R}}$ is
    linearly independent.
\end{proposition}

\begin{proof}
    Consider the linear map of differentiation $D$.
    We notice that for each function in the family,
    $D(e^{\lambda x}) = \lambda e^{\lambda x}$. Thus, each function in the
    family has its own distinct eigenvalue, for which it is an eigenvector.
    Since the eigenvalues are distinct, they each can be identified with
    different eigenspaces. Since a set of eigenvectors each drawn from
    eigenspaces identified by different eigenvalues is linearly independent,
    the family of functions is linearly independent.
\end{proof}

\section{A variation on Fibonacci}

\begin{proposition}
    The recurrence defined by
    \begin{align*}
        a_1 = a_0 &= 1 \\
        a_n &= 2 a_n + a_{n-1}
    \end{align*}
    has the closed form
    \begin{equation*}
        a_n = - \frac{\sqrt{2}}{4}
            \left( \lambda_1^n (1 - \lambda_2) + \lambda_2^n (\lambda_1 - 1)\right)
    \end{equation*}
    where
    ${\lambda_2} = 1 + \sqrt{2}$ and ${\lambda_1} = 1 - \sqrt{2}$
\end{proposition}

\begin{proof}
    We will construct the closed form from a linear system based on the
    recurrence.

    Consider the following two equations.
    \begin{align*}
        a_{n+1} &= 2 a_n + a_{n-1} \\
        a_n &= a_{n-1} + 0 a_{n-2}
    \end{align*}
    We can write this homogeneous system of equations in matrix form.
    \begin{equation*}
        \left(
            \begin{array}{c}
                a_{n+1} \\
                a_n
            \end{array}
        \right)
        =
        A
        \left(
            \begin{array}{c}
                a_n \\
                a_{n-1}
            \end{array}
        \right)
    \end{equation*}
    where
    $$
    A =
    \left(
        \begin{array}{c c}
            2 & 1 \\
            1 & 0
        \end{array}
    \right)
    $$

    Solving $\Delta_A = 0$, we find that the eigenvalues of $A$ are
    \begin{align*}
        \lambda_1 &= 1 - \sqrt{2} \\
        \lambda_2 &= 1 + \sqrt{2}
    \end{align*}

    Solving for $x_i$ in the equation $(A - \lambda_i Id) x_i = 0$, for $i \in
    \{1, 2\}$, we find an eigenvector for each eigenvalue.
    \begin{align*}
        x_1 &=
        \left(
            \begin{array}{c}
                1 - \sqrt{2} \\
                1
            \end{array}
        \right) \\
        x_2 &=
        \left(
            \begin{array}{c}
                1 + \sqrt{2} \\
                1
            \end{array}
        \right)
    \end{align*}

    Let $\Lambda = \diag{(\lambda_1, \lambda_2)}$ and $S = (x_1\, x_2)$. By the
    definitions of eigenvalues and eigenvectors, we have $AS = S\Lambda$.

    Since a set of eigenvectors each drawn from different eigenspaces is
    linearly independent, $S$ is invertible, and we can write
    \begin{align*}
        A &= S \Lambda S^{-1} \\
        A^n &= S \Lambda^n S^{-1} \\
        A^n u_1 &= S \Lambda^n S^{-1} u_1 \\
        u_{n+1} &= S \Lambda^n S^{-1} u_1
    \end{align*}
    where $u_i = \left(\begin{array}{c} a_i \\ a_{i-1} \end{array}\right)$.

    The last step is due to the fact that we designed $A$ precisely to have the
    action of taking such vectors to their successors, according to the
    recurrence.

    Computing the right-hand side of the previous equation, we have
    \begin{equation*}
        \left(
            \begin{array}{c}
                a_{n+1} \\
                a_n
            \end{array}
        \right)
        =
        - \frac{\sqrt{2}}{4}
        \left(
            \begin{array}{c}
                \lambda_1^{n+1} (1 - \lambda_2)
                + \lambda_2^{n+1} (\lambda_1 - 1) \\
                \lambda_1^{n} (1 - \lambda_2)
                + \lambda_2^{n} (\lambda_1 - 1)
            \end{array}
        \right)
    \end{equation*}

    Extracting just an equation for $a_n$, we have
    \begin{equation*}
        a_n = \lambda_1^{n} (1 - \lambda_2) + \lambda_2^{n} (\lambda_1 - 1)
    \end{equation*}
\end{proof}

\end{document}
