\documentclass{article}

\usepackage{amsmath,amssymb,amsthm}

\usepackage[margin=2cm]{geometry}

\author{Jacob Errington (260636023)}
\title{Assignment \#3\\Algebra II (MATH 251)}
\date{26 January 2015}

\renewcommand{\a}{\alpha}
\renewcommand{\b}{\beta}

\newcommand{\R}{\mathbb{R}}

\begin{document}

\maketitle

\begin{enumerate}
    \item Determine which are functions are linear maps.
        \begin{enumerate}
            \item $T : \R^3 \to \R^2$, $(x_1, x_2, x_3) \mapsto (1 + x_1, x_2)$ is not a linear map.
                \begin{proof}
                    Suppose $T$ is a linear map.
                    \begin{align*}
                        (\a (1 + x_1), \a x_2) &= \a (1 + x_1, x_2) \\
                                               &= \a T(x_1, x_2, x_3) \\
                                               &= T(\a (x_1, x_2, x_3)) \\
                                               &= T(\a x_1, \a x_2, \a x_3) \\
                                               &= (1 + \a x_1, \a x_2)
                    \end{align*}

                    In general, it is not true that
                    $(\a (1 + x_1), \a x_2) = (1 + \a x_1, \a x_2)$,
                    so the above is a contradiction.
                \end{proof}

            \item $T : \R^3 \to \R^2$, $(x_1, x_2, x_3) \mapsto (x_3, x_1 + x_2)$ is a linear map.
                \begin{proof}
                    We will check that the structure is preserved.
                    \begin{description}
                        \item[Scalar multiplication.]
                            \begin{align*}
                                \a T(x_1, x_2, x_3) &= \a (x_3, x_1 + x_2) \\
                                                    &= (\a x_3, \a (x_1 + x_2)) \\
                                                    &= (\a x_3, \a x_1 + \a x_2) \\
                                                    &= T(\a x_1, \a x_2, \a x_3) \\
                                                    &= T(\a (x_1, x_2, x_3))
                            \end{align*}
                            Scalar multiplication is preserved.
                        \item[Addition.]
                            \begin{align*}
                                T((x_1, x_2, x_3) + (y_1, y_2, y_3)) &= T(x_1 + y_1, x_2 + y_2, x_3 + y_3) \\
                                                                     &= (x_3 + y+3, (x_1 + y_1) + (x_2 + y_2)) \\
                                                                     &= (x_3 + y_3, (x_1 + x_2) + (y_1 + y_2)) \\
                                                                     &= (x_3, x_1 + x_2) + (y_3, y_1 + y_2) \\
                                                                     &= T(x_1, x_2, x_3) + T(y_1, y_2, y_3)
                            \end{align*}
                    \end{description}
                    The mapping preserves the structure, so it is a linear map.
                \end{proof}

            \item $T : \R^3 \to \R^2$, $(x_1, x_2, x_3) \mapsto (x_3, {x_1}^2 + {x_2}^2)$ is not a linear map.
                \begin{proof}
                    We will show that the scalar multiplication is not
                    preserved, by performing operations ``out of order'', which
                    would not matter if it were preserved.
                    \begin{description}
                        \item[Mapping first.]
                            \begin{align}
                                \a T(x_1, x_2, x_3) &= \a (x_3, {x_1}^2 + {x_2}^2) \notag \\
                                                    &= (\a x_3, \a {x_1}^2 + \a {x_2}^2)
                                \label{eqn:1cmappingfirst}
                            \end{align}

                        \item[Multiplication first.]
                            \begin{align}
                                T(\a (x_1, x_2, x_3)) &= T(\a x_1, \a x_2, \a x_3) \notag \\
                                                      &= (\a x_3, (\a x_1)^2 + (\a x_2)^2) \notag \\
                                                      &= (\a x_3, \a^2 {x_1}^2 + \a^2 {x_2}^2)
                                \label{eqn:1cmultiplyfirst}
                            \end{align}
                    \end{description}

                    It is necessary that the results from
                    \eqref{eqn:1cmappingfirst} and \eqref{eqn:1cmultiplyfirst} be
                    the same, if $T$ is a linear map. Therefore, $T$ must not
                    be a linear map.
                \end{proof}

            \item $T : \R[t]_2 \to \R[t]_3$, $f(t) \mapsto t^2 + f(t)$ is not a linear map.
                \begin{proof}
                    The scalar multiplication is not preserved. We will proceed as in the previous proof.
                    \begin{description}
                        \item[Mapping first.]
                            \begin{equation}
                                \a T(f(t)) = \a (t^2 + f(t)) = \a t^2 + \a f(t)
                                \label{eqn:1dmappingfirst}
                            \end{equation}
                        \item[Multiplication first.]
                            \begin{equation}
                                T(\a f(t)) = t^2 + \a f(t)
                                \label{eqn:1dmultiplyfirst}
                            \end{equation}
                    \end{description}
                    Again as in the previous proof, the results don't match, which is a necessary condition for being a linear map.
                \end{proof}

            \item $T : \R[t]_2 \to \R[t]_3$, $f(t) \mapsto t f(t) + t^2 f^\prime (t)$ is a linear map.
                \begin{proof}
                    We will check that the properties of a linear map are satisfied by $T$.
                    \begin{description}
                        \item[Scalar multiplication.]
                            \begin{align*}
                                \a T(f(t)) &= \a (t f(t) + t^2 f^\prime (t)) \\
                                           &= \a t f(t) + \a t^2 f^\prime (t) \\
                                           &= t (\a f(t)) + t^2 (\a f^\prime (t)) \\
                                           &= T(\a f(t))
                            \end{align*}
                            Scalar multiplication is preserved.

                        \item[Addition.]
                            \begin{align*}
                                T(f(t) + g(t)) &= t (f(t) + g(t)) + t^2 (f^\prime (t) + g^\prime (t)) \\
                                               &= t f(t) + t g(t) + t^2 f^\prime (t) + t^2 g^\prime (t) \\
                                               &= t f(t) + t^2 f^\prime (t) + t g(t) + t^2 g^\prime (t) \\
                                               &= T(f(t)) + T(g(t))
                            \end{align*}
                            Addition is preserved.
                    \end{description}
                \end{proof}
        \end{enumerate}

    \item In order to find the differentiation matrix ${}_B [D]_B$ with respect
        to the basis $B = (e^x, xe^x, x^2 e^x)$, we apply the differentiation
        map $D$ to each element of the basis to construct a basis $B^\prime$,
        whose elements we express as linear combinations of the elements of
        $B$. Those coefficients give us the columns of the matrix ${}_B D_B$.
        \begin{align*}
            D(e^x) &= e^x \\
            D(x e^x) &= e^x + x e^x \\
            D(x^2 e^x) &= 2x e^x + x^2 e^x \\
            \\
            e^x &= 1 e^x + 0 x e^x + 0 x^2 e^x \\
            e^x + x e^x &= 1 e^x + 1 xe^x + 0 x^2 e^x \\
            2x e^x + x^2 e^x &= 0 e^x + 2 x e^x + 1 x^2 e^x
        \end{align*}

        \begin{equation*}
            {}_B [D]_B =
            \left(
                \begin{array}{c c c}
                    1 & 1 & 0 \\
                    0 & 1 & 2 \\
                    0 & 0 & 1
                \end{array}
            \right)
        \end{equation*}

\end{enumerate}
\end{document}

