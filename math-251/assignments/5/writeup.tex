\documentclass{article}

\usepackage{amsmath,amssymb,amsthm}
\usepackage[margin=3.6cm]{geometry}

\newtheorem{proposition}{Proposition}
\newtheorem{lemma}{Lemma}
\DeclareMathOperator{\im}{Im}

\newcommand{\R}{\mathbb{R}}

\author{Jacob Errington (260636023)}
\title{Assignment \#5\\Honours Algebra II (MATH 251)}
\date{9 February 2015}

\begin{document}

\maketitle

\section{Particular homogeneous systems}

\emph{Find a homogeneous system of linear equations whose solution space is $\im{T}$ for the following $T$.}

$$
T : \R^3 \to \R^4
$$

First, we write the matrix representing $T$.

$$ A = \left(
    \begin{array}{c c c}
        3 & 4 & 2 \\
        1 & 2 & 0 \\
        2 & 1 & 3 \\
        -1 & 5 & -7
    \end{array}
\right) $$
and we row-reduce it to reduced row echelon form
$$ \tilde{A} = \left(
    \begin{array}{c c c}
        1 & 0 & 2 \\
        0 & 1 & -1 \\
        0 & 0 & 0 \\
        0 & 0 & 0
    \end{array}
\right)$$
We will use $\tilde{A}$ to solve the equation $\tilde{A} \vec{x} = \vec{0}$.
The solution-set of that equation is the kernel of $A$, and thus $\ker{T}$.
\begin{eqnarray*}
\left\{
    \begin{aligned}
        x_1 &= -2 r \\
        x_2 &= r \\
        x_3 &= r
    \end{aligned}
\right.
\end{eqnarray*}
So the basis for the kernel consists of exactly one vector, and is $\{(-2, 1, 1)\}$.

To extend the basis for the kernel into a basis for $\R^3$, we need to add the
elementary vectors $e_2, e_3$. To find the basis for $\im{T}$, we apply $T$ to
those missing vectors.
\begin{align*}
    T(e_2) &= (4, 2, 1, 5) \\
    T(e_3) &= (2, 0, 3, -7)
\end{align*}

Linear combinations of these vectors make up $\im{T}$. However, we need to find
a homogeneous system whose solution space is $\im{T}$. In other words, we must
find a linear map $S$, represented by a matrix $B$, whose kernel is $\im{T}$.
The homogeneous system of equations corresponding to $B$ is the answer.

We construct the following system by explicitly writing out the linear combinations of $T(e_2)$ and $T(e_3)$.
\begin{eqnarray*}
    \left\{
        \begin{aligned}
            x_1 & -2r & -4s &= 0 \\
            x_2 & & -2s &= 0 \\
            x_3 & -3r & -s &= 0 \\
            x_4 & +7r & -5s &= 0
        \end{aligned}
    \right.
\end{eqnarray*}

To stand in for the parameters $r$ and $s$, we will augment the system with two more variables to build the matrix $B$.
$$\left(
    \begin{array}{c c c c c c}
        1 & 0 & 0 & 0 & -2 & -4 \\
        0 & 1 & 0 & 0 & 0 & -2 \\
        0 & 0 & 1 & 0 & -3 & -1 \\
        0 & 0 & 0 & 1 & 7 & -5
    \end{array}
\right)$$

The solution set of the corresponding homogeneous system of equations equals $\im{T}$.
\begin{eqnarray*}
    \left\{
        \begin{aligned}
            x_1 &  &  &  & -2 x_5 & -4 x_6 &= 0 \\
             & x_2 &  &  &  & -2 x_6 &= 0 \\
             &  & x_3 &  & -3 x_5 & -1 x_6 &= 0 \\
             &  &  & x_4 & + 7 x_5 & -5 x_6 &= 0
        \end{aligned}
    \right.
\end{eqnarray*}

\section{Fibonacci numbers as determinants}

\begin{proposition}
    Consider the matrix whose main diagonal entries are all $1$, whose diagonal
    entries above the main diagonal are all $1$, whose diagonal entries
    below the main diagonal are all $-1$, and whose remaining entries are $0$.
    The determinant of such a matrix of size $n$ is equal to the $n$th
    Fibonacci number.
\end{proposition}

\begin{proof}
    To show that the specially-formed determinant of size $n$ equals the $n$th
    Fibonacci number, we will prove that its cofactor expansion along the first
    column yields a sum of exactly two terms, themselves specially-formed
    determinants of size $n-1$ and $n-2$. Then by induction it will follow that
    the proposition holds.

    We verify by hand that the base cases hold: the specially-formed
    determinants of size $1$ and $2$ are trivially seen to be equal to the
    first and second Fibonacci numbers, respectively.

    Induction hypothesis: the specially-formed determinants of size $n-1$ and $n-2$
    are equal to the corresponding Fibonacci numbers.

    Let the specially-formed determinant of size $n$ be $D$. We will perform
    cofactor expansion on it along the first column. There are two nonzero
    entries in the first column, and we will show that the corresponding
    cofactors are the specially-formed determinants of size $n-1$ and $n-2$.

    Let $D^{ij}$ refer to the $ij$ cofactor of $D$.

    $D^{11}$ is clearly seen to be the specially-formed determinant of size
    $n-1$.

    $D^{21}$'s top row will have only one nonzero element: its top-left entry
    will be $1$ due to the structure of $D$. So $D^{21} = (D^{21})^{11}$, i.e.
    $D^{21}$ is equal to its $1,1$ cofactor, which is precisely the
    specially-formed determinant of size $n-2$.

    We have shown that $D$ is the sum of the specially-formed determinants of
    size $n-1$ and $n-2$, which by the induction hypothesis are the
    corresponding Fibonacci numbers.
    Therefore, $D$ is the $n$th Fibonacci number.
\end{proof}

\section{Determinants and totients}

\begin{lemma}
    Let $(i,j)$ represent the greatest common divisor of $i$ and $j$ and
    define $p_{ij}$ as follows
    $$
    p_{ij} =
    \begin{cases}
        1 & \text{ if } j|i \\
        0 & \text{ otherwise}
    \end{cases}
    $$

    Then,
    $$(i, j) = \sum_{k=1}^n {p_{ik} p_{jk} \phi(k)}$$
\end{lemma}

\begin{proof}
    Let $s = \sum_{k=1}^n {p_{ik} p_{jk} \phi(k)}$.

    A term in the sum $s$ will be nonzero if-and-only-if $k|i$ and $k|j$, and
    $k$ ranges over all the possible divisors of $i$ and $j$, which are
    precisely all the divisors of $(i, j)$. Thus, by the Gauss formula, $s =
    (i, j)$.
\end{proof}

\begin{proposition}
    $$
    \det {(d_{ij})} = \prod_{i=1}^n {\phi{(i)}}
    $$
    where $d_{ij} = (i, j)$.
\end{proposition}

\begin{proof}
    Consider the $n \times n$ matrix $P = (p_{ij})$, and note that $P$ is
    triangular, since $j|i$ requires $j \leq i$ (as we are using nonnegative
    integers only). We also note that the main diagonal of $P$ consists only of
    $1$, since the divides relation is reflexive.

    We examine the product $P\Phi P^T$, where $\Phi$ is the $n \times n$
    diagonal matrix, whose $i$th diagonal element is $\phi{(i)}$.

    With these definitions, we can establish
    $$P\Phi P^T = (p_{ij})\Phi (p_{ji}) = (p_{ij} \phi{(j)}) (p_{ji})$$
    since a product $AD$ where $D$ is diagonal amounts to multiplying the $j$th
    column of $A$ by the $j$th entry of $D$.

    We can write the entries of this product as sums
    $$\left(\sum_{t=k}^n {a_{ik} \phi{(k)} a_{jk}}\right)$$
    Each of these entries, however, are by lemma, $(i, j)$, which establishes
    the following equality
    $$P\Phi P^T = (d_{ij})$$

    We may then take the determinant of both sides. Since $P$ is triangular
    with $1$ everywhere on its main diagonal, its determinant is $1$. Due to
    the multiplicative property of the determinant and since the determinant of
    a diagonal matrix is the product of its diagonal entries, we conclude
    $$\det {(d_{ij})} = \det \Phi = \prod_{i=1}^n {\phi{(i)}}$$
\end{proof}

\end{document}

