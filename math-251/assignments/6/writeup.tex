\documentclass{article}

\usepackage{amsmath,amssymb}

\DeclareMathOperator{\im}{Im}

\author{Jacob Errington(260636023)}
\title{Assignment \#6\\Honours Algebra 2 (MATH 215)}
\date{\today}

\begin{document}

\maketitle

\begin{description}

    \item[Problem \#1] 
        First we'll find the dual basis to the basis $B = ((5, 2), (7, 3))$. We know
        that if $V$ is the matrix whose columns consist of $B$ vectors and $U$ is the
        matrix whose columns are the vectors of $B^*$, then $U^T V = Id$. So the
        vectors of $B^*$ are the rows of $V^{-1}$.

        By row-reduction, we find 
        $$V^{-1} = 
        \frac{1}{11}
        \begin{array}{c c}
            3 & -7 \\
            2 & -1
        \end{array}
        $$
        meaning that the dual basis is $(\frac{1}{11}(3, -7), \frac{1}{11}(2, -1))$, 
        which we can express in terms of the standard dual basis as
        $$
        B^* = (\frac{1}{11}(3x_1 + -7x_2), \frac{1}{11}(2x_1 - x_2))
        $$

    \item[Problem \#3]
        First, we show that every linear map defined on a subspace of $W \subset V$ can
        be extended to a linear map over $V$.

        We consider $W$ to be a proper subset of $V$, so if $\dim V = n$, then it has
        dimension at most $r = n-1$. Let the basis of $W$ be $B_W = (v_1, \cdots,
        v_r)$. This can be completed to a basis of $V$, 
        $B_V = (v_1, \cdots, v_r, v_{r+1} \cdots, v_n)$. The ``extra'' vectors used for
        the completion can be used as a basis for a vector space $W^\prime$ which is in
        some sense complementary to $W$. It follows that $V = W \oplus W^\prime$.

        If we have $S : W \to W$ and $R : W^\prime \to W^\prime$, then from the above
        observation, we can form a $T(v) = S(v) + R(v)$, an extending $S$ over $V$.

        Let $T: V \to W$ be injective. Choose $a \in V^*$ and $b \in W^*$ such that
        $T^*(a) = b$, i.e. $b = T \circ a$. We split the codomain of $T$, namely $W$,
        into two subspaces, such that $W = T(V) \oplus W^\prime$. In particular, this
        means that each $w \in W$ can be represented as $w = T(v) + w^\prime$. Due to
        the injectivity of $T$, the mapping from $w$ to $v$ is well-defined, i.e. each
        $w$ is mapped only to one $v$, so we may define 
        $b(w) = b(T(v) + w^\prime) = a(v)$. Thus, we confirm the surjectivity of $T^*$,
        since $(b \circ T)(v) = b(T(V)) = a(v)$.

        Let $T : V \to W$ be surjective. Suppose $b \in \ker {T^*}$,
        meaning that $(b \circ T)(v) = 0\, \forall v \in V$. Since $T$ is surjective,
        its image is its codomain, so $b$ is the zero map, such that $b(W) = 0$. Thus,
        $\ker{T*}$ consists only of the zero map and is trivial, implying that $T^*$ is
        injective.
\end{description}

\end{document}
