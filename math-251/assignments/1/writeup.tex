\documentclass{article}

\usepackage{amsmath, amssymb}

\author{Jacob Errington (260636023)}
\title{Assignment \#1\\Honours Algebra II (MATH 251)}
\date{due on 12 January 2015}

\newcommand{\R}{\mathbb{R}}
\newcommand{\F}{\mathbb{F}}

\begin{document}

\maketitle

\begin{enumerate}
    \item $V$ is not a vector space since the distributivity of a vector over a
        sum of scalars is not satisfied.

        \begin{align*}
            (\alpha + \beta) (x_1, x_2) &= ((\alpha + \beta) x_1, x_2) \\
                                        &= (\alpha x_1 + \beta x_1, x_2) \\
                                        &\neq \alpha (x_1, x_2) + \beta (x_1,
            x_2)
        \end{align*}

        %\begin{itemize} %When I used to think it was actually a vector space
        %    \item The set forms an abelian group under its addition, being
        %        defined element-wise using the addition of the underlying field
        %        (in this case the reals). Its neutral element is simply $(0,
        %        0)$ and additive inverses are given by taking the additive
        %        inverses of the components, i.e. $-(x_1, x_2) = (-x_1, -x_2)$.

        %    \item The neutral element with respect to scalar multiplication is
        %        the real $1$, since $1 (x_1, x_2) = (1 x_1, x_2) = (x_1, x_2)$.

        %    \item When more than one scalar is multiplying the vector, it does
        %        not matter whether we multiply the scalars together first or
        %        scale the vector by the first one and then by the second one:
        %        $(\alpha \beta) (x_1, x_2) = (\alpha \beta x_1, x_2) = \alpha
        %        (\beta x_1, x_2)$.
        %\end{itemize}

    \item Which of the following subsets of the vector space $\R \to \R$ are
        its subspaces?

        \begin{enumerate}
            \item The functions attaining a fixed nonzero value $a$ at a point
                $b$ do not form a subspace of $\R \to \R$ since their set is
                not closed under scalar multiplication. Indeed, multiplying by
                real zero gives a function that is zero everywhere, including
                the specified point $b$.

            \item The functions vanishing at a specified point $b$ form a
                subspace of $\R \to \R$, since:
                \begin{enumerate}
                    \item adding two functions vanishing at a point $b$ yields
                        another function vanishing at $b$, so addition is
                        closed;
                    \item multiplying a function vanishing at a point $b$ by
                        any scalar $\alpha$ results in yet another function
                        vanishing at $b$.
                \end{enumerate}

            \item Functions vanishing on a specified set $S \subset \R$ form a
                subspace by the same reasoning as in the case of a single
                vanishing point.

            \item Functions with only finitely many discontinuity points form a
                subspace.

                Suppose a function $f : \R \to \R$ with $a$ discontintuities
                and $g : \R \to \R$ with $b$ discontinuities.  $f + g$ has at
                most $a + b$ discontinuities, which is also finite.

                As for scalar multiplication, it does not affect the number of
                discontinuities, so scalar multiplication is closed.

            \item Functions vanishing everywhere outside a finite set dependent
                on the function (equivalently, functions with nonzero values
                only on a finite set) form a subspace of $\R \to \R$.

                First, we note that functions that are zero \emph{everywhere}
                satisfy this property of vanishing everywhere outside a finite
                set since the empty set is finite.

                Consider $f : \R \to \R$ nonzero for $a$ values in $\R$, and $g
                : \R \to \R$ nonzero for $b$ values in $\R$. Their sum will
                have at most $a + b$ and at least $0$ values $z \in \R$ such
                that $(f + g)(z) \neq 0$. Therefore, addition is closed.

                As for scalar multiplication, multiplication by zero is the
                only operation that affects the number of nonzero
                function-values, making the function zero everywhere. As
                discussed previously, such functions are still in the subset of
                interest. Therefore, scalar multiplication is closed.
        \end{enumerate}

    \item Prove that the external direct sum $U \oplus V$ of vector spaces of
        $\F$ is a vector space over $\F$ as well.

        We will check that the axioms of a vector space are satisfied by this
        set.

        \begin{itemize}
            \item The external direct sum forms an abelian group under
                addition, since addition is performed component-wise and the
                constituent vector spaces form abelian groups under addition.

                The neutral element of addition is $(0_U, 0_V)$, made up of
                the neutral elements of addition in the constituent vector
                spaces.

                Likewise, we can find the additive inverse of an element in the
                external direct sum by taking the additive inverses of the
                constituents.
                \begin{equation*}
                    (u, v) + (-u, -v) = (u - v, v - v) = (0_U, 0_V)
                \end{equation*}

            \item The neutral element with respect to scalar multiplication is
                simply the multiplicative identity of the underlying field:
                \begin{equation*}
                    1 (u, v) = (1 u, 1 v) = (u, v)
                \end{equation*}
                In other words, since $1_\F$ is the neutral element with
                respect to scalar multiplication in each of the constituent
                vector spaces, and scalar multiplication is defined
                distibutively thereover, $1_\F$ is also the neutral element
                with respect to scalar multiplication in the external direct
                sum.

            \item
                \begin{align*}
                    (\alpha \beta) (u, v) &= ((\alpha \beta) u, (\alpha \beta)
                    v) \\
                        &= (\alpha (\beta u), \alpha (\beta v)) \\
                        &= \alpha (\beta u, \beta v) \\
                        &= \alpha (\beta (u, v))
                \end{align*}

            \item
                \begin{align*}
                    (\alpha + \beta) (u, v) &= ((\alpha + \beta) u, (\alpha +
                    \beta) v) \\
                        &= (\alpha u + \beta v, \alpha v + \beta v) \\
                        &= (\alpha u, \alpha v) + (\beta u, \beta v) \\
                        &= \alpha (u, v) + \beta (u, v)
                \end{align*}

            \item
                \begin{align*}
                    \alpha ( (u_1, v_1) + (u_2, v_2) ) &= \alpha (u_1 + u_2, v_1
                    + v_2) \\
                    &= (\alpha (u_1 + u_2), \alpha (v_1 + v_2)) \\
                    &= (\alpha u_1 + \alpha u_2, \alpha v_1 + \alpha v_2) \\
                    &= (\alpha u_1, \alpha v_1) + (\alpha u_2, \alpha v_2) \\
                    &= \alpha (u_1, v_1) + \alpha (u_2, v_2)
                \end{align*}
        \end{itemize}
\end{enumerate}

\end{document}
