\documentclass[11pt]{article}

\author{Jacob Thomas Errington}
\title{Set theory}
\date{}

\usepackage[geometry]{jakemath}
\let\P\undefined
\makebb{P}

\begin{document}

\maketitle

\section{Axiomatic set theory}

\begin{defn}{(Well-order.)}
    A set together with a binary relation $(W, \leq)$ is a \emph{well-order}
    if it is a linear order such that any subset $A \subseteq W$ has a minimal
    element.
\end{defn}

\begin{defn}{(Transitive set.)}
    A set $A$ is \emph{transitive} if
    %
    \begin{equation*}
        x \in y \land y \in A \implies x \in A
    \end{equation*}
    %
    Equivalently, this means that every element of $A$ is a subset of $A$.
\end{defn}

\begin{defn}{(Ordinal.)}
    An \emph{ordinal} is a transitive well-ordered set.
\end{defn}

\begin{defn}{(Successor ordinal.)}
    Let $\alpha$ be an ordinal.
    Then we write $\alpha^+$ or $\alpha + 1$ to denote
    $\alpha \union \setof{\alpha}$, the \emph{successor} of $\alpha$.

\end{defn}

\begin{rem}
    For example, every natural number except zero is a successor ordinal.

    Why? We define the natural numbers inductively, with the ordinal successor
    operation. Zero is defined $0 = \emptyset$ and $n + 1$ is defined by
    $n + 1 = n \union \setof{n}$.
\end{rem}

\begin{rem}
    The natural number $n$ is a set of $n$ elements.
    This can be by induction.
    Zero is the empty set, and so contains zero elements as required.
    In the step case, we have $n$ being a set of $n$ elements.
    Note that $n \notin n$ by the Axiom of Regularity,
    so $\setof{n}$ is disjoint from $n$.
    Hence, $n + 1 = n \union \setof{n}$ has $n + 1$ elements.
\end{rem}

\begin{defn}{(Limit ordinal.)}
    Any ordinal that is not a successor is a \emph{limit ordinal}, e.g.
    %
    \begin{itemize}
        \item zero
        \item the set of natural numbers, denoted $\omega$.
        \item the set of real numbers, denoted $\omega_1$.
    \end{itemize}
\end{defn}

\begin{thm}{(Isomorphism of well-orders to ordinals.)}
    Let $(W, \leq)$ be a well-order.
    Then there exists an ordinal number $\alpha$ and a unique isomorphism
    %
    \begin{equation*}
        \phi : W \to \alpha
    \end{equation*}
\end{thm}

\begin{cor}{(Order type.)}
    The ordinal $\alpha$ to which a given well-order $W$ is isomorphic is
    unique.
\end{cor}

\begin{proof}
    Suppose not.
    Then we have ordinals $\alpha$ and $\beta$ such that $\alpha \neq \beta$ as
    well as isomorphisms $\phi_\alpha : W \to \alpha$
    and $\phi_\beta : W \to \beta$.
    By composing, we obtain an isomorphism $\phi : \alpha \to \beta$, which
    contradicts the trichotomy theorem for ordinal numbers.
\end{proof}

\begin{defn}
    The ordinal $\alpha$ isomorphic to a given well-order $W$ is called its
    \emph{order type}.
\end{defn}

This establishes that ordinals are the \emph{canonical well-orders}.

\begin{cor}{(Trichotomy theorem of well-orders.)}
    If $(W_1, \leq)$ and $(W_2, \leq)$ are well-orders, then either
    %
    \begin{enumerate}
        \item $W_1$ is isomorphic to $W_2$, or
        \item $W_1$ is isomorphic to an initial segment of $W_2$, or
        \item an initial segment of $W_1$ is isomorphic to $W_2$.
    \end{enumerate}
\end{cor}

\begin{proof}
    This follows from the unique isomorphism to ordinals and the corresponding
    trichotomy theorem for ordinals.
\end{proof}

\begin{prop}{(Knaster-Tarski Fixed-Point Theorem.)}
    Suppose $L$ is a complete lattice and $f : L \to L$ is a continuous
    monotone function.
    Then $f$ has a fixed point $x \in L$ such that $x = f(x)$.
\end{prop}

\begin{proof}
    Define the set
    %
    \begin{equation*}
        A = \setof{x \in L \sth x \leq f(x)}
    \end{equation*}
    %
    and let $a = \Join A$.

    We claim $a \leq f(a)$.
    Since $a$ is an upper bound for $A$,
    we have that both $x \leq f(x)$ and $x \leq a$.
    By monotonicity, deduce that $f(x) \leq f(a)$, so $x \leq f(a)$.
    Hence, $f(a)$ is an upper bound for $A$ as well.
    But since $a$ is the \emph{least} upper bound, we have $a \leq f(a)$.

    Therefore, $a \in A$ by definition.
    By monotonicity, we have $f(a) \leq f(f(a))$,
    so $f(a) \in A$ as well.
    Since $a$ is an upper bound of $A$, we have $f(a) \leq a$.
    By antisymmetry, $a = f(a)$, so $a$ is fixed by $f$.
\end{proof}

\begin{prop}{(Banach's Lemma.)}
    For any functions
    %
    \begin{align*}
        f &: X \to Y \\
        g &: Y \to X
    \end{align*}
    %
    there exists a set $C$ such that
    %
    \begin{equation*}
        g(Y \setminus f(C)) = X \setminus C
    \end{equation*}
\end{prop}

\begin{proof}
    Define the map $A \mapsto X \setminus g(Y \setminus f(A))$.
    Suppose $A \subseteq B$ and observe the deduction
    %
    \begin{align*}
        A &\subseteq B \\
        f(A) &\subseteq f(B) \\
        Y \setminus f(A) &\supseteq Y \setminus f(B) \\
        g(Y \setminus f(A)) &\supseteq g(Y \setminus f(B)) \\
        X \setminus g(Y \setminus f(A))
            &\subseteq X \setminus g(Y \setminus f(A))
    \end{align*}
    %
    Hence, the map is monotone. It is furthermore continuous because taking the
    image of a function preserves unions (suprema).

    Hence by the fixed-point theorem, there exists $C$ such that
    %
    \begin{equation*}
        C = X \setminus g(Y \setminus f(C))
    \end{equation*}
    %
    as required.
\end{proof}

\begin{prop}{(Bernstein-Cantor-Schr\"oder Theorem.)}
    Suppose $f : X \to Y$ and $g : Y \to X$ are injections.
    Then there exists a bijection $h : X \to Y$.
\end{prop}

\begin{proof}
    By the Banach Lemma, there exists a set $A$ such that
    %
    \begin{equation*}
        A = X \setminus g(Y \setminus f(A))
    \end{equation*}

    Note that $g(Y \setminus f(A)) = X \setminus A$ by the construction of $A$.
    By the injectivity of $g$, we have
    %
    \begin{equation*}
        Y \setminus f(A) = g\inv(X \setminus A)
    \end{equation*}
    %
    so $Y = g\inv(X \setminus A) \union f(A)$.

    Therefore, define the function $h : X \to Y$ by
    %
    \begin{equation*}
        h(x) = \begin{cases}
            f(x) &\quad\text{if } x \in A \\
            g\inv(x) &\quad\text{if } x \in X \setminus A
        \end{cases}
    \end{equation*}
    %
    and note that it is a bijection.
\end{proof}

\subsection{Cardinals}

\begin{defn}{(Cardinal number.)}
    An ordinal number $\alpha$ is called a \emph{cardinal} if it is not
    equinumerous with any ordinal $\beta < \alpha$.
    %
    In other words, there exists no bijection $f : \beta \to \alpha$.
    %
    For example, each natural number is a cardinal by the pigeonhole principle.
    The set of natural numbers is a cardinal, denoted $\aleph_0$, since it is
    infinite yet all its elements are finite.
\end{defn}

\begin{defn}{(Cardinality.)}
    The least ordinal equinumerous with a given set $X$ is called the
    \emph{cardinality} of $X$
\end{defn}

\begin{rem}
    Axiom of Choice is required to see that every set has a cardinality.
    From the axiom, we have the Well-Ordering theorem which states that every
    set can be well-ordered.
    Let $X$ be an arbitrary set, and $(W, \leq)$ be a well-order on it.
    The well-order is isomorphic to a unique ordinal $\alpha$.
    Form the set of ordinals smaller than $\alpha$ and equinumerous with
    $\alpha$.
    The ordinal numbers being themselves well-ordered allows us to pick the
    least ordinal equinumerous with $\alpha$
\end{rem}

\begin{prop}{(Cantor's Theorem.)}
    There is no surjection $f : X \to \pset{X}$.
\end{prop}

\begin{proof}
    Suppose there is a surjection $f : X \to \pset{X}$.
    %
    Define the set
    %
    \begin{equation*}
        A = \setof{
            x \in X \sth
            x \notin f(x)
        }
    \end{equation*}
    %
    Note that $A \subset X$.
    Since $f$ is a surjection, we have $A \in \rng{f}$,
    so let $x \in X$ be such that $f(x) = A$.

    \begin{itemize}
        \item
            Suppose $x \in A = f(x)$.
            But then by definition of $A$, we have that $x \notin f(x)$.
            %
        \item
            Suppose $x \notin A = f(x)$.
            But then by definition of $A$, we have that $x \in f(x)$.
    \end{itemize}

    This is a contradiction, so $f$ cannot be a surjection.
\end{proof}

\begin{defn}
    For a set of cardinals $K$, we say that its \emph{limit} is $\kappa$ and
    write $\kappa = \lim K$ if $\kappa = \Union K$ is the union of the set.
\end{defn}

\begin{prop}{(Limit of cardinals is a cardinal.)}
    Suppose $K$ is a set of cardinals.
    Then $\kappa = \lim K$ is a cardinal.
\end{prop}

\begin{proof}
    Write $K = (\kappa_i \sth i \in I)$ for some index set $I$, so
    $\kappa = \lim_{i < I} \kappa_i$.
    Suppose $\kappa$ is not a cardinal.
    Then there exists a bijection $f : \lambda \to \kappa$
    for some $\lambda < \kappa$.
    %
    Since $\lambda < \Union \kappa_i$,
    there exists some $i \in I$
    such that $\lambda < \kappa_i$.
    Hence there exists a surjection $g : \kappa_i \to \lambda$.
    %
    Composing $f$ and $g$, we obtain a surjection $\kappa_i \to \kappa$.
    %
    Since $\kappa_i < \kappa$, there exists a surjection $\kappa \to \kappa_i$.

    With surjections in both directions, we apply the
    Bernstein-Cantor-Schr\"oder theorem to obtain a bijection
    $h : \kappa \to \kappa_i$.

    By composing the bijection $h$ with the bijection $f$, we obtain a
    bijection $\lambda \to \kappa_i$.
    But $\lambda < \kappa_i$, so this contradicts the assumption that
    $\kappa_i$ is a cardinal, meaning that it is equinumerous with no ordinal
    less than it.
\end{proof}

\section{The axiom of choice}

\begin{prop}{(Characterizations of Choice.)}
    The following are equivalent.
    %
    \begin{enumerate}
        \item
            The Axiom of Choice
            %
        \item
            Zorn's Lemma
            %
        \item
            The Well-Ordering Theorem
            %
    \end{enumerate}
\end{prop}

\begin{proof}
    \begin{description}
        \item[$1 \implies 2$.]
            Suppose not. Then we have a poset $\P$ in which every chain has an
            upper bound, $\P$ itself is unbounded.

            Let $T$ be the set of chains in $\P$.
            Construct the function $b : T \to \P$ which sends each chain to one
            of an upper bound strictly greater than all elements in the chain.
            The construction of this function uses choice.

            Define the transfinite sequence $(a_\alpha)$ by
            %
            \begin{align*}
                a_0 &= b(\emptyset) \\
                a_\alpha &= b(\setof{a_\beta \sth \beta < \alpha})
            \end{align*}
            %
            Note that successor stages are the same as limit stages.

            This sequence is defined over all ordinals, which form a proper
            class. Hence, it would imply that $\P$ is also a proper class,
            which contradicts that $\P$ is a set.

        \item[$2 \implies 3$.]
            Take an arbitrary set $X$.
            %
            The property of $R \subset Y \times Y$ being a well-order on $Y$ is
            a definable property, so we form the set $\mathcal{W}$ of all
            well-orders on \emph{subsets} of $X$.
            %
            We define a partial order $\leq$ on $\mathcal{W}$ by saying that
            $U \prec V$ if $V$ is an end-extension of $U$.

            Take an arbitrary chain $\mathcal{C} \subset \mathcal{W}$ and note
            that the union of its elements gives an upper bound for it. We
            apply Zorn's Lemma to obtain a maximal well-ordering $(W, \leq)$ on
            $X$.

            We claim that $W = X$.
            It is obvious that $W \subseteq X$,
            so it suffices to show that $X \subseteq W$.
            Suppose not, then there exists $x \in X$ such that $x \notin W$.
            But then we can define
            $\leq^\prime = \leq \union \setof{(y, x) : y \in W}$
            and note that $(W \union \setof{x}, \leq^\prime)$ is an
            end-extension of $W$, which contradicts the maximality of $W$.

            Hence, any set can be well-ordered.

        \item[$3 \implies 1$.]
            Let $\mathcal{X} = \setof{X_i \sth i \in I}$ be an indexed family
            of nonempty sets.
            Using the Axiom of Replacement, form an indexed family of
            well-orders by the Well-Ordering Theorem.
            Using Replacement a second time, take the unique minimal element of
            each well-order.
            This procedure builds a choice function on the family
            $\mathcal{X}$.
    \end{description}
\end{proof}

\section{Forcing}

\begin{defn}{(Model of set theory.)}
    We say that a set $M$ together with a membership relation $R$ models a
    statement $\phi$ in the language of set theory if using $R$ in place of
    $\in$ in $\phi$ gives a true sentence in $M$.
\end{defn}

\begin{defn}{(Countable transitive model.)}
    A model $(M, R)$ is \emph{transitive} if its underlying set $R$ is
    transitive, i.e. if $x \in y$ and $y \in M$, then $x \in M$.
    In other words, everything contained by elements of $M$ are themselves
    elements of $M$, namely sets.
    Hence, the sets in $M$ always contain \emph{other sets} and not weird
    things.

    A model is \emph{countable} if its underlying set $M$ is countable.
\end{defn}

Unless explicitly stated, anywhere a model $M$ pops up, it should be understood
as a countable transitive model satisfying enough of ZFC.

\begin{defn}{(Preservation of cardinals.)}
    A forcing notion $\P$ is said to \emph{preserve cardinals} if for any
    $\P$-generic filter $G \subset \P$ we have that if
    %
    \begin{equation*}
        M \models \stmt{$\alpha$ is a cardinal}
    \end{equation*}
    %
    then
    %
    \begin{equation*}
        M[G] \models \stmt{$\alpha$ is a cardinal}
    \end{equation*}
\end{defn}

\end{document}
