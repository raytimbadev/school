\documentclass[11pt,letterpaper]{article}

%%%%% METADATA %%%%%%%%%%%%%%%%%%%%%%%%%%%%%%%%%%%%%%%%%%%%%%%%%%%%%%%%%%%%%%%%

\author{Jacob Thomas Errington}
\title{Assignment \#3 \\ Honours set theory -- MATH 488}
\date{7 March 2017}

%%%%% MATH %%%%%%%%%%%%%%%%%%%%%%%%%%%%%%%%%%%%%%%%%%%%%%%%%%%%%%%%%%%%%%%%%%%%

\usepackage{amsmath,amssymb,amsthm}

% For double brackets
\usepackage{stmaryrd}

\newcommand{\interp}[1]{\left\llbracket #1 \right\rrbracket}

\newtheorem{prop}{Proposition}

\newcommand{\parens}[1]{\left(#1\right)}
\newcommand{\Union}{\bigcup}

\newcommand{\N}{\mathbb{N}}

% Some sets
\newcommand{\V}{\mathrm{V}}
\newcommand{\WF}{\mathrm{WF}}
\renewcommand{\L}{\mathrm{L}}
\newcommand{\Ord}{\mathbf{Ord}}
\DeclareMathOperator{\SeqOp}{Seq}
\newcommand{\Seq}[1]{\SeqOp{\parens{#1}}}
\DeclareMathOperator{\PowersetOp}{\mathcal{P}}
\newcommand{\Powerset}[1]{\PowersetOp{\parens{#1}}}
\DeclareMathOperator{\DefOp}{Def}
\newcommand{\Def}[1]{\DefOp{\parens{#1}}}

%%%%% FORMATTING %%%%%%%%%%%%%%%%%%%%%%%%%%%%%%%%%%%%%%%%%%%%%%%%%%%%%%%%%%%%%%

\usepackage[margin=2.0cm]{geometry}

\renewcommand{\thesection}{Question \#\arabic{section}}
\newcommand{\question}{\section}

% ordinary phi sucks.
\renewcommand{\phi}{\varphi}

%%%%% DOCUMENT %%%%%%%%%%%%%%%%%%%%%%%%%%%%%%%%%%%%%%%%%%%%%%%%%%%%%%%%%%%%%%%%

\begin{document}

\maketitle

\question{Club sets and $L$-structures}

\begin{prop}
    Let $\kappa$ be an uncountable regular cardinal and let $L$ be a countable
    language. Suppose $\parens{M_\alpha : \alpha < \kappa)}$ is an increasing
    and continuous sequence of $L$-structures.
    Let $M = \Union_{\alpha < \kappa}$.
    Then
    \begin{equation*}
        S = \{ \alpha \in \kappa : M_\alpha \prec M \}
    \end{equation*}
    is a club set.
\end{prop}

\begin{proof}
    First we show closure. Take arbitrary $A \subseteq S$ and let
    $\gamma = \lim A$. We want to show that $\gamma \in S$. Since $\kappa$ is
    regular, we can deduce that $\gamma < \kappa$. We need to see that
    $M_\gamma \prec M$. We apply the Tarski-Vaught Test. Take an arbitrary
    $L$-formula $\phi$ and suppose that there exists an $a \in M$ such that
    $M \models \phi(a)$. Since $\phi$ uses finitely many parameters from
    $M_\gamma$, we can find a $\beta < \gamma$ such that $\beta \in S$ and each
    parameter in $\phi$ is in is in $M_\beta$. Now since $M_\beta \prec M$,
    there is some $b \in M_\beta$ such that $M \models \phi(b)$. But
    $M_\beta \subseteq M_\gamma$, so $b \in M_\gamma$. Hence the Tarski-Vaught
    criterion is satisfied, so $M_\gamma \prec M$. Hence $\gamma \in S$. Since
    $\gamma$ is an arbitrary limit, this shows that $S$ is closed under taking
    limits.

    Next we show unboundedness. Take arbitrary $\alpha \in S$. Construct a
    sequence $(A_\gamma : \gamma < \kappa)$ with $A_0 = \alpha$, and whenever
    $\phi(\bar x, x)$ is an $L$-formula such that
    $M \models \exists x : \phi(\bar a, x)$,
    then there exists $a \in M_{i + 1}$
    such that $M \models \phi(\bar a , a)$.
    At limit stages, take the union of preceding $A_i$.
    By construction, this satisfies the Tarski-Vaught criterion.
    Now notice that we perform choice countably many times, once for each
    formula, so we aren't ``growing'' the model too much.
    Next notice that we perform this countable choosing
    for every $\alpha < \kappa$.
    Hence, $\lambda = \lim A_\gamma > \alpha$ and still in $S$.
\end{proof}

\question{Coding formulas in HF}

We establish a coding of formulas. Intuitively this is straightforward because
formulas are finite objects. We assign natural number tags to each kind of
object in a formula, and build up a nested tuple structure for the formula.

\begin{description}
    \item[Constant] $\phi = ``c''$ for a constant $c$ is coded by $(0, c)$;

    \item[Free variable] $\phi = ``x''$ for a variable $x$ is coded by $(1, n)$
        where $n$ is a natural number associated with $``x''$;

        Let $v : \mathrm{Var} \to \N$ denote the injection from variable names
        to natural numbers.

    \item[Connectives] $\phi = ``\psi_1 \land \psi_2''$
        is coded by $(2, \interp{\psi_1}, \interp{\psi_2})$, and other
        connectives use other tags, e.g. $\interp{\lor} = 3$.

    \item[Equality] $\phi = ``\psi_1 = \psi_2''$
        is coded by $(6, \interp{\psi_1}, \interp{\psi_2})$.

    \item[Membership] $\phi = ``\psi_1 \in \psi_2''$
        is coded by $(7, \interp{\psi_1}, \interp{\psi_2})$.

    \item[Quantifiers] ~

        $\phi = ``\exists x \in z : \psi''$
            is coded by $(8, v(x), v(z), \interp{\psi})$.

        $\phi = ``\exists x : \psi''$
            is coded by $(9, v(x), \interp{\psi})$.

        The universal quantifier uses tags $10$ and $11$.
\end{description}

This recursively defines the operator $\interp{\cdot}$ mapping formulas to
elements of HF, since tuples are in HF. Then, to say ``$\phi$ is a formula'' in
the language of set theory, we write $``\exists z : \interp{\phi} = z''$, i.e.
that $\phi$ has such a construction sequence. This is $\Sigma_1$ and hence
$\Delta_1$ because these construction sequences are unique.

For satisfaction, we just look at the semantics to construct a formula in set
theory corresponding to the meaning of $``x \models \phi(y)''$. Equality,
membership, logical connectives, and constants can be ported over wholesale,
and all unbounded quantifiers present in $\phi$ are converted to bounded
quantifiers over $x$. This results in a $\Delta_0$ formula.

\question{The language of set theory}

\begin{enumerate}
    \item $``\V = \WF''$

        To state this in the language of set theory, we will construct $\V$ and
        $\WF$ using existential quantification, and then state that they are
        equal.

        Let $\Seq{n}$ mean ``is a sequence of length $n$''.

        \begin{enumerate}
            \item Constructing $\V$. First we will see how to say
                $``x \in \V_\alpha''$, with $x$ and $\alpha$ being free
                variables.

                \begin{align*}
                    & \alpha \in \Ord \implies \\
                    & \exists s :
                        s \in \Seq{\alpha + 1} \\
                    & \land s(0) = \emptyset \\
                    & \land
                    \parens{
                        \forall \lambda :
                        (
                            \lambda \in \alpha
                            \land
                            \lambda \text{ is a limit }
                        )
                        \implies
                        s(\lambda) = \Union_{\beta < \lambda} s(\beta)
                    } \\
                    & \land
                    \parens{
                        \forall \beta :
                        \beta + 1 \leq \alpha
                        \implies
                        s(\beta + 1) = \Powerset{s(\beta)}
                    } \\
                    & \land
                    x \in s(\alpha)
                \end{align*}

                Then, we can construct $\V$ by taking a union over ordinals.

                \begin{equation*}
                    \forall x :
                    (\exists \alpha : x \in \V_\alpha)
                    \iff
                    x \in \V
                \end{equation*}

            \item Next, constructing $\WF$. This is very similar, since
                $x \in \WF$ is defined by the formula
                $\exists \alpha : \alpha \in \Ord \land x \in \V_\alpha$. Hence
                we can repeat the same construction for $\V_\alpha$ to build
                $\WF$ by
                \begin{align*}
                    \forall x :
                    (\exists \alpha : x \in \V_\alpha)
                    \iff
                    x \in \WF
                \end{align*}

        \end{enumerate}

        But these constructions work out to the same formulas! So to write
        $\V = \WF$ in the language of set theory is simply to write that any
        set whose members belong to a level of the von~Neumann hierarchy is
        equal to itself.

        \begin{equation*}
            \forall A :
            \parens{
                \forall x : x \in A
                \iff
                \parens{
                    \exists \alpha :
                    \alpha \in \Ord
                    \land
                    x \in \V_\alpha
                }
            }
            \implies
            A = A
        \end{equation*}

        But all things are equal to themselves by reflexivity, so the left-hand
        side of the implication doesn't really do anything.

    \item $``\V = \L''$

        We can reuse the same construction for $\V$, but now we must construct
        $L$. The construction is essentially the same, but when looking at the
        successor stage in the construction, rather than use the powerset, we
        use $\Def{s(\beta), s(\beta)}$, which we know has a (at most
        $\Delta_1$) formula in the language of set theory.

        Therefore, we some defining formula $\phi(A)$ for $\V$ and a defining
        formula $\psi(B)$ for $\L$. (Specifically, if we have $\phi(A)$, for
        some $A$ then $A$ \emph{is} $\V$.) So to write $\V = \L$, we can simply
        write
        \begin{equation*}
            \forall A :
            \forall B :
            \phi(A)
            \implies
            \psi(B)
            \implies
            A = B
        \end{equation*}
        to say that ``anything that is $\V$ is equal to anything that is $\L$''
        but of course there is just one such thing of each kind, so this means
        $``\V = \L''$.

\end{enumerate}

\end{document}
