\documentclass[11pt]{article}

\author{Jacob Thomas Errington}
\title{Assignment \#5\\Honours set theory -- MATH 488}
\date{31 March 2017}

\usepackage[geometry]{jakemath}

\begin{document}

\maketitle

\begin{prop}{(Knaster-Tarski fixed point theorem.)}
    Suppose $X$ is a set and $f : P(X) \to P(X)$ is a monotone function.
    Then, $f$ has a fixed point.
\end{prop}

\begin{proof}
    We want to find $A \in P(X)$ such that $A = f(A)$.
    %
    Let
    %
    \begin{equation*}
        \mathcal{A} = \setof{ A \in P(X) : A \subseteq f(A) }
    \end{equation*}
    %
    be a family of subsets of $X$. Let $A = \Union \mathcal{A}$.
    Since each ``piece'' of $A$ is a subset of its image, the whole $A$ is a
    subset of its image, so we have that $A \subseteq f(A)$, and so
    $A \in \mathcal{A}$.

    Note that by monotonicity we have that $f(A) \subseteq f(f(A))$, so
    %
    \begin{equation}
        \label{eq:rekt}
        f(A) \in \mathcal{A}
    \end{equation}

    Next, we want to see that $f(A) \subseteq A$.
    %
    Take arbitrary $x \in f(A)$
    and suppose for a contradiction that $x \notin A$.
    %
    Since $A$ is a union,
    we deduce that for any $B \in \mathcal{A}$,
    we have $x \notin B$.
    %
    But then since $f(A) \in \mathcal{A}$, we have that $x \notin f(A)$, which
    is a contradiction.
\end{proof}

\begin{prop}{(Banach's lemma.)}
    Suppose $f : X \to Y$ and $g : Y \to X$ are functions.
    Then there exists a set $C \subseteq X$ such that
    \begin{equation}
        \label{eq:wts-2}
        g(Y \setminus f(C)) = X \setminus C
    \end{equation}
\end{prop}

\begin{proof}
    Define $F : P(X) \to P(X)$ by $A \mapsto g \circ f (A)$.
    Then due to the fact that $f$ and $g$ are functions rather than arbitrary
    relations, it is clear that $F$ is monotonically decreasing.
    %
    We apply the fixed-point theorem to obtain a set $A \subseteq X$ such that
    $g \circ f (A) = A$, so $f(A) = g\inv(A) = B \subseteq Y$.

    Another way of thinking about the set $C$ we are looking for is as a piece
    of $X$ such that removing it from $X$ and removing its image from $Y$
    results in subsets of $X$ and $Y$ that are in bijection when $f$ and $g$
    are restricted to these subsets.

    The sets $A$ and $B$ correspond with this intuition, so let
    $C = X \setminus A$. We check \eqref{eq:wts-2}.

    Consider $f(X \setminus A)$. Since $f$ restricted to $A$ is a bijection
    with $B$, the function must send everything outside $A$ to things outside
    $B$, so $f(X \setminus A) \subseteq Y \setminus B$.
    But then, since $g$ restricted to $B$ is a bijection with $A$, things
    outside $B$ must be sent to things outside $A$.
    Hence,
    %
    \begin{equation*}
        g(Y \setminus f(C))
        = g(Y \setminus f(X \setminus A))
        = g(Y \setminus (Y \setminus B))
        = g(B)
        = A
        = X \setminus (X \setminus A)
        = X \setminus C
    \end{equation*}
    as required.
\end{proof}



\end{document}
