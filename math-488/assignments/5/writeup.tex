\documentclass[11pt]{article}

\author{Jacob Thomas Errington}
\title{Assignment \#5\\Honours set theory -- MATH 488}
\date{31 March 2017}

\usepackage[geometry]{jakemath}

\begin{document}

\maketitle

\begin{prop}{(Knaster-Tarski fixed point theorem.)}
    Suppose $X$ is a set and $f : P(X) \to P(X)$ is a monotone function.
    Then, $f$ has a fixed point.
\end{prop}

\begin{proof}
    We want to find $A \in P(X)$ such that $A = f(A)$.
    %
    Let
    %
    \begin{equation*}
        \mathcal{A} = \setof{ A \in P(X) : A \subseteq f(A) }
    \end{equation*}
    %
    be a family of subsets of $X$. Let $A = \Union \mathcal{A}$.
    Since each ``piece'' of $A$ is a subset of its image, the whole $A$ is a
    subset of its image, so we have that $A \subseteq f(A)$, and so
    $A \in \mathcal{A}$.

    Note that by monotonicity we have that $f(A) \subseteq f(f(A))$, so
    %
    \begin{equation}
        \label{eq:rekt}
        f(A) \in \mathcal{A}
    \end{equation}

    Next, we want to see that $f(A) \subseteq A$.
    %
    Take arbitrary $x \in f(A)$
    and suppose for a contradiction that $x \notin A$.
    %
    Since $A$ is a union,
    we deduce that for any $B \in \mathcal{A}$,
    we have $x \notin B$.
    %
    But then since $f(A) \in \mathcal{A}$, we have that $x \notin f(A)$, which
    is a contradiction.
\end{proof}

\begin{prop}{(Banach's lemma.)}
    Suppose $f : X \to Y$ and $g : Y \to X$ are functions.
    Then there exists a set $C \subseteq X$ such that
    \begin{equation}
        \label{eq:wts-2}
        g(Y \setminus f(C)) = X \setminus C
    \end{equation}
\end{prop}

\begin{proof}
    Define $F : P(X) \to P(X)$ by $A \mapsto g(Y \setminus f(X \setminus A))$.
    %
    Notice that $F$ is monotone:
    %
    \begin{align*}
        A &\subseteq B \\
        X \setminus A &\supseteq X \setminus B \\
        f(X \setminus A) &\supseteq f(X \setminus B) \\
        Y \setminus f(X \setminus A) &\subseteq Y \setminus f(X \setminus B) \\
        g(Y\setminus f(X\setminus A) &\subseteq g(Y\setminus f(X\setminus B))
    \end{align*}
    %
    Applying the fixed-point theorem, we find $A$ such that
    %
    \begin{equation*}
        g(Y \setminus f(X \setminus A)) = A
    \end{equation*}
    %
    Let $C = X \setminus A$, and substitute.
    %
    \begin{equation}
        \label{eq:prop-2}
        g(Y \setminus f(C)) = X \setminus C
    \end{equation}
\end{proof}

\begin{cor}{(Bernstein-Cantor-Schr\"oder.)}
    Suppose $f : X \to Y$ and $f : Y \to X$ are injections.
    Then, $X$ is equinumerous with $Y$.
\end{cor}

\begin{proof}
    Applying the lemma, we find $A \subseteq X$ such that
    %
    \begin{equation*}
        g(Y \setminus f(A)) = X \setminus A
    \end{equation*}
    %
    Note that $f \restrictto A$ is a bijection with a subset of $Y$,
    namely $B = Y \setminus f(A)$.
    Note that $g$ is injective and maps $B$ outside $A$.
    Hence, if we look at $(g \restrictto B)\inv$, it takes anything outside $A$
    into $Y$ in a one-to-one fashion.
    So $h = (f \restrictto A) \union (g \restrictto B)\inv$ is a bijection.
\end{proof}

\newcommand{\equisub}{\prec}

Write $A \equisub B$ to mean that $A \sim B^\prime$ for $B^\prime \subseteq B$.

\begin{prop}
    If $A \equisub B$ and $B \equisub A$.
    then, $A \sim B$.
\end{prop}

\begin{proof}
    Let $f_A : A \to B$ map points according to the equidecomposition of $A$
    and $B^\prime$ and let $f_B : B \to A$ map according to the
    equidecomposition from $B$ and $A^\prime$.
    Note then that $f_A(A) = B^\prime$ and $f_B(B) = A^\prime$.

    Define the map $F : P(A) \to P(A)$ by
    \begin{equation*}
        X \mapsto A \setminus f_B(B \setminus f_A(X))
    \end{equation*}
    and note that it is monotonically increasing, so it has a fixed point $X$.

    Let $X^\prime = A \setminus X$, $Y = f_A(X)$, and
    $Y^\prime = B \setminus Y$.

    Notice that
    we have $X = (A \setminus A^\prime) \union f_A(f_B(X))$.
    Hence,
    %
    \begin{equation*}
        X^\prime
        = f_B(B) \intersn (A \setminus f_B(f_A(X))
        = f_B(B) \setminus f_B(f_A(X))
        = f_B(B) \setminus f_B(Y)
        = f_B(Y^\prime)
    \end{equation*}

    This means we can split $A$ into $X$ and $X^\prime$ such that on $X$, we
    use $f_A$ and on $X^\prime$ we use $f_B\inv$. This gives an
    equidecomposition with $B$.
\end{proof}

\begin{prop}
    The circle is countably paradoxical with respect to the action of
    $SO_2(\R)$.
\end{prop}

\begin{proof}
    Define a relation $R$ on points $a$ and $b$ of the circle by saying that
    $aRb$ if $b$ can be obtained from $a$ by a rotation of a rational multiple
    of $2\pi$. Since there are countably many rationals, there are countably
    many equivalence classes. Note that this relation is an equivalence
    relation.

    Using a countable choice principle, form the set $A$ by choosing one
    representative from each equivalence class of this relation.
    Enumerate the rotations by $\setof{r_i}_{i\in\N}$ and let $A_i = r_i(A)$.
    The $A_i$ are pairwise disjoint and their union gives the circle, so they
    form a partition.
    Note that any $A_i$ and $A_j$ are congruent by a (irrational) rotation.
    Hence if we look at the even-numbered $A_i$, we may rotate each of them (by
    different amounts) to recover the full set $\setof{A_i}_{i\in\N}$.
    The same applies to the odd-numbered $A_i$. Hence the circle is
    equidecomposable into two disjoint (countable) subsets, each of them
    congruent to the whole circle.
    Hence the circle is paradoxical.
\end{proof}

\begin{cor}
    The unit ball in $R^2$ is countably paradoxical.
\end{cor}

\begin{proof}
    Each point on the circle can be identified with a point in the interval
    $(0, 2\pi]$ using $\sin$ and $\cos$. Hence, we can extend the equivalence
    relation we defined earlier to include any point in the unit ball.
    Hence, instead of looking at points, we look at line segments that are
    rotations of $(0, 2\pi]$ about the origin.

    But now we have a problem at the origin, because we needs "two origins" to
    reconstruct two balls, yet we only have one.

    To fix this, when we instead look at rotations of \emph{open} segments
    corresponding to $(0, 2\pi)$ to construct the interior of the ball. For the
    boundary, when we apply the argument from the previous proposition, in
    considering the even-numbered rotations, we discount the zeroth one, and
    instead use that point as the new origin by translating it to the center of
    the first ball that we construct. The second ball, built from the
    odd-numbered rotations of the boundary, can use the origin of the original
    ball.

    It is precisely because of this translation that we need to consider the
    full group of isometries and not just the rotations.
\end{proof}

\begin{prop}
    There does not exist a nonzero finitely additive measure defined on all
    subsets of $\R^3$ that is invariant under translations and rotations.
\end{prop}

\begin{proof}
    Suppose that such a measure did exist. From the Banach-Tarski paradox, we
    can decompose the ball into two identical copies using translations and
    rotations. By induction, the ball can be duplicated into $k$ many copies
    for any $k \in \N$ of our choosing. Hence, the measure must assign the same
    measure to one ball as it does to $k$ many of them. This is possible only
    if it assigns zero to them all.
\end{proof}

\end{document}
