\documentclass[11pt,letterpaper]{article}

\author{Jacob Thomas Errington}
\title{Assignment \#4\\Honours set theory -- MATH 488}
\date{10 March 2017}

\usepackage[margin=2.0cm]{geometry}
\usepackage{amsmath,amssymb,amsthm}

\newtheorem{prop}{Proposition}

\newcommand{\inv}{^{-1}}
\newcommand{\Union}{\bigcup}
\newcommand{\Intersn}{\bigcap}
\newcommand{\parens}[1]{\left(#1\right)}
\newcommand{\compl}{\overline}
\newcommand{\intersn}{\cap}
\DeclareMathOperator{\closureOp}{Cl}
\newcommand{\closure}[1]{\closureOp{\parens{#1}}}
\newcommand{\embedsto}{\hookrightarrow}

\begin{document}

\begin{prop}
    Suppose $X$ is a set and $A \subseteq X$. Then, the closure of $A$ in
    $\beta X$ is equal to $\hat A$.
\end{prop}

\begin{proof}
    Denote by $\mathcal{A}$ the set $A$ interpreted as the image of $A$ under
    the principal ultrafilter map.
    First, note that $\mathcal{A} \subseteq \hat A$.
    To see this, take $p \in \mathcal{A}$ a principal ultrafilter generated by
    some $a \in A$, so $\{ a \} \in \mathcal{A}$. Since $\{a\} \subseteq A$, we
    have that $A \in \mathcal{A}$ by the upwards closure property of filters.
    Then by definition of $\hat A$, we have $\mathcal{A} \in \hat A$.
    Next, we know that $\hat A$ is closed, and that a closed set contains
    another set if and only if it contains the closure of that set.
    Hence, $\closure{A} \subseteq \hat A$.

    Next, take arbitrary $p \in \hat A$.
    We want to show that $p \in \closure{A}$.
    To do so, we will show that $p$ is a point of closure of $A$,
    i.e. that every neighbourhood of $p$ contains a point of $A$.
    Take an arbitrary neighbourhood of $p$ and restrict it to a basic
    neighbourhood $\hat B$ such that $B \subseteq X$.
    Then we have that $A \in P$ and $B \in p$, so $A \intersn B \in p$.
    Hence, for any $x \in A \intersn B$,
    we have $u(x) \in u(A) \intersn \hat B$,
    where $u : X \embedsto \beta X$ is the principal ultrafilter map.
    In particular, $u(A) \intersn B \neq \emptyset$ for any neighbourhood of
    $p$. This shows that every neighbourhood of $p$ contains a point of $A$.
    Hence $\hat A \subseteq \closure{A}$.
\end{proof}

\begin{prop}
    Let $X$ be a set, $C$ a compact Hausdorff space, and $f : X \to C$ a
    function. Suppose $p$ is an ultrafilter on $X$. Then, there exists a unique
    point $z \in C$ such that for every neighbourhood $U$ of $z$ we have
    \begin{equation*}
        \{ x \in X : f (x) \in U \} \in p
    \end{equation*}
\end{prop}

\begin{proof}
    Suppose not. Then for all $z \in C$, there is a neighbourhood $U_z$ of $z$
    such that $\{ x \in X : f(x) \in U_z \} \notin p$.

    The collection of all such $U_z$ forms a cover of $C$. Hence there exists
    finite $Z_0 \subseteq C$ such that $\{U_z\}_{z \in Z_0}$ is a subcover of
    $C$. Let $A_z = f\inv (U_z)$. Then,
    \begin{equation*}
        \Union_{z \in Z_0} A_z
        = \Union_{z \in Z_0} f\inv (U_z)
        = f\inv \parens{ \Union_{z \in Z_0} U_z }
        = f\inv (C)
        = X
    \end{equation*}
    which shows that $X$ can be finitely partitioned. We take the complement of
    both sides. Since each piece of the partition is not in the ultrafilter,
    each piece's complement is. Hence, the intersection of all the complements,
    which is finite, is also in the ultrafilter, but this is empty set.
    \begin{align*}
        \compl{ \Union_{z \in Z_0} A_z }
        = \Intersn_{z \in Z_0} \compl{A_z}
        = \compl{X}
        = \emptyset
        \in p
    \end{align*}

    This is a contradiction.

    Next, we look at uniqueness. Suppose $z_1, z_2 \in C$ are such that
    $z_1 \neq z_2$. Since $C$ is Hausdorff, we have disjoint neighbourhoods
    $U_{z_1} \ni z_1$ and $U_{z_2} \ni z_2$. By definition of the limit,
    \begin{equation*}
        f\inv (U_{z_1}) \intersn f\inv (U_{z_2}
        = f\inv (U_{z_1} \intersn U_{z_2})
        = \emptyset
        \in p
    \end{equation*}
    which is a contradiction.
\end{proof}

\end{document}
