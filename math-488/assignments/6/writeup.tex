\documentclass[11pt]{article}

\author{Jacob Thomas Errington}
\title{Assignment \#6\\Honours Set Theory -- MATH 488}
\date{10 April 2017}

\usepackage[geometry]{jakemath}
\renewcommand{\P}{\mathbb{P}}
\renewcommand{\phi}{\varphi}

\DeclareMathOperator{\collOp}{Coll}
\newcommand{\coll}[1]{\collOp{\parens{#1}}}

\begin{document}

\maketitle

\begin{prop}
    Let $M$ be a CTM and $\P \in M$ be a forcing notion.
    If $G \subseteq \P$ is $\P$-generic over $M$, then $G$ is
    an ultrafilter.
\end{prop}

\begin{proof}
    Define $C_p = \setof{ q \in \P : q \leq p }$ the principal ideal generated
    by $p$
    and $C^\prime_p = \setof{ q \in \P : q \incompatible p }$.
    Let $D_p = C_p \union C^\prime_p$.

    Note that for any $p \in \P$, the set $D_p$ is dense.
    To see this, consider an arbitrary $r \in \P$.
    If $r$ is compatible with $p$, then we will have some
    $q \in C_p \subseteq D_p$ such that $q \leq r$.
    If $r \incompatible p$, then $r \in C^\prime_p$.

    Now consider some arbitrary $p \in \P$ and suppose $p \notin G$.
    Since $D_p$ is dense, we have that $G \intersn D_p \neq \emptyset$.
    Since $G$ is is a filter, it is upwards closed, which implies that
    $G \intersn C_p = \emptyset$. (Otherwise, we would have $p \in G$, a
    contradiction.)
    Hence, $G \intersn C^\prime_p \neq \emptyset$.
    Let $p^\prime \in G \intersn C^\prime_p$.

    Suppose there exists $G^\prime \supset G$ such that $p \in G^\prime$.
    This is a contradiction, as two incompatible elements of $\P$, namely
    $p^\prime$ and $p$, belong to $G^\prime$.
    Therefore, $G$ is maximal and hence an ultrafilter.
\end{proof}

\begin{prop}
    Let $M$ be a CTM and $\P \in M$ be a forcing notion.
    Then,
    $G \subseteq \P$ is $\P$-generic over $M$ if and only if $G$ is a filter
    and for every maximal antichain $A \subseteq \P$ such that $A \in M$ we
    have that $G \intersn A \neq \emptyset$.
\end{prop}

\begin{proof}
    Suppose $G \subseteq \P$ is generic.
    Take an arbitrary maximal antichain $A$ such that $A \in M$.
    Suppose for a contradiction that $G \intersn A = \emptyset$.
    Let $a \in A$ and consider $D_a$ as in the previous proof.
    Since $D_a$ is dense, we have $G \intersn D_a \neq \emptyset$.
    Since $a \notin G$, then $G \intersn C^\prime_a \neq \emptyset$.
    But $C^\prime_a$ is the set of all elements in $\P$ that are incompatible
    with $a$, so $C^\prime_a \subseteq A$. This is a contradiction.

    Suppose $G \subseteq \P$ is a filter and that $G$ meets every maximal
    antichain in $\P$ that can be formed in $M$.
    Take an arbitrary dense set $D \subseteq \P$.
    Well-order
    $D = (p_\alpha : \alpha < \lambda)$.

    Then, we inductively construct a maximal antichain $A$ by going through the
    sequence and selecting each $p_\alpha$ that is incompatible with all
    elements we have so far put in $A$. We start by putting $p_0 \in A$.

    We claim that this antichain is maximal.
    Suppose not.
    Then there is $p \notin A$ incompatible with everything in $A$.
    Therefore $p$ must not have been in $D$ to begin with, else we would have
    put it in $A$.
    Then by density we can find some $p_\alpha \in D$ such that
    $p_\alpha \leq p$.
    Since $p$ is incompatible with everything in $A$, so is $p_\alpha$,
    so $p_\alpha \in A$.
    As $p$ is compatible with $p_\alpha$, this contradicts the assumption that
    $p$ was incompatible with everything in $A$.

    Since $G$ meets every maximal antichain, it meets $A \subseteq D$.
    Since $D$ is an arbitrary dense subset of $\P$, this means $G$ is a
    $\P$-generic filter.
\end{proof}

\begin{prop}
    Let $\P$ be a forcing notion.
    Suppose $p \in \P$ and $\phi(x)$ is a formula in the forcing language.
    If $p \forces \exists x : \phi(x)$,
    then there exists a name $\dot x$ such that $p \forces \phi(\dot x)$.
\end{prop}

\begin{proof}
    Suppose $p \forces \exists x : \phi(x)$.
    By definition, the set
    %
    \begin{equation*}
        E = \setof{ q \leq p \sth \exists \dot x : q \forces \phi(\dot x) }
    \end{equation*}
    %
    is dense below $p$.

    By the same inductive procedure as in the previous proof, we build a
    maximal antichain $A \subseteq E$ (below $p$). In other words, for any
    $q \leq p$, we have that $q$ is compatible with some $r \in A$. (In fact,
    this $r$ is unique.)

    For each $q \in E$, we have that
    %
    \begin{equation*}
        \exists \dot x : q \forces \phi(\dot x)
    \end{equation*}
    %
    so write $\dot x_q$ for the name witnessing the existential corresponding
    to $q$.

    We would like to construct a name $\dot y$ that is forced to be equal to
    $\dot x_q$ for each $q \in A$.
    With such a name in hand, we would have
    %
    \begin{equation*}
        q \forces \phi(\dot x_q) \land \dot x_q = \dot y
    \end{equation*}
    %
    from which we may deduce $q \forces \phi(\dot y)$.
    Let's see how we can build $\dot y$.

    Note that each $q \in A$ identifies a ``cone'' below it in $E$ of
    conditions that force the existence of a name such that $\phi$ holds for
    that name. We want to look at each cone and form a name $E_q$ whose domain
    consists of all those $\dot z$ in the domain of $\dot x_q$ that are forced
    to belong to $\dot x_q$ by an $r \leq q$.
    More precisely, we define
    %
    \begin{equation*}
        E_q = \setof{
            (\dot z, r) \sth
            r \leq q
            \land
            r \forces \dot z \in \dot x_q
            \land
            \dot z \in \dom{\dot x_q}
        }
    \end{equation*}

    Then, we claim that $\dot y = \Union_{q \in A} E_q$ is as needed,
    i.e. that each $q \in A$ forces $\phi(\dot y)$.

    To see this, we look at an arbitrary $q \in A$ and an arbitrary generic
    filter $G \ni q$ and check that the evaluations are the same, specifically:
    %
    \begin{equation*}
        \dot y / G \qmark{=} \dot x_q / G
    \end{equation*}

    Take an arbitrary element $z \in \dot y / G$.
    It must have the form $\dot z / G$
    for some $r \in G$.
    %
    Since $\dot y$ is a union,
    $(\dot z, r)$ comes from some $E_s$,
    so $r \leq s$ and
    %
    \begin{equation}
        \label{eq:r-forces-z-in-x-s}
        r \forces \dot z \in \dot x_s
    \end{equation}
    %
    Since $A$ is an antichain, we must have that $s = q$;
    otherwise, we would have two incompatible elements belonging to $G$.
    %
    By rewriting $s$ to $q$ in \eqref{eq:r-forces-z-in-x-s}, we have
    $r \forces \dot z \in \dot x_q$.
    %
    But then, since $r \in G$, we have $z \in x_q$.
    This demonstrates that $\dot y / G \subseteq \dot x_q$.

    Next, take an arbitrary element $z \in \dot x_q / G$.
    It must have the form $\dot z / G$
    for some $\dot z \in \dot x_q$,
    and furthermore there must be some $p \in G$ such that
    $p \forces z \in x_q$.
    Since $p, q \in G$, let $r \in G$ such that $r \leq p$ and $r \leq q$.
    But then since $r \leq q$
    and $r \forces z \in x_q$
    and $\dot z \in \dom{x_q}$,
    we have that $(\dot z, r) \in \dot y$ by definition.

    Finally, we have a name $\dot y$ such that $q \forces \phi(\dot y)$ for
    each $q$ in the maximal (below $p$) antichain $A$ that we started with.
    However, we want to show that $p \forces \phi(\dot y)$, so
    suppose not.

    Then there exists some $r \leq p$ such that $r \forces \neg \phi(\dot y)$.
    By the density of $E$ below $p$, find $q^\prime \in E$ such that
    $q^\prime \leq r$ and
    such that there exists a name $\dot x$ and $q^\prime \forces \phi(\dot x)$.
    %
    Since $q^\prime$ is stronger than $r$, we also have
    $q^\prime \forces \neg \phi(\dot y)$.
    %
    Hence, for every $q \in A$, since $q \forces \phi(\dot y)$, we must have
    that $q^\prime \incompatible q$.
    But this contradicts the maximality of $A$.
\end{proof}

\begin{prop}
    Let $\kappa$ be an uncountable cardinal.
    Write $\P = \coll{\omega, \kappa}$ for the partial order of finite partial
    functions from $\omega$ to $\kappa$, ordered by reverse inclusion.
    If $G \subset \P$ is a $\P$-generic filter on $\verum$, then
    %
    \begin{equation*}
        \verum[G] \models \stmt{$\kappa$ is countable}
    \end{equation*}
    %
    and all cardinals in $\verum$ that are bigger than $\kappa$ remain
    cardinals in $\verum[G]$.
\end{prop}

\newcommand{\m}{\verum[G]}

\begin{proof}
    We want to show that in $\m$, there exists a surjection from $\omega$ to
    $\kappa$.

    Let $g = \Union G$.
    We will see that $g : \omega \to \kappa$ is a total function, and
    furthermore is a surjection.

    First, define
    %
    \begin{equation*}
        D_n = \setof{ f \in \P \sth n \in \dom{f} }
    \end{equation*}
    %
    for any $n \in \omega$.

    For each $n \in \omega$, the set $D_n$ is dense.
    To see this take $f \in \P$.
    %
    \begin{itemize}
        \item
            If $n \in \dom{f}$, then there's nothing to do as $f \in D_n$.
            %
        \item
            If $n \notin \dom{f}$,
            then we have $D_n \ni f^\prime = f[n \mapsto 0] \leq f$.
            %
    \end{itemize}

    Since $G \intersn D_n \neq \emptyset$ for all $n$, we have that
    $n \in \dom{g}$ for all $n \in \omega$, so $g$ is in fact a total function.

    Second, we would like $g$ to be a surjective function, so we encode this
    property as a dense set.
    %
    Define
    %
    \begin{equation*}
        J_{\alpha} = \setof{
            f \in \P \sth \exists n \in \omega : f(n) = \alpha
        }
    \end{equation*}
    %
    for any $\alpha < \kappa$.

    To see that such a set is dense, take arbitrary $f \in \P$.
    %
    \begin{enumerate}
        \item
            If there exists $n \in \omega$ such that $f(n) = \alpha$,
            then there's nothing to do as $f \in J_\alpha$.
            %
        \item
            Suppose there is no $n \in \omega$ such that $f(n) = \alpha$.
            Since $f$ has a finite domain, let $m \in \omega$ be such that
            $m > n$ for all $n \in \dom{f}$.
            Define $f^\prime \in J_{\alpha}$ such that
            %
            \begin{equation*}
                f^\prime = f[m \mapsto \alpha]
            \end{equation*}
            %
            and notice that $f^\prime \leq f$.
    \end{enumerate}

    Since $G \intersn J_\alpha$ for all $\alpha < \kappa$, we have that
    for every $\alpha < \kappa$,
    there exists $n \in \omega$ such that $g(n) = \alpha$.
    %
    Consequently, $g$ is surjective, so $\kappa \leq \omega$, and hence
    $\kappa$ must be countable.

    Next, we want to see that every cardinal $\lambda > \kappa$ in $\verum$ is
    still a cardinal in $\m$. To do this, we will generalize the idea of a
    poset being CCC to the idea of $\kappa$-CC to mean that there is no
    antichain of cardinality $\kappa^+$ in the poset. In particular, we claim
    that $\P$ is $\kappa$-CC.

    To see this, suppose not, so $A = ( p_\alpha \sth \alpha < \kappa^+ )$ is
    an antichain in $\P$.
    %
    We define
    %
    \begin{equation*}
        B = \setof{ \dom{p_\alpha} \sth \alpha < \kappa^+ }
    \end{equation*}
    %
    and note that there are $\kappa^+$-many (finite) domains in this family
    because for any $N \subseteq \omega$ we have that
    $\setof{ f \in \coll{\omega, \kappa} \sth \dom{f} = N }$
    has cardinality $\kappa$.

    We apply the Sunflower Lemma to find a $\Delta$-system
    $B^\prime \subset B$ with cardinality $\kappa^+$ and root
    $R \subset B^\prime$.
    %
    For each $b \in B^\prime$, choose $p_b \in A$ such that $\dom{p_b} = b$
    and notice that
    %
    \begin{equation*}
        \abs{
            \setof{f \in \coll{\omega, \kappa} \sth \dom{f} = R}
        }
        = \kappa
    \end{equation*}
    %
    Hence, there exists $f : R \to \kappa$ such that for $\kappa^+$-many
    $b \in B^\prime$, we have that $p_b \restrictto R = f$.
    Let $b_1, b_2 \in B^\prime$ and notice that $p_{b_1}$ is compatible
    with $p_{b_2}$; their union is a valid finite partial function, since they
    agree on $R$, and forms a lower bound for them.
    But $p_{b_1}$ and $p_{b_2}$ belong to an antichain, so this is a
    contradiction.

    Therefore, $\P = \coll{\omega, \kappa}$ is $\kappa$-CC.
    Next we want to generalize the result that CCC forcing notions preserve
    uncountable cardinals to a result that $\kappa$-CC forcing notions
    preserve cardinals bigger than $\kappa$.

    Suppose $\beta \in V$ and
    $V \models \stmt{$\beta > \kappa$ and $\beta$ is a cardinal}$.
    Suppose for a contradiction that
    $\m \models \stmt{$\beta > \kappa$ but $\beta$ is not a cardinal}$.
    Then, we have an ordinal $\alpha < \beta$ and a function $f \in \m$ such
    that $\m \models \stmt{$f : \alpha \to \beta$ and $f$ is a bijection}$.

    Since $f \in \m$, there exists a name $\dot f$ such that $f = \dot f / G$.
    Hence, by the Forcing Theorem, there exists $p \in G$ such that
    %
    \begin{equation*}
        p \forces
        \dot f : \check \alpha \to \check \beta
        \text{ and $\dot f$ is a bijection}
    \end{equation*}

    Define the function $F : \alpha \to \pset{\beta}$ such that

    \begin{equation*}
        F(\gamma) = \setof{
            \delta < \beta
            \sth
            \exists q \leq p : q \forces \dot f(\check \gamma) = \check \delta
        }
    \end{equation*}

    We first claim that $\m \models f(\gamma) \in F(\gamma)$.
    To see this, suppose that $\m \models f(\gamma) = \delta$.
    Then by the Forcing Theorem there exists $q \in G$ (assume without loss of
    generality that $q \leq p$) such that
    %
    \begin{equation*}
        q \forces \dot f(\check \gamma) = \check \delta
    \end{equation*}
    %
    so $\delta \in F(\gamma)$ by definition.

    We next claim that $\abs{F(\gamma)} \leq \kappa$.
    Notice that if $\delta_1, \delta_2 \in F(\gamma)$, then there are
    $q_1, q_2 \leq p$ such that
    %
    \begin{align*}
        q_1 &\forces \dot f(\check \gamma) = \delta_1 \\
        q_2 &\forces \dot f(\check \gamma) = \delta_2
    \end{align*}
    %
    but then $q_1 \incompatible q_2$ by the functionality of $f$.
    %
    Therefore, for any $\delta \in F(\gamma)$,
    choose $q_\delta$
    such that $q_\delta \forces f(\check \gamma) = \check \delta$
    so $\setof{q_\delta \sth \delta \in F(\gamma)}$ forms an antichain in
    $\coll{\omega, \kappa}$.
    %
    This antichain must have cardinality $\kappa$ since we showed that
    $\coll{\omega,\kappa}$ is $\kappa^+$-CC.

    Finally, in $\verum$ we can see that
    %
    \begin{equation*}
        \abs{\Union_{\gamma < \alpha} F(\gamma)}
        \leq \abs{\alpha} \cdot \kappa
        = \abs{\alpha} < \beta
    \end{equation*}
    %
    Hence, subtracting this union from $\beta$ leaves us with a nonempty set.
    Let
    %
    \begin{equation*}
        \delta \in \beta \setminus \Union_{\gamma < \alpha} F(\gamma)
    \end{equation*}
    %
    Then, $\m \models \delta \notin \rng{f}$.
    Hence $f$ is not a surjection in $\m$, which contradicts that $\m$ thinks
    that $f$ is a bijection.
\end{proof}

\begin{prop}
    Suppose $\P$ is a $\sigma$-closed forcing notion and $G \subset \P$ is a
    generic filter. Then the forcing does not add any reals.
\end{prop}

\begin{proof}
    Take an arbitrary subset $A \subseteq \omega$ in $\m$.
    We want to show that this subset is already in $\verum$.

    Since $A \in \m$, we must have a name for it, so $\dot A / G = A$.
    Define the set
    %
    \begin{equation*}
        E = \setof{
            p \in \P
            \sth
            \exists C \subseteq \omega, C \in V: p \forces \dot A = C
        }
    \end{equation*}
    %
    to be those conditions forcing that $A$ was already in $\verum$.

    We claim that $E$ is dense in $\P$. To see this, let $q \in \P$.
    We define a countable decreasing sequence of conditions $(p_n)_{n\in\N}$
    such that
    %
    \begin{align*}
        p_i &\forces \check \i \in \dot A \\
            &\text{or} \\
        p_i &\forces \check \i \notin \dot A
    \end{align*}
    %
    which is possible because generic filters are also ultrafilters.
    As for ensuring that the sequence is decreasing, we use the downwards
    directedness of the generic filter to accomplish this at each step.
    We furthermore use downward-directedness to ensure that $q \geq p_0$.
    %
    By the $\sigma$-closedness of $\P$, there exists a lower bound $p \in \P$
    for the sequence, such that $p \leq p_i$ for all $i \in \omega$.
    Hence, this $p \in E$ and $p \leq q$. This demonstrates the density of $E$.

    Since $E$ is dense, it meets $G$, so there exists $p \in G$ such that
    $p \forces \dot A = C$ for some $C \subseteq \omega$ such that
    $C \in \verum$.

    Hence, $\dot A / G \in V$, so the forcing does not add any reals.
\end{proof}

\end{document}
