\documentclass[11pt]{article}

\author{Jacob Thomas Errington}
\title{Assignment \#6\\Honours Set Theory -- MATH 488}
\date{10 April 2017}

\usepackage[geometry]{jakemath}
\renewcommand{\P}{\mathbb{P}}
\renewcommand{\phi}{\varphi}

\begin{document}

\maketitle

\begin{prop}
    Let $M$ be a CTM and $\P \in M$ be a forcing notion.
    If $G \subseteq \P$ is $\P$-generic over $M$, then $G$ is
    an ultrafilter.
\end{prop}

\begin{proof}
    Define $C_p = \setof{ q \in \P : q \leq p }$
    and $C^\prime_p = \setof{ q \in \P : q \perp p }$.
    Let $D_p = C_p \union C^\prime_p$.

    Note that for any $p \in \P$, the set $D_p$ is dense.
    To see this, consider an arbitrary $r \in \P$.
    If $r$ is compatible with $p$, then we will have some
    $q \in C_p \subseteq D_p$ such that $q \leq r$.
    If $r \perp p$, then $r \in C^\prime_p$.

    Now consider some arbitrary $p \in \P$ and suppose $p \notin G$.
    Since $D_p$ is dense, we have that $G \intersn D_p \neq \emptyset$.
    Since $G$ is is a filter, it is upwards closed, which implies that
    $G \intersn C_p = \emptyset$. (Otherwise, we would have $p \in G$, a
    contradiction.)
    Hence, $G \intersn C^\prime_p \neq \emptyset$.
    Let $p^\prime \in G \intersn C^\prime_p$.

    Suppose there exists $G^\prime \supset G$ such that $p \in G^\prime$.
    This is a contradiction, as two incompatible elements of $\P$, namely
    $p^\prime$ and $p$, belong to $G^\prime$.
    Therefore, $G$ is maximal and hence an ultrafilter.
\end{proof}

\begin{prop}
    Let $M$ be a CTM and $\P \in M$ be a forcing notion.
    Then,
    $G \subseteq \P$ is $\P$-generic over $M$ if and only if $G$ is a filter
    and for every maximal antichain $A \subseteq \P$ such that $A \in M$ we
    have that $G \intersn A \neq \emptyset$.
\end{prop}

\begin{proof}
    Suppose $G \subseteq \P$ is generic.
    Take an arbitrary maximal antichain $A$ such that $A \in M$.
    Suppose for a contradiction that $G \intersn A = \emptyset$.
    Let $a \in A$ and consider $D_a$ as in the previous proof.
    Since $D_a$ is dense, we have $G \intersn D_a \neq \emptyset$.
    Since $a \notin G$, then $G \intersn C^\prime_a \neq \emptyset$.
    But $C^\prime_a$ is the set of all elements in $\P$ that are incompatible
    with $a$, so $C^\prime_a \subseteq A$. This is a contradiction.

    Suppose $G \subseteq \P$ is a filter and that $G$ meets every maximal
    antichain in $\P$ that can be formed in $M$.
    Take an arbitrary dense set $D \subseteq \P$.
    Well-order
    $D = (p_\alpha : \alpha < \lambda)$.

    Then, we inductively construct a maximal antichain $A$ by going through the
    sequence and selecting each $p_\alpha$ that is incompatible with all
    elements we have so far put in $A$. We start by putting $p_0 \in A$.

    We claim that this antichain is maximal.
    Suppose not.
    Then there is $p \notin A$ incompatible with everything in $A$.
    Therefore $p$ must not have been in $D$ to begin with, else we would have
    put it in $A$.
    Then by density we can find some $p_\alpha \in D$ such that
    $p_\alpha \leq p$.
    Since $p$ is incompatible with everything in $A$, so is $p_\alpha$,
    so $p_\alpha \in A$.
    As $p$ is compatible with $p_\alpha$, this contradicts the assumption that
    $p$ was incompatible with everything in $A$.

    Since $G$ meets every maximal antichain, it meets $A \subseteq D$.
    Since $D$ is an arbitrary dense subset of $\P$, this means $G$ is a
    $\P$-generic filter.
\end{proof}

\begin{prop}
    Let $\P$ be a forcing notion.
    Suppose $p \in \P$ and $\phi(x)$ is a formula in the forcing language.
    If $p \forces \exists x : \phi(x)$,
    then there exists a name $\dot x$ such that $p \forces \phi(\dot x)$.
\end{prop}

\begin{proof}
    Suppose $p \forces \exists x : \phi(x)$.
    By definition, the set
    %
    \begin{equation*}
        A = \setof{ q \leq p \sth \exists \dot x : q \forces \phi(\dot x) }
    \end{equation*}
    %
    is dense below $p$.

    By the same inductive procedure as in the previous proof, we build a
    maximal antichain $C \subseteq A$ (below $p$).
    Form the set of names witnessing the existentials in $C$,
    and let $\dot x$ be the union of this set of witnesses.
    We claim that $p \forces \phi(\dot x)$.
\end{proof}

\end{document}
