\documentclass[11pt]{article}

\usepackage[geometry]{jakemath}

\author{Jacob Thomas Errington}
\title{Van der Waerden Theorem}
\date{}

\begin{document}

\maketitle

\begin{prop}
    For any $n, m \in \N$, there exists $N \in \N$ such that for any $d > N$,
    if $d$ is partitioned into $m$ many pieces, then one of the pieces contains
    an arithmetic progression of size $n$.
\end{prop}

\begin{proof}
    Suppose we have a combinatorial line $L \subset n^d$ generated by the
    variable word $w$.
    Let $a$ be the number of wildcards $*$ present in $w$.
    Let $b$ be the sum of the constant components.
    Then each $l \in L$ identifies a function $f_l(i) = i \cdot a + b$.
    Note each $l \in L$ is in fact associated with the integer $i_l$ such that
    $w[i_l] = l$, so we can construct the set
    \begin{equation*}
        M = \setof{ f_l(i_l) \sth l \in L }
    \end{equation*}
    This set is precisely an arithmetic progression of $n$ items.

    By the finitary Hales-Jewett theorem, there exists an $N \in \N$ such that
    for any $d > N$, if $n^d$ is finitely partitioned into $m$ many pieces,
    then one of the pieces contains a combinatorial line.

    We set the $N$ we're looking for to the one given by the Hales-Jewett
    theorem, and the for any $d > N$ we use the remainder of the theorem to
    obtain a combinatorial line by choosing an arbitrary partition of $n^d$
    into $m$ many pieces.
\end{proof}

\end{document}
