\documentclass[11pt]{article}

\usepackage[geometry]{jakemath}

\author{Jacob Thomas Errington}
\title{Hales-Jewett Corollaries}
\date{}

\newcommand{\Seq}{\operatorname{Seq}}

\begin{document}

\maketitle

\begin{prop}{(Van der Waerden.)}
    For any $n, m \in \N$, there exists $N \in \N$ such that for any $d > N$,
    if $d$ is partitioned into $m$ many pieces, then one of the pieces contains
    an arithmetic progression of size $n$.
\end{prop}

\begin{proof}
    Suppose we have a combinatorial line $L \subset n^d$ generated by the
    variable word $w$.
    Let $a$ be the number of wildcards $*$ present in $w$.
    Let $b$ be the sum of the constant components.
    Then each $l \in L$ identifies a function $f_l(i) = i \cdot a + b$.
    Note each $l \in L$ is in fact associated with the integer $i_l$ such that
    $w[i_l] = l$, so we can construct the set
    \begin{equation*}
        M = \setof{ f_l(i_l) \sth l \in L }
    \end{equation*}
    This set is precisely an arithmetic progression of $n$ items.

    By the finitary Hales-Jewett theorem, there exists an $N \in \N$ such that
    for any $d > N$, if $n^d$ is finitely partitioned into $m$ many pieces,
    then one of the pieces contains a combinatorial line.

    We set the $N$ we're looking for to the one given by the Hales-Jewett
    theorem, and the for any $d > N$ we use the remainder of the theorem to
    obtain a combinatorial line by choosing an arbitrary partition of $n^d$
    into $m$ many pieces.
\end{proof}

\begin{prop}{(Finitary Hales-Jewett.)}
    For any $n, m \in \N$, there exists $N \in \N$ such that for any $d > N$,
    if $n^d$ is partitioned into $m$ many pieces, then one of the pieces
    contains a combinatorial line.
\end{prop}

\begin{proof}
    Suppose not.
    Then for every $N \in \N$, there exists $d > N$ and a ``bad partition'' of
    $n^d$ into $m$ many pieces such that none of the pieces contains a
    combinatorial line.

    For simplicity, if I ignore the $d$ and assume that for every $N \in \N$,
    there exists a bad partition of $n^N$ into $m$ many pieces such that none
    of the pieces contains a combinatorial line, then I can take the union of
    across pieces of all the bad partitions to form a partition of $\Seq(n)$
    into $m$ many pieces.
    (What I mean is that if we enumerate the pieces by $P_{i,N}$ for the
    $i$\th{} piece of the bad partition for length $N$, then we can partition
    $\Seq(n)$ by taking the pieces $P_i = \Union_{k < \omega} P_{i,k}$.)
    %
    Then by the infinitary Hales-Jewett theorem, there is a
    combinatorial line in one of the pieces. This line consists of words of
    length $l$, meaning that it is contained by a piece of the bad partition at
    stage $l$. This is a contradiction.

    However, I cannot see how to relax the assumption that there exists a bad
    partition for every $N$ back to the assumption that for every $N$, there is
    some $d > N$ with a bad partition.
\end{proof}

\end{document}
