\documentclass{beamer}

\usepackage[utf8]{inputenc}

\author{Jacob Thomas Errington}
\title{
    Implementing lazy functional languages on stock hardware:
    the Spineless Tagless G-machine
}
\institute{McGill University}
\date{27 October 2016}

\usepackage{listings}
\usepackage{amsmath,amssymb,amsthm}
\usepackage{tikz}

\usetikzlibrary{arrows,shapes,calc}

\tikzstyle{every picture}+=[remember picture]
\everymath{\displaystyle}

\DeclareMathOperator{\ErrorOp}{\mathtt{error}}
\newcommand{\Error}[1]{\ErrorOp{\, \text{#1}}}

\begin{document}

\frame{\titlepage}

\begin{frame}
    \frametitle{About me}

    \begin{itemize}
        \item Third-year Mathematics \& Computer Science student.
        \item Last semester:
            COMP 520 (Compiler Design),
            COMP 527 (Logic and Computation),
            COMP 531 (Advanced Theory of Computing).
        \item Our compiler Goto: written in Haskell using some advanced
            techniques.
        \item Research interests: applications of logic to programming
            languages, techniques for ensuring code correctness by
            construction.
    \end{itemize}
\end{frame}

\begin{frame}
    \frametitle{About this presentation}

    A whirlwind tour of how the Glasgow Haskell Compiler (GHC), as presented in
    the paper, represents functional features, and how Haskell code is
    translated into STG.

    \tableofcontents
\end{frame}

\section{Introduction}

\subsection{Motivation}

\begin{frame}[fragile,shrink]
    \frametitle{Motivating example}

    \begin{lstlisting}
module Main where

data Nat = Z | S Nat

instance Show Nat where
    show n = show (natToInt n) where
        natToInt Z = 0
        natToInt (S n) = 1 + natToInt n

three, five :: Nat
three = S (S (S Z))
five = S (S (S (S (S Z))))

plus :: Nat -> Nat -> Nat
plus Z m = m
plus (S n) m = S (plus n m)

main :: IO ()
main = print (three `plus` five)
-- "8" appears on standard out
    \end{lstlisting}
\end{frame}

\begin{frame}
    \frametitle{Features of Haskell}

    \begin{itemize}
        \item
            Can define recursive, algebraic datatypes.
        \item
            Can implement recursive functions.
        \item
            Can pattern match on datatypes; values ``remember'' how they were
            constructed.
        \item
            Overloading via typeclasses: the print function uses the
            \texttt{Show} instance.
        \item
            Purely functional: side effects marked via the IO type constructor
        \item
            Lazy: we can forget about the evaluation order for the most part,
            and ``let the computer figure it out''.
    \end{itemize}

    We would like to compile a language with these kinds of features to
    efficient native code.
\end{frame}

\subsection{Background}

\begin{frame}
    \frametitle{Transforming functional languages}

    \begin{itemize}
        \item
            Simple means simple.

        \item
            Lazy languages are more amenable to transformation than strict
            languages.
    \end{itemize}
\end{frame}

\begin{frame}
    \frametitle{$\beta$-reduction}

    A strict functional language uses the \alert{call-by-value} reduction
    strategy for its underlying lambda calculus. A lazy functional language
    uses \alert{call-by-name}.

    \begin{block}{Call-by-name}
        Function application is performed before evaluating the arguments.
    \end{block}

    \begin{block}{Call-by-value}
        Arguments are evaluated before function application.
    \end{block}
\end{frame}

\begin{frame}
    \begin{equation}
        (\lambda x. 5) (\Error{``oh no''})
        \label{eq:cbn-vs-cbv}
    \end{equation}

    \begin{block}{Question}
        What does equation \ref{eq:cbn-vs-cbv} reduce to in call-by-name? In
        call-by-value?
    \end{block}
\end{frame}

\begin{frame}
    \frametitle{Consequences of call-by-name reduction}

    As a transformation, $\beta$-reduction (applying a function to its
    arguments) can always be performed statically in a lazy language.

    In a strict language, it can only be performed if the function
    \alert{evaluates its argument}.
\end{frame}

\subsection{Compiling Haskell}

\begin{frame}
    \frametitle{Compilation route for Haskell}

    \begin{enumerate}
        \item Source language: Haskell, a strongly-typed, non-strict,
            purely-functional language.
        \item Haskell to Core
            \begin{itemize}
                \item Syntactic sugar: eliminated
                    % Do notation, proc notation, etc. is replaced by its
                    % equivalent semantics given by functions.
                \item Overloading: made explicit
                    % Typeclasse instances are normally plumbed around
                    % implicitly by the compiler, which makes writing code more
                    % pleasant, but at some point, we have to know which
                    % instance is being used.
                \item Pattern matching: simplified
                    % All equations for a function are unified into a single
                    % equation using a case expression, and each case
                    % expression is made to analyze only one constructor deep
            \end{itemize}

        \item Many analyses and transformations are performed on the Core.

        \item Core to STG: simple translation.

        \item STG to Abstract C. (This is where we cross the
            functional/imperiative boundary.)

        \item Abstract C to the backend: native code generator, C code
            generator, LLVM, etc.
    \end{enumerate}

    We will focus on the representations of major functional features and give
    a brief overview of the STG language and how Haskell is translated to it.
\end{frame}

\section{Representations}

\begin{frame}
    \frametitle{Three key questions}

    \begin{enumerate}
        \item What are the representations of
            \begin{itemize}
                \item function values?
                \item data values?
                \item unevaluated expressions?
            \end{itemize}

        \item How is function application performed?

        \item How is case analysis performed?
    \end{enumerate}
\end{frame}

\subsection{Functions}

\begin{frame}
    \huge{Functions}
\end{frame}

\begin{frame}
    \frametitle{Functions}

    A function is a suspended computation: when applied to arguments, the
    computation is performed.

    \begin{block}{Compact representation of a closure}
        \begin{itemize}
            \item
                a (pointer to a) block of \alert{static code} (shared between
                all instances of the function); and
            \item
                zero or more pointers for the \alert{free variables}.
        \end{itemize}
    \end{block}

    % Many Lisp systems as well as SML of New Jersey use this flat
    % representation.

    \begin{block}{Entering a closure}
        \begin{itemize}
            \item
                set the \alert{environment pointer} (a distinguished register)
                to the beginning of the closure representation; and
            \item
                jump to the static code given in the code pointer.
        \end{itemize}
        Free variables are accessed by offsetting the environment pointer.
        Arguments are accessed by a usual calling convention.
    \end{block}
\end{frame}

\begin{frame}
    \frametitle{A closure}

    \begin{figure}
        \begin{tabular}{|c|}
            \hline
            \tikz[baseline]{\node (codeptr) {code pointer}} \\ \hline
            free variable $1$ \\ \hline
            \textellipsis \\ \hline
            free variable $n$ \\ \hline
        \end{tabular}

        \begin{tikzpicture}[overlay]
            \node [left=3cm] at (codeptr) (f) {closure};
            \node [right=2cm] at (codeptr) (code) {\texttt{shared code}};

            \draw[->] (f) -- (codeptr);
            \draw[->] (codeptr) -- (code);
        \end{tikzpicture}

        \caption{
            A closure contains a pointer to some code, which computes the value
            of the closure. This code can be shared amongst all instances of
            the closure. The code expects that a distinguished register called
            the environment pointer to be pointing at the beginning of the
            closure. By offsetting the environment pointer, the code in the
            closure can access the free variables captured in the closure.
        }
    \end{figure}
\end{frame}

\begin{frame}[fragile]
    \frametitle{Thunks}

    Recall the inductive equation for \texttt{plus}.
    \begin{lstlisting}
plus (S n) m = S (plus n m)
    \end{lstlisting}
    If we reduce \texttt{plus (S Z) (S Z)} what do we get? The right-hand side
    of the equation captured in a \alert{thunk}.

    \begin{itemize}
        \item Also a closure.
            % Thunks can capture free variables, but have no arguments.
        \item \alert{Forcing} a thunk: enter the same way as a function.
            % A continuation is placed on the stack
            % The closure is entered. If the closure is a thunk (unevaluated),
            % then it arranges for an update to be performed when the
            % evaluation completes; otherwise, it just returns the computed
            % value.
        \item Self-updating.
            % The thunk knows its location on the heap (it's in the env ptr),
            % so it can overwrite itself with the computed value, which will
            % also be a closure that can be jumped into. If the computed value
            % is bigger than the thunk, then the thunk is overwritten with an
            % indirection.
    \end{itemize}
\end{frame}

\newcommand{\makeboxes}[2]{
    \def\name{#1}
    \def\num{#2}
    \pgfmathsetmacro{\ylim}{0.5}
    \pgfmathsetmacro{\xsep}{1}

    \pgfmathsetmacro{\xnum}{\num - 1}
    \foreach \y in {0, \ylim} {
        \foreach \x in {0,...,\xnum}{
            \draw (\x * \xsep, \y) -- (\x * \xsep + \xsep, \y);
        }
    }

    \foreach \x in {0,...,\num} {
        \draw (\x * \xsep, 0) -- (\x * \xsep, \ylim);
        \pgfmathsetmacro{\xpos}{\x * \xsep + \xsep * 0.5}
        \pgfmathsetmacro{\ypos}{\ylim * 0.5}
        \node (\name-box-\x) at (\xpos, \ypos) {};
    }
}

\begin{frame}[shrink]
    \begin{figure}
        \begin{tikzpicture}
            %%%%% BEFORE UPDATING

            \def\posname{top}
            \makeboxes{\posname}{3}

            \draw (\posname-box-1) ++(0, 0.2)
                node[anchor=south] {Before updating};

            \draw[->] (\posname-box-0) -- ++(0, -2) -- ++(0.5, 0)
                node[anchor=west] {Code pointer};

            \draw[->] (\posname-box-1) -- ++(0, -1);
            \draw[->] (\posname-box-2) -- ++(0, -1);
            \draw (\posname-box-1) ++(0.5, -1)
                node[anchor=north] {Free variables};

            %%%%% AFTER UDPATE (LARGE VALUE)

            \def\posname{botleft}
            \tikzset{xshift=-3cm, yshift=-3cm}
            \makeboxes{\posname}{2}

            \draw[->] (\posname-box-0) -- ++(0, -1) -- ++(0.5, 0)
                node[anchor=west] {Indirection};

            \draw[->] (\posname-box-1) -- ++(1, 0)
                node[anchor=west] {Value};

            %%%%% AFTER UPDATE (SMALL VALUE)

            \def\posname{botright}
            \tikzset{xshift=6cm}
            \makeboxes{\posname}{2}

            \draw[->] (\posname-box-0) -- ++(0, -1) -- ++(0.5, 0)
                node[anchor=west] {\texttt{S} code};

            \draw[->] (\posname-box-1) -- ++(1, 0)
                node[anchor=west, align=center] {smaller nat\\thunk};
        \end{tikzpicture}

        \caption{
            Different heap objects.
            From left to right: a large data value, a thunk, a small data
            value.
        }
    \end{figure}
\end{frame}

\begin{frame}
    \begin{block}{Question}
        How can the updating of thunks be optimized or eliminated?
    \end{block}
\end{frame}

\begin{frame}
    \frametitle{Extra uses of thunks}

    \begin{itemize}
        \item Loop detection.
        \item Synchronization.
    \end{itemize}
\end{frame}

\begin{frame}
    \frametitle{Three key questions}

    \begin{enumerate}
        \item What is the representations of
            \begin{itemize}
                \item a function value? \alert{A closure.}
                \item a data value? \alert{A closure.}
                \item an unevaluated expression? \alert{A closure.}
            \end{itemize}

        \item How is function application performed?

        \item How is case analysis performed?
    \end{enumerate}

    \pause

    \huge{Heap objects have a uniform representation: \alert{closures}.}
\end{frame}

\subsection{Function application}

\frame{\huge{Function application}}

\begin{frame}
    \frametitle{Function application}
    \begin{block}{Considerations}
        \begin{itemize}
            \item
                Higher-order language: functions are first-class, polymorphism.
            \item
                Currying by default.
        \end{itemize}
    \end{block}
    \begin{block}{Eval-apply}
        \begin{enumerate}
            \item Evaluate the function.
            \item Evaluate the argument. (Skipped in non-strict languages.)
            \item Apply the function to the argument.
        \end{enumerate}
    \end{block}
    \begin{block}{Push-enter}
        \begin{enumerate}
            \item Push arguments onto evaluation stack.
            \item Tail-call (\emph{enter}) the function.
        \end{enumerate}
        Evaluation stack allocated in contiguous chunks on the heap.
    \end{block}
\end{frame}

\begin{frame}[fragile]
    \frametitle{Eval-apply versus push-enter}
    \begin{lstlisting}
const :: a -> (b -> a)
const x y = x
-- currying is nice

apply3 :: (a -> b -> c -> d) -> a -> b -> c -> d
apply3 f x y z = f x y z
-- push-enter is nice
    \end{lstlisting}
\end{frame}

\begin{frame}
    \frametitle{Three key questions}

    \begin{enumerate}
        \item What is the representation of
            \begin{itemize}
                \item a function value? A closure.
                \item a data value? A closure.
                \item an unevaluated expression? A closure.
            \end{itemize}

        \item How is function application performed? \alert{Push-enter}.

        \item How is case analysis performed?
    \end{enumerate}
\end{frame}

\subsection{Algebraic data structures}

\frame{\huge{Algebraic data structures}}

\begin{frame}[fragile]
    \begin{itemize}
        \item Build up data structures with \alert{contructors}.
        \item Take apart data structures with \alert{case analysis}.
    \end{itemize}

    \begin{lstlisting}
data Tree a = Empty | Branch a (Tree a) (Tree a)

tree :: Tree ()
tree = Branch () (Branch () Empty Empty) Empty

size :: Tree a -> Int
size t = case t of
    Empty -> 0
    Branch _ t1 t2 -> 1 + size t1 + size t2

treeSize :: Int
treeSize = size tree -- gives 2
    \end{lstlisting}
\end{frame}

\begin{frame}
    \frametitle{Case expressions = evaluation}

    \begin{itemize}
        \item Pattern matching \emph{is} evaluation:
            \texttt{case t of ...} must evaluate $t$ to determine what
            constructor was used.
        \item Datatype-specific \alert{return convention} determines how to
            the knowledge of the alternative is used, e.g. by returning a
            constructor tag or by passing in a jump table to return into the
            code associated with an alternative.
    \end{itemize}
\end{frame}

\begin{frame}
    \frametitle{Three key questions}

    \begin{enumerate}
        \item What is the representations of
            \begin{itemize}
                \item a function value? A closure.
                \item a data value? A closure.
                \item an unevaluated expression? A closure.
            \end{itemize}

        \item How is function application performed? Push-enter.

        \item How is case analysis performed? With datatype-specific
            \alert{return conventions}.
    \end{enumerate}
\end{frame}

\section{The STG language}

\begin{frame}
    \Huge{The STG language}
\end{frame}

\begin{frame}
    \frametitle{The STG language}

    \begin{itemize}
        \item An austere, purely-functional language.
        \item Has a formal operational semantics.
    \end{itemize}

    \begin{center}
        \begin{tabular}{|l l|}
            \hline Construct & Operational reading \\ \hline
            Function application & Tail call \\
            Let expression & Heap allocation \\
            Case expression & Evaluation \\
            Constructor application & Return to continuation \\
            \hline
        \end{tabular}
    \end{center}
\end{frame}

\begin{frame}
    \frametitle{Important properties of STG}

    \begin{itemize}
        \item
            All function and constructor arguments are \alert{simple variables}
            or \alert{constants}.
            % Corresponds to the operational reality that arguments must be
            % *prepared* prior to a call. To satisfy this, when translating
            % from Core, additional let bindings are added.
        \item All constructors and built-in operations are \alert{saturated}.
            % Guaranteed by eta-expansion. In general, it is not possible to
            % saturate a function in a higher-order language due to the
            % interplay of currying/polymorphism.
        \item Pattern matching is performed only by \texttt{case} expressions.
            % The value scrutinized by a case expression can be an arbitrary
            % value. A restriction to a variable would just involve building
            % and immediately entering a closure.
        \item There is a special form of binding, to a \alert{lambda form}:
            \begin{equation*}
                f = \{ v_1, \ldots, v_n \}
                \backslash \pi
                \{ x_1, \ldots, x_n \}
                \to
                e
            \end{equation*}
            % free variables are explicitly named
            % whether the closure should be updated is explicitly marked
        \item STG supports \alert{unboxed values}.
    \end{itemize}
\end{frame}

\begin{frame}[fragile]
    \frametitle{Example STG program}

    \begin{lstlisting}
g = {} \n {f, xs} ->
  case xs of
    Nil {} -> Nil {}
    Cons {y, ys} ->
      let fy = {f, y} \u {} -> f {y}
          myf = {f, ys} \u {} -> g {f, ys}
      in Cons {fy, mfy}
    \end{lstlisting}

    \begin{block}{Question}
        What well-known function does this program implement?
    \end{block}
\end{frame}

\begin{frame}
    \frametitle{Translation to STG}

    \begin{enumerate}
        \item Replace binary application with multiple application:
            \begin{equation*}
                (\cdots((f e_1) e_2)\cdots) e_n
                \implies f \{ e_1, \ldots, e_n \}
            \end{equation*}
            % The semantics are the same as curried application, but the STG
            % machine just applies all available arguments at once rather than
            % one at a time.

        \item Saturate all constructors and built-in operations.

        \item Introduce \texttt{let}-bindings for non-atomic function arguments
            and lambda functions.

        \item Convert the RHS of each \texttt{let} binding into a lambda form.
    \end{enumerate}
\end{frame}

\section{Conclusion}

\begin{frame}
    \frametitle{Conclusion}

    \begin{itemize}
        \item Covered only about half of the paper!

        \item Closures are used pervasively as representations.

        \item A complex language like Haskell can be straightforwardly be converted
            to a simple language like STG.

        \item STG has a formal operational semantics (detailed in the second half
            of the paper) given in terms of a state transition system. This guides
            the implementation of the compilation of STG into Abstract C.
    \end{itemize}
\end{frame}

\begin{frame}
    \frametitle{Question}

    \alert{Strictness analysis} attempts to determine which paramters a
    function will always evaluate. How can strictness analysis be used to
    optimize a function?
\end{frame}

\end{document}
