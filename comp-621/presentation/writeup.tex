\documentclass{beamer}

\usepackage[utf8]{inputenc}

\author{Jacob Thomas Errington}
\title{
    Implementing lazy functional languages on stock hardware:
    the Spineless Tagless G-machine
}
\institute{McGill University}
\date{27 October 2016}

\usepackage{listings}
\usepackage{amsmath,amssymb,amsthm}
\usepackage{tikz}

\usetikzlibrary{arrows,shapes,calc}

\tikzstyle{every picture}+=[remember picture]
\everymath{\displaystyle}

\DeclareMathOperator{\ErrorOp}{\mathtt{error}}
\newcommand{\Error}[1]{\ErrorOp{\, \text{#1}}}

\begin{document}

\frame{\titlepage}

\begin{frame}
    \frametitle{About me}

    \begin{itemize}
        \item Third-year Mathematics \& Computer Science student.
        \item Last semester:
            COMP 520 (Compiler Design),
            COMP 527 (Logic and Computation),
            COMP 531 (Advanced Theory of Computing).
        \item Our compiler Goto: written in Haskell using some advanced
            techniques.
        \item Research interests: applications of logic to programming
            languages, techniques for ensuring code correctness by
            construction.
    \end{itemize}
\end{frame}

\begin{frame}
    \frametitle{About this presentation}

    A whirlwind tour of the backend of the Glasgow Haskell Compiler (GHC), as
    presented in the paper \emph{Implementing lazy functional languages on
    stock hardware: the Spineless Tagless G-machine} (Peyton Jones 1992).

    \tableofcontents
\end{frame}

\section{Introduction}

\subsection{Motivation}

\begin{frame}[fragile,shrink]
    \frametitle{Motivating example}

    \begin{lstlisting}
module Main where

data Nat = Z | S Nat

instance Show Nat where
    show n = show (natToInt n) where
        natToInt Z = 0
        natToInt (S n) = 1 + natToInt n

three, five :: Nat
three = S (S (S Z))
five = S (S (S (S (S Z))))

plus :: Nat -> Nat -> Nat
plus Z m = m
plus (S n) m = S (plus n m)

main :: IO ()
main = print (three `plus` five)
-- "8" appears on standard out
    \end{lstlisting}
\end{frame}

\begin{frame}
    \frametitle{Features of Haskell}

    \begin{itemize}
        \item
            Can define recursive, algebraic datatypes.
        \item
            Can implement recursive functions.
        \item
            Can pattern match on datatypes; values ``remember'' how they were
            constructed.
        \item
            Overloading via typeclasses: the print function uses the
            \texttt{Show} instance.
        \item
            Purely functional: side effects marked via the IO type constructor
        \item
            Lazy: we can forget about the evaluation order for the most part,
            and ``let the computer figure it out''.
    \end{itemize}

    We would like to compile a language with these kinds of features to
    efficient native code.
\end{frame}

\subsection{Background}

\begin{frame}
    \frametitle{Transforming functional languages}

    \begin{itemize}
        \item
            Simple means simple.

        \item
            Lazy languages are more amenable to transformation than strict
            languages.
    \end{itemize}
\end{frame}

\begin{frame}
    \frametitle{$\beta$-reduction}

    A strict functional language uses the \alert{call-by-value} reduction
    strategy for its underlying lambda calculus. A lazy functional language
    uses \alert{call-by-name}.

    \begin{block}{Call-by-name}
        Function application is performed before evaluating the arguments.
    \end{block}

    \begin{block}{Call-by-value}
        Arguments are evaluated before function application.
    \end{block}
\end{frame}

\begin{frame}
    \begin{equation}
        (\lambda x. 5) (\Error{``oh no''})
        \label{eq:cbn-vs-cbv}
    \end{equation}

    \begin{block}{Question}
        What does equation \ref{eq:cbn-vs-cbv} reduce to in call-by-name? In
        call-by-value?
    \end{block}
\end{frame}

\begin{frame}
    \frametitle{Consequences of call-by-name reduction}

    As a transformation, $\beta$-reduction (applying a function to its
    arguments) can always be performed statically in a lazy language.

    In a strict language, it can only be performed if the function
    \alert{evaluates its argument}.
\end{frame}

\subsection{Compiling Haskell}

\begin{frame}
    \frametitle{Compilation route for Haskell}

    \begin{enumerate}
        \item Source language: Haskell, a strongly-typed, non-strict,
            purely-functional language.
        \item Haskell to Core
            \begin{itemize}
                \item Syntactic sugar: eliminated
                    % Do notation, proc notation, etc. is replaced by its
                    % equivalent semantics given by functions.
                \item Overloading: made explicit
                    % Typeclasse instances are normally plumbed around
                    % implicitly by the compiler, which makes writing code more
                    % pleasant, but at some point, we have to know which
                    % instance is being used.
                \item Pattern matching: simplified
                    % All equations for a function are unified into a single
                    % equation using a case expression, and each case
                    % expression is made to analyze only one constructor deep
            \end{itemize}

        \item Many analyses and transformations are performed on the Core.

        \item Core to STG: simple translation.

        \item STG to Abstract C. (This is where we cross the
            functional/imperiative boundary.)

        \item Abstract C to the backend: native code generator, C code
            generator, LLVM, etc.
    \end{enumerate}

    We will focus on the STG language and its translation to Abstract C, as
    well as how certain language features are represented in hardware.
\end{frame}

\section{The design space}

\begin{frame}
    \frametitle{Three key questions}

    \begin{enumerate}
        \item What are the representations of
            \begin{itemize}
                \item function values?
                \item data values?
                \item unevaluated expressions?
            \end{itemize}

        \item How is function application performed?

        \item How is case analysis performed?
    \end{enumerate}
\end{frame}

\subsection{Representations}

\begin{frame}
    \huge{Representations}
\end{frame}

\begin{frame}
    \frametitle{Functions}

    A function is a suspended computation: when applied to arguments, the
    computation is performed.

    \begin{block}{Compact representation of a closure}
        \begin{itemize}
            \item
                a (pointer to a) block of \alert{static code} (shared between
                all instances of the function); and
            \item
                zero or more pointers for the \alert{free variables}.
        \end{itemize}
    \end{block}

    % Many Lisp systems as well as SML of New Jersey use this flat
    % representation.

    \begin{block}{Entering a closure}
        \begin{itemize}
            \item
                set the \alert{environment pointer} (a distinguished register)
                to the beginning of the closure representation; and
            \item
                jump to the static code given in the code pointer.
        \end{itemize}
        Free variables are accessed by offsetting the environment pointer.
        Arguments are accessed by a usual calling convention.
    \end{block}
\end{frame}

\begin{frame}
    \frametitle{A closure}

    \begin{figure}
        \begin{tabular}{|c|}
            \hline
            \tikz[baseline]{\node (codeptr) {code pointer}} \\ \hline
            free variable $1$ \\ \hline
            \textellipsis \\ \hline
            free variable $n$ \\ \hline
        \end{tabular}

        \begin{tikzpicture}[overlay]
            \node [left=3cm] at (codeptr) (f) {closure};
            \node [right=2cm] at (codeptr) (code) {\texttt{shared code}};

            \draw[->] (f) -- (codeptr);
            \draw[->] (codeptr) -- (code);
        \end{tikzpicture}

        \caption{
            A closure contains a pointer to some code, which computes the value
            of the closure. This code can be shared amongst all instances of
            the closure. The code expects that a distinguished register called
            the environment pointer to be pointing at the beginning of the
            closure. By offsetting the environment pointer, the code in the
            closure can access the free variables captured in the closure.
        }
    \end{figure}
\end{frame}

\begin{frame}[fragile]
    \frametitle{Thunks}

    Recall the inductive equation for \texttt{plus}.
    \begin{lstlisting}
plus (S n) m = S (plus n m)
    \end{lstlisting}
    If we reduce \texttt{plus (S Z) (S Z)} what do we get? The right-hand side
    of the equation captured in a \alert{thunk}.

    \begin{itemize}
        \item Also a closure.
            % Thunks can capture free variables, but have no arguments.
        \item \alert{Forcing} a thunk: enter the same way as a function.
            % A continuation is placed on the stack
            % The closure is entered. If the closure is a thunk (unevaluated),
            % then it arranges for an update to be performed when the
            % evaluation completes; otherwise, it just returns the computed
            % value.
        \item Self-updating.
            % The thunk knows its location on the heap (it's in the env ptr),
            % so it can overwrite itself with the computed value, which will
            % also be a closure that can be jumped into. If the computed value
            % is bigger than the thunk, then the thunk is overwritten with an
            % indirection.
    \end{itemize}
\end{frame}

\newcommand{\makeboxes}[2]{
    \def\name{#1}
    \def\num{#2}
    \pgfmathsetmacro{\ylim}{0.5}
    \pgfmathsetmacro{\xsep}{1}

    \pgfmathsetmacro{\xnum}{\num - 1}
    \foreach \y in {0, \ylim} {
        \foreach \x in {0,...,\xnum}{
            \draw (\x * \xsep, \y) -- (\x * \xsep + \xsep, \y);
        }
    }

    \foreach \x in {0,...,\num} {
        \draw (\x * \xsep, 0) -- (\x * \xsep, \ylim);
        \pgfmathsetmacro{\xpos}{\x * \xsep + \xsep * 0.5}
        \pgfmathsetmacro{\ypos}{\ylim * 0.5}
        \node (\name-box-\x) at (\xpos, \ypos) {};
    }
}

\begin{frame}[shrink]
    \begin{figure}
        \begin{tikzpicture}
            %%%%% BEFORE UPDATING

            \def\posname{top}
            \makeboxes{\posname}{3}

            \draw (\posname-box-1) ++(0, 0.2)
                node[anchor=south] {Before updating};

            \draw[->] (\posname-box-0) -- ++(0, -2) -- ++(0.5, 0)
                node[anchor=west] {Code pointer};

            \draw[->] (\posname-box-1) -- ++(0, -1);
            \draw[->] (\posname-box-2) -- ++(0, -1);
            \draw (\posname-box-1) ++(0.5, -1)
                node[anchor=north] {Free variables};

            %%%%% AFTER UDPATE (LARGE VALUE)

            \def\posname{botleft}
            \tikzset{xshift=-3cm, yshift=-3cm}
            \makeboxes{\posname}{2}

            \draw[->] (\posname-box-0) -- ++(0, -1) -- ++(0.5, 0)
                node[anchor=west] {Indirection};

            \draw[->] (\posname-box-1) -- ++(1, 0)
                node[anchor=west] {Value};

            %%%%% AFTER UPDATE (SMALL VALUE)

            \def\posname{botright}
            \tikzset{xshift=6cm}
            \makeboxes{\posname}{2}

            \draw[->] (\posname-box-0) -- ++(0, -1) -- ++(0.5, 0)
                node[anchor=west] {\texttt{S} code};

            \draw[->] (\posname-box-1) -- ++(1, 0)
                node[anchor=west, align=center] {smaller nat\\thunk};
        \end{tikzpicture}

        \caption{
            Different heap objects.
            From left to right: a large data value, a thunk, a small data
            value.
        }
    \end{figure}
\end{frame}

\begin{frame}
    \begin{block}{Question}
        How can the updating of thunks be optimized or eliminated?
    \end{block}
\end{frame}

\begin{frame}
    \frametitle{Extra uses of thunks}

    \begin{itemize}
        \item Loop detection.
        \item Synchronization.
    \end{itemize}
\end{frame}

\begin{frame}
    \frametitle{Three key questions}

    \begin{enumerate}
        \item What is the representations of
            \begin{itemize}
                \item a function value? \alert{A closure.}
                \item a data value? \alert{A closure.}
                \item an unevaluated expression? \alert{A closure.}
            \end{itemize}

        \item How is function application performed?

        \item How is case analysis performed?
    \end{enumerate}

    \pause

    \huge{Heap objects have a uniform representation: \alert{closures}.}
\end{frame}

\subsection{Function application}

\frame{\huge{Function application}}

\begin{frame}
    hello
\end{frame}

\end{document}
