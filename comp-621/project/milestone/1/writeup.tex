\documentclass[letterpaper,11pt]{article}

\newcommand{\mcfas}{\textsc{McFAS}}
\newcommand{\matlab}{\textsc{Matlab}}
\newcommand{\oatlab}{\textsc{Oatlab}}

\author{Jacob Thomas Errington (260636023)}
\title{\mcfas{} -- Milestone \#1}
\date{15 November 2015}

\usepackage[margin=2.0cm]{geometry}
\usepackage{listings}
\usepackage{hyperref}

\newcommand{\codesnip}{\texttt}

\begin{document}

\maketitle

\section{Current state of \mcfas{}}

We spent quite some time designing \oatlab{}, the language that \mcfas{}
operates on. We based ourselves on the simple \matlab{}-like language given in
assignment \#1, but we augmented it in some ways.

\begin{itemize}
  \item
    Whereas the language in assignment \#1 is merely a sequence of statements,
    an \oatlab{} program consists of a sequence of top-level function
    declarations. Program execution begins at the distinguished function whose
    name matches the name of the file (as in \matlab{}).

  \item
    Unlike \matlab{}, \oatlab{} does not have a module system; the entire
    program must be contained within a single file.

  \item
    Rather than hardcode some interesting functions into the grammar as in
    assignment \#1, we instead support true function calls as expressions. A
    hypothetical \oatlab{} interpreter would have a number of builtin functions
    such as \codesnip{write} or \codesnip{print}.

  \item
    Unlike in assignment \#1 but like in \matlab{}, \oatlab{} has a notion of
    ranges. Furthermore, \oatlab{} has both \codesnip{while}-loops and
    \codesnip{for}-loops.
\end{itemize}

Rather than give a detailed EBNF grammar here, we refer the reader to the
definition of the syntax tree in our code
\footnote{\url{https://github.com/tsani/mcfas/blob/bfa460884a54d79baef3262a0a11bc89ff162180/mcfas-backend/Language/Oatlab/Syntax.hs\#L25}}.

As discussed in the proposal, we are expressing the syntax tree as a signature.
However, we are doing something slightly more advanced than taking a functor
fixed point to recover the type of syntax trees. The reason is that in a more
complex language, in which for example there is a distinction between
statements and expressions, we cannot use the same datatype to represent both.
(In principle, one \emph{can} use the same datatype for everything, but this
opens up the possibility of putting statements as the operands of expressions.
Why bother using a strongly-typed language if that could happen?)
If we represent the syntax tree using multiple datatypes, how can we extract
the signature of the family of datatypes? A 2009 paper presented
at ICFP dicusses this problem in much more detail, and identifies a solution to
it\footnote{
  Alexey Rodriguez, Stefan Holdermans, Andres Löh, Johan Jeuring. Generic
  programming with fixed points for mutually recursive datatypes. ICFP 2009.
}.

We recreate a part of the scheme developed in the paper, using more modern
features available in GHC. (The compiler has changed quite a bit since 2009.)
In essence, we develop a theory of \emph{higher-order} functors, $F$-algebras,
and folds. This allows us to represent \oatlab{} syntax trees as a higher-order
fixed points of a higher-order functor.

Our framework will heavily make use of the notion of \emph{annotations} on
parts of the syntax tree. In Goto, an annotation scheme was developed that
operates on simple functors
\footnote{\url{https://github.com/tsani/goto/blob/master/libgoto/Language/Common/Annotation.hs\#L33}};
we have now extended it to operate on higher-order functors
\footnote{\url{https://github.com/tsani/mcfas/blob/bfa460884a54d79baef3262a0a11bc89ff162180/mcfas-backend/Language/Oatlab/Syntax.hs\#L108}}.

With just this small part of the framework implemented, we are already able to
reap some very interesting benefits. For example, we express a function to
pretty-print a single node in the AST as a higher-order $F$-algebra
\footnote{\url{https://github.com/tsani/mcfas/blob/bfa460884a54d79baef3262a0a11bc89ff162180/mcfas-backend/Language/Oatlab/Syntax.hs\#L169}},
and using a generalized recursion scheme, we can pretty-print a whole AST
\footnote{\url{https://github.com/tsani/mcfas/blob/bfa460884a54d79baef3262a0a11bc89ff162180/mcfas-backend/Language/Oatlab/Syntax.hs\#L255}}.
Another stunning result is that we can strip all annotations from an annotated
syntax tree using a $\sim 15$ character expression
\footnote{\url{https://github.com/tsani/mcfas/blob/bfa460884a54d79baef3262a0a11bc89ff162180/mcfas-backend/Language/Oatlab/Syntax.hs\#L251}}.

% First, we develop a simple datatype that is used to identify which part of the
% syntax tree we are in.
% \begin{lstlisting}[language=Haskell]
% data AstNode
%   = ProgramDeclNode
%   | TopLevelDeclNode
%   | VarDeclNode
%   | StatementNode
%   | ExpressionNode
%   | IdentifierNode
% \end{lstlisting}
% Using the \codesnip{DataKinds} language extension, this datatype is promoted to
% a \emph{kind}, and its values are promoted to \emph{types}. Consequently,
% \codesnip{ProgramDeclNode} for example, could appear on the type level. Then,
% we can define the syntax tree (with explicit recursion) as a single datatype
% \emph{indexed} by values of type \codesnip{AstNode}.
% \begin{lstlisting}[language=Haskell]
% data OatlabAst' :: AstNode -> * where
%   ProgramDecl
%     :: [OatlabAst' TopLevelDeclNode] -> OatlabAst' ProgramDeclNode
%   FunctionDecl
%     :: OatlabAst' IdentifierNode
%     -> [OatlabAst' IdentifierNode]
%     -> [OatlabAst' StatementNode]
%     -> OatlabAst' TopLevelDeclNode
%   Branch
%     :: OatlabAst' ExpressionNode
%     -> [OatlabAst' StatementNode]
%     -> Maybe [OatlabAst' StatementNode]
%     -> OatlabAst' StatementNode
%   ...
% \end{lstlisting}
% The major advantage of using type indices in this way is that, supposing we
% have a value of type \codesnip{OatlabAst' ProgramDeclNode}, pattern matching on
% this value requires only analyzing the \codesnip{ProgramDecl} construtor: none
% of the other constructors could have produced a value of this type.

\section{Upcoming work}

Next, we will design some specialized recursion schemes that implement a
dataflow fixedpoint solver, and we will implement forwards and backwards
analyses that use it. We will also investigate whether it is possible, using
the \codesnip{TypeFamilies} extension, to express that the type of annotation
at a particular node depends on the type index that identifies what part of the
tree the node is in. This way, the programmer can specify, in the type of their
analysis, that it will annotate, say, statements with in-sets and out-sets, and
that other types of nodes will not be annotated.

Sadly, we did not write a parser for \oatlab{} programs yet. The reason is that
we are still unsure of what the final form of the language will be. Our
pretty-printer emits code that vaguely resembles JavaScript; we have not yet
decided whether we will make it look more like \matlab{}.

We are thinking about writing an interpreter for \oatlab{} programs, so that we
can verify that the transformations written using \mcfas{} preserve the
semantics of the programs. However, this might be an unnecessarily large
undertaking.

\end{document}
