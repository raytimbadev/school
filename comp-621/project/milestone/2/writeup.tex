\documentclass[letterpaper,11pt]{article}

\newcommand{\mcfas}{\textsc{McFAS}}
\newcommand{\matlab}{\textsc{Matlab}}
\newcommand{\oatlab}{\textsc{Oatlab}}

\author{Jacob Thomas Errington (260636023)}
\title{\mcfas{} -- Milestone \#2}
\date{22 November 2015}

\usepackage[margin=2.0cm]{geometry}
\usepackage{listings}
\usepackage{hyperref}

\newcommand{\codesnip}{\texttt}

\begin{document}

\maketitle

\section{Current state of \mcfas{}}

Sadly, not a tremendous amount of concrete progress has been made since the
last milestone. In the last milestone, we wanted to have finished writing the
specialized recursion schemes for dataflow fixpoint solution as well as have
finished implementing a forward and backward analysis using these recursion
schemes. Instead, we got stuck in figuring out how to represent analyses and in
how to get the types to line up for the forward analysis $F$-algebra.

Luckily, we now have determined how analyses will be represented.
They are represented as a record that contains the answers to all six of
Laurie's analysis steps: two of the steps are answered directly in the type of
the analysis, and the rest are represented in fields of the record. In fact,
the record that we designed contains nothing specific to \oatlab{} and could
in principle be used to represent analyses for any other languages that are
represented as the fixed point of a higher-order functor.

We have also written the bulk of the forward analysis $F$-algebra: what is
missing are the tricky cases for conditionals and loops. In implementing these
cases, we will probably need to revise our analysis record to include a field
containing a function allowing the $F$-algebra to compare approximations for
equality. This will be essential to decide whether a fixed point has been
reached.

The reason it was so difficult to get the types right in the $F$-algebra is
that we needed to introduce many auxiliary datatypes and type-level functions
to handle the fact that statements work differently from other kinds of nodes
in the syntax tree. (Statements have an in-set and compute an out-set in a
forwards analysis. The in-set of the first statement in a function is computed
by a field in the analysis record.) However, this work pays off because the
passing around of in-sets and out-sets is made very obvious in the code.

Another thing that was tricky to deal with is that fact that catamorphisms
implement a bottom-up recursion. How then can we implement a forward analysis,
which is intuitively top-down? The solution is that when the $F$-algebra
encounters a statement node, rather than immediately computing a new statement
node, (which cannot be done because the $F$-algebra doesn't have access to
the in-set,) a \emph{function} is computed instead. Essentially, by returning a
function, we can defer passing in the in-set to whatever nodes in the syntax
tree contain statements. These are precisely function declarations as well as
control flow structures. It is then up to these constructs to feed an in-set to
the first statement in the list, extract its out-set, feed it to the next
statement, etc. We can imagine each statement in the list having been
transformed into a function, and the nodes that handle lists of statements must
then compose them all together. (I think I will include a diagram of this idea
in my presentation, since I think it's a pretty sweet idea.)

\section{Upcoming work}

The next item on the agenda is the presentation, in a week from now. By then,
it is imperative that we complete the forward analysis $F$-algebra and
implement an analysis with it such as reaching definitions. Ideally by then,
we would also have completed the backward analysis $F$-algebra and implemented
an analysis with it such as liveness analysis.

We also really ought to implement a parser for \oatlab{}. This is not strictly
necessary, but it is extremely tedious to write out the syntax trees by hand
currently.

\end{document}
