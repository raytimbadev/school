% Sample file for COMP 621 Project Propsal

\documentclass{article}

\title{Functionally analyzing imperative programs}
\date{Fall 2016\\Project \#7}
\author{Jacob Thomas Errington}

\begin{document}

\maketitle

\clearpage

\tableofcontents

\listoffigures % comment out if you have no figures
\listoftables  % comment out if you have no tables

\clearpage

% Replace each subsection below with the appropriate contents.
% Feel free to add more (sub)sections if you want. Write clearly
% and concisely. 

\section{Introduction and Motivation}

In this section you should introduce the problem at a high-level and give
motivation for why it is important or interesting.   You can assume that
the reader has some familiarity with compilers and/or architectures, but
you should not assume that they are an expert in your particular topic.

\section{Background and Related Work}

Describe other approaches that have been taken and highlight what parts
of these approaches will be incorporated in the project, and what new
approaches might be taken.

In order to complete this section properly, you must do a literature
search on your topic and talk to the project mentor for relevant references.
You must cite appropriate references and include a bibliography.

If you cite any documents, you should use bibtex, which is an adjunct to
latex for generating bibliographies.  

Here is an example citation \cite{POPL::DeFouwGC1998}.   You just use the 
label given in the bibtex entry for this paper.   

\section{Specific Problem Statement}

As precisely as possible, state the problem that will be investigated.
You may want to break it down into several well-defined subproblems.
The more precisely you can state the problem, the more likely you are
to acheive a good result.

\section{Solution Strategy}

Outline your solution to the problem.   You need not have worked out all
the details,  but you should some plausible ideas.   Try to break down
your solution into several well-defined components.

\section{Experimental Framework}

Most projects will require some experiments to validate the approach for
some selected benchmarks.  In this section, you should outline your
experimental approach.  What sorts of experiments do you plan to conduct?
How will you present and analyze your data?   What sort of results would
you consider positive?

\section{Schedule of Activities}

Based on the solution strategy, develop a schedule for accomplishing the
final goal.  For this schedule, you should indicate what has already been
done and give a proposed schedule of activities leading towards a final
project report (be sure to give the name of each group member who will be
responsible for each activity).

When applicable, make sure that you include sufficient time to understand
exisiting code, develop and thoroughly test any new code you may need to
write, and perform your experiments.  Each of these phases often takes
longer than expected, so beware!

When a team has more than one person, it is important to specify
the role of each individual in sufficient detail.

\section{Expected Results and Evaluation Method}

You may want to plan what results you would like to obtain, and how to
evaluate them.  This may be useful as a guidline for your project activities.

% Comment out the next 2 command lines if you have no bibliography:
% However, your should have one!
\bibliographystyle{plain}
% You can add more entries into example.bib, or add your own .bib file
% and add it to the bibliography list below
\bibliography{example}

\end{document}
