\documentclass{beamer}

\usepackage[utf8]{inputenc}

\title{McFAS\\functionally analyzing imperative programs}
\author{Jacob Thomas Errington}
\institute{McGill University}
\date{29 November 2016}

\usepackage{tikz}
\usepackage{listings}

\begin{document}

\frame{\titlepage}

\begin{frame}
  \frametitle{The expression problem}

  \pause

  \begin{block}{Object-oriented programming}
    Models are \emph{horizontally} extensible.
  \end{block}

  \begin{block}{Functional programming}
    Models are \emph{vertically} extensible.
  \end{block}

  \pause

  \begin{block}{Consequences}
    \begin{description}
      \item[OOP.] Design patterns.
      \item[FP.] Advanced techniques.
    \end{description}
  \end{block}

\end{frame}

\begin{frame}
  What if we had analysis framework leveraging FP?

  \pause

  \alert{McFAS}
\end{frame}

\begin{frame}
  \frametitle{Outline}

  \begin{itemize}
    \item Oatlab
    \item Design of McFAS
    \item Example analysis
    \item Related work
  \end{itemize}
\end{frame}

\section{Oatlab}

\begin{frame}[fragile]
  \frametitle{Oatlab language}

  \begin{itemize}
    \item Very simple imperative language.
    \item Dynamically typed.
    \item Looks more like JavaScript than MATLAB.
  \end{itemize}

  \begin{lstlisting}
function dispRange(n) {
  for i <- 1 : n {
    disp(i);
  }
}
  \end{lstlisting}
\end{frame}

\section{Design of McFAS}

\begin{frame}
  \frametitle{Analysis as a datatype}

  Our goal is to represent an analysis succintly in a record.

  \pause

  \begin{description}
    \item[Approximation:] type parameter.
    \item[Precise statement:] documentation.
    \item[Merge operator:] field in the record.
    \item[Analysis direction:] type index.
    \item[Dataflow equation:] field in the record.
    \item[Initial conditions:] precondition \& field in the record.
  \end{description}
\end{frame}

\begin{frame}
  \frametitle{AST representation}

  \pause

  \begin{itemize}
    \item Strategy used in Goto was a mess.
    \item Adapt
      \emph{
        Generic Programming with Fixed Points for Mutually Recursive Datatypes
      }.
      ICFP 2009.
    \item Building on
      \emph{
        Functional Programming with Bananas, Lenses, Envelopes and Barbed Wire
      }
  \end{itemize}

  \pause

  \begin{block}{Benefits}
    \begin{itemize}
      \item One datatype to rule them all.
      \item Indexed annotations.
      \item General recursion schemes.
    \end{itemize}
  \end{block}
\end{frame}

\begin{frame}
  \frametitle{Concrete consequences}

  \begin{itemize}
    \item Specifying AST traversals is straightforward.
    \item Traversal of an annotated AST is effectively an analysis.
    \item Idea: store last computed flow sets in the annotation.
  \end{itemize}
\end{frame}

\begin{frame}
  \frametitle{Example: reaching definitions}
\end{frame}

\section{Conclusion}

\begin{frame}
  \frametitle{Future work}

  \begin{itemize}
    \item Improve fixed-point solver: better error messages and bug fixes.
    \item Implement backwards analysis runner.
  \end{itemize}
\end{frame}

\begin{frame}
  \begin{block}{Related work}
    \begin{itemize}
      \item
        Many static analysis tools for imperative languages are themselves
        written in imperative languages, e.g. McSAF, Soot.
      \item
        Some formal verification tools for imperative languages are however
        implemented using functional programming languages, e.g. \emph{Frama-C}
        is written in OCaml and formally verifies C programs.
    \end{itemize}
  \end{block}
\end{frame}

\begin{frame}
  \frametitle{Conclusion}
\end{frame}

\end{document}
