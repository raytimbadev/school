\documentclass[letterpaper,11pt]{article}

\author{Jacob Thomas Errington}
\title{Assignment \#2\\Program analysis \& transformation -- COMP 621}
\date{25 October 2016}

\usepackage[margin=2.0cm]{geometry}
\usepackage{amsmath,amssymb}
\usepackage{listings}

\newcommand{\matlab}{\textsc{Matlab}}

\DeclareMathOperator{\genOp}{gen}
\DeclareMathOperator{\killOp}{kill}
\DeclareMathOperator{\insetOp}{in}
\DeclareMathOperator{\outsetOp}{out}
\DeclareMathOperator{\immOp}{Imm}
\DeclareMathOperator{\effOp}{Eff}

\newcommand{\gen}[1]{\genOp{\left(#1\right)}}
\newcommand{\killset}[1]{\killOp{\left(#1\right)}}
\newcommand{\inset}[1]{\insetOp{\left(#1\right)}}
\newcommand{\outset}[1]{\outsetOp{\left(#1\right)}}
\newcommand{\imm}[1]{\immOp{\left(#1\right)}}
\newcommand{\eff}[1]{\effOp{\left(#1\right)}}

\begin{document}

\maketitle

\section{Code coverage}

Our approach is to use a global variable called
\texttt{g\char`_\char`_linemap} of type \texttt{containers.Map} to hold an
association from line numbers to integers, indicating whether a given line is
executed or not. We consider executable lines to be those containing
statements, so for each statement, we inject a statement like
\texttt{g\char`_\char`_linemap(':X') = 1} before it. By injecting our code
before each statement, we avoid issues with \texttt{return}, \texttt{continue},
and \texttt{break} statements which prevent the subsequent line from executing.

This approach is sadly \emph{too} simple, as it fails to account for the loop
statement in a \texttt{for}-loop: naively trying to prepend a statement to the
loop initializer will fail because a loop initializer may contain only
\emph{one} statement. In this case, we inject the statement that marks the loop
initializer as covered as the first statement of the loop body.

Now we have one last problem: the map is initially empty and our injected code
only marks lines as having been covered, so we are unable to determine which
lines did \emph{not} get executed. To remedy this, our analysis sets up the
initial state of the global map at the ``entry point'' of the file. I use scare
quotes because a file comes in two varieties in \matlab{}: functions and
scripts. For a function-file, we inject the intialization logic at the
beginning of the first function in the file. For a script, we inject at the
beginning of the file. In both cases, a \texttt{global} declaration is needed
at the beginning of every scope (I say scope to refer to either a function or a
script) to allow us to access the map.

Here is a recap of the different situations and how our transformation handles
them.
\begin{description}
    \item[For-loop.]
        Inject the coverage marker for the loop initializer in the body of the
        loop.

    \item[Any statement.]
        Inject the coverage marker before the statement.

    \item[Beginning of a scope.]
        Inject a global declaration for the line map.

    \item[Beginning of the first function or of a script.]
        Inject the initialization logic of the line map, setting all executable
        lines (i.e. those lcontaining statements) as having not been covered,
        unless that line has been covered already.
\end{description}

The ``unless'' clause in the initialization logic is essential! Otherwise if
the function is invoked multiple times, then its coverage information would be
erased. Instead, by checking whether each line has known coverage status, we
effectively take the union of the coverage statuses across multiple executions
of the same function within a single session, as required.

To test our implementation, we used a slightly modified version of the program
we wrote for the first assignment, since it makes use of most language features
of \matlab{}. We also use the simple \texttt{for\char`_script} example that was
provided to verify that things work as expected in a script too.

\section{Implementing an analysis -- Possibly not defined}

We chose to implement the possibly not defined (PND) analysis.

\begin{description}
    \item[Precise statement.]
        A variable $v$ is said to be \emph{possibly not defined} at a program
        point $p$ if there exists a path leading to $p$ along which the
        variable $v$ may not have been assigned a value.

    \item[Approximation domain.]
        To track whether a variable has been given a value has a value, we use
        a set of variables: a variable is in the set if and only if it has a
        definite value.

    \item[Analysis direction.]
        Because we are interested in the paths \emph{leading up to} a certain
        program point, this is a \emph{forwards} analysis.

    \item[Merge operator.]
        A variable is not defined if it is not in the set. If we have one path
        into a join node along which the variable \emph{is} defined and
        another path into the join node along which the variable is \emph{not}
        defined, then we would like it to be \emph{not} defined in the join
        node. Hence, we use set intersection.

    \item[Flow equations.]
        There are a few important cases to consider.

        \begin{description}
            \item[Case] $S$ : \texttt{x = <expr>}

                Then,
                \begin{align*}
                    \gen{S} &= \{x\} \\
                    \killset{S} &= \emptyset
                \end{align*}

            \item[Case] $S$ : \texttt{x(y) = <expr>}

                Then,
                \begin{align*}
                    \gen{S} &= \{x\} \\
                    \killset{S} &= \emptyset
                \end{align*}
        \end{description}
        and
        $\outset{S} = \left(\inset{S} \cup \gen{S}\right) \setminus \killset{S}$,
        but the kill-set is always empty so we could omit it from the equation.

        In all other cases, the in-set is equal to the out-set.

    \item[Initial conditions.]
        At the start of every function, the in-set consists of all top-level
        functions and all parameters to the function.

        At the start of a script, the in-set is empty.
\end{description}

Notice that this analysis is in fact essentially the same thing as a definite
assignment analysis.

(A variable $v$ is said to be \emph{definitely assigned} at a program point $p$
if along all paths leading up to $p$, $v$ is assigned to.)

In fact, \emph{possibly not defined} analysis is just the negation of
\emph{definitely assigned} analysis. We take advantage of this by performing a
definite assignment analysis (which is really what our six steps are doing) and
then negating the result: when we print our results, we scan each statement to
extract its variables, and for each variable in the statement not present in
our set of definitely assigned variables, we say that it is possibly not
defined.

We were unable to get McSAF to include the top-level definitions and parameters
of functions as the initial flows in functions, despite trying a few different
things. A non-working approach is included in our code. Discounting this, our
analysis appears to work correctly. We tested it on the simple
\texttt{for\char`_script.m} to verify that scripts work as expected, on
\texttt{function\char`_and\char`_for.m} to try to get parameters and top-level
definitions to be propagated, and on \texttt{kernel.m} (our code from
assignment 1) to verify that other language features work as expected.

\section{Effectively bound variables}

The analysis we are interested in exploring is a variant of a free variable
analysis, in the context of nested functions. The goal of this analysis is to
determine which variables are \emph{effectively bound} within a nested
function, so as to determine whether the function is a closure or not. If the
function is not closing over any variables, then it can safely be lifted to the
top-level. If the function is closing over some variables, then it can be
lifted up one or more levels, sometimes with modifications to its signature and
call sites.

Lifting functions is beneficial in several ways. First, compiling top-level
functions, which depend only on their arguments (and possibly global state), is
significantly simpler than compiling closures, which must be treated specially.
(In particular, the transformation we described is essentially performed at
compile-time and is invisible to the programmer.)

\begin{figure}[ht]
    \begin{lstlisting}
function test()

function isEven = foo(x)
    isEven = mod(x, 2) == 0;
end

function B = removeEven(A)
    B = []
    k = 1;
    for j = 1:length(A)
        if isEven(A(j))
            B(k) = A(j)
            k = k + 1
        end
    end
end

end
    \end{lstlisting}

    \caption{
        A motivating example for free variable analysis. The variable
        \texttt{isEven} is free in the function \texttt{removeEven}. Performing
        the proposed lifting refactor will add an additional parameter
        \texttt{isEven} to the \texttt{removeEven} function's signature. But
        now we have in fact generalized this function; it has become the
        well-known \texttt{filter} function!
    }
    \label{lst:motivation}
\end{figure}

The proposed ``lifting'' refactor is quite complex, since it would also require
a kind analysis to ensure that the call sites of the function to lift are
passed the appropriate things. In listing \ref{lst:motivation} for example, it
would be necessary to use a function handle when calling the lifted function.

Rather than look into the general case of lifting arbitrary functions, possibly
changing their signatures and call sites, we will focus on lifting functions so
that their signatures and call sites do not need to be modified. This is the
essential purpose of the effectively bound variables analysis.

\begin{figure}[ht]
    \begin{lstlisting}
function foo(a)
    % x is effectively bound; it derives from a
    x = a;
    function bar(b)
        % y is not effectively bound; it derives from x which is
        % defined in the parent scope.
        y = x + 2;
        function baz()
            % z is effectively bound; it derives from x which is
            % defined two scopes up.
            z = x + 2;
        end
    end
end
    \end{lstlisting}

    \caption{
        This listing demonstrates what is meant by ``effectively bound''. If a
        nested function contains only effectively bound variables, then it can
        be lifted to be become a sibling of its parent function with \emph{no
        change to its signature}.
    }
    \label{lst:effectively-bound}
\end{figure}

Before we can give the Six Steps breakdown of the analysis, we must establish
one more definition. We say that a variable is \emph{immediately bound} if it
is a formal argument of a function or it is derived from a formal argument of a
function.

Here we give a Six Steps breakdown of the effectively bound variables analysis.

\begin{description}
    \item[Precise statement.]
        A variable $v$ is \emph{effectively bound} at a program point $p$
        within the scope of an immediately enclosing function $f$ if either
        \begin{itemize}
            \item
                it is defined in $f$'s signature; or
            \item
                it is derived from other effectively bound variables in the
                scope determined by $f$; or
        \end{itemize}
        and it is not immediately bound in the parent scope of $f$.

    \item[Approximation domain.]
        Sets of variables.

    \item[Analysis direction.]
        Because we are interested in how certain values are derived, this is a
        forwards analysis.

    \item[Merge operator.]
        We use set union: if along one path, a variable is derived from bound
        variables but in another it is free, then we must consider the variable
        to be potentially free in the join node, and hence everywhere in the
        scope.

    \item[Flow equations.]
        Nested functions cannot appear within control flow statements, so there
        is not much to do for flow equations. The only consideration is
        assignments. Our analysis considers all assignments to decide whether
        the variable being assigned to is effectively bound in its current
        scope.

        \begin{description}
            \item[Case] $S$ : \texttt{global x}

                We consider all global variables effectively bound; it's as if
                they are infinitely many scopes higher.

            \item[Case] $S$ : \texttt{x = C}

                If $C$ is a constant, then $x$ is effectively bound.

            \item[Case] $S$ : \texttt{x = g(...)}

                If $g$ is immediately bound and all its arguments are
                immediately bound, then $x$ is immediately bound.

                If $g$ is effectively bound and all its arguments are
                effectively bound, then $x$ is effectively bound.
        \end{description}

        Then, we must process the nested functions. Each nested function starts
        out with a set of effectively bound variables given by the set of
        effectively bound variables of its parent function minus the
        immediately bound variables of its parent function. This makes sense:
        if the nested function refers to something that its parent function is
        computing from its arguments, then it is not safe to lift the nested
        function into a sibling of the parent function. However, we can add the
        immediately bound variables of the function of the parent function's
        parent function!

        Suppose that $f_0$ is the function we wish to enter, and $f_i$ is its
        $i$th parent function. Let $\eff{f}$ denote the effectively bound
        variables of $f$ and let $\imm{f}$ defnote the immediately bound
        variables of $f$. Then, upon entering a function, we have the following
        equation.
        \begin{equation*}
            \effOp^*{(f_0)} = (\eff{f_1} \setminus \imm{f_1}) \cup \imm{f_2}
        \end{equation*}
        where $\effOp^*{(f)}$ denotes the initial effectively bound variables
        of $f$.

    \item[Initial conditions.]
        We must decide on $\effOp^*{(f)}$ for every top-level function $f$.
        This is simply the set of all built-in \matlab{} functions.

        We must also decide on the in-set for the first statement of each
        block: this is simply the empty set.
\end{description}

This is a rough sketch of the analysis; it does not consider mututally
recursive families of nested functions, which we think can be lifted if some
additional considerations are made.

\end{document}
