\documentclass[letterpaper,11pt]{article}

\author{Jacob Thomas Errington}
\title{Assignment \#2\\Program analysis \& transformation -- COMP 621}
\date{25 October 2016}

\newcommand{\matlab}{\textsc{Matlab}}

\begin{document}

\section{Code coverage}

Our approach is to use a global variable called \texttt{g__linemap} of type
\texttt{containers.Map} to hold an association from line numbers to integers,
indicating whether a given line is executed or not. We consider executable
lines to be those containing statements, so for each statement, we inject a
statement like \texttt{g__linemap(':X') = 1} before it. By injecting our code
before each statement, we avoid issues with \texttt{return}, \texttt{continue},
and \texttt{break} statements which prevent the subsequent line from executing.

This approach is sadly \emph{too} simple, as it fails to account for the loop
statement in a \texttt{for}-loop: naively trying to prepend a statement to the
loop initializer will fail because a loop initializer may contain only
\emph{one} statement. In this case, we inject the statement that marks the loop
initializer as covered as the first statement of the loop body.

Now we have one last problem: the map is initially empty and our injected code
only marks lines as having been covered, so we are unable to determine which
lines did \emph{not} get executed. To remedy this, our analysis sets up the
initial state of the global map at the ``entry point'' of the file. I use scare
quotes because a file comes in two varieties in \matlab{}: functions and
scripts. For a function-file, we inject the intialization logic at the
beginning of the first function in the file. For a script, we inject at the
beginning of the file. In both cases, a \texttt{global} declaration is needed
at the beginning of every scope (I say scope to refer to either a function or a
script) to allow us to access the map.

Here is a recap of the different situations and how our transformation handles
them.
\begin{description}
    \item[For-loop.]
        Inject the coverage marker for the loop initializer in the body of the
        loop.

    \item[Any statement.]
        Inject the coverage marker before the statement.

    \item[Beginning of a scope.]
        Inject a global declaration for the line map.

    \item[Beginning of the first function or of a script.]
        Inject the initialization logic of the line map, setting all executable
        lines (i.e. those lcontaining statements) as having not been covered,
        unless that line has been covered already.
\end{description}

The ``unless'' clause in the initialization logic is essential! Otherwise if
the function is invoked multiple times, then its coverage information would be
erased. Instead, by checking whether each line has known coverage status, we
effectively take the union of the coverage statuses across multiple executions
of the same function within a single session, as required.

To test our implementation, we used a slightly modified version of the program
we wrote for the first assignment, since it makes use of most language features
of \matlab{}. We also use the simple \texttt{for_script} example that was
provided to verify that things work as expected in a script too.
