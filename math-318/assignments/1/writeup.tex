\documentclass{article}

\usepackage{amsmath}
\usepackage{amssymb}

\author{Jacob Errington (260636023)}
\date{19 September 2014}
\title{Mathematical Logic -- Assignment \#1}

\newcommand*{\problem}{}

\newcommand{\R}{\mathbb{R}}
\newcommand{\N}{\mathbb{N}}
\newcommand{\Q}{\mathbb{Q}}

\begin{document}

\maketitle

\begin{enumerate}
    \item \problem{Decide whether the following are true or false.}
        \begin{enumerate}
            \item $1 \in 1$

                False. A set cannot be an element of itself.

            \item $1 \subseteq 1$

                True. A set is a subset of another if all its elements are also elements of the
                other. Since the two sets in question are identical, they have the same elements,
                and are therefore subsets of each other.

            \item $1 \in \left\{ 1, 2 \right\}$

                True.

            \item $1 \subseteq \left\{ 1, 2 \right\}$

                False. We can construct the natural numbers from the primitive operations on sets
                inductively with $0 = \emptyset$ and $\mathrm{succ}(n) = n \cup \left\{ n \right\}$.  Therefore, $1 =
                \{0, \{ 0 \}\}$ and $2 = \left\{1, \left\{ 1 \right\}\right\}$. Neither of the elements of $1$ are equal
                to $1$ or to $2$. Thus, the statement is false.

        \end{enumerate}

    \item \problem{Draw the following sets in $\R^2$.}
        \begin{enumerate}
            \item $\left\{(x, y) \in \R^2 : x < y + 1 \, \mathrm{and} \, x^2 > y\right\}$

                See appendix 1.

            \item $\left\{(x, y) \in \R^2 : \left( x^2 + y^2 \leq 1 \, \mathrm{and} \, x \geq y\right)
                \, \mathrm{or} \, x^2 + y^2 \geq 2\right\}$

                See appendix 2.

        \end{enumerate}

    \item 
        \begin{enumerate}
            \item \problem{Compute the transitive closure $T$ of the following relation $R$ on 
                $A = \left\{1, 2, 3, 4\right\}$.}
                \begin{align*}
                    R                 &= \{(1, 2), (2, 3), (3, 1), (4, 4)\} \\
                    R \circ R         &= \{(1, 3), (3, 2), (2, 1), (4, 4)\} \\
                    R \circ R \circ R &= \{(1, 1), (2, 2), (3, 3), (4, 4)\} 
                \end{align*}
                Let $R^\prime$ be the union of the above three compositions. Further compositions yield only
                elements already contained in $R^\prime$, so $\bigcup_{n=1}^\infty R^n = R^\prime = T$.
                
            \item \problem{Is $T$ an equivalence relation?}

                We verify the properties of an equivalence relation.
                \begin{description}
                    \item[reflexivity] The pairs $\{(1,1),(2,2),(3,3),(4,4)\} \subseteq
                    T$ indicate that each element in $A$ is related to itself by $T$.

                    \item[symmetricity] We can see that for all $k_1, k_2 \in \{1, 2, 3\}.\, (k_1, k_2)
                        \in T \, \mathrm{and} \, (k_2, k_1) \in T$.

                    \item[transitivity] The transitive closure of a relation is guaranteed to be
                        transitive.
                \end{description}
                From the above verification, $T$ is an equivalence relation.

            \item \problem{Compute the equivalence class of 1.}

                $\left[1\right]_T = \{1, 2, 3\}$
        \end{enumerate}

    \item \problem{For each of the following relations, decide whether it is an equivalence
        relation.}
        \begin{enumerate}
            \item $E$ on $\N$ such that $x E y$ if $x < y$

                This relation is not an equivalence relation since it is not symmetric. If $y$ is
                greater than $x$, then $x$ cannot be greater than $y$.

            \item $E$ on $\N$ such that $x E y$ if $x^2 = y^2$

                Since squaring is an injective function on $\N$, i.e. no two different natural
                numbers have the same square, $x^2 = y^2 \iff x = y$, which we already
                know to be an equivalence relation.

            \item $E$ on $\R$ such that $x E y$ if $x - y \notin \Q$

                This relation is not reflexive, since $x - x = 0 \in \Q$.
        \end{enumerate}

    \item
        \begin{enumerate}
            \item \problem{Compute the composition of the following functions.}

                $
                    \left( \begin{array}{cccccc}
                            1 & 2 & 3 & 4 & 5 & 6 \\
                            6 & 5 & 1 & 4 & 2 & 3
                    \end{array} \right)
                    \left( \begin{array}{cccccc}
                            1 & 2 & 3 & 4 & 5 & 6 \\
                            1 & 3 & 6 & 2 & 4 & 5
                    \end{array} \right) =
                    \left( \begin{array}{cccccc}
                            1 & 2 & 3 & 4 & 5 & 6 \\
                            6 & 1 & 3 & 5 & 4 & 2
                    \end{array} \right)
                $

            \item \problem{Compute the inverse of the following permutation.}
                
                $\left( \begin{array}{cccccc}
                        1 & 2 & 3 & 4 & 5 & 6 \\
                        2 & 5 & 4 & 1 & 6 & 3
                \end{array} \right)^{-1} = 
                \left( \begin{array}{cccccc}
                        1 & 2 & 3 & 4 & 5 & 6 \\
                        4 & 1 & 6 & 3 & 2 & 5
                \end{array} \right)$
        \end{enumerate}
\end{enumerate}
\end{document}
