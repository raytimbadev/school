\documentclass[letterpaper,11pt]{article}

\author{Jacob Thomas Errington}
\title{Assignment \#2\\Formal verification -- COMP 525}
\date{7 February 2017}

\usepackage[margin=2.0cm]{geometry}
\usepackage{tikz}
\usepackage{amsmath,amsthm,amssymb}

\renewcommand{\thesection}{Question \arabic{section}}

\newcommand{\Z}{\mathbb{Z}}

\begin{document}

\maketitle

\section{Formalizing linear-time properties}

We consider the set of atomic propositions $AP = \{x = 0, x > 1\}$ and a
nonterminating sequential program $P$ that manipulates $x$. We formalize the
following informal linear-time properties. Let $\Sigma = 2^{AP}$.

\begin{enumerate}
        \renewcommand{\labelenumi}{(\alph{enumi})}
        \setcounter{enumi}{4}
    \item ``$x$ exceeds one only finitely many times''

        This is formalized by the $\omega$-regular language
        $\Sigma^* (\Sigma \setminus \{ \{x>1\} \})$.
        (We do not exclude $\{x>1, x=0\}$ since it is impossible anyway.)

    \item ``$x$ exceeds one infinitely often''

        We use the $\omega$-regular language
        $\Sigma^* (\Sigma^* \{x>1\})^\omega$.

    \item ``the value of $x$ alternates between zero and two''

        There is no way to determine using only the given atomic propositions
        that $x$ has the value two. We make the following assumption:
        $x \in \Z_3$. Hence $x > 1 \iff x = 2$.

        Then, we use the $\omega$-regular language
        $(\{x=0\}\{x>1\})^\omega + (\{x>1\}\{x=0\})^\omega$.

    \item ``true''

        To construct a family of sets of atomic propositions that is always
        satisfied, it suffices to take the whole set, and subtract the
        impossible case $\{x=0, x>1\}$.

        This gives the $\omega$-regular language
        $(\Sigma \setminus \{x=0, x>1\})^\omega$.
\end{enumerate}

\end{document}
