\documentclass[letterpaper,11pt]{article}

\author{Jacob Thomas Errington}
\title{Assignment \#2\\Formal verification -- COMP 525}
\date{7 February 2017}

\usepackage[margin=2.0cm]{geometry}
\usepackage{tikz}
\usepackage{amsmath,amsthm,amssymb}

\newtheorem{prop}{Proposition}

\renewcommand{\thesection}{Q\arabic{section}}

\newcommand{\Z}{\mathbb{Z}}
\newcommand{\union}{\cup}
\newcommand{\intersn}{\cap}
\newcommand{\up}{\uparrow}
\DeclareMathOperator{\Pref}{Pref}
\DeclareMathOperator{\FairPaths}{FairPaths}

\begin{document}

\maketitle

\section{(\# 3.5) Formalizing linear-time properties}

We consider the set of atomic propositions $AP = \{x = 0, x > 1\}$ and a
nonterminating sequential program $P$ that manipulates $x$. We formalize the
following informal linear-time properties. Let $\Sigma = 2^{AP}$.

\begin{enumerate}
        \renewcommand{\labelenumi}{(\alph{enumi})}
        \setcounter{enumi}{4}
    \item ``$x$ exceeds one only finitely many times''

        This is formalized by the $\omega$-regular expression
        $\Sigma^* (\Sigma \setminus \{ \{x>1\} \})$.
        (We do not exclude $\{x>1, x=0\}$ since it is impossible anyway.)

    \item ``$x$ exceeds one infinitely often''

        We use the $\omega$-regular expression
        $\Sigma^* (\Sigma^* \{x>1\})^\omega$.

    \item ``the value of $x$ alternates between zero and two''

        There is no way to determine using only the given atomic propositions
        that $x$ has the value two. We make the following assumption:
        $x \in \Z_3$. Hence $x > 1 \iff x = 2$.

        Then, we use the $\omega$-regular expression
        $(\{x=0\}\{x>1\})^\omega + (\{x>1\}\{x=0\})^\omega$.

    \item ``true''

        To construct a family of sets of atomic propositions that is always
        satisfied, it suffices to take the whole set, and subtract the
        impossible case $\{x=0, x>1\}$.

        This gives the $\omega$-regular expression
        $(\Sigma \setminus \{x=0, x>1\})^\omega$.
\end{enumerate}

\section{(\# 3.6) Classifying linear-time properties}

\begin{enumerate}
        \renewcommand{\labelenumi}{(\alph{enumi})}
    \item ``$A$ should never occur''

        This can be formalized by the $\omega$-regular expression
        \begin{equation*}
            P = (\Sigma \setminus \{ \{A\}, \{A, B\} \})^\omega
        \end{equation*}

        This is an invariant, since there exists a formula $\phi = \neg A$ such
        that
        \begin{equation*}
            P = \left\{
                S_0 S_1 S_2 \ldots \in \Sigma^\omega
                | \forall j \geq 0 : A_j \models \phi
            \right\}
        \end{equation*}

    \item ``$A$ should occur exactly once''

        Let $Q = \Sigma \setminus \{ \{A\}, \{A, B\} \}$. Then, this property
        can be formalized by the $\omega$-regular expression
        \begin{equation*}
            P = Q^* \{A\} Q^\omega
        \end{equation*}

        This is not an invariant, because intuitively there's nothing we can
        say about the states individually that lets us know whether the
        property is satisfied or not. This is not a safety property either.
        Suppose we have an arbitrary run $\sigma \notin P$. Notice that the
        string $\{B\}^\omega$ is a possible choice for $\sigma$. However, any
        finite prefix of this string has as a possible extension the string
        containing exactly one ``letter'' that is a set containing $A$. This is
        not a liveness property either. Suppose we have an arbitrary
        $\hat \sigma \in \Sigma^*$. A possible choice for this $\hat \sigma$
        may have more than one letter containing $A$. Hence it is not the case
        that an arbitrary finite word can be extended to an infinite run
        satisfying the property.

    \item ``$A$ and $B$ alternate infinitely often''

        For any index $i$ at which $A$ is true, there exists an index
        $i^\prime \geq i$ such that $B$ is true, and vice versa.
        Let $\bar A = \{ \{A\}, \{A, B\} \}$
        and $\bar B = \{ \{B\}, \{A, B\} \}$.
        Then, we formalize the property with the $\omega$-regular expression
        \begin{equation*}
            P = \Sigma^* (\bar A \Sigma^* \bar B \Sigma^*)^\omega
        \end{equation*}

        This is a liveness property. Take an arbitrary $\hat \sigma \in
        \Sigma^*$. Then it can clearly be extended to a $\sigma \in P$.

    \item ``$A$ should eventually be followed by $B$''

        We understand this as a somewhat weaker version of the previous
        property. We adopt the notation introduced in the previous exercise,
        and
        let $\hat A = \{ \emptyset, \{A\} \}$
        and $\hat B = \{ \emptyset, \{B\} \}$.
        We formalize this with the
        $\omega$-regular expression
        \begin{equation*}
            P
            = (\hat B)^\omega
            + \Sigma^* (\bar A (\Sigma^* \{A, B\})^* \Sigma^* \{B\})^\omega
        \end{equation*}

        This expression encodes the notion that if a letter involving $A$ is
        encountered, that at some point later in the string, a letter involving
        $B$ is found. This is tricky because the letter $\{A, B\}$ can serve
        both to terminate a forwards search for a $B$ and to``kick off''
        another search for $B$.
\end{enumerate}

\section{(\# 3.11) Closure properties of liveness and safety properties}

\begin{prop}
    Let $P$ and $P^\prime$ be liveness properties over $AP$. Then,
    \begin{enumerate}
        \item $P \union P^\prime$ is a liveness property.
        \item $P \intersn P^\prime$ is not, in general, a liveness property.
    \end{enumerate}
\end{prop}

\begin{proof}
    We show each statement separately.

    \begin{enumerate}
        \item
            Take arbitrary $\hat \sigma \in \Sigma^*$. Since $P$ is a liveness
            property, then $\hat \sigma \up \intersn P \neq \emptyset$.
            Likewise, $\hat \sigma \up \intersn P^\prime \neq \emptyset$.
            Observe
            \begin{align*}
                \hat \sigma \up \intersn (P \union P^\prime)
                = (\hat \sigma \up \intersn P)
                \union (\hat \sigma \up \intersn P^\prime)
                \neq \emptyset
            \end{align*}
            Hence, an arbitrary finite word can be extended into a run that
            satisfies $P$ or $P^\prime$, so
            $\hat \sigma \in \Pref(P \union P^\prime)$ which establishes that
            $\Pref(P \union P^\prime) = \Sigma^*$ and that $P \union P^\prime$
            is a liveness property.

        \item
            Consider that
            $P = \Sigma^* \{B\}^\omega$ and
            $P^\prime = \Sigma^* \{A\{^\omega\}$ are liveness properties.
            Notice that $P \intersn P^\prime = \emptyset$,
            so $\Pref(P \intersn P^\prime) = \emptyset \neq \Sigma^*$, which
            establishes that $P \intersn P^\prime$ is not a liveness property.
    \end{enumerate}
\end{proof}

\section{(\#3.17) Satisfaction under fairness assumptions}

\begin{prop}
    The trasition system $TS$ does not satisfy the (informally stated) property
    ``eventually $a$'' under the fairness assumptions $\mathcal{F}$.
\end{prop}

\begin{proof}
    Consider the execution
    \begin{equation*}
        \rho = (s_3 \beta s_4 \alpha s_5 \alpha s_4 \beta s_6 \beta)^\omega
    \end{equation*}
    This execution satisfies the required fairness conditions, namely if
    the action $X$ is enabled infinitely often, then that action is taken
    infinitely often, where $X = \alpha, \beta$. This means that $\rho$ is a
    fair path. However, this run never reaches the state $s_1$ in which the
    desired atomic proposition $a$ is true. Hence, the system does not satisfy
    the property ``eventually $a$'', since it might ``never $a$''.
\end{proof}

\end{document}
