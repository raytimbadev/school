\documentclass[letterpaper,11pt]{article}

\author{Jacob Thomas Errington (260636023)}
\date{29 September 2015}
\title{Assignment \#1 -- Algorithm Design}

\usepackage{amsthm,amsmath,amssymb}
\usepackage[margin=2.0cm]{geometry}
\usepackage{algorithmicx,algpseudocode}

\newtheorem{proposition}{Proposition}

\newcommand{\val}{\!\mathrm{val}}

\begin{document}

\maketitle

\begin{description}
    \item[Question \#1]

        \begin{proposition}
            For every flow $f$, $\val{(f)}$ equals the sum of the flows carried
            by the edges leading into the sink.
        \end{proposition}

        \begin{proof}
            Consider the cut $(A, B)$ defined by the following.

            \begin{align*}
                A &= \{ v \in V | v \neq t \} \\
                B &= \{ t \}
            \end{align*}
            where $t$ is the vertex designated as the sink of the flow network.

            First, we examine $f^\mathrm{out}(A)$.

            \begin{align*}
                f^\mathrm{out}(A) &= f^\mathrm{in}(B) \\
                                  &= \sum_{\substack{e = (u, v) \\ u \in A \\ v \in B}} f(e) \\
                                  &= \sum_{e \text{ into } t} f(e)
            \end{align*}
            The summation range is simplified as $B$ is a singleton containing
            $t$.

            Next, we similarly examine $f^\mathrm{in}(A)$.

            \begin{align*}
                f^\mathrm{in}(A) &= f^\mathrm{out}(B) \\
                                 &= \sum_{\substack{e = (u, v) \\ u \in B \\ v \in A}} f(e) \\
                                 &= \sum_{e \text{ leaving } t} f(e) \\
                                 &= 0
            \end{align*}
            The simplification of the summation range is justified as in the
            examination of $f^\mathrm{out}(A)$. The final simplification to
            zero is due to an overarching assumption on flow networks, namely
            that there are no edges entering the sink $t$.

            Finally, we apply the above results to the definition of
            $\val{(f)}$.

            \begin{align*}
                \val{(f)} &= f^\mathrm{out}(A) - f^\mathrm{in}(A) \\
                          &= \sum_{e \text{ into } t} f(e)
            \end{align*}

        \end{proof}


\end{description}

\end{document}
