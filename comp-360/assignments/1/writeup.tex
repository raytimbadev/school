\documentclass[letterpaper,11pt]{article}

\author{Jacob Thomas Errington (260636023)}
\date{29 September 2015}
\title{Assignment \#1 -- Algorithm Design}

\usepackage{amsthm,amsmath,amssymb}
\usepackage[margin=2.0cm]{geometry}
\usepackage{algorithmicx,algpseudocode}

\newtheorem{proposition}{Proposition}

\newcommand{\val}{\!\mathrm{val}}

\begin{document}

\maketitle

\begin{description}
    \item[Question \#1]

        \begin{proposition}
            For every flow $f$, $\val{(f)}$ equals the sum of the flows carried
            by the edges leading into the sink.
        \end{proposition}

        \begin{proof}
            Consider the cut $(A, B)$ defined by the following.

            \begin{align*}
                A &= \{ v \in V | v \neq t \} \\
                B &= \{ t \}
            \end{align*}
            where $t$ is the vertex designated as the sink of the flow network.

            First, we examine $f^\mathrm{out}(A)$.

            \begin{align*}
                f^\mathrm{out}(A) &= f^\mathrm{in}(B) \\
                                  &= \sum_{\substack{e = (u, v) \\ u \in A \\ v \in B}} f(e) \\
                                  &= \sum_{e \text{ into } t} f(e)
            \end{align*}
            The summation range is simplified as $B$ is a singleton containing
            $t$.

            Next, we similarly examine $f^\mathrm{in}(A)$.

            \begin{align*}
                f^\mathrm{in}(A) &= f^\mathrm{out}(B) \\
                                 &= \sum_{\substack{e = (u, v) \\ u \in B \\ v \in A}} f(e) \\
                                 &= \sum_{e \text{ leaving } t} f(e) \\
                                 &= 0
            \end{align*}
            The simplification of the summation range is justified as in the
            examination of $f^\mathrm{out}(A)$. The final simplification to
            zero is due to an overarching assumption on flow networks, namely
            that there are no edges entering the sink $t$.

            Finally, we apply the above results to the definition of
            $\val{(f)}$.

            \begin{align*}
                \val{(f)} &= f^\mathrm{out}(A) - f^\mathrm{in}(A) \\
                          &= \sum_{e \text{ into } t} f(e)
            \end{align*}

        \end{proof}

    \item[Question \#2]
        There are five types of packages to be delivered by four trucks, with
        capacities $5$, $4$, $4$, and $3$ respectively. There are three
        packages of each type. No truck may carry two of the same type of
        package.

        We set up a maximum flow problem as follows in order to optimize the
        delivery of the packages subject to the above constraints.

        \begin{itemize}
            \item
                The vertex set $V$ of the graph is composed of: the source
                vertex $s$, a set $P \subset V$ of vertices representing the
                types of packages, a set $T \subset V$ of vertices representing
                the trucks, and the sink vertex $t$. Since there are $5$ types
                of packages, $|P| = 5$, and likewise, $|T| = 4$.

            \item
                For each vertex representing a package type $p \in P$, we
                create an edge $(s, p)$ with capacity $3$, representing that
                there are three packages of each type.

            \item
                For each vertex representing a package type $p \in P$ and for
                each vertex representing a truck $q \in T$, we create an edge
                $(p, q)$ with capacity $1$, representing that no truck may
                carry two packages of the same type.

            \item
                Finally, for each vertex representing a truck $q \in T$, we
                create an edge $(q, t)$. The capacities on these edges match
                the respective capacities of the trucks as given in the problem
                statement.
        \end{itemize}

    \item[Question \#3]

        \begin{proposition}
            If an edge $e = (u, v)$ where $u \in A$, $v \in B$, and $(A, B)$ is
            a minimum cut, then every maximum flow $f$ uses the full capacity
            of $e$, i.e. $f(e) = c_e$.
        \end{proposition}

        \begin{proof}
            By contradiction.

            Suppose there is such an edge such that
            \begin{equation}
                f(e) < c_e
                \label{eq:max-flow-contradiction}
            \end{equation}

            From this assumption, $e$ is a forward edge in the residual graph
            due to $f$.

            From the construction of $A$ as the set of all vertices reachable
            from $s$, there is an $s \to u$ path. Since $e$ has unused
            capacity, we can append $e$ to the $s \to u$ path, forming an $s
            \to v$ path. Hence, $v \in A$, contradicting the assumption that
            $v \in B$, since $(A, B)$ is a partition.
        \end{proof}

    \item[Question \#4]

        We can decide whether there is a unique minimum cut in a flow network
        on a graph $G = (V, E)$ in polynomial time in the following way.

        We first find a minimum cut $(A, B)$ on $G$ and an associated maximum
        flow $f$. Let $F$ be the set of all edges $e = (u, v)$ where $u \in A$
        and $v \in B$. For each $e \in F$, consider a graph $G_e$ built
        from $G$ simply by replacing $e$ with $e^\prime = (u, v)$ but having
        $c_{e^\prime} = c_e + 1$. We then find a maximum flow $f_e$ on
        $G_e$ and an associated minimum cut $(A_e, B_e)$. Let
        $\mathcal{G} = \{ G_e | e \in F \}$.

        \begin{proposition}
            If for all $G_e in \mathcal{G}$, the maximum flow $f_e$ on $G_e$ is
            greater than $f$, then $(A, B)$ is a unique minimum cut on $G$
        \end{proposition}

        \begin{proof}
            Since $F$ is finite, our proof simply chooses distinct
            $e = (u, v) \in F$ until there are none left.

            If $u \in A_e$ and $v \in B_e$, then
            $\val{(f_e} = \val{(f)} + 1$.
            and we conclude nothing yet about
            the uniqueness of the minimum cut $(A, B)$.

            Else, if $u \in A_e \iff v \in A_e$, then
            $\val{(f_e} = \val{(f)}$. Furthermore, $(A, B) \neq (A_e, B_e)$,
            for if they were equal, then $u \in A_e$ and $v \in B_e$, which
            contradict our assumption. Hence, the minimum cut $(A, B)$ is not
            unique, since we have found a distinct yet equivalent one.

            If we exhaust $F$, then we can conclude that the minimum cut
            $(A, B)$ is unique.
        \end{proof}

        This proof gives rise to a natural algorithm whose running time is at
        most $|F|$ times the running time of an algorithm to compute the
        maximum flow and minimum cut of a flow network. Since there are
        polynomial time algorithms to do so, the algorithm that emerges from
        this proof is also polynomial, as $|F| \in O(|E|)$. (We can discount
        the construction of the graphs $G_e \in \mathcal{G}$; since two such
        graphs are never needed simultaneously, they may be built by modifying
        $G$ and then undoing this modification.)

    \item[Question \#7]

        Consider a variant of the maximum flow problem in which vertices as
        well as edges are given capacities. The capacity $c_v$ of a vertex $v$
        is such that $f^\mathrm{in}(v) \leq c_v$.

        This variant may be solved by applying the following transformation
        resulting in a regular maximum flow problem.

        For each vertex $v$, we create a new vertex $v^\prime$. Each edge
        $e = (u, v)$ for some $u$ is changed into a new edge
        $e^\prime = (u, v^\prime)$. An edge $e_v = (v^\prime, v)$ is created
        and given capacity $c_v$.

        In the case of the source vertex, there is no incoming flow, so this
        transformation is unnecessary. In other words, the restriction imposed
        by a vertex capacity is only effectful when the vertex has incoming
        edges.

        In the case of the sink vertex, the transformation is applied as usual.
        Note that the newly created vertex $v^\prime$ in this case is not a
        sink vertex.

\end{description}

\end{document}
