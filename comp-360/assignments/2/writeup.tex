\documentclass[letterpaper,11pt]{article}

\usepackage{amsthm,amsmath,amssymb}
\usepackage[margin=2.0cm]{geometry}
\usepackage{graphicx,float}

\newtheorem{proposition}{Proposition}

\newcommand{\R}{\mathbb{R}}

\author{Jacob Thomas Errington}
\title{Assignment \#2 \\ Algorithm Design -- COMP 360}
\date{13 October 2015}

\begin{document}

\maketitle

\begin{enumerate}

    \item TODO


    \item The following linear program determines the minimum weight of food
        to pack while satisfying all of Gerlinde Kaltenbrunner's requirements.

        \begin{align*}
            \text{minimize} \quad
                & x_A + x_B + x_C + x_D + x_E                                                         \\
            \text{subject to} \quad
                & T    \leq (5000)(21)                                                                \\
                & 0.50 \leq \frac{2500x_A + 1500x_B + 2500x_C + 3500x_D + 1000x_E}{T} \leq 0.65       \\
                & 0.20 \leq \frac{3500x_A + 100x_B  + 100x_C  + 1800x_D + 2900x_E}{T} \leq 0.35       \\
                & 0.15 \leq \frac{500x_A  + 3500x_B + 2500x_C + 200x_D  + 500x_E }{T} \leq 0.20       \\
                & x_A  \geq 0                                                                         \\
                & x_B  \geq 0                                                                         \\
                & x_C  \geq 0                                                                         \\
                & x_D  \geq 0                                                                         \\
                & x_E  \geq 0                                                                         \\
            \text{where} \quad
                & T    = (2500 + 3500 + 500) x_A + (1500 + 100 + 3500) x_B + (2500 + 100 + 2500) x_C  \\
                &      + (3500 + 1800 + 200) x_D + (1000 + 2900 + 500) x_E
        \end{align*}

    \item

        \begin{enumerate}
            \item Draw the feasible region of the linear program.

                \begin{figure}[H]
                    \centering
                    \includegraphics[width=0.5\textwidth]{plot.pdf}
                \end{figure}

            \item The following list has the a vertex of the feasible region in
                each row of the left column and its defining two equations in
                the right column.

                \begin{align*}
                    (0, 2)  &\quad (3, 5) \\
                    (4, 3)  &\quad (2, 5) \\
                    (3, 1)  &\quad (2, 4) \\
                    (0, 0)  &\quad (1, 4) \\
                    (-1, 1) &\quad (1, 3)
                \end{align*}

            \item TODO

        \end{enumerate}

    \item

        \begin{proposition}
            If $x,x^\prime \in \R^n$ are feasible solutions to a given linear
            program, then all points on the line between $x$ and $x^\prime$ are
            also solutions to that linear program.
        \end{proposition}

        \begin{proof}
            First suppose all constraints in the linear program are of the form
            $$
            x \cdot a \leq b
            $$
            where $a \in \R^n$ and $b \in \R$.

            Second, all points on the line between $x$ and $x^\prime$ are identified by
            $\alpha \in [0,1]$ as follows.
            $$
            \alpha x + (1-\alpha)x^\prime
            $$

            Third, we examine some resulting inequalities.
            \begin{align}
                (1 - \alpha)x^\prime \cdot a & \leq (1 - \alpha) b \label{ineq:oneminusalphax} \\
                \alpha x \cdot a \leq \alpha b \label{ineq:alphax}
            \end{align}

            Finally, we consider the following equality, which is a
            reformulation of what it means to lie on the line between $x$ and
            $x^\prime$.
            \begin{align}
                (\alpha x \cdot a + (1 - \alpha) x^\prime) \cdot a
                &= \alpha x \cdot a + (1 - \alpha) \cdot a \notag \\
                &= \alpha x \cdot a + x^\prime \cdot a - \alpha x^\prime \cdot a \label{eq:qed}
            \end{align}

            Adding up inequalities \eqref{ineq:oneminusalphax} and
            \eqref{ineq:alphax}, we arrive at the statement we wish to prove,
            which completes the proof.

            Alternatively, suppose that a point on the line between $x$ and
            $x^\prime$ were not a feasible solution. Geometrically, that would
            be a contradiction of the convexity of the feasible region. Hence,
            all points on a line between any two feasible solutions must be
            feasible solutions.
        \end{proof}

    \item For a graph in which each edge is given a load and the load of a
        vertex is given by the sum of the loads of its edges, we seek to find
        an assignment of edges loads that sum to $1$ but minimize the maximal
        vertex load. To do so with linear programming, the set of variables
        consists of the edge loads as well as an additional variable to
        represent the maximal vertex load due to the current assignment of edge
        loads.

        \begin{enumerate}
            \item For a graph on four vertices $a$, $b$, $c$, $d$ with edges
                $(a, b)$, $(a, c)$, $(b, c)$, $(b, d)$, $(c, d)$, solutions
                to the following linear program provide assignments to the
                edge loads that minimize the maximum load on any vertex.

                \begin{align*}
                    \text{minimize} \quad &
                        M \\
                    \text{subject to} \quad
                        & M \geq x_{ab} + x_{ac} \\
                        & M \geq x_{ab} + x_{bc} + x_{bd} \\
                        & M \geq x_{ac} + x_{bc} + x_{cd} \\
                        & M \geq x_{bd} + x_{cd} \\
                        & x_{ab} + x_{ac} + x_{bc} + x_{bd} + x_{cd} = 1 \\
                        & M \geq 0 \\
                        & x_{ab} \geq 0 \\
                        & x_{ac} \geq 0 \\
                        & x_{bc} \geq 0 \\
                        & x_{bd} \geq 0 \\
                        & x_{cd} \geq 0
                \end{align*}

            \item The dual of the above linear program is the following.

                \begin{align*}
                    \text{maximize} \quad
                        & (0, 0, 0, 0, 1, -1) y \\
                    \text{subject to} \quad &
                    \left(
                        \begin{array}{cccccc}
                            1  & 1  & 1  & 1  & 0 & 0 \\
                            -1 & -1 & 0  & 0  & 1 & -1 \\
                            -1 & 0  & -1 & 0  & 1 & -1 \\
                            0  & -1 & -1 & 0  & 1 & -1 \\
                            0  & -1 & 0  & -1 & 1 & -1 \\
                            0  & 0  & -1 & -1 & 1 & -1
                        \end{array}
                    \right)
                    y
                    \leq
                    \left(
                        \begin{array}{c}
                            1 \\
                            0 \\
                            0 \\
                            0 \\
                            0 \\
                            0
                        \end{array}
                    \right)
                \end{align*}

                In order, the rows of the dual matrix are due to $M$, $x_{ab}$,
                $x_{ac}$, $x_{bc}$, $x_{bd}$, $x_{cd}$.

            \item Now, we formulate the problem for a general graph
                $G = (V, E)$.

                The variables of the program in this case are $x_{uv}$
                for each $\{u, v\} \in E$, i.e. the loads given to each edge.
                We define the load on a vertex as
                $$
                l_v = \sum_{\{u, v\}} {l_{\{u, v\}}}
                $$

                The linear program is the following.
                \begin{alignat*}{3}
                    \text{minimize} \quad
                        & M \\
                    \text{subject to} \quad
                        & M \geq l_v & \text{for } v \in V \\
                        & \sum_{e \in E} l_e = 1 \\
                        & M \geq 0 \\
                        & l_e \geq 0 & \text{for } e \in E
                \end{alignat*}
        \end{enumerate}

\end{enumerate}

\end{document}
