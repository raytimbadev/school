\documentclass[letterpaper,11pt]{article}

\author{Jacob Thomas Errington (260636023)}
\title{Assignment 3\\Algorithm Design -- COMP 360}
\date{10 November 2015}

\usepackage[margin=2.0cm]{geometry}
\usepackage{amsmath,amssymb,amsthm}
\usepackage{algorithmicx,algorithm,algpseudocode}

\begin{document}

\begin{enumerate}
    \item
        --

    \item
        We notice that the constraints $|x_i| \geq 1$, $x_i \geq -1$, and
        $x_i \leq 1$ imply that $x_i \in \{-1, 1\}$, which in turn allows us to
        solve integer programs. In general, solving integer programs is
        $NP$-complete, so this augmented form of linear programming is thus
        also $NP$-complete.

    \item
        \begin{enumerate}
            \item
                This problem is $NP$-complete.

                A YES certificate for this problem is a truth assignment. We
                can verify it by first seeing if there are $M$ variables
                assigned to \texttt{true} and then checking that the assignment
                satisfies $\phi$.

                To show completeness we will reduce SAT to this problem. Given
                is a CNF $\phi$ of $n$ variables. We run the oracle for each 
                $i \in \{1, \cdots, n\}$. If for some $i$ it outputs YES, then
                we output YES. Else, we output NO.

            \item
                This problem is $NP$-complete.

                A YES certificate for this problem is a subset
                $S \in \{1, \cdots, n\}$. Given such a certificate, it suffices
                to compute the sum and check that it is equal to one of 
                $M - 1$, $M$, or $M + 1$.

                To show completeness we we will reduce the subset sum problem
                to this problem. Given are positive integers
                $A = \{a_1, \cdots, a_n\}$ and some $M$. We wish to decide
                whether there is $S \subseteq A$ such that
                $\sum_{s \in S} s = M$. It suffices to use our oracle three
                times: once on $M-1$, once on $M$, and once on $M+1$.
                If all of them come out YES, we output YES. Else, we
                output NO.

            \item
                This problem is in $P$. Here is a polytime algorithm for
                solving it.

                \begin{algorithm}[ht]
                    \caption{
                        A polytime algorithm to decide whether there exists a
                        truth assignment for a given CNF that satisfies none of
                        its clauses.
                    }

                    \begin{algorithmic}
                        \Require{A CNF $\phi$.}
                        \Ensure{
                            Whether there exists a truth assignment satisfying
                            none of the clauses of $\phi$.
                        }

                        \State Initialize an empty dictionary $T$.

                        \For{each clause $C$ in $\phi$}
                            \For{each literal $x$ in $C$}
                                \If{$x$ is of the form $\neg x^\prime$}
                                    \Comment{
                                        We need to assign \texttt{true} to
                                        $x^\prime$ to make $x$ false.
                                    }
                                    \If{$x^\prime \in T$ and $T[x^\prime] = 0$}
                                        \Comment{
                                            But a previous iteration assigned
                                            \texttt{false} to $x^\prime$!
                                        }
                                        \State \Return \texttt{false}
                                    \Else
                                        \State $T[x^\prime] \gets 0$
                                    \EndIf
                                \Else
                                    \Comment{
                                        We need to assign \texttt{false} to
                                        $x^\prime$ to make $x$ false.
                                    }
                                    \If{$x^\prime \in T$ and $T[x^\prime] = 1$}
                                        \Comment{
                                            But a previous iteration assigned
                                            \texttt{true} to $x^\prime$!
                                        }
                                        \State \Return \texttt{false}
                                    \Else
                                        \State $T[x^\prime] \gets 1$
                                    \EndIf
                                \EndIf
                            \EndFor
                        \EndFor

                        \State \Return \texttt{true}
                        \Comment{
                            $T$ is the assignment ensuring that no clause is
                            satisfied.
                        }
                    \end{algorithmic}
                \end{algorithm}
                
            \item
                Since the number of two-colorings on a graph is unique up to
                a renaming of the colors, it suffices to color the graph and
                count the number of vertices of each color to see whether the
                graph admits a two-coloring with a given number of vertices
                having a certain color. Hence, the problem is in $P$.

            \item
                This problem is in $P$.

                Our algorithm is a simple modified version of the two-coloring
                algorithm in which we using only the color $R$ when we have no
                other choice. This in turn will result in a $3$-coloring where
                the number of vertices colored $R$ is minimal. If that number
                is at most $100$, output YES. Else, output NO.

            \item
                This problem is $NP$-complete.

                A certificate for this problem is twofold: it consists of a
                path $P$ and two sets of vertices $A$ and $B$. To verify the
                certificate, we travel along the path $P$ checking that each
                new vertex we visit has not been visited before, e.g. by
                marking them as we visit them, and that we finally return to
                the start point having visited all vertices. This shows that
                there is a hamiltonian cycle. Then we check that the graph is
                bipartite by ensuring that no vertex from $A$ is adjacent to
                another vertex from $A$, and likewise for $B$.

                To show completeness, we will reduce the general hamiltonian
                cycle problem to this problem. We can in fact transform an
                arbitrary graph $G$ into a bipartite graph by splitting each
                vertex $v_i$ into four vertices, each one connected to the
                next. All the edges that used to go to $v_i$ will go to the
                first vertex and to the last vertex of this split. The result
                is that we can partition the vertex set of this new graph into
                the set of all first and third vertices resulting from the
                split and into the set of all second and fourth vertices
                resulting from the split. Running the oracle on this new graph
                will thus only determine whether the graph contains a
                hamiltonian cycle; the fact that the algorithm also determine
                bipartiteness is now irrelevant since we know the graph to be
                bipartite anyway. If the new graph contains a hamiltonian
                cycle, then the original one must as well, since we can
                collapsing the new graph according to the way we constructed it
                preserves hamiltonian cycles.
        \end{enumerate}
\end{enumerate}
\end{document}
