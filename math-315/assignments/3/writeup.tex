\documentclass[letterpaper,11pt]{article}

\author{Jacob Thomas Errington (260636023)}
\title{Assignment \#3\\Ordinary Differential Equations -- MATH 315}
\date{24 November 2015}

\usepackage{amsmath,amssymb,amsthm}
\usepackage[margin=2.0cm]{geometry}

\DeclareMathOperator{\D}{D}
\newcommand{\R}{\mathbb{R}}
\newcommand{\intd}[1]{{\ensuremath \; \mathrm{d}#1}}

\newcommand{\deriv}[1]{{\ensuremath \frac{\mathrm{d}}{\mathrm{d}#1}}}
\newcommand{\nderiv}[2]{{\ensuremath \frac{\mathrm{d}^#1}{\mathrm{d}#2^#1}}}

\begin{document}

\maketitle

\begin{enumerate}
    \item
        \begin{enumerate}
            \item
                To find the annihilator for the nonhomogeneous term, we will
                find annihilators for each term in the sum and compose them.

                \begin{enumerate}
                    \item
                        The term $3xe^x$ has annihilator $(\D{} - 1)^2$.

                    \item
                        The term $2e^{-2x}$ has annihilator $\D{} + 2$.

                    \item
                        The term $x \sin{2x}$ has annihilator $(\D^2{} + 4)^2$.
                \end{enumerate}

                Hence, the full annihilator is the following.
                $$
                (\D{} - 1)^2 (\D{} + 2)(\D^2{} + 4)^2
                $$

            \item
                We find the roots of the following polynomial to to find the
                basis for the set of solutions to the annihilated equation.
                $$
                (r - 1)^2 (r + 2) (r^2 + 4)^2 (r^3 + 2 r^2 + 4r + 8)
                $$

                The solutions and their multiplicities are the following.
                \begin{align*}
                    r_1 &=  1  &\quad (2)\\
                    r_2 &= -2  &\quad (2)\\
                    r_3 &= -4  &\quad (1)\\
                    r_4 &=  2i &\quad (3)\\
                    r_5 &= -2i &\quad (3)
                \end{align*}

                From these solutions, we get the following basis for the
                solution set of the nonhomogeneous equation.
                \begin{align*}
                    y_1 &= e^x \\
                    y_2 &= x e^x \\
                    y_3 &= e^{-2x} \\        % HOM
                    y_4 &= x e^{-2x} \\
                    y_5 &= e^{-4x} \\
                    y_6 &= e^{2ix} \\        % HOM
                    y_7 &= x e^{2ix} \\
                    y_8 &= x^2 e^{2ix} \\
                    y_9 &= e^{-2ix} \\       % HOM
                    y_{10} &= x e^{-2ix} \\
                    y_{11} &= x^2 e^{-2ix} \\
                \end{align*}

                The solution to the corresponding homogeneous equation is the
                following.
                $$
                y_H = c_1 e^{-2x} + c_2 e^{-2ix} + c_3 e^{2ix}
                $$

                We notice that components $\#3$, $\#6$, and $\#9$ from the
                basis of the solution set of the nonhomogeneous equation
                correspond to terms from the general solution of the
                complementary homogeneous equation.
                The particular solution is thus
                $$
                y_p = c_1 e^x
                    + c_2 x e^x
                    + c_4 x e^{-2x}
                    + c_5 e^{-4x}
                    + c_7 x e^{2ix}
                    + c_8 x^2 e^{2ix}
                    + c_{10} x e^{-2ix}
                    + c_{11} x^2 e^{-2ix}
                $$

            \item
                The general solution to the equation is given by the basis we
                found from the preceding question. Specifically, it is
                following.
                \begin{align*}
                    y &=
                    c_{1} e^x +
                    c_{2} x e^x +
                    c_{3} e^{-2x} +
                    c_{4} x e^{-2x} +
                    c_{5} e^{-4x} \\
                    &+ c_{6} e^{2ix} +
                    c_{7} x e^{2ix} +
                    c_{8} x^2 e^{2ix} +
                    c_{9} e^{-2ix} +
                    c_{10} x e^{-2ix} +
                    c_{11} x^2 e^{-2ix}
                \end{align*}
        \end{enumerate}

    \item
        We wish to find a function $z = z(x)$ such that its effect on the
        derivatives of $y$ is to cancel out the nonconstant coefficients.

        The effect of this substitution on the derivatives of $y$ is the
        following.
        \begin{align*}
            \deriv{x} y &= \deriv{z} y(z) \deriv{x} z \\
            \nderiv{2}{x} y &= \nderiv{2}{z} y(z) \left(\deriv{x} z\right)^2
        \end{align*}

        Substituting these back into the original differential equation, we
        establish a certain number of constraints that $z$ must satisfy in
        order to cancel out the nonconstant coefficients.
        \begin{align*}
            \left(\deriv{x} z\right)^2 &= \frac{1}{x^2} \\
            \nderiv{2}{x} z &= \frac{1}{x^2} \\
            \deriv{x} z &= \frac{1}{x}
        \end{align*}

        We observe that $z(x) = \ln x$ is a suitable choice of substitution.
        The equation becomes
        $$ \nderiv{2}{z} y(z) + (\alpha - 1) \deriv{z} y(z) + 3 y(z) = 0 $$
        which we can write in terms of the differential operator $\D{}$ as
        $$ \left( \D^2{} + (\alpha - 1) \D{} + 3 \right)y = 0 $$

        This polynomial in $\D{}$ has two roots (counting multiplicity), whose
        values will affect the behaviour of solutions at infinity. The form of
        the polynomial indicates that the following cases can arise according
        to the value of $\alpha$.

        \begin{description}
            \item[Distinct negative real roots.]

                The form of solutions will be
                \begin{equation*}
                    y = c_1 e^{r_1 z} + c_2 e^{r_2 z}
                    = c_1 e^{r_1 \ln x} + c_2 e^{r_2 \ln x}
                \end{equation*}

                As $x \to \infty$, $\ln x \to \infty$, so
                $r_1 \ln x \to - \infty$ and $r_2 \ln x \to -\infty$ since
                $r_1 < 0$ and $r_2 < 0$. Thus, for any choices of $c_1 \in \R$
                and $c_2 \in \R$ not both zero, the solution will approach
                zero.

                This case arises for $\alpha > 1 + 2 \sqrt 3$.

            \item[Distinct positive real roots.]

                The form of solutions is the same as before, but since
                $r_1 > 0$ and $r_2 > 0$, the solutions will grow without bound.

                This case arises for $\alpha < 1 - 2 \sqrt 3$.

            \item[Repeated negative real roots.]

                The form of solutions will be
                \begin{equation*}
                    y = c_1 e^{rz} + c_2 z e^{rz}
                    = c_1 e^{r \ln x} + c_2 \left(\ln x\right) e^{r \ln x}
                    = c_1 x^r + c_2 x^r \ln x
                \end{equation*}

                The first term will approach zero as shown in our analysis of
                distinct negative real roots. Since $r < 0$, the second term
                will also approach zero.

                This case occurs for $\alpha = 1 - 2 \sqrt 3$.

            \item[Repeated positive real roots.]

                The form of solutions will be the same as for repeated negative
                real roots. However, since $r > 0$, both terms will approach
                infinity.

                This case occurs for $\alpha = 1 + 2 \sqrt 3$.

            \item[Complex roots.]

                When the discriminant of the polynomial in $\D{}$ is negative,
                complex roots arise, introducing factors of the form
                $\cos{(\mu_j \ln x)} + i \sin{(\mu_j \ln x)}$ to the terms that
                make up the solutions. This in turn results in an oscillatory
                behaviour for all solutions. The range of the osciallation for
                each term is determined by a factor $x^\lambda$. For
                $\lambda > 0$, the limit oscillates between $-\infty$ and
                $\infty$. For $\lambda < 0$, the limit converges to $0$. For
                $\lambda = 0$, the limit oscillates between $-c_j$ and $c_j$.
                This analysis of the oscillation bounds is valid for each term
                in the solution in isolation, but since these limits do not
                exist for certain values of $\lambda$, it is not valid when we
                consider the limiting value of their sum.

                First, we must determine which values of $\alpha$ result in
                complex roots. We know on what values of $\alpha$ we have real
                roots, so the difference of the reals by the set of those
                values of $\alpha$ will give the values of $\alpha$ for which
                we obtain complex valued roots.
                \begin{equation*}
                    \R \setminus
                    (-\infty, 1 - 2 \sqrt 3]
                    \cup
                    [1 + 2 \sqrt 3, \infty)
                    =
                    (1 - 2 \sqrt 3, 1 + 2 \sqrt 3)
                \end{equation*}

                Next, we must determine how the choice of $\alpha$ affects
                $\lambda$, i.e. the real part of the root. From the real part
                of the quadratic formula applied to the polynomial in $\D{}$,
                we obtain the following result.
                \begin{equation*}
                    \lambda = \frac{1 - \alpha}{2}
                \end{equation*}

                Hence, $\lambda = 0$ requires that $\alpha = 1$. So for
                $\alpha > 1$, the limit will converge. for
                $1 - s \sqrt 3 < \alpha < 1$, the limit will oscillate.
        \end{description}

    \item
        To solve the nonhomogeneous second-order differential equation using
        the method of variation of parameters, we must first solve the
        complementary homogeneous equation. To do so, we will make the
        substitution $z(x) = \ln x$ to eliminate the nonconstant coefficients
        and solve the resulting characteristic equation. We obtain the
        following general solution.
        \begin{equation*}
            y_H = c_1 x^2 + c_2 x^3
        \end{equation*}

        Next, our goal is to find functions $u_1(x)$ and $u_2(x)$ such that
        \begin{equation*}
            y_p = y_1(x) u_1(x) + y_2(x) u_2(x)
        \end{equation*}
        where $y_1(x) = x^2$ and $y_2(x) = x^3$.

        We assume that $u_1^\prime y_1 + u_2^\prime y_2 = 0$. Hence,
        \begin{align*}
            y_p^\prime &= u_1 y_1^\prime + u_2 y_2^\prime \\
            y_p^{\prime\prime}
            &= u_1^\prime y_1^\prime + u_1 y_1^{\prime\prime}
            + u_2^\prime y_2^\prime + u_2 y_2^{\prime\prime}
        \end{align*}

        Substituting these back into the original differential equation causes
        two terms to cancel out as they are solutions to the complementary
        homogeneous equation, leaving us with
        \begin{equation*}
            u_1^\prime y_1^\prime + u_2^\prime + y_2^\prime = \frac{g(t)}{p(t)}
        \end{equation*}
        where $g(t) = x^4 \sin x$.

        We need one more assumption to make things workable, which is that the
        nonhomogeneous term be the constant function $p(t) = 1$. To achieve
        this we will divide across the entire equation by $x^2$. The
        nonhomogeneous term thus becomes $h(x) = x^2 \sin x$.

        Now we can use the following system of equations to solve for $u_1(x)$
        and $u_2(x)$.
        \begin{align*}
            u_1^\prime y_1^\prime + u_2^\prime + y_2^\prime &= h(x)
            u_1^\prime y_1 + u_2^\prime y_2 &= 0
        \end{align*}

        First, we isolate $u_1$ in the second equation and substitute it into
        the first equation. This gives a single equation for $u_2$ that we can
        solve by integrating.
        \begin{align*}
            u_2^\prime &= \frac{y_1 h(x)}{y_1 y_2^\prime - y_1^\prime y_2} \\
            u_2 &=
                \int{\frac{(x^2)(x^2 \sin x)}{(x^2)(2x^2) - (x^3)(2x)}}\intd{x} \\
                &= \int{\sin x}\intd{x} \\
                &= -\cos x + C
        \end{align*}
        We choose $C = 0$ since we only need one value for $u_2$.

        Next, we substitute this function for $u_2$ back into the second
        equation and solve for $u_1$ using integration by parts.
        \begin{align*}
            u_1^\prime &= \frac{x^3 \cos x}{x^2} = x \cos x \\
            u_1 &= \int{x \cos x}\intd{x} \\
                &= x \sin x + \cos x + C
        \end{align*}
        Again we choose $C = 0$.

        Hence, a particular solution to the differential equation is
        \begin{equation*}
            y_p = (x^2)(x \sin x + \cos x) - (x^3)(\cos x)
        \end{equation*}

        Adding this particular solution to the general solution of the
        homogeneous equation gives the general solution of the nonhomogeneous
        equation.
        \begin{equation*}
            y = (x^2)(x \sin x + \cos x) - (x^3)(\cos x)
                + c_1 x^2 + c_2 x^3
        \end{equation*}
\end{enumerate}

\end{document}
