\documentclass{article}

\usepackage{amsmath}

\author{Jacob Errington (260626023)}
\date{12 February 2015}
\title{Assignment \#2\\COMP 273}

\newcommand{\E}[1]{\times 10^{#1}}

\begin{document}

\maketitle

See the attached \texttt{.circ} file for the 3-bit adder.

\section*{Theoretical questions}

\begin{enumerate}
    \item $2\,\mathrm{GHz} = 2\,\mathrm{Gs}^{-1} = 2\E{9}\,\text{ticks/second}$.

        Two billion ticks per second.

    \item Dimensional analysis is the only real problem solving technique.
        $$ 2\E{9} \text{ ticks/s} \times 2\E{-9} \text{ s} = 4 \text{ ticks} $$

    \item A cycle is the amount of time or number of ticks required to complete
        a single stage in the CPU, in the worst case.

    \item We can visualize the pipelining of the CPU by stacking the concurrent
        paths at $t=0\text{ s}$.

\begin{verbatim}
FDEWFDEW
 FDEWFDEW
  FDEWFDEW
    FDEWFDEW
\end{verbatim}
        where
        \begin{description}
            \item[F.] Fetch.
            \item[D.] Decode.
            \item[E.] Execute.
            \item[W.] Write
        \end{description}
        assuming that at $t=0 \text{ s}$ the processor was not in the middle of anything.

        Then, at $t=1\text{ s}$, the pipeline will look like the following:

\begin{verbatim}
FDEW|
WFDE|W
EWFD|EW
DEWF|DEW
\end{verbatim}

        where the vertical bars show $t=1\text{ s}$. Conveniently, each
        character in these fancy diagrams represents one cycle, so we get the following analysis:
        \begin{equation*}
            \frac{\frac{1\text{ s}}{2 \text{ ns/cycle}}}{4\text{ cycles/instruction}} 
                =  125 000 000\text{ instructions}
        \end{equation*}

        But wait, there's more!

        Line $N$ in these diagrams is offset by $N-1$ cycles, so only line 1
        will perform an integer number of instructions, precisely $125 000 000$
        instructions as we saw above. The other lines will perform one
        instruction less, plus some fractional part equal to the
        $1 - \frac{\text{number of offset cycles}}{4}\text{ instructions}$.

        So let's add all that up:
        \begin{align*}
            125 000 000 &+ \\
            124 999 999 &+ \frac{3}{4} \\
            124 999 999 &+ \frac{1}{2} \\
            124 999 999 &+ \frac{1}{4} \\
                        &= 499 999 998.5
        \end{align*}

    \item In a classical CPU, each instruction must be completely finished
        before the subsequent one can begin. Therefore the math is easy: we just do some divisions.
        \begin{equation*}
            \frac{\frac{1\text{ s}}{2 \text{ ns/cycle}}}{4\text{ cycles/instruction}} 
                =  125 000 000\text{ instructions}
        \end{equation*}
        which is roughly four times less than in a four-stage pipeline CPU, as we would expect.

    \item The pipeline CPU offers a certain amount of concurrency that is
        unmatched by the classical CPU in that it allows itself to be maximally
        "busy". The circuitry required to perform each of the different stages
        is separate to a certain extent: in a classical CPU this goes to waste,
        whereas in a pipeline CPU while one part of the CPU is performing the
        decode, another is performing the fetch, and yet another is actually
        carrying out the execution. That is why the pipeline CPU is generally
        faster than a classical CPU.


\end{enumerate}

\end{document}
