\documentclass[12pt]{article}
\usepackage[utf8]{inputenc} 

\title{\vspace{-3em}Application for HGEN 396 Project}
\author{}
\date{\vspace{-3em}20 November 2014}

\begin{document}

\maketitle

\begin{description}
    \itemsep0em
    \item[Supervisor's Name] Simon Gravel
    \item[Supervisor's Email] \texttt{simon.gravel@mcgill.ca}
    \item[Supervisor's Phone]
    \item[Supervisor's Website] \texttt{http://simongravel.lab.mcgill.ca/Home.html}
    \item[Supervisor's department] Human Genetics
    \item[Course number] HGEN 396 (Human Genetics)
    \item[Term] Winter 2014-2015
    \item[Project Duration] from Monday, 5 January 2015 to Tuesday, 14 April 2015
    \item[Project title] New method for unconstrained global optimization, with applications to genetics
    \item[Project description] Many applications in bioinformatics and population genetics require extensive optimization, especially for model fitting in a maximum-likelihood framework. Optimization of these problems is typically difficult due to the discontinuous nature of the functions to optimize, the high number of variables, and the existence of "sloppy" directions. Despite much existing research into the topic of optimization, available methods fall short in terms of practical convergence to the optimum. Widely-used population-genetics programs such as dadi suffer from this poor convergence and require users to expend thousands of hours of CPU time. Recent progress in constrained optimization suggests that better unconstrained optimizers can be designed for realistic problems encountered in population genetics. The aim of this project is to implement such an unconstrained global optimizer for likelihood functions of several variables, ideally comparable or superior to existing methods such as simulated annealing, test the optimizer of realistic genetic datasets, and make the method available to the community as a freely available package that will interface with widely used population genetics inference software.
    \item[Prerequisite] 1 term completed at McGill + CGPA of 3.0 or higher; or permission of instructor.
    \item[Grading scheme] Final grade shall be based on an evaluation of laboratory (or equivalent) performance (40\%), a final written report (50\%), and an oral presentation (10\%) by the supervisor. Details of the evaluation scheme will be provided by HGEN to supervisors and applicants.
    \item[Project status] This project is taken. The professor has no more 396 projects this term.
    %\item[How students can apply / Next steps] After all of the parts of this application forms are completed and the hard copy is signed by the professor and the student, bring the application form and a copy of your unofficial transcript to the Department of Human Genetics (to the attention of Dr.  Patricia N. Tonin, Stewart Biology Building N5.13) during office hours.  
    \item[Ethics, safety, and training] Supervisors are responsible for the ethics and safety compliance of undergraduate students.  This project involves: None of the above (NEITHER animal subjects, nor human subjects, nor biohazardous substances, nor radioactive materials, nor handling chemicals, nor using lasers)
    \item[Student's Name] Jacob Errington
    \item[Student's McGill ID] 260636023
    \item[Student's Email] \texttt{jacob.errington@mail.mcgill.ca}
    \item[Student's Phone] (514) 503-3100
    \item[Student's Program] B.Sc. Maj. Comp. Sci. \& Math.
    \item[Student's Level] U1
    \item[Student's signature] 

    I certify that this course is with a different supervisor and on a different topic than any previous 396 course I have taken. I have not applied for another 396 course in this term. \\ \\

    \item[Supervisor's Signature] I give my permission for the student identified above to register for this project under my supervision \\ \\

    \item[Unit chair/director/designate's name]
    \item[Unit chair/director/designate's signature] 
    
    I certify that this project conforms to departmental requirements for 396 courses. \\ \\
\end{description}
\end{document}
