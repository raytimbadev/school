\documentclass[11pt,letterpaper]{article}

\author{Jacob Thomas Errington (260636023)}
\title{Assignment \#4\\Logic and computation -- COMP 527}
\date{24 March 2016}

\usepackage[margin=2.0cm]{geometry}
\usepackage{amsmath,amsthm,amssymb,proof}

\newtheorem{theorem}{Theorem}
\newtheorem{lemma}{Lemma}

\DeclareMathOperator{\opNode}{Node}
\DeclareMathOperator{\opTree}{tree}
\DeclareMathOperator{\opNat}{nat}
\DeclareMathOperator{\opTrue}{true}
\DeclareMathOperator{\opRec}{\mathtt{rec}}
\DeclareMathOperator{\opLet}{\mathtt{let}}
\DeclareMathOperator{\opIn}{\mathtt{in}}
\DeclareMathOperator{\opWith}{\mathtt{with}}
\DeclareMathOperator{\opInit}{\mathrm{init}}

\newcommand{\emptyTree}{\mathrm{Empty}}
\newcommand{\nodeTree}[3]{\opNode{#1\,#2\,#3}}
\newcommand{\tree}{\opTree{}}
\newcommand{\nat}{\opNat{}}
\newcommand{\with}{\opWith{}}
\newcommand{\case}[3]{#1\,#2\mapsto #3}
\newcommand{\orCase}{\mid}
\newcommand{\rec}[1]{\opRec{\,#1\,}}
\newcommand{\reducesTo}{\implies}
\newcommand{\olet}{\opLet{}\,}
\newcommand{\oin}{\opIn{}\,}
\newcommand{\proves}{\vdash}
\newcommand{\init}{\opInit{}}
\newcommand{\seq}{\implies}
\newcommand{\imp}{\supset}
\renewcommand{\t}{\opTrue{}}

\begin{document}

\maketitle

\section{Induction principles for binary trees}

Consider the following definition of binary trees.

\begin{align*}
    \infer{\emptyTree : \text{tree}}{}
    &
    \quad
    &
    \infer{\nodeTree{n}{t_1}{t_2} : \tree}{
        n : \nat & t_1 : \tree & t_2 : \tree
    }
\end{align*}

\begin{enumerate}
    \item Here is an induction rule for binary trees.

        \begin{equation*}
            \infer[\tree E^{n, t_1, t_2, ih_1, ih_2}]{A(t) \t}{
                t : \tree
                &
                A(\emptyTree) \t
                &
                \infer*{A(\nodeTree{n}{t_1}{t_2}) \t}{
                    \infer{n : \nat}{}
                    &
                    \infer{t_1 : \tree}{}
                    &
                    \infer{t_2 : \tree}{}
                    &
                    \infer[ih_1]{A(t_1) \t}{}
                    &
                    \infer[ih_2]{A(t_2) \t}{}
                }
            }
        \end{equation*}

    \item Here is an induction rule for binary trees annotated with proof
        terms.

        \begin{equation*}
            \infer[\tree E^{n, t_1, t_2, ih_1, ih_2}]{
                \rec{t} \with
                \case{f}{\emptyTree}{M_E}
                \orCase
                \case{f}{\nodeTree{n}{t_1}{t_2}}{M_N}
                : A(t)
            }{
                t : \tree
                &
                M_E : A(\emptyTree)
                &
                \infer*{M_N : A(\nodeTree{n}{t_1}{t_2})}{
                    \infer{n : \nat}{}
                    &
                    \infer{t_1 : \tree}{}
                    &
                    \infer{t_2 : \tree}{}
                    &
                    \infer[ih_1]{f\, t_1 : A(t_1)}{}
                    &
                    \infer[ih_2]{f\, t_2 : A(t_2)}{}
                }
            }
        \end{equation*}

    \item Here is a reduction rule for recursion on binary trees.

        \begin{equation*}
            \rec{\emptyTree}
            \with \case{f}{\emptyTree}{M_E}
            \orCase \case{f}{\nodeTree{n}{t_1}{t_2}}{M_N} \reducesTo M_E
        \end{equation*}

        \begin{align*}
            &\rec{\nodeTree{m}{s_1}{s_2}}
            \with \case{f}{\emptyTree}{M_E}
            \orCase \case{f}{\nodeTree{n}{t_1}{t_2}}{M_N} \\
            \reducesTo
            &\olet r_1 = \rec{s_1}
                \with \case{f}{\emptyTree}{M_E}
                \orCase \case{f}{\nodeTree{n}{t_1}{t_2}}{M_N} \oin \\
            &\olet r_2 = \rec{s_2}
                \with \case{f}{\emptyTree}{M_E}
                \orCase \case{f}{\nodeTree{n}{t_1}{t_2}}{M_N} \oin \\
            &\left[m/n, s_1/t_1, s_2/t_2, r_1/f\,t_1, r_2/f\,t_2\right]M_N
        \end{align*}

    \item The proposed rules preserve typing.

        \begin{proof} By examining the operation of the rules.
            \begin{description}
                \item[Base case.] Using the first reduction rule.

                    \begin{tabular}{c r}
                        $
                        \rec{\emptyTree}
                        \with \case{f}{\emptyTree}{M_E}
                        \orCase \case{f}{\nodeTree{n}{t_1}{t_2}}{M_N}
                        : A(\emptyTree)
                        $
                        &
                        ass. \\
                        $
                        M_E : A(\emptyTree)
                        $
                        &
                        by $\tree E$
                    \end{tabular}

                \item[Step case.] Using the second reduction rule.

                    \begin{tabular}{c r}
                        $ % ***
                        \rec{\nodeTree{m}{s_1}{s_2}}
                        \with \case{f}{\emptyTree}{M_E}
                        \orCase \case{f}{\nodeTree{n}{t_1}{t_2}}{M_N}
                        : A(\nodeTree{m}{s_1}{s_2})
                        $
                        &
                        ass. \\
                        $ % ***
                        \nodeTree{m}{s_1}{s_2} : \tree,
                        m : \nat, s_1 : \tree, s_2 : \tree
                        $
                        &
                        by $\tree E$ \\
                        $ % ***
                        M_E : A(\emptyTree)
                        $
                        &
                        ~ \\
                        $ % ***
                        n : \nat, t_1 : \tree, t_2 : \tree,
                        ih_1 : A(t_1), ih_2 : A(t_2)
                        \proves M_N : A(\nodeTree{n}{t_1}{t_2})
                        $
                        &
                        ~ \\
                        $ % ***
                        ih_1 : A(s_1), ih_2 : A(s_2)
                        \proves
                        \left[m/n, s_1/t_1, s_2/t_2\right]
                        M_N : A(\nodeTree{m}{s_1}{s_2})
                        $
                        &
                        subst. \\
                        $ % ***
                        \proves
                        \left[
                            m/n, s_1/t_1, s_2/t_2, r_1/f\, t_1, r_2/f\, t_2
                        \right]
                        M_N : A(\nodeTree{m}{s_1}{s_2})
                        $
                        &
                        subst. \\
                    \end{tabular}
            \end{description}
        \end{proof}

    \item Here is a primitive recursive program to compute the size of a binary
        tree.

        \begin{equation*}
            \rec{t}
            \with \case{f}{\emptyTree}{z}
            \orCase \case{f}{\nodeTree{n}{t_1}{t_2}}{s(f\, t_1 + f\, t_2)}
        \end{equation*}

\end{enumerate}

\section{Sequent calculus}

\subsection{Practice}

See figures \ref{fig:seq-calc-1} and \ref{fig:seq-calc-2}.

\begin{figure}[ht]
    \centering
    Let $\Gamma = A \land (B \lor C)$ and $\Gamma^\prime = \Gamma, A, B \lor C$.
    \begin{equation*}
        \infer[\land L_l]{\Gamma \seq A \land B \lor A \land C}{
            \infer[\land L_r]{\Gamma, A \seq A \land B \lor A \land C}{
                \infer[\lor L^{u,v}]{\Gamma^\prime \seq A \land B \lor A \land C}{
                    \infer[\lor R_l]{\Gamma^\prime, u:B \seq A \land B \lor A \land C}{
                        \infer[\land R]{\Gamma^\prime, u:B \seq A \land B}{
                            \infer[\init]{\Gamma^\prime, u:B \seq A}{}
                            &
                            \infer[\init]{\Gamma^\prime, u:B \seq B}{}
                        }
                    }
                    &
                    \infer[\lor R_r]{\Gamma^\prime, v:C \seq A \land B \lor A \land C}{
                        \infer[\land R]{\Gamma^\prime, v:C \seq A \land C}{
                            \infer[\init]{\Gamma^\prime, v:C \seq A}{}
                            &
                            \infer[\init]{\Gamma^\prime, v:C \seq C}{}
                        }
                    }
                }
            }
        }
    \end{equation*}
    \caption{
        Sequent calculus proofs can be very large sometimes.
    }
    \label{fig:seq-calc-1}
\end{figure}

\begin{figure}[ht]
    \centering
    Let $\Gamma = (D \imp B) \imp C \imp D, B \imp A, (C \imp D) \imp B$.
    \begin{equation*}
        \infer[\imp L]{\Gamma \seq A}{
            \infer[\imp L]{\Gamma \seq B}{
                \infer[\imp R^u]{\Gamma \seq C \imp D}{
                    \infer[\imp L]{\Gamma, u:C \seq D}{
                        \infer[\imp R^v]{\Gamma, u:C \seq (D \imp B) \imp C}{
                            \infer[\init]{\Gamma, u:C, v:D \imp B \seq C}{}
                        }
                        &
                        \infer[\init]{\Gamma, u:C, D \seq D}{}
                    }
                }
                &
                \infer[\init]{\Gamma, B \seq B}{}
            }
            &
            \infer[\init]{\Gamma, A \seq A}{}
        }
    \end{equation*}
    \caption{
        I can't tell which is crazier; this one, or the previous one?
    }
    \label{fig:seq-calc-2}
\end{figure}

\section{Optimization}

\begin{lemma}
    For all propositions $A$, $$\Gamma, A \seq A$$
\end{lemma}

\begin{proof}
    By structural induction on $A$.

    \begin{description}
        \item[Base case.] $A = P$, i.e. $A$ is atomic. Then by $init^\prime$ we
            have $\Gamma, A \seq A$.

        \item[Base case.] $A = \top$

            By $\top R$, $\Gamma \seq \top$ and by weakening
            \begin{equation*}
                \Gamma, \top \seq \top
            \end{equation*}

        \item[Base case.] $A = \bot$

            By $\bot L$, $\Gamma, \bot \seq C$ for any $C$. Instantiating $C$
            to $\bot$ completes the construction.

        \item[Case] $A = B \land C$

            \begin{equation*}
                \infer[\land R]{\Gamma, B \land C \seq B \land C}{
                    \infer[\land L_l]{\Gamma, B \land C \seq B}{
                        \deduce[\mathcal{D}]{\Gamma, B \land C, B \seq B}{}
                    }
                    &
                    \infer[\land L_r]{\Gamma, B \land C \seq C}{
                        \deduce[\mathcal{E}]{\Gamma, B \land C, C \seq C}{}
                    }
                }
            \end{equation*}

            We apply the induction hypothesis on $\Gamma, B \seq B$ and use
            weakening to build the derivation $\mathcal{D}$. Likewise we apply
            the induction hypothesis on $\Gamma, C \seq C$ and use weakening to
            build $\mathcal{E}$.


        \item[Case] $A = B \imp C$

            \begin{equation*}
                \infer[\imp R]{\Gamma, B \imp C \seq B \imp C}{
                    \infer[\imp L]{\Gamma, B \imp C, B \seq C}{
                        \deduce[\mathcal{D}]{\Gamma, B \imp C, B \seq B}{}
                        &
                        \deduce[\mathcal{E}]{\Gamma, B \imp C, C \seq C}{}
                    }
                }
            \end{equation*}

            We proceed by using the induction hypothesis and weakening as in
            the case of conjunction.

        \item[Case] $A = B \lor C$

            \begin{equation*}
                \infer[\lor L^{u, v}]{\Gamma, B \lor C \seq B \lor C}{
                    \infer[\lor R_l]{\Gamma, B \lor C, u:B \seq B \lor C}{
                        \deduce[\mathcal{D}]{\Gamma, B \lor C, u:B \seq B}{}
                    }
                    &
                    \infer[\lor R_r]{\Gamma, B \lor C, v:C \seq B \lor C}{
                        \deduce[\mathcal{E}]{\Gamma, B \lor C, v:C \seq C}{}
                    }
                }
            \end{equation*}

            We proceed by using the induction hypothesis and weakening as in
            the case of conjunction.

    \end{description}
\end{proof}

\section{Proving cut}

Omitted.

\end{document}
