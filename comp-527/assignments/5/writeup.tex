\documentclass[letterpaper,11pt]{article}

\author{Jacob Thomas Errington (260636023)}
\title{Assignment \#5\\Logic and computation -- COMP 527}
\date{12 April 2016}

\usepackage[margin=2.0cm]{geometry}
\usepackage{amsmath,amssymb,amsthm}
\usepackage{proof}

\newtheorem{prop}{Proposition}
\newtheorem{lemma}{Lemma}

\newcommand{\seq}{\implies}
\newcommand{\imp}{\supset}
\newcommand{\step}{\rightarrow}
\newcommand{\R}{\mathcal{R}}
\newcommand{\proves}{\vdash}

\newcommand{\nand}{\barwedge}

\DeclareMathOperator{\fst}{\mathtt{fst}}
\DeclareMathOperator{\snd}{\mathtt{snd}}
\DeclareMathOperator{\initOp}{init}
\newcommand{\init}{\initOp{}}

\begin{document}

\maketitle

\section{Inversion lemmas}

\begin{prop}
    If $\Gamma \seq A \imp B$, then $\Gamma, A \seq B$.
\end{prop}

\begin{proof} ~


    \begin{center}
        \begin{tabular}{c r}
            $\Gamma, A \imp B, A \seq A$
            &
            by init and weakening \\
            $\Gamma, A, A \imp B, B \seq B$
            &
            by init and weakening \\
            $\Gamma, A, A \imp B \seq B$
            &
            by $\imp L$ \\
            $\Gamma, A \seq B$
            &
            by cut
        \end{tabular}
    \end{center}
\end{proof}

\section{A sequent calculus for $\nand$}

Here are sequent calculus rules for $\nand$.

\begin{equation*}
    \infer[\nand R]{\Gamma \seq A \nand B}{
        \Gamma, A, B \seq p
    }
\end{equation*}

\begin{equation*}
    \infer[\nand L]{\Gamma, A \nand B \seq C}{
        \Gamma, A \nand B, p \seq C
        &
        \Gamma, A \nand B \seq A
        &
        \Gamma, A \nand B \seq B
    }
\end{equation*}

\begin{prop}
    The right rule is invertible.
\end{prop}

\begin{proof} ~

    \begin{center}
        \begin{tabular}{c r}
            $\Gamma \seq A \nand B$
            &
            by assumption \\
            $\Gamma, A, B, A \nand B \seq A$
            &
            by init \\
            $\Gamma, A, B, A \nand B \seq B$
            &
            by init \\
            $\Gamma, A, B, A \nand B, C \seq C$
            &
            by init \\
            $\Gamma, A, B, A \nand B \seq C$
            &
            by $\nand L$ \\
            $\Gamma, A, B \seq C$
            &
            by cut
        \end{tabular}
    \end{center}
\end{proof}

\begin{prop}
    The left rule is not invertible.
\end{prop}

\begin{proof}
    If the left rule were invertible, then it could be eagerly applied without
    reaching a dead end when trying to prove a true statement.

    First, we will show $A \nand B \seq B \nand A$.

    \begin{equation*}
        \infer[\nand R]{A \nand B \seq B \nand A}{
            \infer[\nand L]{A \nand B, A, B \seq p}{
                \infer[\init\; q \sim p]{A \nand B, A, B, q \seq p}{}
                &
                \infer[\init]{A \nand B, A, B \seq A}{}
                &
                \infer[\init]{A \nand B, A, B \seq B}{}
            }
        }
    \end{equation*}

    However, if we apply the left rule first, we are backed into a dead end.

    \begin{equation*}
        \infer[\nand L]{A \nand B \seq B \nand A}{
            \infer[\init\; p \sim B \nand A]{A \nand B, p \seq B \nand A}{}
            &
            \infer{A \nand B \seq A}{???}
            &
            \infer{A \nand B \seq B}{???}
        }
    \end{equation*}

    If the rule were invertible, we would not have this problem. Hence, the
    left rule is not invertible.
\end{proof}

\section{Weak normalization}

We consider an extension of the simply-typed $\lambda$ calculus with products,
and prove weak normalization for it.

First, we add the following reducibility candidate.
\begin{equation*}
    \mathcal{R}_{A \times B} = \{ M
        | M \;\text{halts}\;
        \;\text{and}\; \fst{M} \in \mathcal{R}_A
        \;\text{and}\; \snd{M} \in \mathcal{R}_B
    \}
\end{equation*}

Next, we show an additional case for the backwards closure lemma.

\begin{lemma}
    If $M \step M^\prime$ and $M^\prime \in \R_C$, then $M \in \R_C$.
\end{lemma}

\begin{proof} by induction on $C$

    \begin{description}
        \item[Case] $C = A \times B$

            \begin{center}
                \begin{tabular}{c r}
                    $M^\prime \in \R_{A \times B}$
                    &
                    by assumption \\
                    %
                    $\fst M^\prime \in \R_A$
                    &
                    by congruence \\
                    %
                    $\snd M^\prime \in \R_B$
                    &
                    by congruence \\
                    %
                    $\fst M \in \R_A$
                    &
                    by I.H. \\
                    %
                    $\snd M \in \R_B$
                    &
                    by I.H. \\
                    %
                    $M$ halts
                    &
                    by $\step^*$ definition \\
                    %
                    $M \in \R_{A \times B}$
                    &
                    by $\R_{A \times B}$ since $M$ halts
                \end{tabular}
            \end{center}
    \end{description}
\end{proof}

Finally, we show additional cases for the fundamental lemma.

\begin{lemma}
    If $\Gamma \proves M : A$ and $\sigma \in \R_\Gamma$,
    then $[\sigma]M \in \R_A$
\end{lemma}

\begin{proof} by induction on $\Gamma \proves M : A$

    \begin{description}
        \item[Case] $\fst{}$

            \begin{equation*}
                \mathcal{D} = \infer[\times E_l]{\Gamma \proves \fst M : A}{
                    \deduce[\mathcal{D}^\prime]{\Gamma \proves M : A \times B}{}
                }
            \end{equation*}

            \begin{center}
                \begin{tabular}{c r}
                    $\sigma \in \R_\Gamma$ &
                    by assumption \\
                    %
                    $[\sigma]M \in \R_{A \times B}$ &
                    by I.H. on $\mathcal{D}^\prime$ \\
                    %
                    $\fst{([\sigma]M)} \in \R_A$ &
                    by $\R_{A \times B}$ definition \\
                    %
                    $[\sigma]\fst M \in \R_A$ &
                    by subsitution definition
                \end{tabular}
            \end{center}

        \item[Case] $\snd{}$

            This proof is essentially the same as for $\fst{}$, so we will omit
            it.

        \item[Case] $\times I$

            \begin{equation*}
                \mathcal{D} =
                \infer[\times I]{\Gamma \proves (M, N) : A \times B}{
                    \deduce[\mathcal{D}_1]{\Gamma \proves M : A}{}
                    &
                    \deduce[\mathcal{D}_2]{\Gamma \proves N : B}{}
                }
            \end{equation*}

            \begin{center}
                \begin{tabular}{c r}
                    $\sigma \in \R_\Gamma$ &
                    by assumption \\
                    %
                    $[\sigma]M \in \R_A$ &
                    by I.H. on $\mathcal{D}_1$ \\
                    %
                    $[\sigma]N \in \R_B$ &
                    by I.H. on $\mathcal{D}_2$ \\
                    %
                    $[\sigma]M$, $[\sigma]N$ halt &
                    by $\R_A$, $\R_B$ definitions \\
                    %
                    $([\sigma]M, [\sigma]N)$ halts &
                    by $\step^*$ and ``halt'' definition \\
                    %
                    $\fst{([\sigma]M, [\sigma]N)} \in \R_A$ &
                    by backwards closure lemma \\
                    %
                    $\snd{([\sigma]M, [\sigma]N)} \in \R_B$ &
                    by backwards closure lemma \\
                    %
                    $([\sigma]M, [\sigma]N) \in \R_{A \times B}$ &
                    by $R_{A \times B}$ definition \\
                    %
                    $[\sigma](M, N) \in \R_{A \times B}$ &
                    by subtitution definition
                \end{tabular}
            \end{center}
    \end{description}
\end{proof}

\end{document}
