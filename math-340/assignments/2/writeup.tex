\documentclass[12pt,letterpaper]{article}

\author{Jacob Thomas Errington}
\date{15 February 2017}
\title{Assignment \#2\\Discrete structures 2 -- MATH 340}

\usepackage[margin=2.0cm]{geometry}
\usepackage{amsmath,amssymb,amsthm}
\usepackage{tikz}

\usetikzlibrary{graphs}

%%%%% Set up sections to be questions

\renewcommand{\thesection}{Question \arabic{section}}
\newcommand{\question}{\section}

%%%%% Set up math notations

\newtheorem{prop}{Proposition}
\DeclareMathOperator{\degOp}{deg}
\newcommand{\parens}[1]{\left(#1\right)}
\renewcommand{\deg}[1]{\degOp{\parens{#1}}}
\DeclareMathOperator{\chromOp}{\chi}
\newcommand{\chrom}[1]{\chromOp{\parens{#1}}}
\newcommand{\half}{\frac{1}{2}}

\begin{document}

\maketitle

\question{Euler's formula}

\begin{prop}
    Suppose $G$ is a planar graph such that every vertex in $G$ has degree at
    least five, and at least one vertex in $G$ has degree eight.
    Then, $G$ has at least 15 vertices.
\end{prop}

\begin{proof}
    Let $n = |V(G)|$ and $e = |E(G)|$.
    From Euler's formula we have the upper bound on the number of edges
    \begin{equation}
        \label{eq:euler-lower-bound}
        3n - 6 \geq e
    \end{equation}
    From the assumptions, we know that every vertex has degree at least five,
    so by the handshaking lemma
    \begin{equation}
        \label{eq:handshake-lower-bound}
        \sum_{v \in V(G)} \deg{v} = 2 e \geq 5 n
    \end{equation}
    However, we also know that at least one vertex has degree eight, so the
    lower bound in \eqref{eq:handshake-lower-bound} on the edges can be bumped
    up by three, giving
    \begin{equation}
        \label{eq:shake-better-lower-bound}
        2 e \geq 5 n + 3
    \end{equation}
    We combine the bounds in \eqref{eq:shake-better-lower-bound} and
    \eqref{eq:euler-lower-bound} by transitivity to obtain
    \begin{equation*}
        2(3n - 6) \geq 5n + 3
    \end{equation*}
    Some algebraic manipulation gives the desired bound $n \geq 15$.
\end{proof}

\begin{prop}
    Suppose $G$ is a triangulation of the plane. Then, the number of faces of
    $G$ is even.
\end{prop}

\begin{proof}
    Since every face in a triangulation is bounded by exactly three edges, we
    get the following relation between the number of faces and the number of
    edges.
    \begin{equation*}
        \label{eq:tri-face-edge-rel}
        2 |E| = 3 |F|
    \end{equation*}
    Using this relation, we can eliminate $E$ in Euler's formula to get.
    \begin{equation*}
        |V| - \frac{3}{2} |F| + |F| = 2
    \end{equation*}
    Algebraic manipulation gives $|F| = 2|V| - 4$, which is even for any size
    of $V$.
\end{proof}

\question{Coloring planar graphs}

\begin{prop}
    Suppose that a planar graph $G = (V, E)$ has no $K_3$ subgraph.
    Then, $\chrom{G} \leq 4$.
\end{prop}

\begin{proof}
    Not having a $K_3$ subgraph means that the graph contains no triangles, or
    cycles of length three. Hence, every face in $G$ is bounded by at least $4$
    edges.

    Suppose that every face in $G$ were bounded by \emph{exactly} four edges.
    Then $4|F| = 2|E|$.
    Substituting into Euler's formula gives $|V| - |E| + \half |E| = 2$, which
    can be rewritten as $2|V| - 4 = |E|$. However, our graph does not have each
    face bounded by \emph{exactly} four edges, but rather by \emph{at least}
    four edges, which gives us an upper bound on the number of edges.
    \begin{equation}
        \label{eq:edge-upper-bound}
        2|V| - 4 \geq |E|
    \end{equation}

    Combining this bound with the handshaking lemma gives
    \begin{equation}
        \label{eq:bounded-handshake}
        \sum_{v \in V} \deg{v} = 2 |E| \leq 4 |V| - 8 < 4 |V|
    \end{equation}

    This shows that the average degree is less strictly less than $4$, so there
    must be an edge with degree at most $3$.

    This shows that $G$ is $3$-degenerate.
    Thus by a theorem, $\chrom{G} \leq 4$.
\end{proof}

The following proposition is \emph{false.}

\begin{prop}
    If a planar graph has no $K_4$ subgraph, then $\chrom{G} \leq 3$.
\end{prop}

To see that the proposition is false, consider the graph in figure
\ref{fig:counterexample}. Since the bottom vertices are a triangle, they each
receive different colors. Then, the middle two vertices each belong to separate
triangles, and are assigned different colors. Finally, the top vertex must be
assigned a color different from the middle two, and the bottom center vertex.
This is the fourth color.

\tikzset{
    mynode/.style={
        inner sep=0,
        draw,
        circle,
        fill=black!15,
        minimum width=1em,
    },
}

\newcommand{\mknode}[1]{\node[mynode] (#1) {} ;}

\begin{figure}[ht]
    \begin{center}
        \begin{tikzpicture}
            \matrix[row sep=1em, column sep=1em]{
                          &           & \mknode A &           &           \\
                          & \mknode B &           & \mknode C &           \\
                \mknode D &           & \mknode E &           & \mknode F \\
            } ;

            \graph[use existing nodes]{
                A -- B -- D -- E -- F -- C -- A -- E;
                E -- B ;
                E -- C ;
                D --[bend right=30] F;
            } ;
        \end{tikzpicture}
    \end{center}
    \caption{
        A counterexample showing that a graph having no $K_4$ subgraph cannot
        in general be $3$-colored. The graph shown above has chromatic number
        $4$.
    }
    \label{fig:counterexample}
\end{figure}

\end{document}
