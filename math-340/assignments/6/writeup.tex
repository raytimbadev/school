\documentclass[11pt]{article}

\title{Assignment \#6 (Optional)\\Discrete structures 2 -- MATH 340}
\author{Jacob Thomas Errington}
\date{17 April 2017}

\usepackage[questions,geometry]{jakemath}

\begin{document}

\maketitle

\question{Fruit salad}

Let $s(n)$ be the number of ways to make a fruit salad consisting of
strawberries, apples, bananas, and pineapples, with $n$ pieces of
fruit, where
%
\begin{itemize}
    \item strawberries are used by the half-dozen;
    \item apples are an even number;
    \item there are at most five bananas; and
    \item there is at most one pineapple.
\end{itemize}

First, we define a generating function for each constraint.
%
\begin{description}
    \item[Strawberries.]
        Let $f_s(x)$ denote the number of counts for strawberries.
        %
        \begin{equation*}
            f_s(x) = \infsum_{k=0} x^{6k} = \frac{1}{1 - x^6}
        \end{equation*}

    \item[Apples.]
        Let $f_a(x)$ denote the number of counts for apples.
        %
        \begin{equation*}
            f_a(x) = \infsum_{k=0} x^{2k} = \frac{1}{1 - x^2}
        \end{equation*}

    \item[Bananas.]
        Let $f_b(x)$ denote the number of counts for bananas.
        %
        \begin{equation*}
            f_b(x)
            = 1 + x + x^2 + x^3 + x^4 + x^5
            = \frac{1-x^6}{1 - x}
        \end{equation*}

    \item[Pineapples.]
        Let $f_p(x)$ denote the number of counts for pineapples.
        %
        There can be only zero or one pineapple, so
        %
        \begin{equation*}
            f_p(x) = 1 + x = \frac{1 - x^2}{1 - x}
        \end{equation*}
\end{description}

Taking the product of these gives an ordinary generating function for ways to
select the counts of fruit, subject to each constraint, to produce the ordinary
generating function for the fruit salad.
%
\begin{equation*}
    S(x)
    =
    \frac{1}{\cancel{1 - x^6}}
    \cdot \frac{1}{\cancel{1 - x^2}}
    \cdot \frac{\cancel{1 - x^6}}{1 - x}
    \cdot \frac{\cancel{1 - x^2}}{1 - x}
    = \frac{1}{(1-x)^2}
\end{equation*}

Next, we want to find an explicit formula for the coefficients,
since $s(n) = [x^n]S(x)$. These can be found by differentiation.
%
Notice
%
\begin{equation*}
    \frac{1}{(1-x)^2}
    = \deriv{x} { \frac{1}{1-x} }
    = \deriv{x} { \infsum_{n=0} x^n }
    = \infsum_{n=1} n x^{n-1}
    = \infsum_{n=0} (n + 1) x^n
\end{equation*}
%
So there are $n + 1$ ways to make a fruit salad with $n$ pieces of fruit
subject to the above constraints, i.e. $s(n) = n + 1$.

\question{Round table}

Let $r(n)$ be the number of different ways to seat $n$ people around a round
table. The idea is that because the table is round, two arrangements of people
are considered equivalent if one arrangement is obtained as a rotation of the
other, e.g. $(12345) \equiv (23451)$ are the same seating arrangement, for
$n=5$. The goal is to count the number of such equivalence classes for each
$n$.
%
Note that each equivalence class has exactly $n$ members, and there are
$n\fact$ permutations of $[n]$. Dividing by $n$ gives the number of equivalence
classes, so there are $(n-1)\fact$ seating arrangements at a round table.
%
Hence, the exponential generating function has the series representation
%
\begin{equation*}
    R(x)
    = \infsum_{n=0} (n-1)\fact \frac{x^n}{n\fact}
    = \infsum_{n=0} \frac{x^n}{n}
\end{equation*}

Differentiating, we find
%
\begin{equation*}
    R^\prime(x)
    = \infsum_{n=1} \frac{\cancel{n} x^{n-1}}{\cancel{n}}
    = \infsum_{n=0} x^n
    = \frac{1}{1-x}
\end{equation*}

Integrating, we find
%
\begin{equation*}
    R(x) = \int R^\prime(x) \intd x
    = \int \frac{1}{1-x} \intd x
    = - \ln (1-x)
\end{equation*}
%
for a closed form representation of the exponential generating function.

\question{Sum of cubes}

Using generating functions, we can find a closed form equation to evaluate
%
\begin{equation*}
    s(n) = \sum_{k=0}^n (k-1)k(k+1)
\end{equation*}

We begin with $\frac{1}{1-x} = \sum x^n$ and differentiate three times.
%
\begin{align*}
    \deriv{x} \deriv{x} \deriv{x} \frac{1}{1-x}
    &= \deriv{x} \deriv{x} \deriv{x} \infsum_{n=0} x^n \\
    %
    \frac{1}{(1-x)^4} &= \infsum_{n=0} (n+1)(n+2)(n+3) x^n \\
    \frac{x^2}{(1-x)^4} &= \infsum_{n=0} (n+1)(n+2)(n+3) x^{n+2}
\end{align*}
%
Notice that
%
\begin{equation*}
    [x^n] \frac{x^2}{(1-x)^4} = (n-1)n(n+1)
\end{equation*}

Using convolution with $\frac{1}{1-x} = \sum x^n$ we obtain the partial sums.
%
\begin{equation*}
    [x^n] \frac{1}{1-x} \cdot \frac{x^2}{(1-x)^4}
    = [x^n] \frac{x^2}{(1-x)^5}
    = \sum_{k=0}^n (k-1)k(k+1)
\end{equation*}

We perform a partial fraction decomposition on $\frac{x^2}{(1-x)^5}$.
%
\begin{equation*}
    \frac{x^2}{(1-x)^5}
    = \cancel{\frac{0}{1-x}}
    + \cancel{\frac{0}{(1-x)^2}}
    + \frac{1}{(1-x)^3}
    + \frac{-2}{(1-x)^4}
    + \frac{1}{(1-x)^5}
\end{equation*}
%
Then, we can compute the corresponding series by differentiating.
%
\begin{align*}
    \frac{1}{(1-x)^3}
    &= \infsum_{n=0} \frac{(n+1)(n+2)}{2} x^n \\
    %
    \frac{-2}{(1-x)^4}
    &= \infsum_{n=0}
        (-\cancel{2}) \frac{(n+1)(n+2)(n+3)}{\cancel{2} \cdot 3}
        x^n \\
    %
    \frac{1}{(1-x)^5}
    &= \infsum_{n=0} \frac{(n+1)(n+2)(n+3)(n+4)}{2 \cdot 3 \cdot 4} x^n
\end{align*}

Summing these series, we obtain
%
\begin{align*}
    \frac{x^2}{(1-x)^5}
    &= \infsum_{n=0} \frac{(n+1)(n+2)}{2} x^n \\
    &+ \infsum_{n=0} \frac{-(n+1)(n+2)(n+3)}{3} x^n \\
    &+ \infsum_{n=0} \frac{(n+1)(n+2)(n+3)(n+4)}{24} x^n \\
    %
    &= \infsum_{n=0} \frac{1}{24} (n-1)n(n+1)(n+2) x^n
\end{align*}
%
which gives the closed form
%
\begin{equation*}
    s(n) = \frac{1}{24} (n-1)n(n+1)(n+2)
\end{equation*}

\end{document}
