\documentclass[11pt]{article}

\author{Jacob Thomas Errington}
\title{Assignment \#4\\Discrete structures 2-- MATH 340}
\date{22 March 2017}

\usepackage[geometry,questions]{jakemath}

\begin{document}

\maketitle

\question{Birthday paradox}

We are interested in the event ``there exists a pair of people with the same
birthday''.
To make things simpler, we look at the complement ``no two people have the same
birthday''.
Then, we cook up the following process, by numbering the $n$ people
$\range{n}$.
%
\begin{enumerate}
    \item
        Look at person one. Any birthday will do since nobody else has been
        chosen yet, so the probability of not having a common birthday with a
        previously chosen person is $\frac{d}{d}$ where $d$ is the number of
        days in the year.
        %
    \item
        Look at person two. The probability that they do not share a birthday
        with person one is $\frac{d-1}{d}$.
        %
    \item Look at person $k$. The probability that they do not share a birthday
        with any of the $k-1$ previous people is $\frac{d-(k-1)}{d}$.
\end{enumerate}
%
The probabiity that all these events occur at once is given by the product, so
we have that the probability that no two people out of $n$ share a birthday
among $d$ days is
%
\begin{equation*}
    p
    = \overbrace{
        \frac{d}{d} \times
        \frac{d-1}{d} \times
        \cdots \times
        \frac{d-n+1}{d}
    }^{\text{$n$ factors in total}}
    = \parens{1 - \frac{1}{d}}
        \times \cdots
        \times \parens{1 - \frac{1}{n-1}}
\end{equation*}
%
We take a logarithm on both sides to turn the product into a sum.
%
\begin{equation*}
    \ln p
    = \sum_{k=1}^{n-1} {
        \ln{\parens{1 - \frac{k}{d}}}
    }
    = \cdots
\end{equation*}
%
and use the fact that $\ln{1 + x} \approx x$ for small values of $x$ to
eliminate the logarithms on the right-hand side
%
\begin{equation*}
    \cdots
    = \sum_{k=1}^{n-1} {
        \frac{-k}{d}
    }
    = \frac{-1}{d} \sum_{k=1}^{n-1} k
    = \frac{-n(n-1)}{2d}
\end{equation*}
%
This gives a quadratic equation for $n$,
%
\begin{equation*}
    0 = \frac{1}{2d} n^2 + \frac{1}{2d} n - \ln p
\end{equation*}
%
Solving for $n$, we get
%
\begin{equation*}
    n
    = \frac{
        \frac{1}{2d}
        \pm
        \sqrt{
            \frac{1}{4d^2} - 4 \cdot \frac{1}{2d} \cdot \ln p
        }
    }{
        2 \cdot \frac{1}{2d}
    }
\end{equation*}
%
We are given that $d = 365$ and $1 - p = \frac{999}{1000}$, so we substitute to
find that
%
\begin{equation*}
    n = 71.54
\end{equation*}
%
Computing the probability backwards from $n = 71$, we find that the probability
that a collision occurs is indeed above $\frac{999}{1000}$. However, we also
tried $70$ and $69$, and found that $70$ is the least $n$ that gives a
sufficient probability.

\end{document}
