\documentclass[11pt]{article}

\author{Jacob Thomas Errington}
\title{Assignment \#4\\Discrete structures 2-- MATH 340}
\date{22 March 2017}

\usepackage[geometry,questions]{jakemath}
\usepackage{cancel}

\begin{document}

\maketitle

\question{Birthday paradox}

We are interested in the event ``there exists a pair of people with the same
birthday''.
To make things simpler, we look at the complement ``no two people have the same
birthday''.
Then, we cook up the following process, by numbering the $n$ people
$\range{n}$.
%
\begin{enumerate}
    \item
        Look at person one. Any birthday will do since nobody else has been
        chosen yet, so the probability of not having a common birthday with a
        previously chosen person is $\frac{d}{d}$ where $d$ is the number of
        days in the year.
        %
    \item
        Look at person two. The probability that they do not share a birthday
        with person one is $\frac{d-1}{d}$.
        %
    \item Look at person $k$. The probability that they do not share a birthday
        with any of the $k-1$ previous people is $\frac{d-(k-1)}{d}$.
\end{enumerate}
%
The probabiity that all these events occur at once is given by the product, so
we have that the probability that no two people out of $n$ share a birthday
among $d$ days is
%
\begin{equation*}
    p
    = \overbrace{
        \frac{d}{d} \times
        \frac{d-1}{d} \times
        \cdots \times
        \frac{d-n+1}{d}
    }^{\text{$n$ factors in total}}
    = \parens{1 - \frac{1}{d}}
        \times \cdots
        \times \parens{1 - \frac{1}{n-1}}
\end{equation*}
%
We take a logarithm on both sides to turn the product into a sum.
%
\begin{equation*}
    \ln p
    = \sum_{k=1}^{n-1} {
        \ln{\parens{1 - \frac{k}{d}}}
    }
    = \cdots
\end{equation*}
%
and use the fact that $\ln{1 + x} \approx x$ for small values of $x$ to
eliminate the logarithms on the right-hand side
%
\begin{equation*}
    \cdots
    = \sum_{k=1}^{n-1} {
        \frac{-k}{d}
    }
    = \frac{-1}{d} \sum_{k=1}^{n-1} k
    = \frac{-n(n-1)}{2d}
\end{equation*}
%
This gives a quadratic equation for $n$,
%
\begin{equation*}
    0 = \frac{1}{2d} n^2 + \frac{1}{2d} n - \ln p
\end{equation*}
%
Solving for $n$, we get
%
\begin{equation*}
    n
    = \frac{
        \frac{1}{2d}
        \pm
        \sqrt{
            \frac{1}{4d^2} - 4 \cdot \frac{1}{2d} \cdot \ln p
        }
    }{
        2 \cdot \frac{1}{2d}
    }
\end{equation*}
%
We are given that $d = 365$ and $1 - p = \frac{999}{1000}$, so we substitute to
find that
%
\begin{equation*}
    n = 71.54
\end{equation*}
%
Computing the probability backwards from $n = 71$, we find that the probability
that a collision occurs is indeed above $\frac{999}{1000}$. However, we also
tried $70$ and $69$, and found that $70$ is the least $n$ that gives a
sufficient probability.


\question{Balls and bins 1}

\begin{prop}
    Suppose $n^{\frac{3}{2}}$ balls are dropped into $n$ bins uniformly at
    random.
    Let $M^*$ be the random variable representing the maximum number of balls
    in a bin.
    An upper bound on the expectation of this variabe is
    \begin{equation*}
        \E{M^*} = 1 + 3 \sqrt{\frac{\ln n}{\sqrt n}} + n
    \end{equation*}
\end{prop}

\begin{proof}
    Let $X_i$ be a random variable counting the number of balls in bin $i$.
    Note that each $X_i$ is binomially distributed:
    a trial is the action of dropping a ball, and its ``success'' probability
    is $p = \frac{1}{n}$.
    %
    Then the expected number of balls in any bin is
    \begin{equation*}
        \E{X_i} = n^{\frac{3}{2}} \frac{1}{n} = \sqrt{n}
    \end{equation*}
    %
    We will apply the nice form of the Chernoff bound, with
    $\delta = 3 \sqrt{\frac{\ln n}{\sqrt n}}$.
    Let $r^* = 1 + \delta$.
    %
    \begin{equation*}
        \P{X_i \geq r^* \sqrt{n}}
        \leq e^{
            \frac{-1}{3} \cancel{\sqrt{n}} 3^2 \frac{\ln n}{\cancel{\sqrt n}}
        }
        = \frac{1}{n^3}
    \end{equation*}
    %
    Then to compute a bound on the probability that the maximum $M^*$ is not
    less than $r^*$ via the union bound.
    %
    \begin{equation*}
        \P{M^* \geq r^*}
        = \P{\Union_i X_i \geq r^*}
        \leq \sum_{i=1}^n \cancelto{\frac{1}{n^3}}{\P{X_i \geq r^*}}
        = n \frac{1}{n^3} = \frac{1}{n^2}
    \end{equation*}
    %
    Then we can use this information to compute a bound on the expectation of
    $M^*$.
    %
    \begin{align*}
        \E{M^*}
        &= \sum_{k=0}^n {k \P{M^* = k}} \\
        &= \sum_{k=0}^{r^*} {
            \cancelto{\leq r^*}{k} \P{M^* = k}
        }
        + \sum_{k=r^*}^{n} {
            \cancelto{\leq n}{k} \P{M^* = k}
        } \\
        &\leq
        r^* \cancelto{\leq 1}{
            \sum_{k=0}^{r^*} \P{M^* = k}
        }
        + n \sum_{k=r^*}^n { \P{M^* = k } } \\
        &\leq r^* + n \frac{1}{n^2} \\
        &= 1 + 3 \sqrt{\frac{\ln n}{\sqrt n}} + n
    \end{align*}
    %
    which is better than the trivial upper bound, for large values of $n$
    (greater than $5$ or $6$).
\end{proof}

\end{document}
