\documentclass[11pt]{article}

\usepackage[geometry,questions]{jakemath}

\author{Jacob Thomas Errington}
\title{Assignment \#5\\Discrete structures 2 -- MATH 340}
\date{7 April 2017}

\begin{document}

\maketitle

\question{Combinatorial identities}

\begin{prop}
    We have
    %
    \begin{equation}
        \label{eq:identity}
        %
        \binom{n+1}{m+1} = \sum_{k=m}^n \binom{k}{m}
    \end{equation}
\end{prop}

\begin{proof}{(Algebraic.)}
    
\end{proof}

\begin{proof}{(Combinatorial.)}
    We know that the number of solutions to $\sum_{i=1}^m x_i = n$ is
    $\binom{n+m-1}{m-1}$ and intuitively gives the number of ways we may
    distribute $n$ items among $k$ containers.

    Another way to look at this is to imagine an algorithm that counts these
    different distributions of items among the containers.

    A naive such procedure could operate recursively, by knowing two pieces of
    information at each stage: the number of containers in consideration and
    the number of items to distribute among them.
    We reach a base case when there is a single container into which any number
    of items must be distributed, as there is a unique way in which to perform
    this distribution.
    The algorithm would first assign $0$ items to the first container,
    and then recursively determine how to distribute all $n$ remaining items
    among the remaining $m-1$ containers;
    then, the algorithm could assign $1$ item to the first container, and
    inductively compute how to distribute the $n-1$ items among the remaining
    $m-1$ containers; and so on.
    Once these counts have all been determined, the procedure adds all the
    counts and produces the final sum.

    This procedure establishes a bijective proof that
    %
    \begin{equation*}
        \binom{n + m - 1}{m - 1}
        = \sum_{i=0}^n \binom{(n - i) + (k - 1) - 1}{(k - 1) - 1}
        = \sum_{i=0}^n \binom{n + k - 2 - i}{k - 2}
    \end{equation*}

    Then, perform the substitution $m = k - 2$ and $n^\prime = n + m$ to
    obtain
    %
    \begin{equation*}
        \binom{n^\prime + 1}{m + 1}
        =
        \sum_{i=0}^n \binom{n^\prime - i}{m}
        = 
        \sum_{i=m}^{n^\prime} \binom{i}{m}
    \end{equation*}
    %
    as required.
\end{proof}

\end{document}
