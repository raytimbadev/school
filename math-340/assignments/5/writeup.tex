\documentclass[11pt]{article}

\usepackage[geometry,questions]{jakemath}

\author{Jacob Thomas Errington}
\title{Assignment \#5\\Discrete structures 2 -- MATH 340}
\date{7 April 2017}

\begin{document}

\maketitle

\question{Combinatorial identities}

\begin{prop}
    We have
    %
    \begin{equation}
        \label{eq:identity}
        %
        \binom{n+1}{m+1} = \sum_{k=m}^n \binom{k}{m}
    \end{equation}
\end{prop}

\begin{proof}{(Algebraic.)}
    We proceed by induction on $n$.

    In the base case, we have $n = m$, so the sum degenerates into a single
    term in which the top and bottom of the binomial coefficient are the same.
    Similarly the left-hand side has equal top and bottom. Hence both evaluate
    to one.

    In the step case, suppose that
    %
    \begin{equation*}
        \binom{n+1}{m+1} = \sum_{k=m}^n \binom{k}{m}
    \end{equation*}
    %
    We want to show that
    %
    \begin{equation*}
        \binom{(n+1) + 1}{m+1} = \sum_{k=m}^{n+1} \binom{k}{m}
    \end{equation*}

    Starting from the right-hand side, we split the sum, use the induction
    hypothesis, and apply Pascal's identity.
    %
    \begin{equation*}
        \sum_{k=m}^{n+1} \binom{k}{m}
        =
        \binom{n+1}{m} + \sum_{k=m}^n \binom{k}{m}
        =
        \binom{n+1}{m} + \binom{n+1}{m+1}
        =
        \binom{(n+1)+1}{m+1}
    \end{equation*}
    %
    as required.
\end{proof}

\begin{proof}{(Combinatorial.)}
    We know that the number of solutions to $\sum_{i=1}^k x_i = n$ is
    $\binom{n+k-1}{k-1}$ and intuitively gives the number of ways we may
    distribute $n$ items among $k$ containers.

    Another way to look at this is to imagine an algorithm that counts these
    different distributions of items among the containers.

    A naive such procedure could operate recursively, by knowing two pieces of
    information at each stage: the number of containers in consideration and
    the number of items to distribute among them.
    We reach a base case when there is a single container into which any number
    of items must be distributed, as there is a unique way in which to perform
    this distribution.
    The algorithm would first assign $0$ items to the first container,
    and then recursively determine how to distribute all $n$ remaining items
    among the remaining $k-1$ containers;
    then, the algorithm could assign $1$ item to the first container, and
    inductively compute how to distribute the $n-1$ items among the remaining
    $k-1$ containers; and so on.
    Once these counts have all been determined, the procedure adds all the
    counts and produces the final sum.

    This procedure establishes a bijective proof that
    %
    \begin{equation*}
        \binom{n + k - 1}{k - 1}
        = \sum_{i=0}^n \binom{(n - i) + (k - 1) - 1}{(k - 1) - 1}
        = \sum_{i=0}^n \binom{n + k - 2 - i}{k - 2}
    \end{equation*}

    Then, perform the substitution $m = k - 2$ and $n^\prime = n + m$ to
    obtain
    %
    \begin{equation*}
        \binom{n^\prime + 1}{m + 1}
        =
        \sum_{i=0}^n \binom{n^\prime - i}{m}
        =
        \sum_{i=m}^{n^\prime} \binom{i}{m}
    \end{equation*}
    %
    as required.
\end{proof}

\question{Labelled trees}

\begin{prop}
    Let $f : \nset \to \nset$ and $T_f$ be a labelled tree on $n$ vertices
    constructed by the procedure seen in class.
    Suppose $T_f$ contains a vertex of degree at least $k$.
    Then $f$ takes on at most $n - k + 2$ different values.
\end{prop}

\begin{proof}
    First remark that if $k < 3$, the upper bound is simply $n$ which is a
    trivial bound. We assume that $k \geq 3$ for the remainder of the proof.

    The size of the image of such a function can be found by looking at the
    number of ``duplicates'' in the graph. We say that a point $y \in \nset$
    has $k$ duplicates under $f$ if $\abs{f\preimage{y}} = k + 1$.
    The number of duplicates in $f$ is the sum of the number of duplicates
    across all points in the image of $f$.

    Now consider $T_f$.
    The tree consists of two types of vertices.
    We have \emph{backbone} vertices, corresponding to those vertices belonging
    to a cycle in the graph of $f$; and \emph{appendage} vertices,
    corresponding to those vertices not belonging to cycles.

    The backbone of the tree corresponds to the ``bijective part'' of $f$.
    Formally, if we restrict $f$ to its backbone, then we have a bijection.

    Backbone vertices are tricky because the ones on the ``ends'' of the tree
    have only one edge due to the cycle, whereas the ``inner'' backbone
    vertices have two edges due to edges from cycles in the graph of $f$.

    Suppose an end backbone vertex labelled $y$ has degree at least $k \geq 3$.
    Then, there are at least $k - 1 \geq 2$ appendage vertices adjacent to it.
    By the construction, appendage vertices come directly from the graph of
    $f$.
    This means that there are at least $k$ values in the domain of $f$
    --
    at least $k-1$ due to the appendage vertices plus exactly one because the
    backbone edge is in a cycle
    --
    that are mapped to $y$ under $f$.
    Consequently, these $k-1$ appendage vertices are not going elsewhere, so
    there are $k-1$ elements in the codomain of $f$ that are unmapped.
    Hence, the size of the image of $f$ is at most $n - k + 1$.

    The same argument applies for inner backbone vertices, but we have to
    subtract $2$ instead of one, since two edges come from the cycle.

    Finally, appendage vertices have exactly one edge that corresponds with an
    outgoing edge in the graph of $f$. Hence they are treated the same was as
    end backbone vertices, and result in a bound of $n - k + 1$.

    The most restrictive of these bounds is the one resulting from inner
    backbone edges, and this bound is precisely the one we are seeking.
\end{proof}

\question{Catalan numbers 1}

\begin{prop}
    The number of orderings of $N = \range{2n}$ such that
    %
    \begin{enumerate}
        \item
            the odd numbers appear in order,
        \item
            the even numbers ppear in order, and
        \item
            $2k-1$ precedes $2k$ for every $k \in \range{n}$
    \end{enumerate}
    %
    is counted by Catalan numbers.
\end{prop}

\begin{proof}{(By bijection.)}
    We construct a bijection between valid orderings and sequences of $+$ and
    $-$ such that each partial sum is nonnegative.

    Take an arbitrary valid ordering. Rewrite each odd number to $+$ and each
    even number of $-$ to form a sequence.
    Now suppose for a contradiction that $i$ is the least position in the
    sequence that gives a negative partial sum.
    By minimality, the $i$\th element is a $-$, and we have $i = 2k + 1$, such
    that $k$ elements in the sequence are $+$ and $k + 1$ elements are $-$.
    Hence, the $i$\th element in the ordering has value $2(k + 1)$.
    Now we look at $k$ more closely.
    \begin{description}
        \item[Case] $k = 0$.
            Then, we have the value $2$ appearing before $1$ in the original
            ordering, which contradicts property 3.
        \item[Case] $k > 0$.
            Then, the greatest odd number in the ordering before position $i$
            is $2k - 1$. But then we have $2k + 2$ preceding $2k + 1$ in the
            ordering, which contradicts property 3.
    \end{description}

    Hence, this transformation is correct, and the sequence produced by it does
    satisfy the property that all its partial sums are nonnegative.

    It is easy to see that this transformation is bijective, because it is
    invertible: given a sequence of $+$ and $-$, rewrite the $+$s into ordered
    odd numbers and the $-$s into ordered even numbers. By the property that
    the partial sums of the sequence are positive, we will always have more odd
    numbers before even numbers, so property three is guaranteed.
    (A similar simple argument by contradiction as above can be used as well.)
\end{proof}

\question{Catalan numbers 2}

Consider the sequence $+++--+--++--$.

\tikzset{
    dyckUp/.style={
        color=red!50,
    },
    dyckDown/.style={
        color=blue!50,
    },
}

The corresponding Dyck path is given in the following diagram.

\begin{center}
    \begin{tikzpicture}
        \draw[step=1,color=black!50] (0, 0) grid (12, 3);
        \coordinate (prev) at (0,0) ;
        \foreach \dir in {1,1,1,0,0,1,0,0,1,1,0,0} {
            \ifnum\dir=1
            \coordinate (dep) at (1, 1);
            \draw[dyckUp] (prev) -- ++(dep) coordinate (prev) ;
            \else
            \coordinate (dep) at (1, -1);
            \draw[dyckDown] (prev) -- ++(dep) coordinate (prev) ;
            \fi
        }
    \end{tikzpicture}
\end{center}

The corresponding rooted plane tree on 7 vertices is given in the following
diagram.

\begin{center}
    \begin{tikzpicture}
        \Tree
            [.$\cdot$
                [.$\cdot$
                    [.$\cdot$ $\cdot$ ]
                    $\cdot$
                ]
                [.$\cdot$ $\cdot$ ]
            ]
    \end{tikzpicture}
\end{center}

To find the corresponding decomposition of an 8-gon into triangles requires
more work. First, we find the corresponding planted trivalent tree on $14$
vertices.
Then, we draw the triangulation of the 8-gon. The following diagram illustrates
both constructions.

\begin{tikzpicture}[
        shorten >=0,
        x=1.5in,
        y=1.5in,
    ]
    \foreach \n in {0,1,2,3,4,5,6,7} {
        \draw ({22.5 + 45 * \n}:1) coordinate (\n)
            -- ({22.5 + 45 * (\n + 1)}:1) ;

        \ifnum\n=0{
            \node (outerpt\n) at ({45 * \n}:1.1) {root} ;
        }
        \else{
            \node (outerpt\n) at ({45 * \n}:1.1) {x} ;
        }
        \fi

    }

    \tikzset{
        triangle/.style={
            draw=blue!50,
        },
    }

    \path[triangle]
        (7) -- (4)
        (4) -- (0)
        (0) -- (2)
        (2) -- (4)
        (6) -- (4)
        ;

    \centroid{A}{(7),(0),(4)}
    \node[draw=black,circle] (p2) at (A) {} ;

    \centroid{B}{(7),(6),(4)}
    \node[draw=black,circle] (p6) at (B) {} ;

    \centroid{C}{(4),(5),(6)}
    \node[draw=black,circle] (p7) at (C) {} ;

    \centroid{D}{(0),(2),(4)}
    \node[draw=black,circle] (p3) at (D) {} ;

    \centroid{E}{(2),(3),(4)}
    \node[draw=black,circle] (p5) at (E) {} ;

    \centroid{F}{(0),(1),(2)}
    \node[draw=black,circle] (p4) at (F) {} ;

    \graph[use existing nodes]{
        outerpt0 --
        p2 -- {
            p6 -- {
                outerpt7,
                p7 -- {
                    outerpt6,
                    outerpt5,
                },
            },
            p3 -- {
                p4 -- {
                    outerpt1,
                    outerpt2,
                },
                p5 -- {
                    outerpt3,
                    outerpt4,
                },
            },
        } ;
    };
\end{tikzpicture}

\question{Generating functions}

\begin{enumerate}
    \item
        Consider the following recurrence.
        %
        \begin{equation*}
            f(n) = \begin{cases}
                3   &\quad\text{for } n = 0 \\
                10  &\quad\text{for } n = 1 \\
                6f(n-1) - 8f(n-2)
                    &\quad\text{for } n \geq 2
            \end{cases}
        \end{equation*}

        We define the formal power series
        %
        \begin{equation*}
            F(x) = \sum_{n=0}^\infty f(n)x^n
        \end{equation*}
        %
        and proceed by the method of generating functions.
        %
        \begin{align*}
            \frac{F(x) - 3}{x} &= \sum_{n=0}^\infty f(n+1) x^n \\
            6 \frac{F(x) - 3}{x} - 8 F(x)
                &= \sum_{n=0}^\infty (6 f(n+1) - 8 f(n)) x^n \\
                &= \sum_{n=0}^\infty f(n+2) x^n \\
                &= \frac{F(x) - f(0) - f(1) x}{x^2}
        \end{align*}

        Solving for $F(x)$ we find
        %
        \begin{equation*}
            F(x) = \frac{3 - 8x}{(2x-1)(4x-1)} = \cdots
        \end{equation*}
        %
        which we rewrite by performing a partial fraction decomposition
        %
        \begin{equation*}
            \cdots
            =
            \frac{-1}{2x-1} + \frac{-2}{4x - 1}
            =
            \frac{1}{1-2x} + \frac{2}{1-4x}
            \cdots
        \end{equation*}
        %
        and finally rewrite as series
        %
        \begin{equation*}
            \cdots
            =
            \sum_{n=0}^\infty{
                (2^n + 2 \cdot 4^n)x^n
            }
        \end{equation*}

        Hence, the closed form of the recurrence is
        %
        \begin{equation*}
            f(n) = 2^n + 2 \cdot 4^n
        \end{equation*}

    \item
        Consider the recurrence
        %
        \begin{equation*}
            f(n) = \begin{cases}
                0   &\quad\text{for } n = 0 \\
                2   &\quad\text{for } n = 1 \\
                4 f(n-1) - 4 f(n-2)
                    &\quad\text{for } n \geq 2
            \end{cases}
        \end{equation*}

        We define a formal power series $F$ as before, and apply the method of
        generating functions to find that
        %
        \begin{equation*}
            F(x) = \frac{2x}{(2x - 1)^2} = \cdots
        \end{equation*}
        %
        which we decompose by partial fractions as before to obtain
        %
        \begin{equation*}
            \cdots = \frac{1}{2x - 1} + \frac{1}{(2x-1)^2}
        \end{equation*}

        The first of these terms is easy to write as a series, but the latter
        requires additional work. Observe the following derivation.
        %
        \begin{align*}
            \frac{1}{1-2x} &= \sum_{n=0}^\infty 2^n x^n \\
            \deriv{x} \frac{1}{1-2x}
                = \frac{\cancel{-}1}{(1-2x)^2} \cdot (\cancel{-} 2)
                &= \sum_{n=1}^\infty 2^n n x^{n-1} \\
            \frac{1}{(1-2x)^2} &=
                \sum_{n=1}^\infty 2^{n-1} n x^{n-1}
                =
                \sum_{n=0}^\infty 2^n (n+1) x^n
        \end{align*}

        Finally, we can write the series we need.
        %
        \begin{equation*}
            F(x)
            = \sum_{n=0}^\infty f(n) x^n
            = \sum_{n=0}^\infty (2^n (n+1) - 2^n) x^n
        \end{equation*}

        So the closed form of the recurrence is
        %
        \begin{equation*}
            f(n) = 2^n (n+1) - 2^n
        \end{equation*}
\end{enumerate}

\end{document}
