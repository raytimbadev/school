\documentclass[letterpaper,11pt]{article}

\author{Jacob Thomas Errington}
\title{Assignment \#1\\Discrete structures 2 -- MATH 340}
\date{27 January 2017}

\usepackage{amsthm,amsmath,amssymb}
\usepackage[margin=2.0cm]{geometry}
\usepackage{tikz}

\usetikzlibrary{graphs}

\newcommand{\questionname}{\textit}

\newcommand{\match}[2]{B_#1 &\mapsto G_#2}
\newcommand{\preference}[4]{P_#1 &: P_#2 > P_#3 > P_#4}

\tikzset{
    roommates/.style={row sep=2em, column sep=2em}
}

\begin{document}

\maketitle

\begin{enumerate}
    \item
        \questionname{Stable matching algorithm.}

        Applying the Boy Proposal Algorithm with the given preference lists
        gives the following matching.

        \begin{align*}
            \match{1}{4} \\
            \match{2}{1} \\
            \match{3}{2} \\
            \match{4}{5} \\
            \match{5}{3}
        \end{align*}

        Other stable matchings do exist. For instance, applying the Girl
        Proposal Algorithm gives the following matching that is different.

        \begin{align*}
            \match{1}{4} \\
            \match{2}{3} \\
            \match{3}{2} \\
            \match{4}{5} \\
            \match{5}{1}
        \end{align*}

    \item
        \questionname{Stable roommates.}

        Consider the following preference list.
        \begin{align*}
            \preference{1}{3}{2}{4} \\
            \preference{2}{4}{3}{1} \\
            \preference{3}{2}{4}{1} \\
            \preference{4}{3}{2}{1}
        \end{align*}

        There does not exist a stable matching for this preference list in a
        roommate-pairing situation. To show this, we will see that every
        matching on this graph is not stable. The number of matchings is only
        three, so this is straightforward. Figure \ref{fig:unstable} shows the
        three matchings and describes their instability.

        \newcommand{\matrixcontent}{
            \node (1) [] {$P_1$} ; \& \node (2) [] {$P_2$} ; \\
            \node (3) [] {$P_3$} ; \& \node (4) [] {$P_4$} ; \\
        }

        \begin{figure}[ht]
            \centering
            \begin{tabular}{c c}
                \begin{tikzpicture}[ampersand replacement=\&]
                    \matrix[roommates] {
                        \matrixcontent
                    } ;
                    \graph {
                        (1) -- (2) ;
                        (3) -- (4) ;
                    } ;
                \end{tikzpicture}
                &
                \begin{tikzpicture}[ampersand replacement=\&]
                    \matrix[roommates] {
                        \matrixcontent
                    } ;
                    \graph {
                        (1) -- (4) ;
                        (3) -- (2) ;
                    } ;
                \end{tikzpicture}
                \\
                \begin{tikzpicture}[ampersand replacement=\&]
                    \matrix[roommates] {
                        \matrixcontent
                    } ;
                    \graph {
                        (1) -- (3) ;
                        (2) -- (4) ;
                    } ;
                \end{tikzpicture}
                &
            \end{tabular}
            \caption{
                The three matchings that are possible on the given set of nodes.
                Conceptually, what happens in this preference setup is that
                ``nobody like $P_1$'', so whoever is paired with $P_1$ will
                happily trade them away. In the top-right case, $P_2$ prefers
                $P_3$ to $P_1$ and $P_3$ prefers $P_2$ to $P_4$, so they swap,
                giving the matching in the top-left corner. In this case, $P_2$
                prefers $P_4$ to $P_3$ and $P_4$ prefers $P_2$ to $P_1$, so
                they swap, giving the bottom matching. In this case, $P_4$
                prefers $P_3$ to $P_2$ and $P_3$ prefers $P_4$ to $P_1$, so
                they swap, returning us to the matching in the top-left corner.
            }
            \label{fig:unstable}
        \end{figure}

\end{enumerate}

\end{document}
