\documentclass[letterpaper,11pt]{article}

\author{Jacob Thomas Errington}
\title{Assignment \#1\\Discrete structures 2 -- MATH 340}
\date{27 January 2017}

\usepackage{amsthm,amsmath,amssymb}
\usepackage[margin=2.0cm]{geometry}
\usepackage{tikz}

\newtheorem{proposition}{Proposition}

\usetikzlibrary{graphs}

\newcommand{\questionname}{\textit}

\newcommand{\match}[2]{B_#1 &\mapsto G_#2}
\newcommand{\preference}[4]{P_#1 &: P_#2 > P_#3 > P_#4}

\newcommand{\nest}[2][c]{
    \begin{tabular}[#1]{@{}c@{}}
        #2
    \end{tabular}
}

\newcommand{\fin}{\mathbf}

\tikzset{
    roommates/.style={row sep=2em, column sep=2em},
    market/.style={row sep=0.5em, column sep=2em}
}

\begin{document}

\maketitle

\begin{enumerate}
    \item
        \questionname{Stable matching algorithm.}

        Applying the Boy Proposal Algorithm with the given preference lists
        gives the following matching.

        \begin{align*}
            \match{1}{4} \\
            \match{2}{1} \\
            \match{3}{2} \\
            \match{4}{5} \\
            \match{5}{3}
        \end{align*}

        Other stable matchings do exist. For instance, applying the Girl
        Proposal Algorithm gives the following matching that is different.

        \begin{align*}
            \match{1}{4} \\
            \match{2}{3} \\
            \match{3}{2} \\
            \match{4}{5} \\
            \match{5}{1}
        \end{align*}

    \item
        \questionname{Stable roommates.}

        Consider the following preference list.
        \begin{align*}
            \preference{1}{3}{2}{4} \\
            \preference{2}{4}{3}{1} \\
            \preference{3}{2}{4}{1} \\
            \preference{4}{3}{2}{1}
        \end{align*}

        There does not exist a stable matching for this preference list in a
        roommate-pairing situation. To show this, we will see that every
        matching on this graph is not stable. The number of matchings is only
        three, so this is straightforward. Figure \ref{fig:unstable} shows the
        three matchings and describes their instability.

        \newcommand{\matrixcontent}{
            \node (1) [] {$P_1$} ; \& \node (2) [] {$P_2$} ; \\
            \node (3) [] {$P_3$} ; \& \node (4) [] {$P_4$} ; \\
        }

        \begin{figure}[ht]
            \centering
            \begin{tabular}{c c c}
                \begin{tikzpicture}[ampersand replacement=\&]
                    \matrix[roommates] {
                        \matrixcontent
                    } ;
                    \graph {
                        (1) -- (2) ;
                        (3) -- (4) ;
                    } ;
                \end{tikzpicture}
                &
                \begin{tikzpicture}[ampersand replacement=\&]
                    \matrix[roommates] {
                        \matrixcontent
                    } ;
                    \graph {
                        (1) -- (4) ;
                        (3) -- (2) ;
                    } ;
                \end{tikzpicture}
                &
                \begin{tikzpicture}[ampersand replacement=\&]
                    \matrix[roommates] {
                        \matrixcontent
                    } ;
                    \graph {
                        (1) -- (3) ;
                        (2) -- (4) ;
                    } ;
                \end{tikzpicture}
            \end{tabular}
            \caption{
                The three matchings that are possible on the given set of nodes.
                Conceptually, what happens in this preference setup is that
                ``nobody likes $P_1$'', so whoever is paired with $P_1$ will
                happily trade them away. In the left case, $P_2$ prefers
                $P_3$ to $P_1$ and $P_3$ prefers $P_2$ to $P_4$, so they swap,
                giving the matching in the center. In this case, $P_2$
                prefers $P_4$ to $P_3$ and $P_4$ prefers $P_2$ to $P_1$, so
                they swap, giving the right case. In this case, $P_4$
                prefers $P_3$ to $P_2$ and $P_3$ prefers $P_4$ to $P_1$, so
                they swap, returning us to the matching in the left.
            }
            \label{fig:unstable}
        \end{figure}

    \item
        \questionname{Edge coloring.} --

    \item
        \questionname{Counting matchings.}

        \begin{proposition}
            For any graph $G$ with bipartition $(A, B)$ such that
            $A = \{a_1, \ldots, a_n\}$ and $B = \{b_1, \ldots, b_{n+1}\}$ with
            $a_i$ adjacent to $b_1, \ldots, b_{i+1}$ for $i = 1, \ldots, n$,
            there are exactly $2^n$ matchings on $G$ covering $A$.
        \end{proposition}

        \begin{proof}
            First notice that $A$ is simply a finite set on $n$ elements, which
            we will denote by $\fin{n}$. Then, a matching on $G$ covering $A$
            is simply an injective function $f : \fin{n} \to \fin{n+1}$. (To
            each item in $A$, we associate an item in $B$ such that no two
            elements of $A$ are mapped to the same item in $B$.) We can
            count the number of such functions easily by induction on $n$.

            First, consider the case $n = 0$. Notice that the domain is then
            empty.
            There exists only the empty set as a mapping from the empty domain.
            The empty map is vacuously injective.
            Hence, there are $2^0$ injective maps from $\fin{0}$ to $\fin{1}$.

            Next in the step case, suppose there are $2^k$ injective mappings
            from $\fin{k}$ to $\fin{k+1}$. We wish to show that there are
            $2^{k+1}$ injective functions from $\fin{k+1}$ to $\fin{k+2}$.

            Take an arbitrary injective mapping
            $f : \fin{k} \to \fin{k+1}$.
            Next, we define the map $g : \fin{k+1} \to \fin{k+2}$ by the
            following rule.
            \begin{equation*}
                a \mapsto \begin{cases}
                    f(a) &\quad\text{if } a < k + 1 \\
                    b    &\quad\text{if } a = k + 1
                \end{cases}
            \end{equation*}
            There are exactly two choices for the value $b$ that is the image
            of $k + 1$ under $f$. The first choice is the unique element in
            $\fin{k+1}$ that is not the image of anything in $\fin{k}$ under
            $f$. (This element exists and is unique by the pigeonhole
            principle.)
            The alternative choice for $b$ is simply $k+1$.
            Any other value we could choose for $b$ would be in the image of
            $f$, consequently making $g$ not injective.

            Since there are $2^k$ choices for the function $f$ and two ways to
            construct $g$ given $f$, there are $2^{k+1}$ injective maps from
            $\fin{k+1}$ to $\fin{k+2}$.

            Hence by induction, for any $n$, the number of injective functions
            from $\fin{n}$ to $\fin{n+1}$ is $2^n$. By the correspondence
            between matchings covering $A$ on bipartite graphs of the form of
            $G$ and injective functions on finite sets, there are $2^n$
            matchings covering $A$ on $G$.
        \end{proof}

    \item
        \questionname{K\"onig's Theorem.} --

    \item
        \questionname{Matching markets.}

        Consider the following market with buyers $(A,B,C,D)$ and sellers
        $(X,Y,Z,W)$ with the following valuations.

        \begin{center}
            \begin{tabular}{c | c c c c}
                ~ & X & Y & Z & W \\ \hline
                A & 6 & 4 & 6 & 6 \\
                B & 6 & 5 & 7 & 2 \\
                C & 4 & 1 & 7 & 5 \\
                D & 3 & 1 & 6 & 3
            \end{tabular}
        \end{center}

        Figure $\ref{fig:market}$ shows an algorithmically generated sequence
        of pricings for the sellers and the graphs that these pricings give
        rise to according to buyer satisfaction.

        \newcommand{\marketmatrix}{
            \matrix[market]{
                \node (A) {$A$} ; \& \node (X) {$X$} ; \\
                \node (B) {$B$} ; \& \node (Y) {$Y$} ; \\
                \node (C) {$C$} ; \& \node (Z) {$Z$} ; \\
                \node (D) {$D$} ; \& \node (W) {$W$} ; \\
            }
        }

        \newcommand{\marketgraph}[1]{
            \begin{tikzpicture}[ampersand replacement=\&]
                \marketmatrix ;
                \graph[use existing nodes, right anchor=west, left anchor=east]{
                    #1
                } ;
            \end{tikzpicture}
        }

        \newcommand{\marketcell}[4]{
            \nest{
                $p_#1 = #2$ \\
                $N(S_#1) = #3$ \\
                \marketgraph{
                    #4
                }
            }
        }

        \newcommand{\marketcellok}[3]{
            \nest{
                $p_#1 = #2$ \\
                \marketgraph{
                    #3
                }
            }
        }

        \begin{figure}[ht]
            \centering
            \begin{tabular}{c c c c}
                \marketcell{1}{ (0, 0, 0, 0) }{ \{Z\} }{
                    A -- { X, Z, W } ;
                    B -- Z ;
                    C -- Z ;
                    D -- Z ;
                }
                &
                \marketcell{2}{ (0 ,0 ,1 , 0) }{ \{X, Z\} }{
                    A -- { X, W } ;
                    B -- { X, Z } ;
                    C -- { Z } ;
                    D -- { Z } ;
                }
                &
                \marketcell{3}{ (1, 0, 2, 0) }{ \{Z, W\} }{
                    A -- W ;
                    B -- { X, Y, Z } ;
                    C -- { Z, W } ;
                    D -- Z ;
                }
                &
                \marketcellok{4}{ (1, 0, 3, 1) }{
                    A -- { X, W } ;
                    B -- { X, Y } ;
                    C -- { Z, W } ;
                    D -- Z ;
                }
            \end{tabular}
            \caption{
                The pricings produced according to the buyer satisfaction
                algorithm. Recall that each graph has an edge between a buyer
                and a seller if the trade represented by the edge maximizes
                buyer satisfaction. The set $N(S_i)$ is the constricting set
                at stage $i$ of the algorithm. In the final stage, a perfect
                matching exists.
            }
            \label{fig:market}
        \end{figure}

\end{enumerate}

\end{document}
