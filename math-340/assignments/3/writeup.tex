\documentclass[letterpaper,11pt]{article}

\author{Jacob Thomas Errington}
\title{Assignment \#3\\Discrete structures 2 -- MATH 340}
\date{8 March 2017}

\usepackage[margin=2.0cm]{geometry}
\usepackage{amsmath,amssymb,amsthm}

\newtheorem{prop}{Proposition}
\DeclareMathOperator{\Prob}{P}
\renewcommand{\P}[1]{\Prob{\parens{#1}}}
\DeclareMathOperator{\prob}{p}
\newcommand{\p}[2][]{%
    \def\temp{#2}
    \ifx\temp\empty
        \prob{\parens{#2}}
    \else
        \prob_{#1}{\parens{#2}}
    \fi
}
\DeclareMathOperator{\Expect}{\mathbb{E}}
\newcommand{\E}[1]{\Expect{\left[#1\right]}}
\DeclareMathOperator{\Var}{\mathbb{V}}
\newcommand{\V}[1]{\Var{\parens{#1}}}

\newcommand{\parens}[1]{\left(#1\right)}
\newcommand{\Union}{\bigcup}
\newcommand{\union}{\cup}
\newcommand{\Intersn}{\bigcap}
\newcommand{\intersn}{\cap}
\newcommand{\symdiff}{\,\Delta\,}
\newcommand{\fact}{!\,}
\newcommand{\given}{\;\vert\;}
\newcommand{\compl}{^c}
\newcommand{\inv}{^{-1}}
\newcommand{\compose}{\circ}

\renewcommand{\thesection}{Question \arabic{section}}
\newcommand{\question}{\section}

\begin{document}

\maketitle

\question{Bayes's Theorem}

\begin{enumerate}
    \item
        This problem can be solved straighforwardly without Bayes's theorem
        by enumerating the sample space, $S = \{GG, BG, GB, BB\}$, where $BG$
        for instance denotes the outcome that the first child is male and the
        second is female. As we know that one child is male, this restricts the
        sample space to $S^\prime = \{BG, GB, BB\}$. In this space, the event
        $E = \text{``having one boy and one girl''}$ is true for both the $BG$
        and $GB$ outcomes.
        Hence, $\P{E} = \frac{2}{3}$.

    \item
        Let $Y$ denote the number of heads and $X$ denote the face value of the
        die. We wish to find $\P{X = 6 \given X = Y}$, so we apply Bayes's
        theorem.
        \begin{equation}
            \label{eq:bleh}
            \P{X = 6 \given X = Y} = \frac{
                \P{Y = X \given X = 6} \P{Y = X}
            }{
                \P{X = 6}
            }
        \end{equation}

        First, we know that $\P{Y = X \given X = 6}$ is simply the probability
        of getting heads six times $\P{Y = 6} = \frac{1}{2^6}$.

        Second, $\P{X = 6} = \frac{1}{6}$ is the probability of rolling a six.

        Finally, we find $\P{Y = X}$ by using the Law of Total Probability, as
        the different values of $X$ partition the space.
        \begin{equation*}
            \P{Y = X} = \sum_{i = 1}^6 {
                \P{Y = X \given X = i}
            }
            =
            \sum_{i=1}^6 {
                \frac{1}{2^i}
            }
            = \frac{63}{64}
        \end{equation*}

        Substituting into \eqref{eq:bleh} gives
        \begin{equation*}
            \P{X = 6 \given X = Y}
            = \frac{189}{2048} \sim 9.23
        \end{equation*}
\end{enumerate}

\end{document}
