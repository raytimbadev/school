\documentclass[letterpaper,11pt]{article}

\author{Jacob Thomas Errington\\(260636023)}
\date{2 February 2017}
\title{Assignment \#1 \\ Probability -- MATH 323}

\usepackage[margin=2.0cm]{geometry}
\usepackage{amsmath,amssymb,amsthm}
\usepackage{tikz}

\usetikzlibrary{graphs}

\newtheorem{proposition}{Proposition}
\let\P\undefined
\DeclareMathOperator{\P}{\mathbb{P}}

\newcommand{\half}{\frac{1}{2}}
\newcommand{\parens}[1]{\left(#1\right)}

\newcommand{\questionname}{\textit}
\newcommand{\intersn}{\cap}
\newcommand{\union}{\cup}
\DeclareMathOperator{\trueOp}{true}
\newcommand{\true}{\trueOp{}}

\begin{document}

\maketitle

\begin{enumerate}
    \item
        Three boys $B_{1,2,3}$ and three girls $G_{1,2,3}$ line up in a random
        order for a photograph.

        \begin{proposition}
            The probability that the three girls stand next to each other is
            $0.2$.
        \end{proposition}

        \begin{proof}
            The sample space under consideration is the set of permutations on
            six distinct items, which has cardinality $|S| = 6!$. The event $E$
            whose cardinality we seek to measure is ``three girls
            stand next to each other''. The key insight is that this event can
            be broken into four disjoint events that describe the ordering of
            the items. Figure \ref{fig:lineups} shows these events pictorially.

            \begin{figure}[ht]
                \centering
                \newcommand{\lineup}[6]{
                    \node (#1) {$#1$} ;
                    \&
                    \node (#2) {$#2$} ;
                    \&
                    \node (#3) {$#3$} ;
                    \&
                    \node (#4) {$#4$} ;
                    \&
                    \node (#5) {$#5$} ;
                    \&
                    \node (#6) {$#6$} ;
                    \\
                }
                \begin{tikzpicture}[ampersand replacement=\&]
                    \matrix[column sep=0.25em, row sep=1em]{
                        \lineup{G}{G}{G}{B}{B}{B}
                        \lineup{B}{G}{G}{G}{B}{B}
                        \lineup{B}{B}{G}{G}{G}{B}
                        \lineup{B}{B}{B}{G}{G}{G}
                    } ;
                \end{tikzpicture}
                \caption{
                    The different ways the people can line up. To find the size of
                    the overall event we wish to measure, it suffices to count the
                    number of ways that this diagram can be labelled with
                    subscripts $1$, $2$ and $3$.
                }
                \label{fig:lineups}
            \end{figure}

            As mentioned in the figure caption, it suffices to find out how
            many ways the subscripts can be written into each line of the
            diagram. In each line of the diagram, notice that this is the same
            as asking ``how many ways can be boys the arranged in this line?''
            and ``how many ways can the girls be arranged in this line?'' The
            answer to both these questions is $3!$. Performing both of these
            processes together in a line means multiplying, so for a fixed
            \emph{unlabelled} arrangement of boys and girls, the number of ways
            to assign the subscripts is $(3!)^2$.

            Since there are four fixed unlabelled arrangements of boys and
            girls that comprise the overall event, we multiply this count by
            four to arrive at $|E| = 4(3!)^2$. Finally, we compute the
            probability of the event.

            \begin{equation*}
                \P(E) = \frac{|E|}{|S|} = \frac{4(3!)^2}{6!} = 0.2
            \end{equation*}
        \end{proof}

        \begin{proposition}
            The probability that the boys and girls alternate is $0.1$.
        \end{proposition}

        \begin{proof}
            We reuse the fact from the previous proof that given a fixed
            unlabelled arrangement of boys and girls, the number of ways that
            labels may be assigned to it is $(3!)^2$. There are two possible
            alternation schemes: one begins with a boy and the other begins
            with a girl. Hence, we compute the probability of the event ``the
            boys and girls alternate''.

            \begin{equation*}
                \P(E) = \frac{|E|}{|S|} = \frac{2(3!)^2}{6!} = 0.1
            \end{equation*}
        \end{proof}

    \item
        \questionname{Sampling with and without replacement.}

        \begin{proposition}
            Suppose $10$ cards are drawn with replacement from a standard deck.
            The probability that cards of both colors are selected is
            $\frac{511}{512}$.
        \end{proposition}

        \begin{proof}
            Notice that when sampling with replacement, the fact that we have
            drawn, say, a red card does not affect in any way whether we may
            draw another red card, unlike when sampling without replacement.
            (In the latter case after drawing one red card, the odds of drawing
            another one go down, as the supply of red cards has become
            smaller.)
            In other words, if ten items are sampled with replacement, then
            each of the ten samples is \emph{independent}.

            To approach this problem, we will compute the probability of
            drawing all cards of the same color, and then taking the
            complement.
            The probability of drawing a red card is $\half$, so the
            probability of drawing ten red cards is $\left(\half\right)^{10}$.
            Likewise for black cards. Hence, the probability of drawing all red
            or all black cards is the sum of these probabilities, since these
            events are disjoint.

            Finally, we compute the probability of drawing ten cards of both
            colors by noticing that it is the complement of drawing ten cards
            of the same color.

            \begin{equation*}
                \P(E) = 1 - 2 \parens{\half}^{10} = \frac{511}{512} \sim 0.998
            \end{equation*}
        \end{proof}

        \begin{proposition}
            Suppose 10 cards are drawn without replacement from a standard
            deck. The probability that cards of both colors are selected is X.
        \end{proposition}

        \begin{proof}
            We approach the scenario without replacement similarly by finding
            the probability that all cards drawn have the same color and taking
            the complement. The difference here is that after drawing, say, a
            red card, the probability of drawing another red card is smaller.

            The probability of first drawing a red card is $\frac{26}{52}$, and
            then drawing another one is $\frac{25}{51}$, and so one for a
            product of ten factors. We compute the probability of drawing a red
            card.

            \begin{equation*}
                \P(\text{Red}) = \prod_{i=0}^9 \frac{26 - i}{52 - i}
                = \frac{4}{21}
            \end{equation*}

            By symmetry, we have the same probability for drawing ten black
            cards. Finally, we compute the probability of drawing cards of
            different colors.

            \begin{equation*}
                \P{(\text{Red or Black})} = 1 - \frac{8}{21} = \frac{13}{21}
            \end{equation*}
        \end{proof}

    \item
        Two cards are drawn from a deck. Let $A$ be the event that the first is
        a diamond. Let $B$ be the event that the second is a diamond.

        If we just consider the second card, leaving the first card unknown,
        then the probability $\P{(B)} = \frac{1}{4}$. The probability that the
        first card is a diamond is the same: $\P{(A)} = \frac{1}{4}$.

        The probability of drawing two diamonds is
        \begin{equation*}
            \P{(A \intersn B)} = \frac{13}{52} \frac{12}{51} = \frac{1}{17}
            \neq
            \frac{1}{16} = \P(A) \P(B)
        \end{equation*}
        This shows that the events $A$ and $B$ are not independent.

    \item
        In a town, $40\%$ of the population has HIV. An HIV test comes back
        positive for a person with the disease with probability $0.9$; it comes
        back negative for a person without the disease with probability $0.95$.

        \begin{proposition}
            Suppose a person is chosen uniformly at random in this population
            and that their HIV test comes back positive. Then, that person has
            HIV with probability X.
        \end{proposition}

        \begin{proof}
            Let $H$ be the event of ``one having HIV'' and
            $T$ be the event of ``testing positive for HIV''.
            Then note the probabilities
            \begin{align*}
                \P(H) &= 0.4 \\
                \P(T|H) &= 0.9 \\
                \P(T|H^c) &= 0.05
            \end{align*}

            Using Bayes's theorem, we find that
            \begin{equation}
                \label{eq:afterbayes}
                \P(H|T) = \frac{\P(T|H) P(H)}{\P(T)}
            \end{equation}

            The quantities in the numerator are known, so it suffices to
            compute $\P(T)$. Notice that
            $\P(T) = \P(T \intersn (H \union H^c))$
            since $H$ and $H^c$ are mutually exclusive and
            obviously $\P(H \union H^c) = 1$. Hence by the law of conditional
            probability,
            \begin{align*}
                P(T)
                &= P(T \intersn (H \union H^c))
                = P(T \intersn H) + P(T \intersn H^c)
                = P(T | H) P(H) + P(T | H^c) P(H^c) \\
                &= (0.9) (0.4) + (0.05) (0.6)
                = \frac{39}{100}
            \end{align*}

            Substituting this result into \eqref{eq:afterbayes} gives
            \begin{equation*}
                \P(H|T) = \frac{(0.9) (0.4)}{0.39} = \frac{12}{13} \sim 0.92
            \end{equation*}
        \end{proof}
\end{enumerate}

\end{document}
