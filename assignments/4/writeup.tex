\documentclass[11pt,letterpaper]{article}

\author{Jacob Thomas Errington (260636023)}
\title{Assignment \#4\\Logic and computation -- COMP 527}
\date{24 March 2016}

\usepackage[margin=2.0cm]{geometry}
\usepackage{amsmath,amsthm,amssymb,proof}

\newtheorem{proposition}{Proposition}
\newtheorem{lemma}{Lemma}

\DeclareMathOperator{\opNode}{Node}
\DeclareMathOperator{\opTree}{tree}
\DeclareMathOperator{\opNat}{nat}
\DeclareMathOperator{\opTrue}{true}
\DeclareMathOperator{\opRec}{\mathtt{rec}}
\DeclareMathOperator{\opLet}{\mathtt{let}}
\DeclareMathOperator{\opIn}{\mathtt{in}}
\DeclareMathOperator{\opWith}{\mathtt{with}}

\newcommand{\emptyTree}{\mathrm{Empty}}
\newcommand{\nodeTree}[3]{\opNode{#1\,#2\,#3}}
\newcommand{\tree}{\opTree{}}
\newcommand{\nat}{\opNat{}}
\newcommand{\with}{\opWith{}}
\newcommand{\case}[3]{#1\,#2\mapsto #3}
\newcommand{\orCase}{\mid}
\newcommand{\rec}[1]{\opRec{\,#1\,}}
\newcommand{\reducesTo}{\implies}
\newcommand{\olet}{\opLet{}\,}
\newcommand{\oin}{\opIn{}\,}
\newcommand{\proves}{\vdash}
\renewcommand{\t}{\opTrue{}}

\begin{document}

\maketitle

\section{Induction principles for binary trees}

Consider the following definition of binary trees.

\begin{align*}
    \infer{\emptyTree : \text{tree}}{}
    &
    \quad
    &
    \infer{\nodeTree{n}{t_1}{t_2} : \tree}{
        n : \nat & t_1 : \tree & t_2 : \tree
    }
\end{align*}

\begin{enumerate}
    \item Here is an induction rule for binary trees.

        \begin{equation*}
            \infer[\tree E^{n, t_1, t_2, ih_1, ih_2}]{A(t) \t}{
                t : \tree
                &
                A(\emptyTree) \t
                &
                \infer*{A(\nodeTree{n}{t_1}{t_2}) \t}{
                    \infer{n : \nat}{}
                    &
                    \infer{t_1 : \tree}{}
                    &
                    \infer{t_2 : \tree}{}
                    &
                    \infer[ih_1]{A(t_1) \t}{}
                    &
                    \infer[ih_2]{A(t_2) \t}{}
                }
            }
        \end{equation*}

    \item Here is an induction rule for binary trees annotated with proof
        terms.

        \begin{equation*}
            \infer[\tree E^{n, t_1, t_2, ih_1, ih_2}]{
                \rec{t} \with
                \case{f}{\emptyTree}{M_E}
                \orCase
                \case{f}{\nodeTree{n}{t_1}{t_2}}{M_N}
                : A(t)
            }{
                t : \tree
                &
                M_E : A(\emptyTree)
                &
                \infer*{M_N : A(\nodeTree{n}{t_1}{t_2})}{
                    \infer{n : \nat}{}
                    &
                    \infer{t_1 : \tree}{}
                    &
                    \infer{t_2 : \tree}{}
                    &
                    \infer[ih_1]{f\, t_1 : A(t_1)}{}
                    &
                    \infer[ih_2]{f\, t_2 : A(t_2)}{}
                }
            }
        \end{equation*}

    \item Here is a reduction rule for recursion on binary trees.

        \begin{equation*}
            \rec{\emptyTree}
            \with \case{f}{\emptyTree}{M_E}
            \orCase \case{f}{\nodeTree{n}{t_1}{t_2}}{M_N} \reducesTo M_E
        \end{equation*}

        \begin{align*}
            &\rec{\nodeTree{m}{s_1}{s_2}}
            \with \case{f}{\emptyTree}{M_E}
            \orCase \case{f}{\nodeTree{n}{t_1}{t_2}}{M_N} \\
            \reducesTo
            &\olet r_1 = \rec{s_1}
                \with \case{f}{\emptyTree}{M_E}
                \orCase \case{f}{\nodeTree{n}{t_1}{t_2}}{M_N} \oin \\
            &\olet r_2 = \rec{s_2}
                \with \case{f}{\emptyTree}{M_E}
                \orCase \case{f}{\nodeTree{n}{t_1}{t_2}}{M_N} \oin \\
            &\left[m/n, s_1/t_1, s_2/t_2, r_1/f\,t_1, r_2/f\,t_2\right]M_N
        \end{align*}

    \item The proposed rules preserve typing.

        \begin{proof} By examining the operation of the rules.
            \begin{description}
                \item[Base case.] Using the first reduction rule.

                    \begin{tabular}{c r}
                        $
                        \rec{\emptyTree}
                        \with \case{f}{\emptyTree}{M_E}
                        \orCase \case{f}{\nodeTree{n}{t_1}{t_2}}{M_N}
                        : A(\emptyTree)
                        $
                        &
                        ass. \\
                        $
                        M_E : A(\emptyTree)
                        $
                        &
                        by $\tree E$
                    \end{tabular}

                \item[Step case.] Using the second reduction rule.

                    \begin{tabular}{c r}
                        $ % ***
                        \rec{\nodeTree{m}{s_1}{s_2}}
                        \with \case{f}{\emptyTree}{M_E}
                        \orCase \case{f}{\nodeTree{n}{t_1}{t_2}}{M_N}
                        : A(\nodeTree{m}{s_1}{s_2})
                        $
                        &
                        ass. \\
                        $ % ***
                        \nodeTree{m}{s_1}{s_2} : \tree,
                        m : \nat, s_1 : \tree, s_2 : \tree
                        $
                        &
                        by $\tree E$ \\
                        $ % ***
                        M_E : A(\emptyTree)
                        $
                        &
                        ~ \\
                        $ % ***
                        n : \nat, t_1 : \tree, t_2 : \tree,
                        ih_1 : A(t_1), ih_2 : A(t_2)
                        \proves M_N : A(\nodeTree{n}{t_1}{t_2})
                        $
                        &
                        ~ \\
                        $ % ***
                        ih_1 : A(s_1), ih_2 : A(s_2)
                        \proves
                        \left[m/n, s_1/t_1, s_2/t_2\right]
                        M_N : A(\nodeTree{m}{s_1}{s_2})
                        $
                        &
                        subst. \\
                        $ % ***
                        \proves
                        \left[
                            m/n, s_1/t_1, s_2/t_2, r_1/f\, t_1, r_2/f\, t_2
                        \right]
                        M_N : A(\nodeTree{m}{s_1}{s_2})
                        $
                        &
                        subst. \\
                    \end{tabular}
            \end{description}
        \end{proof}

    \item Here is a primitive recursive program to compute the size of a binary
        tree.

        \begin{equation*}
            \rec{t}
            \with \case{f}{\emptyTree}{z}
            \orCase \case{f}{\nodeTree{n}{t_1}{t_2}}{s(f\, t_1 + f\, t_2)}
        \end{equation*}

\end{enumerate}

\end{document}
