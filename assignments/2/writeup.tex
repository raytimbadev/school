\documentclass[11pt,letterpaper]{article}

\usepackage{proof}
\usepackage{amsmath,amssymb,amsthm}
\usepackage[margin=2.0cm]{geometry}

\newtheorem{lemma}{Lemma}
\DeclareMathOperator{\trueP}{\text{ true}}
\DeclareMathOperator{\fst}{\mathtt{fst}}
\DeclareMathOperator{\snd}{\mathtt{snd}}

\renewcommand{\t}{\trueP{}}
\newcommand{\proves}{\vdash}
\newcommand{\neut}{\uparrow}
\newcommand{\norm}{\downarrow}
\renewcommand{\implies}{\supset}

\begin{document}

\section{Logical equivalence}

We can interpret logical equivalence,
$A \equiv B$
as $A \supset B$ and $B \supset A$, with the following introduction and
elimination rules.

\begin{center}
    \begin{tabular}{c c c}
        $$
        \infer[\equiv I^{u,v}]{A \equiv B \t}{
            \infer*{B\t}{
                \infer[u]{A\t}{}
            }
            &
            \infer*{A\t}{
                \infer[v]{B\t}{}
            }
        }
        $$
        &
        $$
        \infer[\equiv E_L]{A\t}{
            A \equiv B \t
            &
            B \t
        }
        $$
        &
        $$
        \infer[\equiv E_R]{B\t}{
            A \equiv B \t
            &
            A \t
        }
        $$
    \end{tabular}
\end{center}

\newcommand{\equivproofterm}[6]{
    \left \langle
        \lambda #1 : #2 . #3,
        \lambda #4 : #5 . #6
    \right\rangle : #2 \equiv #5
}

\begin{enumerate}
    \item These rules have the following proof terms, which look overall like a
        combination of the terms for conjunctions and implications.

        $$
        \infer[\equiv I^{u,v}]{\equivproofterm{x}{A}{M}{y}{B}{N}}{
            \infer*{M : B}{
                \infer[u]{x : A}{}
            }
            &
            \infer*{N : A}{
                \infer[v]{y : B}{}
            }
        }
        $$

        $$
        \infer[\equiv E_L]{\fst M x : B}{
            M : A \equiv B
            &
            x : A
        }
        $$

        $$
        \infer[\equiv E_R]{\snd M y : A}{
            M : A \equiv B
            &
            y : B
        }
        $$

    \item Using these proof terms, here are the local reduction and expansion
        rules for logical equivalence.

        $$
        \begin{array}{c c c}
            \infer[\equiv E_L]{M : B}{
                \infer[\equiv I^{u, v}]{\equivproofterm{x}{A}{M}{y}{B}{N}}{
                    \deduce[\mathcal{D}]{M:B}{
                        \infer[u]{x:A}{}
                    }
                    &
                    \deduce[\mathcal{E}]{N:A}{
                        \infer[v]{y:B}{}
                    }
                }
                &
                \deduce[\mathcal{F}]{x : A}{}
            }
            &
            \Rightarrow_R
            &
            \deduce[\text{[}\mathcal{F} / u \text{]} \mathcal{D}]{M:B}{}
        \end{array}
        $$

        $$
        \begin{array}{c c c}
            \deduce[\mathcal{D}]{M : A \equiv B}{}
            &
            \Rightarrow_E
            &
            \infer[\equiv I^{u, v}]{
                \equivproofterm{x}{A}{\fst M x}{y}{B}{\snd M y}
            }{
                \infer[\equiv E_L]{\fst M x : B}{
                    \deduce[\mathcal{D}]{M : A \equiv B}{}
                    &
                    \infer[u]{x : A}{}
                }
                &
                \infer[\equiv E_R]{\snd M y : A}{
                    \deduce[\mathcal{D}]{M : A \equiv B}{}
                    &
                    \infer[v]{y : B}{}
                }
            }
        \end{array}
        $$

        These rules show that the proposed extension is locally sound and
        complete.

    \item
        The addition of these three rules necessitates adding three cases to
        the proof by structural induction of the substitution lemma.

        \begin{lemma}[Substitution]
            If $\Gamma, x:A, \Gamma^\prime \proves M : B$
            and $\Gamma \proves N : A$, then
            $$
            \Gamma, \Gamma^\prime \proves [N/x]M : B
            $$
        \end{lemma}


        \begin{proof} By structural induction on a given derivation.

            \begin{description}
                \item[Case] $\equiv I^{u, v}$.
                    $$
                    \mathcal{D} = \infer[\equiv I^{u, v}]{
                        \Gamma, x:A, \Gamma^\prime
                        \proves
                        \equivproofterm{u}{B}{M_1}{z}{C}{M_2}
                    }{
                        \deduce[\mathcal{D}_1]{
                            \Gamma, x:A, \Gamma^\prime, u:B \proves M_1 : C
                        }{}
                        &
                        \deduce[\mathcal{D}_2]{
                            \Gamma, x:A, \Gamma^\prime, v:C \proves M_2 : B
                        }{}
                    }
                    $$

                    \begin{center}
                        \begin{tabular}{l r}
                            $\Gamma \proves N : A$
                            &
                            by assumption. \\
                            $\Gamma, \Gamma^\prime, u:B \proves [N/x]M_1 : C$
                            &
                            by I.H. on $\mathcal{D}_1$. \\
                            $\Gamma, \Gamma^\prime, v:C \proves [N/x]M_2 : B$
                            &
                            by I.H. on $\mathcal{D}_2$. \\
                            $
                            \Gamma, \Gamma^\prime
                            \proves
                            \equivproofterm{u}{B}{[N/x]M_1}{v}{C}{[N/x]M_2}
                            $
                            &
                            by $\equiv I^{u, v}$. \\
                            $
                            \Gamma, \Gamma^\prime
                            \proves
                            [N/x]\equivproofterm{u}{B}{M_1}{v}{C}{M_2}
                            $
                            &
                            by substitution definition. \\
                        \end{tabular}
                    \end{center}

                \item[Case] $\equiv E_L$.
                    $$
                    \mathcal{D} = \infer[\equiv E_L]{
                        \Gamma, x:A, \Gamma^\prime
                        \proves
                        \fst M u : C
                    }{
                        \deduce[\mathcal{D}_1]{
                            \Gamma, x:A, \Gamma^\prime
                            \proves
                            M : B \equiv C
                        }{}
                        &
                        \deduce[\mathcal{D}_2]{
                            \Gamma, x:A, \Gamma^\prime
                            \proves
                            u : B
                        }{}
                    }
                    $$

                    \begin{center}
                        \begin{tabular}{l r}
                            $\Gamma \proves N : A$
                            &
                            by assumption. \\
                            $
                            \Gamma, \Gamma^\prime
                            \proves
                            [N/x]M : B \equiv C
                            $
                            &
                            by I.H. on $\mathcal{D}_1$. \\
                            $
                            \Gamma, \Gamma^\prime
                            \proves
                            [N/x]u : B
                            $
                            &
                            by I.H. on $\mathcal{D}_2$. \\
                            $
                            \Gamma, \Gamma^\prime
                            \proves
                            \fst [N/x]M [N/x]u : C
                            $
                            &
                            by $\equiv E_L$. \\
                            $
                            \Gamma, \Gamma^\prime
                            \proves
                            [N/x] \fst M u : C
                            $
                            &
                            by substitution definition.
                        \end{tabular}
                    \end{center}

                \item[Case] $\equiv E_R$.

                    This proof is essentially the same as for $\equiv E_L$ so
                    we will omit it.
            \end{description}
        \end{proof}

    \item
        We restate introduction and elimination rules for logical equivalence
        with the judgements $A\norm$ and $A\neut$ to show how to construct only
        normal proofs with these rules.

        \begin{center}
            \begin{tabular}{c c c}
                $$
                \infer[\equiv I^{u,v}]{A \equiv B \norm}{
                    \infer*{B \norm}{
                        \infer[u]{A \neut}{}
                    }
                    &
                    \infer*{A \norm}{
                        \infer[v]{B \neut}{}
                    }
                }
                $$
                &
                $$
                \infer[\equiv E_R]{A \neut}{
                    A \equiv B \neut
                    &
                    B \norm
                }
                $$
                &
                $$
                \infer[\equiv E_L]{B \neut}{
                    A \equiv B \neut
                    &
                    A \norm
                }
                $$
            \end{tabular}
        \end{center}

\end{enumerate}

\section{First-order logic proofs}

See the attached verified proofs written for Tutch.

We found the following statements to be unprovable.

\begin{itemize}
    \item
        $(\exists x : \tau . (A \implies P x)
        \implies A \implies \forall x : \tau . P x$

        From the assumptions that we have, it can only be proved that $P a$
        holds for the specific witness $a : \tau$ that was used to construct
        the existential. We cannot prove that $P x$ holds for a general $x$.

    \item
        $((\forall x : \tau . P x) \implies A)
        \implies \exists x : \tau . (P x \implies A)$

        It is impossible to produce a specific witness $a : \tau$ for which
        $P a \implies A$ using nothing more that the fact that
        $(\forall x : \tau . P x) \implies A$. The proof breaks down when we
        try to construct the existential.

    \item
        $\neg (\forall x : \tau . B x) \implies \exists x : \tau . \neg B x$

        Again from nothing more than an assumption involving a universally
        quantified proposition, we can't construct the witness necessary to
        build the existential.
\end{itemize}

\end{document}
