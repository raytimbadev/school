\documentclass{article}

\usepackage{amsmath,amssymb,amsthm}
\usepackage{tikz}
\usepackage[margin=2.0cm]{geometry}

\author{Jacob Thomas Errington (260636023)}
\title{Assignment \#1\\Distributed Systems -- COMP 512}
\date{20 October 2015}

\newcommand{\ms}{\text{ ms}}

\begin{document}

\maketitle

\section{Minis}

\begin{enumerate}
    \item
        In an asynchronous system, a process $p$ requests the time from a time
        server $S$, with a measured round-trip time of $24\ms$, and $p$
        receives $t = 10:54:23:674$ from $S$.

        \begin{enumerate}
            \item
                $p$ would set it's local clock to the received time, plus half
                the round-trip time, i.e.
                $$
                ps
                = t + \frac{T_\text{round}}{2}
                = 10 : 54 : 23 : 674 + 0 : 0 : 0 : 12
                = 10 : 54 : 23 : 698
                $$

            \item
                If $p$'s local time upon receipt of the response from $S$ is
                $ps = 10 : 54 : 24 : 005$, then setting its time to the value
                calculated above would result in a ``jump back in time''. This
                violates the expected monotonicity of the system clock. To
                remedy this, the system can run it's local clock more slowly
                for some time, to have it eventually become synchronized with
                the received time.
        \end{enumerate}

    \item
        There are many ways to marshal program data into a form suitable
        for transmission: XML (similarly JSON), Java's serialization, and
        Google Protocol Buffer, as well as others. For each of these three
        methods, here is an advantage it has over the other two.

        \begin{enumerate}
            \item
                XML and JSON have the advantage of being human-readable,
                text formats; if data is inspected while in transit, it can
                be understood without any additional tooling by a
                developer.

            \item
                Java's built-in serialization has the advantage of being
                extremely easy to use, from a developer's point of view.
                No additional libraries are required, since the
                functionality is built directly into the base Java Runtime
                Environment.

            \item
                Google Protocol Buffer is a binary format, like Java's
                built-in serialization, but is not tied to Java, so it can
                be used to interact with non-Java environments. It is also
                much faster than XML.
        \end{enumerate}
\end{enumerate}

\end{document}
