\documentclass{article}

\author{Jacob Thomas Errington (260636023)}
\title{Assignment \#2\\Distributed Systems -- COMP 512}
\date{28 October 2015}

\usepackage{float}
\usepackage{float}
\usepackage{amsmath,amssymb,amsthm}
\usepackage[margin=2.0cm]{geometry}
\usepackage{algorithm,algorithmicx,algpseudocode}

\begin{document}

\maketitle

\section{Total order multicast}

Given is a group of nodes organized into clusters. Let $g_1, \cdots, g_n$
denote these clusters. Each cluster $g_i$ contains a single distinguished
process denoted $p^\prime_i$ called the control process. Now we can establish
the two kinds of communication groups in this system.

\begin{enumerate}
    \item A \emph{regular communication group} consists of all members of
        a group $g_i$. There are $i$ such groups under consideration.
    \item The \emph{control group} $g^\prime$ consists of all the control
        processes, i.e. $g^\prime = \{p^\prime_1, \cdots, p^\prime_n\}$. There
        is exactly one control group, and it has $n$ members.
\end{enumerate}

Messages are to be transmitted between all nodes in a totally ordered fashion.
To achieve this, the control process of each cluster will act as a sequencer
within its cluster, and the control group will operate using a token ring
approach. Below are the different scenarios that can arise in this scheme.

\begin{enumerate}
    \item
        When a regular process wants to multicast a message, it simply sends it
        to the control process of its cluster.

    \item
        When a control process receives a message from an ordinary process in
        its cluster, it will add it to a queue $Q$.

    \item
        When a regular process receives a message from its associated control
        process, it delivers it in the order given by the sequence number of
        the message.

    \item
        When a control process receives a message from another control process,
        it simply adds it to a queue $P$ prioritized on the sequence numbers of
        received messages. Notice that this priority queue is separate from the
        queue of cluster-local messages.

    \item
        When a control process receives the token from another control process,
        it multicasts its enqueued messages to each of the processes in its
        cluster and to every other control node, assigning sequence numbers to
        the messages according to the sequence number contained by the token.
\end{enumerate}

Of these scenarios, only the last one's implementation is nontrivial, so it is
given in full in algorithm \ref{alg:token-receive}.

\begin{algorithm}[h]
    \caption{A control process receives the token.}
    \label{alg:token-receive}
    \begin{algorithmic}
        \Require{
            The token $t$,
            the remote message queue $P$,
            the cluster-local queue $Q$,
            the next control process ID $j$.
        }
        \Ensure{
            Messages in $Q$ are transmitted to all processes in the local
            cluster and to all other control processes, and messages in $P$ are
            delivered to all processes in the local cluster.
        }
        \State ~
        \While{$|P| > 0$}
            \State $m, seq \gets$ \Call{Dequeue}{P}
            \For{each process $p$ in the local cluster}
                \State \Call{Send}{$p$, $\left\langle m, seq\right\rangle$}
            \EndFor
        \EndWhile
        \State $n \gets t.seq$
        \While{$|Q| > 0$}
            \State $m \gets$ \Call{Dequeue}{Q}
            \For{each process $p$ in the local cluster}
                \State \Call{Send}{$p$, $\left\langle m, n\right\rangle$}
            \EndFor
            \For{each other control process $p^\prime$}
                \State \Call{Send}{$p$, $\left\langle m, n\right\rangle$}
            \EndFor
            \State $n \gets n + 1$
        \EndWhile
        \State $t.seq \gets n$
        \State \Call{SendToken}{$t$, $p_j$}
    \end{algorithmic}
\end{algorithm}

\end{document}
