\documentclass[letterpaper,11pt]{article}

\author{Jacob Thomas Errington}
\title{Studying}
\date{}

\renewcommand{\d}{\mathrm{d}}
\newcommand{\R}{\mathbb{R}}
\newcommand{\parens}[1]{\left(#1\right)}
\newcommand{\abs}[1]{\left|#1\right|}

\usepackage[margin=2.0cm]{geometry}
\usepackage{amsmath,amssymb,amsthm}

\DeclareMathOperator{\curl}{\mathrm{curl}}

\begin{document}

\maketitle

\begin{description}
    \item[Problem 12.]
        Let $\mathcal{C}$ be the curve of intersection of the cylinder
        $x^2 + y^2 = 1$ and the surface $z = xy + 1$, oriented counterclockwise
        around the cylinder. Compute the line integral
        \begin{equation}
            \int_{\mathcal{C}} z(x-1) \d y + y(x+1) \d z
            \label{eq:p12:statement}
        \end{equation}

        Computing the line integral directly would be problematic, because the
        naive parameterization gives an integral in terms of the parameter that
        involves a lot of products of trigonometric functions. Instead, we can
        use the Kelvin-Stokes theorem to transform this into an integral over a
        surface bounded by $\mathcal{C}$.

        Since we are working in $\R^3$, there are infinitely many such
        surfaces. We choose the surface $\mathcal{S}$ identified by the
        parameterization $F : D \to \R^3$ given by
        \begin{equation}
            \parens{
                \begin{array}{c}
                    x \\
                    y
                \end{array}
            }
            \mapsto
            \parens{
                \begin{array}{c}
                    x \\
                    y \\
                    xy + 1
                \end{array}
            }
            \label{eq:p12:prmzn}
        \end{equation}
        where $D = \{(x, y) : x^2 + y^2 = 1\}$ is the unit disk centered at the
        origin.

        Let $\alpha = z(x-1)\d y + y(x+1) \d z$. Then the integral in
        \eqref{eq:p12:statement} over the differential form $\alpha$ is the
        same as an integral of the associated vector field of $\alpha$
        \begin{equation*}
            \int_{\mathcal{C}} \alpha^T \cdot \d l
        \end{equation*}
        which we can rewrite using the Kelvin-Stokes theorem as
        \begin{equation}
            \int_{\mathcal{S}} \curl{\alpha^T} \cdot n \d \mathcal{S}
            \label{eq:p12:afterks}
        \end{equation}

        Compute the curl of the vector field to integrate
        \begin{equation}
            \curl{\alpha^T} = \parens{
                \begin{array}{c}
                    2\\
                    y\\
                    z
                \end{array}
            }
            \label{eq:p12:curlat}
        \end{equation}
        and the normal vector of the surface $\mathcal{S}$
        \begin{equation}
            n = \partial_x F \times \partial_y F = \abs{
                \begin{array}{c c c}
                    e_1 & e_2 & e_3 \\
                    1   & 0   & y   \\
                    0   & 1   & x
                \end{array}
            }
            = \parens{
                \begin{array}{c}
                    y \\
                    -x \\
                    1
                \end{array}
            }
            \label{eq:p12:normal}
        \end{equation}

        Substituting \eqref{eq:p12:normal} and \eqref{eq:p12:curlat} into
        \eqref{eq:p12:afterks} gives
        \begin{equation}
            \int_{\mathcal{S}} 2y - xy + z \d \mathcal{S}
            \label{eq:p12:aftersub}
        \end{equation}

        Finally, we apply the parameterization given in \eqref{eq:p12:prmzn}
        to give a double integral over the parameter space.
        \begin{equation*}
            \int_{-1}^{1} \int_{-\sqrt{1-x^2}}^{\sqrt{1-x^2}} {2y + 1}\,\d y\d x
            = \int_{-1}^{1} {y^2 + y}\bigg\rvert_{-\sqrt{1-x^2}}^{\sqrt{1-x^2}}
            = 2 \int_{-1}^{1} \sqrt{1 - x^2} \, \d x
        \end{equation*}
        The last expression can be evaluated either by a trigonometric
        substitution, or by recognizing that this is simply the area of half of
        the unit disk, which is equal to $\frac{1}{2}\pi$. Hence, the final
        value for the integral is $\pi$.
\end{description}

\end{document}
