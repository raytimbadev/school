\documentclass[letterpaper,11pt]{article}

\author{Jacob Thomas Errington (260636023)}
\title{Assignment \#2\\Honours advanced calculus -- MATH 248}
\date{20 October 2016}

\usepackage[margin=2.0cm]{geometry}
\usepackage{amsmath,amsthm,amssymb}

\newtheorem{prop}{Proposition}

\newcommand{\R}{\mathbb{R}}

\begin{document}

\maketitle

\begin{enumerate}
    \item
        Let $f : \R^2 \to \R$ be defined by
        \begin{equation*}
            f(x, y) = \begin{cases}
                xy &\quad\text{if } -|x| < y < |x| \\
                0  &\quad\text{else}
            \end{cases}
        \end{equation*}

        \begin{enumerate}
            \item
                First, we will compute the first partial derivatives.
                \begin{align*}
                    \partial_x f &= \begin{cases}
                        y &\quad\text{if }
                            -|x| < y < |x| \lor x = 0\\
                        0 &\quad\text{if }
                            y > |x| \lor y < -|x|
                    \end{cases} \\
                    \partial_y f &= \begin{cases}
                        x &\quad\text{if }
                            -|x| < y < |x| \\
                        0 &\quad\text{if }
                            y > |x| \lor y < -|x| \lor x = 0
                    \end{cases}
                \end{align*}

                Notice that these partial derivatives are undefined along the
                boundary between the two functions that make up $f$, except at
                the origin. $f$ is discontinuous along the boundary, due to the
                jump from $xy \neq 0$ to $0$, but at the origin, there is no
                such jump. This is due to the fact that one of the factors in
                $xy$ will be zero as we approach the origin along the standard
                basis vectors. Consequently, the limit that defines the partial
                derivative will converge. However, the question of \emph{what}
                it converges to depends on how exactly the origin is
                approached.

                In $\partial_x f$, the origin is approached along the basis
                vector $(1, 0)$, so the limit that defines the partial
                derivative
                \begin{equation*}
                    \lim_{t \to 0} \frac{(x + t)y - xy}{t}
                    = \lim_{t \to 0} \frac{ty}{t}
                    = y
                \end{equation*}
                converges to $y$.

                In $\partial_y f$ however, the origin is approached along the
                basis vector $(0, 1)$, so the limit that defines the partial
                derivative converges to $0$.

                That is why the $x = 0$ case gives $0$ in $\partial_y$, but
                gives $y$ in the $\partial_x$ case. Of course, \emph{at} the
                origin, the function values of both partial derivatives agree.

                Now let us look at the second partial derivatives.
                \begin{align*}
                    \partial_y \partial_x f &= \begin{cases}
                        1 &\quad\text{if }
                            -|x| < y < |x| \lor x = 0\\
                        0 &\quad\text{if }
                            y > |x| \lor y < -|x|
                    \end{cases} \\
                    \partial_x \partial_y f &= \begin{cases}
                        1 &\quad\text{if }
                            -|x| < y < |x| \\
                        0 &\quad\text{if }
                            y > |x| \lor y < -|x| \lor x = 0
                    \end{cases}
                \end{align*}

                Although these two functions appear suspiciously similar, they
                are different in one major way,
                \begin{equation*}
                    1 = \partial_y \partial_x f(0, 0)
                    \neq
                    \partial_x \partial_y f(0, 0) = 0
                \end{equation*}
                due to the change in placement of the $x = 0$ case in the
                definition.

            \item
                Although this difference appears to contradict the theorem on
                the symmetricity of the Hessian, it does not; the theorem
                requires that the second partial derivatives exist and be
                continuous in some box domain (more generally in some
                neighbourhood). It is not hard to see that there is no
                neighbourhood containing the origin in which these second
                partial derivatives would be continuous.
        \end{enumerate}
\end{enumerate}

\end{document}
