\documentclass[letterpaper,11pt]{article}

\usepackage[margin=2.0cm]{geometry}
\usepackage{amsmath,amssymb,amsthm}

\newtheorem{prop}{Proposition}

\author{Jacob Thomas Errington (260636023)}
\title{Assignment 1\\Honours Advanced Calculus -- MATH 248}
\date{27 September 2016}

\newcommand{\diff}[1]{\,\text{differentiable at}\,#1}
\newcommand{\R}{\mathbb{R}}

\begin{document}

\maketitle

\begin{enumerate}
    \item
        \begin{enumerate}
            \item
                The function $f(x) = \sin{(\frac{1}{x})}$ is not continuous at
                $x = 0$, since $\frac{1}{x} \to \infty$ as $x \to 0$, and
                $\lim_{x\to\infty} {sin(x)}$ does not exist.

            \item
                The function $f(x) = x \sin{(\frac{1}{x})}$ is continuous at
                $x = 0$. Suppose there is a sequence $\{x_i\} \to 0$.
                $f(\{x_i\}) \to 0$ as well, since the factor
                $\sin{(\frac{1}{x_i})}$ remains bounded within $[-1, 1]$ for
                any $i$, but the factor $x_i \to 0$. Consequently, the overall
                limit will tend to $0$.

                Furthermore, the function $f$ is not differentiable at $x=0$.
                Consider the limit that defines the notion of
                differentiability, specialized for $f$.
                \begin{equation*}
                    \lim_{x\to y}{
                        \frac
                        {x \sin{(\frac{1}{x})} - y \sin{(\frac{1}{y})}}
                        {x - y}
                    }
                \end{equation*}

                Since $y=0$, the limit becomes $\sin{(\frac{1}{x})}$ as
                $x \to 0$, which as discussed earlier, does not exist. Hence
                the function is not differentiable, despite being continuous.

            \item
                The function $f(x) = x^2 \sin{(\frac{1}{x})}$ is continuous by
                the same argument as given immediately above. However, it
                differs from the function above as $f$ here is differentiable.
                Consider the limit again: this time there will remain an
                additional factor $x$ in the numerator. Consequently, by the
                same argument that showed that $f$ is continuous in the
                previous problem, we find that the limit for differentiability
                of $f$ in this problem exists
        \end{enumerate}

    \item
        We wish to show that differentiability is a local property.

        \begin{prop}
            Differentiability is a local property, i.e. $f : K \to \R$ is
            differentiable at $y$ if and only if
            $g = f|_{(y-\epsilon, y+\epsilon)}$ is differentiable at $y$.
        \end{prop}

        \begin{proof}
            First we show the forwards direction. Using the sequential
            criterion, this is extremely straightforward. Since $f$ is
            differentiable at $y$, we know that for all sequences
            $\{x_i\} \subseteq K$ converging to values in $\R$, the limit
            \begin{equation*}
                \lim_{n\to\infty}{
                    \frac{f(x_i) - f(y)}{x - y} = f^\prime (x)
                }
            \end{equation*}
            exists and defines the derivative of $f$ at $y$. If we retrict our
            attention to those sequences that are bounded within the domain of
            $g$, then of course $f$ is differentiable there as well and the
            associated limits of those sequences converge. Since the values in
            those limits are bounded, and $f$ operating on that restricted
            domain is simply $g$, we can conclude that $g$ is differentiable
            there as well, by the sequential criterion.

            For the reverse direction, we will employ a similar argument using
            the sequential criterion of differentiability. Assume that $g$ is
            differentiable at $y$. Consider an arbitrary sequence
            $\{x_i\} \subseteq K$ converging to $y$. Now consider
            $\lim_{n\to\infty}{\frac{f(x_i) - g(y)}{x - y}}$. We want to show
            that this limit exists. By definition of a convergent sequence,
            for any $\epsilon > 0$, there exists some index $k$ such that
            $|x_k - y| < \epsilon$. Consequently, we instantiate this universal
            statement taking its $\epsilon$ to be the $\epsilon$ of the
            restricted domain of $g$ and consider a new sequence
            $\{x_i^\prime\}$ constructed from $\{x_i\}$ by dropping all items
            before $k$. This restricted sequence can be used applied to
            instantiate the sequential criterion for $g$
            \begin{equation*}
                \lim_{n\to\infty}{
                    \frac{g(x_i^\prime) - g(y)}{x - y}
                }
            \end{equation*}
            However, we know that $g$ is defined by $f$ in the domain
            $(y-\epsilon, y+\epsilon)$ within which lie each $x_i^\prime$, so
            we may replace $g$ with $f$ to obtain the convergent limit
            \begin{equation*}
                \lim_{n\to\infty}{
                    \frac{f(x_i^\prime) - f(y)}{x - y}
                }
            \end{equation*}
            This is almost the sequential criterion for $f$. All that remains
            is to return to the sequence $x_i$ instead of $x_i^\prime$: we know
            that for any convergent sequent $x_i$, prepending finitely many
            elements to it does not affect its convergence. The relationship
            between our initial sequence $x_i$ and our constructed sequence
            $x_i^\prime$ is precisely that there is some finite number ($k$ to
            be exact) of initial elements in $x_i$ before it is the same as
            $x_i^\prime$. Thus, we can replace $x_i^\prime$ with $x_i$ and
            obtain the convergent limit
            \begin{equation*}
                \lim_{n\to\infty}{
                    \frac{f(x_i) - f(y)}{x - y}
                }
            \end{equation*}
            which is precisely the sequential criterion for $f$, showing that
            $f$ is differentiable at $y$.
        \end{proof}
\end{enumerate}

\end{document}
