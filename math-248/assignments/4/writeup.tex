\documentclass[letterpaper,11pt]{article}

\author{Jacob Thomas Errington}
\title{Assignment \#4\\Honours advanced calculus -- MATH 248}
\date{22 November 2016}

\usepackage[margin=2.0cm]{geometry}
\usepackage{amsmath,amssymb,amsthm}

\newcommand{\makebb}[1]{
  \expandafter\newcommand\csname #1\endcsname{\mathbb{#1}}
}

\makebb{N}
\makebb{R}

\newtheorem{prop}{Proposition}

\newcommand{\union}{\cup}
\newcommand{\intersn}{\cap}
\newcommand{\parens}[1]{\left(#1\right)}

\begin{document}

\maketitle

\section*{\#1 -- Open and closed sets}

Let $A, B \subset \R^n$.

\begin{prop}
  Suppose $A, B$ are open sets. Then,
  \begin{enumerate}
    \item $A \union B$ is an open set.
    \item $A \intersn B$ is an open set.
  \end{enumerate}
\end{prop}

\begin{proof}
  ~

  \begin{enumerate}
    \item
      Take arbitrary $p \in A \union B$. Then $p \in A \lor p \in B$.
      \begin{description}
        \item[Case] $p \in A$.

          Since $A$ is open, $\exists \delta > 0 : Q_\delta(p) \subseteq A$.
          Since $A \subseteq A \union B$, $Q_\delta(p) \subseteq A \union B$ by
          transitivity.

        \item[Case] $p \in B$. The same argument applies, using instead the
          openness of $B$.
      \end{description}

      Therefore,
      \begin{equation*}
        \forall p \in A \union B:
        \exists \delta_p > 0:
        Q_{\delta_p} \subseteq A \union B
      \end{equation*}
      so $A \union B$ is open.

    \item
      Take arbitrary $p \in A \intersn B$. Then $p \in A \land p \in B$. Since
      both these sets are open,
      \begin{align*}
        \exists \delta_A > 0 : Q_{\delta_A}(p) &\subseteq A \\
        \exists \delta_B > 0 : Q_{\delta_B}(p) &\subseteq B
      \end{align*}
      Let $\delta = \min\{\delta_A, \delta_B\}$. Then,
      \begin{align*}
        Q_\delta(p) &\subseteq Q_{\delta_A}(p) \subseteq A \\
        Q_\delta(p) &\subseteq Q_{\delta_B}(p) \subseteq B
      \end{align*}
      so $Q_\delta(p) \subseteq A \intersn B$.

      Therefore,
      \begin{equation*}
        \forall p \in A \union B:
        \exists \delta_p > 0:
        Q_{\delta_p} \subseteq A \intersn B
      \end{equation*}
      so $A \intersn B$ is open.
  \end{enumerate}
\end{proof}

\begin{prop}
  Suppose $A, B$ are closed sets. Then,
  \begin{enumerate}
    \item
      $A \union B$ is closed.

    \item
      $A \intersn B$ is closed.
  \end{enumerate}
\end{prop}

\begin{proof}
  ~

  \begin{enumerate}
    \item
      Take an arbitrary sequence $(x_n)_{n\in\N}$ in $\R^n$.
      Suppose that $\forall n \in \N : x_n \in A \union B$,
      and that $\lim_{n\to\infty} x_n = x \in \R^n$ exists.
      We want to show that $x \in A \union B$.

      Since $(x_n)$ is a convergent sequence, then all its subsequences are
      convergent, and converge to the same limit. Therefore, consider the
      subsequence $(x_{n_k})$ such that $\forall k \in \N : x_{n_k} \in A$.
      Since $A$ is closed, it contains all its limit points, so
      \begin{equation*}
        \lim_{k\to\infty} x_{n_k} = x \in A
      \end{equation*}
      Since $A \subseteq A \union B$, $x \in A \implies x \in A \union B$.
      Hence, for any convergent sequence with terms in $A \union B$ converges
      to a value in $A \union B$, so $A \union B$ is closed.

      (There is a slight problem with this argument as is: it requires that we
      can construct the subsequence $(x_{n_k})$ consisting only of elements of
      $A$. This is not necessarily possible; in the arbitrary sequence $(x_n)$
      there might only be finitely many terms in $A$. In that case however,
      there will be infinitely many terms in $B$, and we can just use the
      variant of the argument where we show that the limit is in $B$ rather
      than in $A$.)

    \item
      Take an arbitrary sequence $(x_n)_{n\in\N}$ in $\R^n$.
      Suppose that $\forall n \in \N : x_n \in A \intersn B$,
      and that $\lim_{n\to\infty} x_n = x \in \R^n$ exists.
      We want to show that $x \in A \intersn B$.

      Since $\forall n \in \N : x_n \in A \intersn B$,
      $x_n \in A \land x_n \in B$. Since $A$ and $B$ are closed, they contain
      they contain their limit points, so
      \begin{align*}
        \lim_{n\to\infty} x_n = x \in A \\
        \lim_{n\to\infty} x_n = x \in B
      \end{align*}
      Hence, $x \in A \intersn B$.
      Therefore, for any convergent sequence with terms in $A \intersn B$
      converges to a limit in $A \intersn B$, so $A \intersn B$ is closed.
  \end{enumerate}
\end{proof}

\begin{prop}
  Suppose $\Omega \subseteq \R^n$ is an open set and $\phi : \Omega \to \R$ is
  a continuous function. Then, $U = \{ x \in \Omega : \phi(x) > 0\}$ is an open
  set.
\end{prop}

\begin{proof}
  Since $\Omega$ is open,
  \begin{equation*}
    \forall \omega \in \Omega : \exists \delta_\omega > 0 :
    Q_{\delta_\omega}(\omega) \subseteq \Omega
  \end{equation*}
  Since $\phi$ is continuous on its domain,
  \begin{equation*}
    \forall x \in \Omega :
    \forall \epsilon > 0 :
    \exists \delta_\epsilon > 0:
    \forall x^* \in \Omega :
    x^* \in Q_{\delta_\epsilon}(x) \implies \phi(x^*) \in Q_\epsilon(\phi(x))
  \end{equation*}

  Take arbitrary $u \in U$.
  Then $\phi(u) > 0$ by definition,
  and $\exists \delta_u > 0 : Q_{\delta_u}(u) \subseteq \Omega$ because
  $u \in U \subseteq \Omega$ and $\Omega$ is open.
  We want to find $\delta_u^\prime > 0$ such that
  $Q_{\delta_u^\prime}(u) \subseteq U$.

  Let $\epsilon = \phi(u)$.
  By continuity of $\phi$,
  \begin{equation*}
    \exists \delta_\epsilon > 0 :
    \forall x^* \in U :
    x^* \in Q_{\delta_\epsilon}(u) \implies \phi(x^*) \in Q_\epsilon(\phi(u))
  \end{equation*}

  Consider $Q_{\delta_\epsilon}(u)$. Every point in it is such that when mapped
  by $\phi$ gives a value in $Q_\epsilon(\phi(u))$. This cube is the open
  interval with half-length $\epsilon = \phi(u)$, centered at $\phi(u)$, i.e.
  $(0, 2\phi(u))$. All points in this interval are greater than zero.
  Therefore, $Q_{\delta_\epsilon}(u) \subseteq U$,
  for any point $u \in U$, we can find a nonempty open cube centered at
  that point that is contained in $U$. Therefore, $U$ is an open set.
\end{proof}

\begin{prop}
  Suppose $C \subseteq \R^n$ is a closed set, and $\phi : C \to \R$ is a
  continuous function. Then, $K = \{ x \in C : \phi(x) \geq 0 \}$ is a closed
  set.
\end{prop}

\begin{proof}
  Take an arbitrary sequence $(x_n)_{n\in\N}$. Suppose that
  $\forall n \in N : x_n \in K$ and that $\lim_{n\to\infty} x_n = x$ exists.
  We want to show that $x \in K$, i.e. $\phi(x) \geq 0$.

  By the sequential criterion, we know that the sequence $(\phi(x_n))_{n\in\N}$
  is convergent, and that its limit is $\phi(x)$.

  Suppose that $\phi(x) < 0$. By the definition of convergence of a sequence,
  \begin{equation*}
    \forall \epsilon > 0 : \exists N_\epsilon : \forall n > N_\epsilon :
    \phi(x_n) \in Q_\epsilon(\phi(x))
  \end{equation*}
  we know that for any choice of size for a ``box'' around $\phi(x)$, there is
  a threshold beyond which all subsequent terms in the sequence $(\phi(x_n))$
  are inside that box. Since this is a one-dimensional box, it is simply an
  open interval. Instatiating $\epsilon$ with $\phi(x)$ gives us in particular
  that beyond some threshold, all subsequent $\phi(x_n)$ are in the interval
  $(-2\phi(x), 0)$, and therefore that they are negative. But each $x_n \in K$
  means that $\phi(x_n) \geq 0$, so this is a contradiction.

  Therefore, $\phi(x) \geq 0$, so $x \in K$.
\end{proof}

\section*{\#2 -- Pointy spheres in high dimensions}

\begin{prop}
  The function $f : \R^n \to \R$ defined by
  \begin{equation*}
    f(x) = \prod_{i=1}^n x_i^2
  \end{equation*}
  subject to the constraint
  \begin{equation*}
    \sum_{i=1}^n x_i^2 = 1
  \end{equation*}
  has maximum value $\parens{\frac{1}{n}}^n$.
\end{prop}

\begin{proof}
  Using the constraint, we can solve for $x_1^2$ in terms of the other
  components
  \begin{equation*}
    x_1^2 = 1 - \sum_{i=2}^n x_i^2
  \end{equation*}
  and substitute this solution into the definition of $f$
  \begin{equation*}
    f(x) = \parens{1 - \sum_{i=2}^n x_i^2} \prod_{i=2}^n x_i^2
  \end{equation*}

  Next, we would like to compute all the partial derivatives of $f$ in order to
  find the critical points. Let $j \in \{2, \ldots, n\}$.
  \begin{equation*}
    f(x) = \parens{1 - \sum_{i=2}^n x_i^2} \prod_{i=2}^n x_i^2
    = \parens{x_j^2 - x_j^4 - x_j^2 \sum_{\substack{i=2 \\ i \neq j}}^n x_i^2}
      \prod_{\substack{i=2 \\ i \neq j}}^n x_i^2
  \end{equation*}

  In this form, computing the $j$th partial derivative is easy.
  \begin{equation*}
    f_{x_j}(x)
    = 2 x_j
      \parens{1 - 2 x_j^2 - \sum_{\substack{i=2 \\ i \neq j}}^n x_i^2}
      \prod_{\substack{i=2 \\ i \neq j}}^n x_i^2
  \end{equation*}
  for $j \neq 1$, and $f_{x_1}(x) = 0$.

  To find the critical points, we must solve a system of $n - 1$ equations.
  Each equation will be of the form $f_{x_j}(x) = 0$.
  In each equation, setting to zero produces $n$ cases: any of the coordinates
  could be zero, or
  \begin{align*}
    1 - 2 x_j^2 - \sum_{\substack{i=2 \\ i \neq j}}^n x_i^2 & = 0 \\
    x_j^2 & = \frac{1}{2}\parens{1 - \sum_{\substack{i=2 \\ i \neq j}} x_i^2}
  \end{align*}

  Now we guess that each $x_i^2$ for $i \in \{2, \ldots, n\} \setminus \{j\}$
  is equal to $\frac{1}{n}$. Computing with this substitution gives us that
  $x_j^2 = \frac{1}{n}$ as well, so assigning $\frac{1}{n}$ to every $x_i^2$
  for $i \in \{2, \ldots, n\}$ is a consistent solution to the system, since
  $j$ is arbitrary.

  However, the system may have multiple solutions. To dispel this, we consider
  the $n-1 \times n-1$ matrix that represents the system: its entries are $2$
  along the main diagonal and $1$ everywhere else. Using row reduction on the
  determinant, we divide through each row by $2$, making $1$ along the diagonal
  and $\frac{1}{2}$ everywhere else: this alters the value of the determinant
  by a factor of $\frac{1}{2^n}$. Then we can add multiples of each row to
  eliminate all the halves, making the identity matrix: this does not alter the
  value of the determinant. Hence, the determinant is nonzero, and equal to
  $\frac{1}{2^n}$. Therefore, the matrix representing the system is invertible,
  and the system has a unique solution, which is precisely the one we found.

  Now we can compute the value of $x_1^2$ using the constraint, by substituting
  $\frac{1}{n}$ for each $x_i^2$ for $i \in \{2, \ldots, n\}$, and we find that
  $x_1 = \frac{1}{n}$ as well.
  Then,
  \begin{equation*}
    f\parens{\overbrace{\frac{1}{n}, \ldots, \frac{1}{n}}^\text{$n$ times}}
    = \prod_{i=1}^n \frac{1}{n} = \frac{1}{n^n}
  \end{equation*}

  Notice that all other critical points have at least one component equal to
  zero, so the function value is zero for all the critical points except those
  in which each $x_i^2 = \frac{1}{2}$. Hence, the maximum of the function is
  $\frac{1}{n^n}$.
\end{proof}

Furthermore, notice that $\lim_{n\to\infty} \frac{1}{n^n} = 0$. As the number
of dimensions increases, the maximum value of $f$ tends to zero. This is
another instance of the very interesting phenomenon that spheres in high
dimensions are in some sense ``pointy''.

\section*{\#4 -- Maximizing the area of a triangle}

\begin{prop}
  Of all triangles with a fixed perimeter $p$, the one with the greatest area
  is equilateral.
\end{prop}

\begin{proof}
  We can phrase this problem as a constrained optimization problem.
  \begin{align*}
    \text{maximize} &\quad \sqrt{s(s-x)(s-y)(s-z)} \\
    \text{subject to} &\quad x + y + z = p
  \end{align*}
  which can be solved straightforwardly using the method of Lagrange
  multipliers. The system of equations to solve becomes
  \begin{align}
    (s-x)(s-y)(s-z) - s(s-y)(s-z) + s(s-x)(s-z) + s(s-x)(s-y) & = 2 \lambda
    \label{eq:partderivx} \\
    (s-x)(s-y)(s-z) + s(s-y)(s-z) - s(s-x)(s-z) + s(s-x)(s-y) & = 2 \lambda
    \label{eq:partderivy} \\
    (s-x)(s-y)(s-z) + s(s-y)(s-z) + s(s-x)(s-z) - s(s-x)(s-y) & = 2 \lambda
    \label{eq:partderivz} \\
    x + y + z & = p
    \label{eq:perimeter}
  \end{align}
  Subtracting \eqref{eq:partderivy} from \eqref{eq:partderivz} gives $y = z$.
  Similarly, subtracting \eqref{eq:partderivx} from \eqref{eq:partderivy} gives
  $x = y$. Hence $x = y = z$ and \eqref{eq:perimeter} imply that each length is
  equal to a third of the permeter, and that the triangle is equilateral.
\end{proof}

\end{document}
