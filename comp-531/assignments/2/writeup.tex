\documentclass[11pt,letterpaper]{article}

\usepackage[margin=2.0cm]{geometry}
\usepackage{amsmath,amssymb,amsthm}

\newtheorem{proposition}{Proposition}

\DeclareMathOperator{\vdeg}{deg}
\newcommand{\degb}[1]{\vdeg_b{#1}}
\newcommand{\Z}{\mathbb{Z}}

\begin{document}

\section{Log-space reductions}

\begin{proposition}
    If both $f$ and $g$ are log-space computable functions, then $f \circ g$ is
    also a log-space computable function.
\end{proposition}

\begin{proof}
    Since $f$ and $g$ are computable, we have Turing machines $M_f$ and $M_g$
    that implement them with the appropriate space complexities. We will
    construct a Turing machine $M$ of logarithmic space complexity implementing
    $f \circ g$.

    The na\"ive strategy for constructing $M$ is to compute $f(w)$ by
    simulating $M_f$, and then feed $f(w)$ as input to a simulation of $M_g$.
    This however does not work since the output of $M_f$ might have length
    polynomial in $w$! It suffices to recognize that we can compute select
    elements of the output of $M_f$ in an on-demand fashion.

    In $M$, we represent the position of the $M_g$ simulation's read head with
    a counter: when the simulation would move the read head to the right, the
    counter is decremented; when the simulation would move the read head to the
    left, the counter is decremented. When the simulation of $M_g$ performs a
    read, $M$ performs a simulation of $M_f$ but ``jams'' the simulated write
    head, so to speak. Consequently, letters that are written by the $M_f$
    simulation overwrite each other. Each time an $M_f$ write is simulated, a
    copy of the counter is decremented; this allows $M$ to keep track of how
    many letters have been written. When this counter reaches zero, then the
    letter under the write head of the $M_f$ simulation can be used as the
    letter under the read head of the $M_g$ simulation.

    The nonconstant extra space required by this procedure is due to the
    counters. There is one primary counter used to represent the read head of
    the $M_g$ simulation, and one copy used when a new letter needs to be
    computed. The size of this counter is logarithmically bounded by the size
    of $f(w)$, which is at most polynomial in $w$. Hence overall, the
    nonconstant size is logarithmically bounded in the size of $w$ as required.
\end{proof}

\end{document}
