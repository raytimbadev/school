\documentclass[11pt,letterpaper]{article}

\author{Jacob Thomas Errington (260636023)}
\title{Assignment \#1\\Advanced theory of computation -- COMP 531}
\date{27 January 2016}

\usepackage[margin=2.0cm]{geometry}

\usepackage{amsmath,amssymb,amsthm}

\newtheorem{proposition}{Proposition}

\begin{document}

\maketitle

\section*{Question 1}

\begin{proposition}
    If $L$ is recognized by a Turing machine $M$ that does not necessarily halt
    but such that $s_M(n) \in O(S(n))$ for $S(n) \geq \log n$, then $L$ can be
    recognized by a Turing machine $M^\prime$ such that
    $s_{M^\prime}(n) = S(n)$ and that always halts.
\end{proposition}

\begin{proof}
    Our strategy is to construct $M^\prime$ so as to simulate $M$ \emph{twice},
    which will allow us to detect space-bounded loops.

    The intuition for this solution comes from a technique to find loops in
    linked lists. We hold onto two cursors in the list. One cursor advances by
    one item, and the other advances by two items. Every iteration, the items
    referred to by these cursors are compared: if they are equal then there is
    a loop. If there is no loop, then one of the cursors will reach the end of
    the list.

    When it comes to Turing machines, we have a graph rather than a list.
    However, we still have this notion of advancing one step at a time
    according to the transition function of the machine, so this algorithm
    still applies. In one of our simulations, we apply the transition function
    twice, and in the other we apply it once. The outcome is that on one tape
    of our machine $M^\prime$ we can trace the evolution of $M$ one step at a
    time, and on another tape of $M^\prime$ we can trace the evolution of $M$
    two steps at a time. It suffices to compare the configuration in one tape
    with the configuration in the other in order to detect loops, in which case
    $M^\prime$ will reject. If one simulation or the other reaches an accepting
    state then $M^\prime$ accepts; else if one or the other simulations reaches
    a rejecting state then $M^\prime$ rejects.

    As for space complexity, $M^\prime$ will use twice as much space as $M$, as
    there are two simulations, as well as some overhead in order to perform the
    comparison of the two work tapes. That overhead will of course be no more
    than $s_M(n)$.
\end{proof}

\section*{Question 2}

\begin{proposition}
    The language $L = \{ a^n b^n | n \geq 0\}$ can be recognized by a
    single-tape Turing machine in time $O(n \log n)$.
\end{proposition}

\begin{proof}
    First without loss of generality we will assume that all inputs are of the
    form $a^* b^*$, which can be verified by a deterministic finite automaton
    in time $O(n)$.

    Our idea is to use a counter. Leaving the counter at a fixed location on
    the tape, however, would result in $O(n^2)$ time complexity due to needing
    to seek from the input to the counter and back for each input character.

    Our insight is to drag the counter along with the input head in order to
    minimize seeking distances.

    Our strategy is to use an extended alphabet
    $\Gamma = \{a, b, a_0, a_1, b_0, b_1\}$.
    The subscripts of sequential characters on the tape will be used to
    represent the value of the counter.

    Incrementing or decrementing the value of the counter that is immediately
    under the cursor takes time $O(\log n)$. Moving the counter over by one
    takes time $O(\log n)$. We must perform these operations $2n$ times. Hence
    overall, this algorithm is $O(n \log n)$.
\end{proof}

Now we will show that for a single-tape Turing machine, this time bound is
optimal for recognizing $L$. To do so, we notice that $L$ is not regular (by a
simple application of the pumping lemma), and we will prove that all
single-tape Turing machines with time complexity $o(n \log n)$ are equivalent
to deterministic finite automata.

I have found some proofs of this, but I am unable to understand them, so I will
leave this part unsolved.

\section*{Question 3}

\begin{proposition}
    If $L$ is a language that can be recognized by a Turing machine in space
    $s_M(n) \in o(\log \log n)$, then L is regular.
\end{proposition}

\begin{proof}
    Suppose not. Then $L$ is a nonregular language that can be recognized by a
    Turing machine $M$ in space asymptotically strictly less than
    $\log \log n$. Of course $M \notin DSPACE(1)$, so for any
    $k \in \mathbb{N}$, there exists some word $x \in L$ requiring more than
    $k$ space to recognize.

    Now suppose $x$ is an input of minimal size, let $n = |x|$. Let $C$ be the
    set of all configurations of $M$ processing $x$. Since
    $M \in DSPACE(s(n))$,
    $$
    |C| \leq 2^{c s(n)} \in o(\log n)
    $$
    where $c$ is a constant that depends on $M$ to account for differences in
    alphabet size.

    Let $S$ be the set of all possible crossing sequences of $M$ processing
    $x$. Note that the length of any crossing sequence cannot exceed $|C|$. (If
    there were a longer crossing sequence, then there would be a repeated
    configuration, which would cause the machine to run forever.) There are at
    most $|C|$ possibilities for each element of any crossing sequence (we
    consider crossing sequences to consist of \emph{configurations}), so the
    number of crossing sequences of $M$ processing $x$ is
    $$
    |S| \leq |C|^{|C|} = {2^{c s(n)}}^{2^{c s(n)}} = 2^{c s(n) 2^{c s(n)}}
    < 2^{2^{2 c s(n)}} = 2^{2^{o(\log \log n)}} = o(n)
    $$

    Then, due to the pigeonhole principle, there will exist indices $i$ and $j$
    where $i < j$ such that $C_i = C_j$ where $C_k$ denotes the crossing
    sequence at position $k$.

    Finally, since those crossing sequences are the same, we can create a new
    string $x^\prime$ by snipping out the cells from $i$ to $j$ while
    preserving the behaviour of $M$. This new string is smaller than $x$, which
    contradicts the minimality of $x$. Thus, this language $L$ cannot exist.
\end{proof}

\section*{Question 4}

\begin{proposition}
    The language $BIN$ consisting of strings of the form
    $$
    b_1 \# b_2 \# \cdots \# b_n
    $$
    where $b_i$ is the binary encoding of the number $i$ can be recognized in
    $DSPACE(\log \log n)$.
\end{proposition}

\begin{proof}
    The Turing machine needs merely to check that for each $b_i$, the next
    binary string is equal to $i + 1$. This can be done by successively
    comparing digits of the binary encoded string. The Turing machine must keep
    track only of which digit is has reached. The number of digits in the
    binary encoding is on the order of $\log n$, so the counter to keep track
    of which digit is currently being processed will be on the order of
    $\log \log n$.
\end{proof}

\begin{proposition}
    The language $BIN$ is not regular.
\end{proposition}

We will not provide a rigorous argument. We know that a deterministic finite
automaton cannot perform an equality check of two arbitrary strings; what is
being done by the Turing machine in our construction is slightly more complex
than that, so it stands to reason that a DFA should be incapable of it as well.

\end{document}
