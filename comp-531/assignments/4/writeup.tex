\documentclass[letterpaper,11pt]{article}

\author{Jacob Thomas Errington (260636023)}
\title{Assignment \#4\\Advanced theory of computation -- COMP 531}
\date{30 March 2016}

\usepackage[margin=2.0cm]{geometry}
\usepackage{amsmath,amssymb,amsthm}

\usepackage{algorithm,algorithmicx,algpseudocode}

\newtheorem{prop}{Proposition}
\newtheorem{thm}{Theorem}

\DeclareMathOperator{\PrOp}{Pr}
\DeclareMathOperator{\poly}{poly}
\renewcommand{\Pr}[1]{\PrOp{\left[\text{#1}\right]}}
\newcommand{\Z}{\mathbb{Z}}

\begin{document}

\maketitle

\section{Graph matchings and the probabilistic method}

\section{Upper bound on random bits}

\section{Communication complexity of disjointness}

\section{Cliques and independent sets}

A clique and an independent set either intersect at a single vertex or are
disjoint; this is a special case of the disjointness problem.

We propose the protocol given in algorithm $\ref{alg:clique-indep}$, in which
we name the players $\#1$ and $\#2$ and we suppose $\#1$ has the clique $C$ and
$\#2$ has the independent set $I$.

\begin{algorithm}
    \caption{Computes $|C \cap I|$}
    \begin{algorithmic}
        \If{$\exists v \in C$ such that $\deg v < \frac{n}{2}$}
            \State $\#1$ sends the name of such a $v$ to $\#2$
            \State $\#2$ sends whether $v \in I$
            \If{$v \in I$}
                \State the protocol ends, concluding $|C \cap I| = 0$
            \Else
                \State recurse on the subgraph determined by $v$ and its
                neighbours
            \EndIf
        \Else
            \State $\#1$ sends an empty message to $\#2$
        \EndIf

        \If{$\exists v \in I$ such that $\deg v \geq \frac{n}{2}$}
            \State $\#2$ sends the name of such a $v$ to $\#1$
            \State $\#1$ sends whether $v \in C$
            \If{$v \in C$}
                \State the protocol ends, concluding $|C \cap I| > 0$
            \Else
                \State recurse on the subgraph determined by $v$ and its
                non-neighbours
            \EndIf
        \Else
            $\#2$ ends the protocol and concludes that $|C \cap I| = 0$
        \EndIf
    \end{algorithmic}
    \label{alg:clique-indep}
\end{algorithm}

Each iteration uses $O(\log n)$ bits and there are $O(\log n)$ iterations due
to the cutting-by-half nature of the recursion, so the overall cost is
$O(\log^2 n)$.

\section{Randomized discrete logarithms}

\begin{prop}
    If an algorithm $A$ solves the discrete logarithm problem for
    $\frac{1}{\poly{n}}$ of the elements of $\Z_p^*$ for a prime $p$ in time
    $O(\poly n)$ where $n$ is the length of $p$, then there exists a randomized
    polytime algorithm that solves the discrete logarithm problem on all
    elements of the finite field with high probability.
\end{prop}

\begin{proof} By randomized self-reduction.

    Intuitively, if we pick an element $y \in \Z_p^*$ at random and give it to
    $A$, then $\Pr{$A$ finds $\log_g y$} = \frac{1}{\poly n}$. It
    suffices to find a way to turn a given $y$ into a random instance
    distributed uniformly over $\Z_p^*$.

    If $0 \leq t < p$ is chosen uniformly at random and $g^x = y$, then
    $g^{t+x} \equiv yg^t$ is independent of $y$ and also uniformly distributed
    over $\Z_p^*$. Thus,
    \begin{align*}
        \log_g y
            &\equiv \log_g{g^{x+t}} - t \\
            &\equiv \log_g{yg^t} - t
    \end{align*}
    so the problem is reduced to finding $\log_g{yg^t}$, which can now be done
    with probability $\frac{1}{\poly n}$.
\end{proof}

\end{document}
