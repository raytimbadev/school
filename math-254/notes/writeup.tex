\documentclass[letterpaper,11pt]{article}

\author{Jacob Thomas Errington}
\title{Notes\\Honours analysis 1 -- MATH 254}
\date{}

\usepackage[margin=2.0cm]{geometry}
\usepackage{amsmath,amssymb,amsthm}

\newtheorem{prop}{Proposition}
\newtheorem{thm}{Theorem}
\newtheorem{lem}{Lemma}
\newtheorem{defn}{Definition}
\newtheorem{axm}{Axiom}
\newtheorem{rem}{Remark}

\newcommand{\makebb}[1]{
  \expandafter\newcommand\csname #1\endcsname{\mathbb{#1}}
}
\makebb{R}
\makebb{N}

\newcommand{\union}{\cup}
\newcommand{\Union}{\bigcup}
\newcommand{\intersn}{\cap}
\newcommand{\Intersn}{\bigcap}

\begin{document}

\maketitle

\section{Properties of real numbers}

\subsection{Supremum and infimum}

\begin{axm}
  The set of real numbers $\R$ is a complete ordered field.
\end{axm}

We already know $\R$ is an ordered field (which we saw as the definition of
$\R$). Completeness is also taken to be an axiom in this course. Concretely,
what this means is that for any set $A \subset \R$ that is bounded above, there
exists a \emph{supremum} $\sup A = u$ that is the least upper bound of $A$,
i.e. for any upper bound $M$ of $A$, $u \leq M$.

\begin{rem}
  Suppose that $u = \sup A$ exists for an arbitrary set $A$. By definition, $u$
  is an upper bound of $A$, so $A$ is bounded above. This is the converse
  property of the completeness axiom.
\end{rem}

\begin{thm}
  For any set $A \subset \R$ that is bounded below, there exists a greatest
  lower bound $u \in \R$ of $A$, i.e. for any lower bound $M$ of $A$, $M \leq
  u$.
  \label{thm:exists-infimum}
\end{thm}

\begin{proof}
  For any set $A \subseteq \R$, define $-A = \{ -x | x \in A \}$. Let $M$ be a
  lower bound of $A$. Then $M$ is an upper bound of $-A$. Hence, $-A$ is
  bounded above, and by the completeness of $\R$, there exists $\sup(-A)$ which
  is its least upper bound. Let $u = \sup(-A)$.

  Take $y \in -A$. Then by definition, $y = -x$ for some $x \in A$. By
  definition of the supremum, $u > y = -x$, so $-u < x$. Since $x$ is
  arbitrary, we have that $-u$ is a lower bound of $A$.

  Suppose there is a lower bound $v$ of $A$ such that $v > -u$. Then for any
  $x \in A$, $v < x$ and $u < x$. $-v$ is an upper bound of $-A$ by reasoning
  just like above. But then $-v < u$ is an upper bound of $-A$, which
  contradicts the definition $u = \sup(-A)$ as being the least upper bound on
  $-A$.
\end{proof}

\begin{defn}[Infimum]
  The greatest lower bound of a set $A$ that is bounded below is called the
  \emph{infimum}, denoted $\inf A$.
\end{defn}

\begin{rem}
  The infimum is guaranteed by theorem \ref{thm:exists-infimum}. Furthermore,
  the proof of the theorem shows a useful relation between the infimum and the
  supremum
  \begin{equation}
    \inf A = - \sup(-A)
    \label{eq:inf-sup-neg}
  \end{equation}
\end{rem}

\begin{defn}[Maximum]
  Suppose $A \subseteq \R$ is a set of reals.
  The \emph{maximum} of $A$, denoted $\max A$, is a maximal element of $A$ if
  one exists, i.e. $M = \max A \in A$ is such that for all $x \in A$,
  $x \leq M$.
\end{defn}

There is a connection between the maximal elements of a set of reals and the
supremum of the set. Before exploring this, we will show that some sets of
reals do not have a maximal element.

\begin{prop}
  The interval $A = [0, 1)$ does not have a maximal element.
\end{prop}

\begin{proof}
  Suppose there exists a maximal element $M \in A$.
  Consider $x = \frac{M + 1}{2}$.
  Since $M < 1$, $M + 1 < 2$ so $x < 1$.
  Obviously $x > 0$.
  Hence $x \in A$.
  But then, consider this derivation.
  \begin{align*}
    M &< 1 \\
    2M &< M + 1 \\
    M &< \frac{M + 1}{2}
  \end{align*}
  Hence, $M < x = \frac{M + 1}{2} < 1$, so there is an element in $A$ greater
  than the maximal element $M$, which is a contradiction.
\end{proof}

Now that we know that some sets do not have maximal elements, we can explore
the connection between the supremum and the maximum.

\begin{thm}
  A set of reals $A \subseteq \R$ has a maximal element if and only if
  $\sup A \in A$.

  If $\max A$ exists, then $\max A = \sup A$.
\end{thm}

\begin{proof}
  Suppose $\sup A \in A$. Suppose further that $A$ has no maximum, i.e. that
  given any element $x \in A$, there exists $y \in A$ such that $x < y$. Then
  there exists in particular some $y \in A$ such that $\sup A < y$. This
  contradicts the fact that $\sup A$ is an upper bound.

  Suppose $\max A$ exists. By definition, $\max A \in A$ and
  $\forall x \in A$, $x \leq \max A$. Hence, $\max A$ is an upper bound of $A$,
  so $A$ is bounded above and the supremum exists by completeness. Next suppose
  there exists $M \in A$ that is an upper bound for $A$ where $M < \max A$.
  But this contradiction the definition of $\max A$ as the maximum.
  Hence $\max A$ is the least upper bound of $A$, so $\max A = \sup A$.
\end{proof}

We can relate the infimum to the supremum in another way: they must respect a
certain ordering. Intuitively, this idea is that the greatest lower bound must
be less than or equal to the least upper bound.

\begin{thm}
  Suppose $A \subseteq \R$, and $\inf A$ and $\sup A$ exist.
  Then,
  \begin{enumerate}
    \item $\inf A > \sup A$ implies that $A = \emptyset$.
    \item $\inf A = \sup A$ if and only if $A$ is a singleton set.
    \item $\inf A < \sup A$ if and only if $A$ contains at least two (distinct)
      elements.
  \end{enumerate}
\end{thm}

\begin{proof}
  Suppose $\inf A > \sup A$.
  By definition, $\forall x \in A: x \leq \sup A$
  and $\forall x \in A: \inf A \leq x$.
  Thus, for all $x \in A$, we have $x \leq \sup A < \inf A \leq x$ which means
  $x < x$. Therefore, there must be no $x \in A$.

  Suppose $\inf A = \sup A$.
  By definition, $\forall x \in A: x \leq \sup A$
  and $\forall x \in A: \inf A \leq x$.
  Taking arbitrary $x \in A$, we establish
  \begin{equation*}
    \sup A = \inf A \leq x \leq \sup A = \inf A
  \end{equation*}
  so $x = \sup A = \inf A$. Since $x$ is arbitrary, we generalize that
  $\forall x \in A: x = \sup A = \inf A$, so
  $A = \{x = \sup A = \inf A\}$ is a singleton set.

  Suppose $A = \{x\}$ is a singleton set. Then
  $x = \sup A \geq y \forall y \in A$, by exhaustively checking this property
  for all one elements of $A$. The same goes for concluding that $\inf A = x$.
  Hence, by transitivity of equality, $\inf A = \sup A$.

  Suppose $A$ contains two distinct elements $x, y$. Assume without loss of
  generality that $x < y$. Then, $\inf A \leq x < y \leq \sup A$, so by
  transitivity, $\inf A < \sup A$.

  Suppose $\inf A < \sup A$. Suppose that $A$ does not have two distinct
  elements. Two cases arise.
  \begin{itemize}
    \item $A$ is a singleton, in which case $\inf A = \sup A$ is a
      contradiction.
    \item $A$ is the empty set. Then it is vacuously bounded above and below by
      any number, so $\inf A$ and $\sup A$ also have arbitrary values, and we
      can select them so that $\sup A < \inf A$, which is also a contradiction.
  \end{itemize}
\end{proof}

\begin{prop}
  If $A, B \subseteq \R$ have suprema, then
  \begin{equation*}
    A \subseteq B \implies \sup A \leq \sup B
  \end{equation*}
  \label{prop:subset-sup-leq}
\end{prop}

\begin{proof}
  Suppose $\sup B < \sup A$.
  Since $A \subseteq B$, then $\forall x: x \in A \implies x \in B$.
  By definition, $\forall x \in B: x \leq \sup B$.
  Then, $\forall x \in A: x \leq \sup B < \sup A$.
  So $\sup B$ is an upper bound of $A$, and is strictly less than $\sup A$.
  This contradicts that $\sup A$ is the least upper bound on $A$.
\end{proof}

\begin{prop}
  If $A, B \subseteq \R$ have suprema,
  then $\sup{A \union B} = \max\{\sup A, \sup B\}$
  exists.
\end{prop}

\begin{proof}
  Let $M = \max \{ \sup A, \sup B \}$.
  By definition, $\sup A \leq M$
  and $\sup B \leq M$.
  Take arbitrary $x \in A \union B$.
  \begin{description}
    \item[Case] $x \in A$.
      Then $x \leq \sup A$ by definition of the supremum,
      and $\sup A \leq M$ by definition of $M$,
      so $x \leq M$ by transitivity.
    \item[Case] $x \in B$. Then $x \leq M$ by the same reasoning.
  \end{description}
  Hence $M$ is an upper bound of $A \union B$.

  Suppose there exists an upper bound $u < M$ of $A \union B$.
  \begin{description}
    \item[Case] $M = \sup A$, so $u < \sup A$.
      Since $\forall x: x \in A \implies x \in A \union B$,
      we have that $\forall x \in A: x \leq u$, which contradicts that $\sup A$
      is the supremum of $A$.
      So this case is impossible.
    \item[Case] $M = \sup B$. Same reasoning, so this case is also impossible.
  \end{description}

  Therefore, $M = \max \{ \sup A, \sup B \} = \sup A \union B$ is the least
  upper bound of $A \union B$.
\end{proof}

The argument of this proof naturally lends itself to induction, but we have the
additional caveat that the union of the family of bounded sets must also be
bounded.  Obviously, a countable union of bounded sets is not always bounded:
consider the union $\Union_{n=0}^\infty [0-n, n+1]$. This union is equal to
$\R$, which is unbounded.

\begin{lem}
  Suppose $A \subseteq B \subseteq \R$, and $B$ is bounded. Then $A$ is
  bounded.
\end{lem}

\begin{proof}
  Since $B$ is bounded, there exist $L, U \in \R$ such that
  $\forall y \in B: L \leq y \leq U$.
  Take arbitrary $x \in A$.
  By assumption, $\forall y: y \in A \implies y \in B$, so $x \in B$.
  Hence $L \leq x \leq U$.
  This shows $\exists L, U \in \R: \forall x \in A: L \leq x \leq U$,
  so $A$ is bounded.
\end{proof}

Let $L_X, U_X \in \R$ represent lower and upper bounds of a bounded set $X$.

\begin{lem}
  Suppose $A, B \subseteq \R$ are bounded,
  then $A \intersn B$ is bounded and
  \begin{align*}
    L_{A \intersn B} &= \max\{L_A, L_B\} \\
    U_{A \intersn B} &= \min\{U_A, U_B\}
  \end{align*}
\end{lem}

\begin{proof}
  Since $A \intersn B \subseteq A$ and $A \intersn B \subseteq B$,
  then by applying the previous lemma twice, we have that for any
  $x \in A \intersn B$,
  \begin{align*}
    L_A \leq &x \leq U_A \\
    L_B \leq &x \leq U_A
  \end{align*}
  Let $M_L = \max \{ L_A, L_B \}$ and $M_U = \min \{ U_A, U_B \}$.
  Then we have $\forall x \in A \intersn B: M_L \leq x \leq M_U$, so the
  intersection is bounded, and the lower and upper bounds are the maximum and
  minimum of the intersected sets.
  Notice that this reasoning applies even if $A \intersn B$ is empty, in that
  case it is trivially bounded by any values.
\end{proof}

This lemma can in fact be considered two lemmas: one for sets that are bounded
above and one for sets that are bounded below. We will consider it as such.

\begin{prop}
  If $A, B \subseteq \R$ have suprema and if $A \intersn B$ is nonempty,
  then $\sup \{ A \intersn B \} \leq \min \{ \sup A, \sup B \}$.
\end{prop}

\begin{proof}
  By the previous lemma, be get that
  $\min \{ U_A, U_B \}$ is an upper bound of $A \intersn B$.
  Since suprema are upper bounds, we specialize the lemma to obtain that
  $\min \{ \sup A, \sup B \}$ is an upper bound of $A \intersn B$.
  Since $A \intersn B$ is bounded, by the lemma, its supremum must exist by
  completeness of $\R$.
  Since the supremum is the least upper bound, we have that
  \begin{equation*}
    \sup \{ A \intersn B \} \leq \min \{ \sup A, \sup B \}
  \end{equation*}
\end{proof}

\begin{prop}
  Suppose $A, B \subseteq \R$ are nonempty sets of reals. (We have no
  boundedness assumptions on $A$ and $B$.)
  If $\forall a \in A \forall b \in B: a \leq b$,
  then $\sup A$ and $\inf B$ exist and $\sup A \leq \inf B$.
\end{prop}

\begin{proof}
  Since $A, B$ are nonempty choose some $a \in A$ and $b \in B$ arbitrarily.
  By assumption, $\forall b \in B: a \leq b$, so $a$ is a lower bound of $B$.
  Similarly, $\forall a \in A: b \leq a$, so $b$ is an upper bound of $A$.
  Then by completeness, $\sup A$ and $\inf B$ must exist.
  Since $a$ is a lower bound of $B$, then by the definition of the infimum,
  $a \leq \inf B$.
  But then since $a$ is arbitrary, we have that
  $\forall a \in A:a \leq \inf B$ is an upper bound of $A$.
  By the definition of the supremum, we deduce that $\sup A \leq \inf B$.
\end{proof}

Now we can examine the notion of suprema and infima of functions.

\begin{defn}
  Let $A \subseteq X \subseteq \R$.
  We say that $f : X \to \R$ is \emph{bounded on $A$} if the image of $A$ under
  $f$ is bounded.

  We we simply say $f$ is bounded, then it is understood that we mean that it
  is bounded \emph{on its domain}.
\end{defn}

This definition is just like the earlier lemma: it is really two definitions,
one for \emph{bounded above} and one for \emph{bounded below}.

\begin{defn}
  The supremum of a function $f : X \to \R$ (on $A$), denoted $\sup_A f$ is
  \begin{equation*}
    \sup_A f = \sup f(A)
  \end{equation*}

  Again, omitting $A$ is to be understood as taking the supremum of the image
  of $f$.
\end{defn}

\begin{prop}
  Suppose $\sup f$ exists. Then $\sup f|_{X^\prime \subseteq X} \leq \sup f$.
\end{prop}

\begin{proof}
  For any $f : X \to \R$ and any $A \subseteq X$, if $A^\prime \subseteq A$,
  then $f(A^\prime) \subseteq f(A)$. This follows easily from functional
  dependency.
  The proof is complete by invoking proposition \ref{prop:subset-sup-leq}.
\end{proof}

\section{Dense and open sets}

\begin{defn}[Open set]
  A subset $U \subseteq \R$ is \emph{open} if
  \begin{equation*}
    \forall a \in U: \exists r \in \R: (a - r, a + r) \subseteq U
  \end{equation*}
\end{defn}

\begin{defn}[Dense set]
  A subset $U \subseteq \R$ is \emph{dense} in $\R$ if for any $x, y \in \R$,
  if $x < y$, then
  \begin{equation*}
    \exists a \in U: x < a < y
  \end{equation*}
\end{defn}

\begin{prop}
  Suppose $U \subseteq \R$ is dense.
  Then $U$ is nonempty and if $U \subseteq V \subseteq \R$ then $V$ is dense.
\end{prop}

\begin{proof}
  Take arbitrary reals $x, y$. Then by density $\exists a \in U$ such that
  $x < a < y$, so $U$ is nonempty.

  Suppose $U \subseteq V \subseteq \R$.
  Take arbitrary $x, y \in \R$.
  Since $U$ is dense, then $\exists a \in U: x < a < y$.
  Since $U \subseteq V$, $a \in U \implies a \in V$.
  Therefore, $\exists a \in V: x < a < y$.
  By definition, $V$ is dense.
\end{proof}

\begin{lem}
  Suppose $\exists x, y \in R: x < y$ and $U \subseteq \R$ is open and dense.
  Then, $\exists x^\prime, y^\prime \in \R$ such that if $x^\prime < y^\prime$,
  then
  \begin{equation*}
    [x^\prime, y^\prime] \subseteq U \intersn (x, y)
  \end{equation*}
  \label{lem:open-dense-interval}
\end{lem}

\begin{proof}
  Since $U$ is dense, $\exists z \in U: x < z < y$.
  Since $U$ is open, $\exists r \in \R: (z-r, z+r) \subseteq U$.
  Since $U$ is dense, there exist reals $x^\prime$ and $y^\prime$
  such that $\max \{z-r, x\} < x^\prime < z$
  and $z < y^\prime < \max \{r, z+r\}$.
  Since $z - r < x^\prime$ and $y^\prime < z + r$,
  $[x^\prime, y^\prime] \subseteq (z - r, z + r) \subseteq U$.
  Since $\max \{z - r, x\} < x^\prime$, $x < x^\prime$.
  Similarly, $y^\prime > y$. Hence $[x^\prime, y^\prime] \subseteq (x, y)$.
  Therefore, $[x^\prime, y^\prime] \subseteq U \intersn (x, y)$.
\end{proof}

\begin{prop}
  Suppose $x_0, y_0 \in \R$ such that $x_0 < y_0$, and
  $\forall n \in \N: \exists U_n \subseteq \R$ such that $U_n$ is open and
  dense.
  Then,
  \begin{equation*}
    \forall n \in \N: \exists x_{n+1}, y_{n+1} \in \R:
    [x_{n+1}, y_{n+1}] \subseteq U_n \intersn (x_n, y_n)
  \end{equation*}
\end{prop}

\begin{proof}
  By induction.
  \begin{description}
    \item[Base case.]
      We have that $U_0$ is open and dense.
      Then by lemma
      \ref{lem:open-dense-interval}
      that $\exists x^\prime, y^\prime : x^\prime < y^\prime$
      and
      \begin{equation*}
        [x^\prime, y^\prime] \subseteq U_1 \intersn (x_0, y_0)
      \end{equation*}
      Taking $x_1 = x^\prime$ and $y_1 = y^\prime$ completes the base case.

    \item[Step case.]
      The induction hypothesis is
      \begin{equation*}
        \exists x_n, y_n \in \R: x_n < y_n
        \land
        \exists x_{n+1}, y_{n+1} \in \R: x_{n+1} < y_{n+1}
        \land
        [x_{n+1}, y_{n+1}] \subseteq U_n \intersn (x_n, y_n)
      \end{equation*}

      Applying lemma \ref{lem:open-dense-interval} for $x_{n+1}$, $y_{n+1}$,
      and $U_{n+1}$ (which is dense and open by assumption) gives us
      \begin{equation*}
        \exists x^\prime, y^\prime \in \R: x^\prime < y^\prime
        \land
        [x^\prime, y^\prime] \subseteq U_{n+1} \intersn (x_{n+1}, y_{n+1})
      \end{equation*}
      Taking $x_{n+2} = x^\prime$ and $y_{n+2} = y^\prime$ completes the step
      case.
  \end{description}
\end{proof}

\end{document}
