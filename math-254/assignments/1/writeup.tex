\documentclass[11pt,letterpaper]{article}

\usepackage[margin=2.0cm]{geometry}

\usepackage{amsmath,amsthm,amssymb}

\newtheorem{prop}{Proposition}

\author{Jacob Thomas Errington (260636023)}
\title{Assignment \#1\\Honours Analysis 1 -- MATH 254}
\date{Tuesday, 20 September 2016}

\begin{document}
    \maketitle

\section*{\#1 -- Closed form by induction}

\begin{prop}
    The following is a closed form for the series.

    \begin{equation*}
        \sum_{i=1}^n \frac{i}{(i+1)!} = \frac{(n+1)! - 1}{(n+1)!}
    \end{equation*}
\end{prop}

\begin{proof}
    By induction.

    \begin{description}
        \item[Base case.]
            Have $n = 1$.

            \begin{equation*}
                \sum_{i=1}^1 \frac{i}{(i+1)!}
                    = \frac{1}{2!} = \frac{2! - 1}{2!}
            \end{equation*}
            as required.

        \item[Step case.]
            Have $n = k$ and want to show the proposition for $n = k+1$.

            The induction hypothesis is
            \begin{equation*}
                \sum_{i=1}^k \frac{i}{(i+1)!} = \frac{(k+1)! - 1}{(k+1)!}
            \end{equation*}

            We want to show
            \begin{equation*}
                \sum_{i=1}^{k+1} \frac{i}{(i+1)!} = \frac{(k+2)! - 1}{(k+2)!}
            \end{equation*}

            So
            \begin{align*}
                  & \sum_{i=1}^{k+1} \frac{i}{(i+1)!} \\
                = & \sum_{i=1}^k \frac{i}{(i+1)!} + \frac{k+1}{(k+2)!} \\
                = & \frac{(k+1)! - 1}{(k+1)!} + \frac{k+1}{(k+2)!} \\
                = & \frac{(k+2)((k+1)! - 1) + (k + 1)}{(k+2)!} \\
                = & \frac{(k+2)! - k - 2 + k + 1}{(k+2)!} \\
                = & \frac{(k+2)! - 1}{(k+2)!}
            \end{align*}
            as required.
    \end{description}
\end{proof}

\section*{\#3 -- Minimal element of a family of sets}

\begin{prop}
    Let $n \in \mathbb{N}$ such that $n > 1$ and let
    $\mathcal{A} = \{A_1, \ldots, A_n\}$ be pairwise distinct sets.
    There exists some minimal set $A \in \mathcal{A}$, i.e. $A$ is a superset
    of no other set in $\mathcal{A}$.
\end{prop}

\begin{proof}
    By induction.

    \begin{description}
        \item[Base case.]
            Have $n = 2$.

            We know that $A_1 \neq A_2$. Three cases arise.

            \begin{enumerate}
                \item $A_1 \subseteq A_2$. Then $A_1$ is minimal.
                \item $A_2 \subseteq A_1$. Then $A_2$ is minimal.
                \item The sets do not contain each other.
                    Then both are minimal.
            \end{enumerate}

            A minimal set has been found, so this completes the base case.

        \item[Step case.]
            Have $n = k$.

            The induction hypothesis is that the family of sets
            $\mathcal{A_k} = \{A_1, \ldots, A_k\}$ has some $A_i$ that is
            minimal.

            We wish to add $A_{k+1}$ to the collection. Three cases arise.

            \begin{enumerate}
                \item $A_i \subseteq A_{k+1}$.
                    Then $A_i$ remains the minimal set so far.
                \item $A_{k+1} \subseteq A_i$.
                    Then $A_{k+1}$ becomes the minimal set.
                \item $A_{k+1}$ and $A_i$ are incomparable.
                    Then $A_i$ remains the minimal set so far. $A_{k+1}$ may in
                    fact also be minimal, but we choose to simply stick with
                    $A_i$.
            \end{enumerate}
    \end{description}
\end{proof}

\section*{\#5 -- A characterization of bijective functions}

\begin{prop}
    $f : A \to B$ is bijective if, and only if,
    \begin{equation*}
        \forall G \in \mathcal{P}(A), f(A \setminus G) = B \setminus f(G)
    \end{equation*}
\end{prop}

\begin{proof}
    First, we will show the forwards direction: suppose $f$ is bijective. Now
    we proceed by contradiction: assume there exists $G \in \mathcal{P}(A)$
    such that $f(A \setminus G) \neq B \setminus f(g)$. Two cases arise.

    \begin{enumerate}
        \item $\exists x \in f(A \setminus G)$ such that
            $x \notin B \setminus f(G)$.

            Due to the surjectivity of $f$,
            $\forall b \in B, \exists a \in A$ such that $f(a) = b$.
            Hence, $\exists y \in A$ such that $f(y) = x$.

            Thus, $f(y) \notin B \setminus f(G)$. We know that $f(y) \in B$, so
            in order for $f(y) \notin B \setminus f(G)$ to be true,
            $f(y) \in f(G)$. By the injectivity of $f$, we deduce that $y \in
            G$ is in the unique element in $A$ mapped to $x$ by $f$.  We also
            have that, $x = f(y) \in f(A \setminus G)$. Again by the
            injectivity of $f$, we deduce that $y \in A \setminus G$, which
            contradicts that $y \in G$.

        \item The other case, in which $\exists x \in B \setminus f(G)$
            such that $x \notin f(A \setminus G)$ is similar. Take $y$ such
            that $f(y) = x$ due to the surjectivity of $f$. In order to satisfy
            $f(y) \notin f(A \setminus G)$, we establish that
            $y \notin A \setminus G$ with the injectivity of $f$. Since we know
            that $y \in A$, we can deduce that $y \in G$ in order to maintain
            that $y \notin A \setminus G$. Consequently, $x = f(y) \in f(G)$.
            However, $f(y) = x \in B \setminus f(G)$, which is a contradiction.
    \end{enumerate}

    Next, we will show the backwards direction: suppose
    $\forall G \in \mathcal{P}(A): f(A \setminus B) = B \setminus f(G)$. We
    want to show that $f$ is bijective. To do so, we will show that with this
    assumption, we can show that $f$ is both injective and surjective.

    \begin{description}
        \item[Injectivity.] We will proceed by contradiction.
            Suppose $f(a) = f(b)$, but $a \neq b$.

            Take some arbitrary $G$ such that $a \in G$. Since $a \neq b$, we
            deduce $b \notin G$. Since $b \notin G$, then
            $b \in A \setminus G$, so
            $f(b) \in f(A \setminus G)$.
            Contrariwise, since $a \in G$,
            $a \notin A \setminus G$, so $a \notin f(A \setminus G$.
            However, this is a contradiction, since $f(a) = f(b)$.

            Thus, $f$ must be injective.

        \item[Surjectivity.] We will proceed again by contradiction.
            Suppose $\exists y \in B:\, \forall x \in A:\, f(x) \neq y$, i.e.
            there is an unmapped element in $B$. An equivalent phrasing is that
            there exists no $A^\prime \subseteq A$ such that $y \in
            f(A^\prime)$.

            Hence, $y \in B \setminus f(G)$ for any nonempty
            $G \in \mathcal{P}(A)$. (If there were some nonempty $G$ such that
            $y \in B \setminus f(G)$, then we could conclude that there is some
            $x \in G$ such that $f(x) = y$ which would contradict that $y$ is
            unmapped to.) However, $y \notin f(A \setminus G)$, which
            contradicts that $f(A \setminus G) = B \setminus f(G)$.

            Thus, $f$ must be surjective.
    \end{description}
\end{proof}

\end{document}
