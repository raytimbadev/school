\documentclass[11pt,letterpaper]{article}

\author{Jacob Thomas Errington (260636023)}
\title{Assignment \#4\\Honours analysis 1 -- MATH 254}
\date{1 November 2016}

\usepackage[margin=2.0cm]{geometry}
\usepackage{amsmath,amssymb,amsthm}

\newcommand{\makebb}[1]{
  \expandafter\newcommand\csname #1\endcsname{\mathbb{#1}}
}
\makebb{R}
\makebb{N}

\newcommand{\abs}[1]{\left|#1\right|}
\newcommand{\absdiff}[2]{\abs{#1 - #2}}

% Definition of convergence.
\newcommand{\convergence}[5]{
  \forall #1 > 0\,
  \exists #2 \in \N\,
  \forall #3 \geq #2:\,
  \absdiff{#4}{#5} < #1
}

% Convergence after instantiating the forall.
\newcommand{\convergenceinst}[5]{
  \exists #2 \in \N\,
  \forall #3 \geq #2:\,
  \absdiff{#4}{#5} < #1
}

\newtheorem{prop}{Proposition}
\newtheorem{lemma}{Lemma}

\begin{document}

\maketitle

\begin{prop}[Problem \#6]
  Suppose $(x_n)$ is a sequence in $\R$.
  Let $A$ be the set of all limits of converging subsequences of $(x_n)$.
  Let $(y_n)$ be a sequence in $\R$ such that $\forall n \in \N: y_n \in A$.
  If $(y_n)$ converges, then
  \begin{equation*}
    \lim_{n\to\infty} {y_n} \in A
  \end{equation*}
\end{prop}

\begin{proof}
  Suppose $(y_n)$ converges.

  Denote $\lim_{n\to\infty} {y_n} = y$.

  Then we want to show that $\exists (x^\prime_n) \subseteq (x_n)$ and
  $(x^\prime_n) \to x^\prime$ as $n \to \infty$ such that $x^\prime = y$.
  In other words, we must construct an explicit subsequence $(x^\prime_n)$ of
  $(x_n)$ that converges to $y$.

  Since each element of $(y_n)$ is the limit of a converging subsequence of
  $(x_n)$, then for each $n \in \N$, there exists a sequence
  $(y_n^{(k)})_{k \in \N}$ which we call the \emph{associated sequence} of
  $y_n$, which converges to $y_n$.

  Consider the sequence $(x^\prime_n)$ given by
  \begin{equation*}
    x^\prime_n = y_n^{(n)}
  \end{equation*}

  We claim that this sequence converges to $y$ as $n \to \infty$.

  Take arbitrary $\epsilon > 0$.

  From the convergence of $(y_n)$ and $y_n^{(k)}$, we know that
  \begin{align}
    \convergence{\epsilon}{M_1}{n}{y_n}{y} \label{eq:cvg-yn} \\
    \convergence{\epsilon}{M_2}{k}{y_n^{(k)}}{y_n} \label{eq:cvg-ynk}
  \end{align}
  and we want to show
  \begin{equation}
    \convergence{\epsilon}{M_3}{n}{y_n^{(n)}}{y} \label{eq:cvg-want}
  \end{equation}

  Notice that in the limit, we have
  \begin{equation}
    \abs{y_n^{(n)} - y}
    = \abs{y_n^{(n)} - y_n + y_n - y}
    \leq \absdiff{y_n^{(n)}}{y_n} + \absdiff{y_n}{y}
    \label{eq:tri-ineq-lim}
  \end{equation}
  and we control both of these quantities.

  Instantiating $\frac{\epsilon}{2}$ for $\epsilon$ in \eqref{eq:cvg-ynk},
  we obtain $\convergenceinst{\frac{\epsilon}{2}}{M_2}{k}{y_n^{(k)}}{y_n}$.
  If $n \geq M_2$, then we're fine, since we can instantiate $\forall k$ with
  $n$ to obtain $\abs{y_n^{(n)} - y_n} < \frac{\epsilon}{2}$.

  Instantiating $\frac{\epsilon}{2}$ for $\epsilon$ in \eqref{eq:cvg-yn},
  we obtain $\convergenceinst{\frac{\epsilon}{2}}{M_1}{n}{y_n}{y}$. There are
  no tricky considerations to be made here.

  Now we must select $M_3$, for the convergence criterion we want to show, in
  \eqref{eq:cvg-want}. Taking $M_3 = \sup\{M_1, M_2\}$, we guarantee that $n$
  in the forall will be greater than $M_2$, so we can -- broadly speaking --
  perform the required instantiation of $k$.

  Specifically, take arbitrary $n \geq M_3$. Then $n \geq M_1$ and
  $n \geq M_2$ by the definition of $M_3$.
  We know that
  \begin{equation*}
    \forall k \geq M_2:\, \absdiff{y_n^{(k)}}{y_n} < \frac{\epsilon}{2}
  \end{equation*}
  %
  Since $n \geq M_2$, we can instantiate $n$ for $k$ to obtain
  \begin{equation*}
    \absdiff{y_n^{(n)}}{y_n} < \frac{\epsilon}{2}
  \end{equation*}
  %
  Next, we know that
  \begin{equation*}
    \forall m \geq M_1:\, \absdiff{y_m}{y} < \frac{\epsilon}{2}
  \end{equation*}
  %
  Since $n \geq M_1$, we can instantiate $n$ for $m$ to obtain
  \begin{equation*}
    \absdiff{y_n}{y} < \frac{\epsilon}{2}
  \end{equation*}
  %
  Finally, by our derivation in \eqref{eq:tri-ineq-lim}, we have
  \begin{equation*}
    \absdiff{y_n^{(n)}}{y}
    \leq \frac{\epsilon}{2} + \frac{\epsilon}{2}
    = \epsilon
  \end{equation*}

  This shows $(x^\prime_n) = (y_n^{(n)}) \to y$ as $n \to \infty$. Since the
  limit of $(y_n)$ is equal to the limit of a subsequence of $(x_n)$, $y$
  satisfies the definition of $A$. Hence, $y \in A$. This completes the proof.
\end{proof}

\begin{prop}[Problem \#9]
  Suppose $(x_n)$ is a sequence in $\R$, and $(y_n)$ is a sequence given by
  \begin{equation*}
    y_n = \frac{x_1 + \cdots + x_n}{n}
  \end{equation*}
  %
  If $(x_n)$ is bounded, then
  \begin{enumerate}
    \item $(y_n)$ is bounded; and
    \item we have the following relation between the sequences' associated
      limits
      \begin{equation*}
        \limsup_{n\to\infty}{x_n}
        \leq
        \liminf_{n\to\infty}{y_n}
        \leq
        \limsup_{n\to\infty}{y_n}
        \leq
        \limsup_{n\to\infty}{x_n}
      \end{equation*}
  \end{enumerate}
\end{prop}

\begin{proof}
  Suppose $(x_n)$ is bounded.

  We have
  \begin{align}
    \exists M^\circ \in \R\, \forall n \in \N:\, x_n &\leq M^\circ
    \label{eq:xn-bounded-above} \\
    \exists M_\circ \in \R\, \forall n \in \N:\, M_\circ &\leq x_n
    \label{eq:xn-bounded-below}
  \end{align}

  Take arbitrary $n \in \N$. Notice that
  \begin{equation*}
    y_n
    = \frac{x_1 + \cdots + x_n}{n}
    \leq \frac{M^\circ + \cdots + M^\circ}{n}
    = \frac{n M^\circ}{n}
    = M^\circ
  \end{equation*}
  which \emph{does not depend on $n$}. Hence $(y_n)$ is bounded above by
  $M^\circ$.

  The same reasoning shows that $M_\circ$ is a lower bound of $(y_n)$.

  Hence, $(y_n)$ is bounded.

  Next, we want to show
  \begin{equation}
    \limsup_{n\to\infty}{y_n}
    \leq
    \limsup_{n\to\infty}{x_n}
  \end{equation}

  Take arbitrary $N \in \N$, let $n \geq N$, and consider the following
  derivation.
  \begin{align*}
    y_n
    &= \frac{1}{n} \sum_{i=1}^n {x_i} \\
    &= \frac{1}{n} \sum_{i=1}^N {x_i}
      + \frac{1}{n} \sum_{N+1}^n {x_i} \\
    &\leq \frac{1}{n} \sum_{i=1}^N {x_i}
      + \frac{n-N}{n} \sup \{x_i\}_{i=N+1}^n \\
    &\leq \frac{1}{n} \sum_{i=1}^N {x_i}
      + \sup \{x_i\}_{i=N+1}^\infty
  \end{align*}

  Essentially, we split the calculation of the average after $N$ terms, and
  replace the second part with its supremum.

  Next, we take the limit superior on both sides.
  \begin{equation*}
    \limsup_{n\to\infty} {y_n}
    \leq \limsup_{n\to\infty} {
      \frac{1}{n} \sum_{i=1}^N {x_i} + \sup \{x_i\}_{i=N+1}^\infty
    }
    = \sup \{x_i\}_{i=N+1}^\infty
  \end{equation*}
  since $\frac{1}{n} \sum_{i=1}^N {x_i} \to 0$ as $n \to \infty$, and the
  second term does not depend on $n$.

  Then since our choice of $N$ was arbitrary, this equation is in fact valid
  for all $N$, so we can rewrite the right-hand side to give
  \begin{equation*}
    \limsup_{n\to\infty} {y_n}
    \leq \lim{N\to\infty} \sup \{ x_n : n > N \}
    = \limsup_{n\to\infty} x_n
  \end{equation*}
  which is exactly the rightmost inequality we wanted to show.

  This will also be true if we negate all the terms in each sequence, giving us
  \begin{align*}
    \limsup_{n\to\infty}{-y_n} &\leq \limsup_{n\to\infty}{-x_n} \\
    -\liminf_{n\to\infty}{y_n} &\leq -\liminf_{n\to\infty}{x_n} \\
    \liminf_{n\to\infty}{x_n} &\leq \liminf_{n\to\infty}{y_n}
  \end{align*}
  which shows the leftmost inequality.

  All that remains is the middle inequality.

  Expanding a definition of $\limsup{}$ and $\liminf{}$ we get
  \begin{equation*}
    \lim_{N\to\infty} \inf \{ y_n : n > N \}
    \leq
    \lim_{N\to\infty} \sup \{ y_n : n > N \}
  \end{equation*}
  as the statement we want to prove.

  Consider the sequences
  \begin{align}
    \left( \inf \{ y_n : n > N \} \right)_{N \in \N} \label{eq:infterms} \\
    \left( \sup \{ y_n : n > N \} \right)_{N \in \N} \label{eq:supterms}
  \end{align}
  which underly the limit inferior and limit superior, respectively.

  Notice that since $(x_n)$ and $(y_n)$ are bounded, these sequences that
  underly their limits inferior and limits superior are monotone and bounded,
  and hence convergent. Hence all the limits inferior and limits superior we've
  been studying in this problem exist. (Probably should have said this sooner.)

  Anyway, notice that each term in \eqref{eq:infterms} is at most its
  corresponding term in \eqref{eq:supterms}, since we know that for any
  nonempty bounded $R \subseteq \R$, $\inf R \leq \sup R$, and we know that
  for any $N \in \N$, $\{ y_n : n > N \}$ is bounded since it is a subset of a
  bounded set.

  Next, by a property of limits seen in class, if for each $n$, $a_n \leq b_n$,
  then $\lim_{n\to\infty} a_n \leq \lim_{n\to\infty}$. This property is
  satisfied by the two sequences we're examining.

  Hence,
  \begin{align*}
    \lim_{N\to\infty} \inf \{ y_n : n > N \} &\leq
    \lim_{N\to\infty} \sup \{ y_n : n > N \} \\
    \liminf_{n\to\infty} y_n &\leq \limsup_{n\to\infty} y_n
  \end{align*}

  Notice that this result is actually a bit more general that in our current
  setup; it did not use any information about $(y_n)$ except that it is
  bounded.
\end{proof}

\end{document}
