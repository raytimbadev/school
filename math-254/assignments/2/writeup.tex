\documentclass[letterpaper,11pt]{article}

\author{Jacob Thomas Errington (260636023)}
\title{Assignment \#2\\Honours Analysis 1 -- MATH 254}
\date{4 October 2016}

\usepackage[margin=2.0cm]{geometry}
\usepackage{amsmath,amssymb,amsthm}
\newcommand{\R}{\mathbb{R}}
\newcommand{\N}{\mathbb{N}}
\renewcommand{\P}{\mathbb{P}}
\newcommand{\Z}{\mathbb{Z}}
\newcommand{\A}{\mathcal{A}}
\newcommand{\Union}{\bigcup}

\newtheorem{prop}{Proposition}
\newtheorem{lemma}{Lemma}
\newtheorem{definition}{Definition}

\begin{document}

\maketitle

\section*{\# 3 -- The algebraic numbers}

\begin{definition}
    The algebraic numbers $\A \subseteq \R$ are given by the following
    description.

    \begin{equation*}
        \A = \left\{
            x | \exists n \in \N, \exists a \in \Z_*^n :
            \sum_{i=0}^n {a_i x^i} = 0
        \right\}
    \end{equation*}
    where $\Z_*^n$ is defined by $\Z_*^0 = \Z \setminus \{0\}$, and
    $\Z_*^n = \Z^{n-1} \times (\Z \setminus \{0\})$ for $n > 0$.
\end{definition}

Verbally, the algebraic numbers are those reals that can be obtained as a root
of a polynomial of some degree $n$ with integer coefficients.  This definition
involving $Z_*^n$ avoids the possibility of the leading coefficient of the
polynomial being zero.

\begin{prop}
    The algebraic numbers are countable.
\end{prop}

\begin{proof}
    We will write the algebraic numbers as a countable sum, from which it will
    follow that they are countable. Informally, all we have to do is ``pull out
    the existentials'' from our definition of $\A$ to produce the countable
    sum. First, let's pull out the $\exists n \in \N$.
    \begin{equation*}
        \A = \Union_{n=0}^\infty \left\{
            x | \exists a \in \Z_*^n :
            \sum_{i=0}^n {a_i x^i} = 0
        \right\}
    \end{equation*}

    Next, we need to ``pull out'' the $\exists a \in \Z)*^n$.
    \begin{equation*}
        \A = \Union_{n=0}^\infty {
            \Union_{a \in \Z_*^n} \left\{
                x | \sum_{i=0}^n {a_i x^i} = 0
            \right\}
        }
    \end{equation*}
    Note that $\Z_*^n$ is countable because it is the cross product of a finite
    number of countable sets.

    For $n \in \N$ and $a \in \Z_*^n$, $\sum_{i=0}^n {a_i x^n} = 0$ has at
    most $n$ roots, so $\left\{ x | \sum_{i=0}^n {a_i x^i} \right\}$ is finite.
    The countable union of finite sets is countable, so for fixed $n \in \N$,
    $\Union_{a \in \Z_*^n} \left\{ x | \sum_{i=0}^n {a_i x^i} \right\}$ is
    countable.  Hence, $\A$ is countable, since it is a countable union of
    countable sets.
\end{proof}

\section*{\#4 -- Some facts about real numbers}

First let's show some (extremely) useful lemmas to help us with our proofs.

\begin{lemma}
    Zero is the absorbing element of multiplication, i.e.
    \begin{equation*}
        \forall a \in \R:\, 0 \cdot a = a \cdot 0 = 0
    \end{equation*}
    \label{lem:zeroabsorb}
\end{lemma}

\begin{proof}
    Consider the true statement $1 = 1 + 0$ obtained by the fact that $0$ is
    the neutral element of addition. Then multiply both sides by an arbitrary
    real $a$.
    \begin{align*}
        1 \cdot a &= (1 + 0)a & ~ \\
        1 \cdot a &= 1 \cdot 0 + 0 \cdot a
            & \quad\text{distributivity of (+)} \\
        0 = 1 \cdot a - 1 \cdot a &= 0 \cdot a
            & \quad\text{by existence of inverse elements for addition}
    \end{align*}

    Since the choice of $a$ was arbitrary, we generalize to
    $\forall a: 0 \cdot a = 0$, which is precisely what we wanted to show.
\end{proof}

\begin{lemma}
    For any $x \in \R$, $-x = -1 \cdot x$.
    \label{lem:minusone}
\end{lemma}

\begin{proof}
    In other words, we want to show that $-1 \cdot x$ is the inverse of $x$. It
    suffices to show that $-1 \cdot x + x = 0$. Here is our deduction.

    \begin{center}
        \begin{tabular}{r l | r}
            ~ & $-1 \cdot x + x$ & ~ \\
            = & $-1 \cdot x + 1 \cdot x$
                & $1$ is the neural element of $(\cdot)$ \\
            = & $(-1 + 1) \cdot x$ & distributivity of $(\cdot)$ over $(+)$ \\
            = & $0 \cdot x$ & $-1$ is the additive inverse of $1$ \\
            = & $0$ & by lemma \ref{lem:zeroabsorb}
        \end{tabular}
    \end{center}

    Hence, $-1 \cdot x$ is the inverse of $x$, so it is equal to $-x$.
\end{proof}

\begin{lemma}
    For any $x \in \R$, $-(-x) = x$.
    \label{lem:doubleneg}
\end{lemma}

\begin{proof}
    Since $-(-x)$ is the inverse of $-x$, we can write
    \begin{equation*}
        -(-x) + (-x) = 0
    \end{equation*}
    Adding $x$ to both sides of the equation, we get
    \begin{equation*}
        -(-x) + (-x) + x = x
    \end{equation*}
    Since $-x$ is the inverse of $x$, we can write
    \begin{equation*}
        -(-x) + 0 = x
    \end{equation*}
    Since $0$ is the neutral element of addition, we can simplify to
    \begin{equation*}
        -(-x) = x
    \end{equation*}
    which is precisely what we wanted to show.
\end{proof}

Now we are ready to attack the main proposition.

\begin{prop}
    Suppose $x, y \in \R$ are arbitrary reals.
    \begin{equation*}
        (-x)(-y) = xy
    \end{equation*}
\end{prop}

\begin{proof}
    Here is our deduction.

    \begin{center}
        \begin{tabular}{r l | r}
            ~ & $(-x)(-y)$ & ~ \\
            = & $(-1) \cdot x \cdot (-1) \cdot y$
                & by lemma \ref{lem:minusone} \\
            = & $((-1)(-1))(xy)$ & by associativity \\
            = & $(-(-1))(xy)$ & by lemma \ref{lem:minusone} \\
            = & $(1)(xy)$ & by lemma \ref{lem:doubleneg} \\
            = & $xy$ & because $1$ is the neutral element of $(\cdot)$
        \end{tabular}
    \end{center}
\end{proof}

Next, we will show an ordering property for reals, but we will need a lemma
first.

\begin{lemma}
    Let $x \in \R$, $x \neq 0$. Then $\frac{1}{x} > 0 \iff x > 0$.
    \label{lem:invsign}
\end{lemma}

\begin{proof}
    Suppose $x \neq 0$. Consider $\frac{1}{x}\cdot\frac{1}{x}$. Clearly, since
    $\frac{1}{x} = \frac{1}{x}$, they are either both positive or both
    negative, so by a property seen in class, $\frac{1}{x}\frac{1}{x}$ is
    positive. We can multiply it with $x$ and obtain some result, say $p$. By a
    result seen in class, $p$, being a product of two things, will
    be positive if either both factors are positive, or both factors are
    negative. Obviously, they cannot be both negative, as we have shown that
    one of those factors is always positive. Consequently, $p > 0 \iff x > 0$.

    Since $\frac{1}{x}$ is the multiplicative inverse of $x$, we can simplify
    $p$.
    \begin{equation*}
        p = x\frac{1}{x}\frac{1}{x} = 1 \frac{1}{x} = \frac{1}{x}
    \end{equation*}
    so $\frac{1}{x} > 0 \iff x > 0$, which is precisely what we wanted to show.
\end{proof}

\begin{prop}
    If $x, y \in \R$ are arbitrary reals, then
    \begin{equation*}
        x < y < 0 \implies \frac{1}{y} < \frac{1}{x} < 0
    \end{equation*}
\end{prop}

\begin{proof}
    Since $x$ is negative (by the transitivity property seen in class),
    $\frac{1}{x}$ is negative by lemma \ref{lem:invsign}. Since $y$ is
    negative, $\frac{y}{x}$ is positive by the result shown in class, since
    both factors are negative.

    Since $x < y$, $y - x \in \P$, so multiplying $y - x$ with another positive
    number will yield a positive number by the closure property of $\P$. Let's
    multiply by $\frac{y}{x}$ and use distributivity to get
    \begin{equation*}
        (y - x)\frac{y}{x}
        = y \frac{y}{x} - x \frac{y}{x}
        = y \frac{y}{x} - x \frac{1}{x} y
        = y \frac{y}{x} - 1 \cdot y
        = y \frac{y}{x} - y \in \P
    \end{equation*}
    using distributivity, associativity of $(\cdot)$, property of
    multiplicative inverse, and $1$ as neutral element of multiplication.

    Then, consider $\frac{1}{y}\frac{1}{y}$. Being the product of equal things,
    it is positive by the result seen in class. Let's multiply it with our
    the result we just computed to get the following positive number by the
    closure property.
    \begin{equation*}
        \left(y \frac{y}{x} - y\right)\frac{1}{y}\frac{1}{y}
        = y \frac{y}{x}\frac{1}{y}\frac{1}{y} - y\frac{1}{y}\frac{1}{y}
        = \left(y \frac{1}{y}\right) \frac{1}{x} \left(y \frac{1}{y}\right)
        - \left(y\frac{1}{y}\right)\frac{1}{y}
        = 1 \cdot \frac{1}{x} \cdot 1 - 1 \cdot \frac{1}{y}
        = \frac{1}{x} - \frac{1}{y} \in \P
    \end{equation*}

    So $\frac{1}{y} < \frac{1}{x}$, which is \emph{almost} what we want to
    show. All that remains is that $\frac{1}{x} < 0$, which follows immediately
    from lemma \ref{lem:invsign} since $x < 0$ by assumption.

    Thus, $\frac{1}{y} < \frac{1}{x} < 0$.
\end{proof}

\end{document}
