\documentclass[letterpaper,11pt]{article}

\author{Jacob Thomas Errington (260636023)}
\title{Assignment \#5\\Honours analysis 1 -- MATH 254}
\date{17 November 2016}

\usepackage[margin=2.0cm]{geometry}
\usepackage{amsmath,amssymb,amsthm}
\usepackage{cases}

\newcommand{\makebb}[1]{
    \expandafter\newcommand\csname #1\endcsname{\mathbb{#1}}
}

\makebb{N}
\makebb{Q}
\makebb{R}
\makebb{Z}

\newcommand{\floor}[1]{\left\lfloor#1\right\rfloor}
\newcommand{\parens}[1]{\left(#1\right)}
\newcommand{\abs}[1]{\left|#1\right|}
\newcommand{\intersn}{\cap}
\newcommand{\union}{\cup}
\newcommand{\half}{\frac{1}{2}}
\newcommand{\quarter}{\frac{1}{4}}

\newtheorem{prop}{Proposition}

\begin{document}

\maketitle

\section*{\#6 -- Continuity}

\begin{prop}
    The function $f : \R \to \R$ defined by
    \begin{equation*}
        f(x) = \floor{x} + \sqrt{x - \floor{x}}
    \end{equation*}
    is continuous everywhere on its domain.
\end{prop}

\begin{proof}
    Note that $\floor{x}$ is continuous everywhere except for $x \in \Z$ due to
    the jump in the function value, i.e. the limit from the left is not equal
    to the limit from the right.
    A similar reasoning applies to $\sqrt{x - \floor{x}}$ the function is
    continuous everywhere except for integers: the limit from the left as $x$
    approches an integer is $1$, but the limit from the right is $0$.
    Therefore, it suffices to show that $f$ is continuous when $x$ is an
    integer, as everywhere else $f$ is a sum/composition of continuous
    functions.

    Suppose $x$ is an integer, and consider
    \begin{equation*}
        \lim_{x^* \to x^-}\parens{\floor{x^*} + \sqrt{x^* - \floor{x^*}}}
    \end{equation*}
    Notice that we can consider a sequence $(x_n)$ converging to $x$ whose
    terms are all less than $x$ to understand this limit. By the definition of
    convergeance, we can ``cut'' the sequence to obtain a subsequence whose
    terms are all between $x - 1$ and $x$.
    Consequently, we can evaluate $\floor{x^*}$ to $x - 1$ in the limit.
    \begin{equation*}
        \lim_{x^* \to x^-}\parens{x^* - 1 + \sqrt{x^* - x + 1}}
        = x - 1 + 1
        = x
    \end{equation*}
    Next, consider
    \begin{equation*}
        \lim_{x^* \to x^+}\parens{\floor{x^*} + \sqrt{x^* - \floor{x^*}}}
    \end{equation*}
    Again, we can consider an underlying sequence consisting only of terms
    between $x$ and $x + 1$, allowing us to evaluate $\floor{x^*}$ to $x$ in
    the limit.
    \begin{equation*}
        \lim_{x^* \to x^-}\parens{x + \sqrt{x^* - x}}
        = x + 0
        = x
    \end{equation*}
    This shows that the left limit is equal to the right limit for any integer
    $x$. Therefore, the limit of the function exists for all integers $x$.

    Next, we must check that the function value is equal to the limit value for
    all integers $x$.
    Suppose again that $x \in \Z$. Then $\floor{x} = x$, so
    \begin{equation*}
        f(x)
        = \floor{x} + \sqrt{x - \floor{x}}
        = x + 0 = x
    \end{equation*}
    which is equal to the limit value.

    Therefore, $f$ is continuous everywhere.
\end{proof}

\section*{\#7 -- Bijiective function that is continuous nowhere}

Consider the function $f : [0, 1] \to [0, 1]$ defined by
\begin{numcases}{f(x) = }
    x & if $x \in \Q \intersn (0, 1] \setminus \{\half\}$ \label{eq:f-1} \\
    1 - x & if $x \in (0, 1) \setminus \Q$ \label{eq:f-2} \\
    0 & if $x = \half$ \label{eq:f-3} \\
    \half & if $x = 0$ \label{eq:f-4}
\end{numcases}

\begin{prop}
    The function $f$ is bijective, but continuous nowhere on its domain.
\end{prop}

\begin{proof}
    First we will show bijectivity. To do so, we will examine every point $y$
    in the codomain, and show that there exists a unique element in the domain
    that maps to $y$. This will establish both injectivity and surjectivity in
    one shot: since every element in the codomain is mapped to by some element
    in the domain, $f$ will be surjective; and since that element mapping to
    $y$ is \emph{unique}, $f$ will be injective.

    \begin{description}
        \item[Case] $y = 0$.
            Consider the preimage of $\{0\}$ under $f$, $A = f^{-1}(\{0\})$.
            The elements of $A$ must match one of the cases in $f$. Notice that
            $A$ is nonempty because $0 \in A$.
            Take arbitrary $x \in A$. We will now proceed to dispel every
            equation besides \eqref{eq:f-3}, i.e. $x = \half$.
            \begin{description}
                \item[Case] $y = x \in \Q \intersn (0, 1] \setminus \{\half\}$,
                    but $0 \notin (0, 1]$, so this is a contradiction.
                \item[Case] $1 - x = y$, so $x = 1 - y$, but since $y = 0$,
                    then $x = 1$. Since $x \notin (0, 1) \setminus \Q$, this
                    case is impossible.
                \item[Case] $y = 0$, then $x = \half$.
                \item[Case] $y = 0 = \half$ is a contradiction, so this case is
                    impossible.
            \end{description}
            Since $x$ is arbitrary, we conclude $\forall x \in A: x = \half$.
            Since $A$ is nonempty, this establishes that $A$ is a singleton.
            Therefore, there is a unique element in the domain of $f$ that maps
            to $0$

        \item[Case] $y = \half$.
            The reasoning in this case is practically the same as for the only
            constant case. Rather than give a detailed case analysis, we will
            instead briefly give an overview of the analysis. The preimage is
            nonempty due to \eqref{eq:f-4}. Case \eqref{eq:f-3} is impossible
            because $0 \neq \half$. Case \eqref{eq:f-2} is impossible because
            $x = \half \notin (0, 1) \setminus \Q$. Case \eqref{eq:f-1} is
            impossible because
            $\half \notin \Q \intersn (0, 1] \setminus \{\half\}$.

            Therefore we conclude that $\forall x \in A: x = 0$, and thus that
            $A$ is a singleton, so there is a unique element in the domain of
            $f$ that maps to $\half$.

        \item[Case] $y \in [0, 1] \setminus \{0, \half\}$.
            Notice that equations \eqref{eq:f-3} and \eqref{eq:f-4} are
            impossible because $y \neq 0$ and $y \neq \half$.

            We further split this case according to whether $y \in \Q$.
            \begin{description}
                \item[Case] $y \in \Q$.
                    Consider the preimage under $f$,
                    $A = f^{-1}(\{y\})$.
                    Note that $A$ is nonempty because choosing $x = y$ gives
                    $x$ matching the requirement for \eqref{eq:f-1}.
                    Take arbitrary $x \in A$.
                    \begin{description}
                        \item[Case] \eqref{eq:f-1} i.e. $y = x$.
                            Since $x = y$, we can infer from
                            $y \in \Q \intersn (0, 1] \setminus \{\half\}$
                            that
                            $x \in \Q \intersn (0, 1] \setminus \{\half\}$.

                        \item[Case] \eqref{eq:f-2} i.e. $y = 1 - x$, so
                            $x = 1 - y$. Since $y \in \Q$, then $x \in Q$ by
                            the closure properties of $\Q$.
                            But this is a contradiction, since for
                            \eqref{eq:f-2}, we must have
                            $x \in (0, 1) \setminus \Q$, so this case is
                            impossible.
                    \end{description}
                    Therefore, we can conclude that $\forall x \in A: x = y$,
                    which establishes that $A$ is a singleton, so there is a
                    unique element mapped to $y$.
                \item[Case] $y \notin \Q$.
                    The argument is the same as in the previous case, so we
                    will not go into as much detail.
                    This time, case \eqref{eq:f-3} is impossible and
                    case \eqref{eq:f-4} is possible. We will establish that the
                    preimage of $y$ is nonempty, and that all its elements are
                    equal to $1 - y$, so it is a singleton. Hence there is a
                    unique element mapping to $y$.
            \end{description}
    \end{description}

    This establishes that $f$ is injective and surjective and thus, bijective.

    Next, we will show that $f$ is continuous nowhere on its domain, i.e.
    \begin{equation*}
        \forall x \in [0, 1]:
        \exists \epsilon > 0:
        \forall \delta > 0:
        \exists x^* \in [0, 1]:
        \abs{x - x^*} < \delta \land \abs{f(x) - f(x^*)} \geq \epsilon
    \end{equation*}

    Take arbitrary $x \in [0,1]$.
    \begin{description}
        \item[Case] $x = 0$.
            Then $f(x) = \half$.
            Choose $\epsilon = \quarter$.
            Take arbitrary $\delta > 0$.
            \begin{description}
                \item[Case] $\delta > \quarter$.
                    Then choose $0 < x^* < \quarter < \delta$
                    such that $x^* \in \Q$.
                    Notice $x^* \in \Q \intersn (0, 1] \setminus \{\half\}$.
                    So $0 < f(x^*) = x^* < \quarter < \delta$.

                    Our goal is to show
                    \begin{equation*}
                        \abs{x - x^*} < \delta
                        \land \abs{f(x) - f(x^*)} \geq \epsilon
                    \end{equation*}
                    Substituting known values gives us a refined goal
                    \begin{equation*}
                        \abs{x^*} < \delta
                        \land
                        \abs{\half - x^*} \geq \quarter
                    \end{equation*}
                    Since $0 < x^* < \quarter < \delta$, we only have to know
                    $\abs{\half - x^*} \geq \quarter$.
                    Here is our deduction.
                    \begin{align*}
                        x^* < \quarter \\
                        -x^* > \quarter \\
                        \half - x^* > \half + \quarter > \quarter = \epsilon
                    \end{align*}
                    So $\abs{\half - x^*} > \epsilon$ as required.

                \item[Case] $\delta < \quarter$.
                    Choose $0 < x^* < \delta < \quarter$
                    such that $x^* \in \R \setminus \Q$.
                    Hence $x^* \in (0, 1) \setminus \Q$,
                    so $f(x^*) = 1 - x^*$.
                    By definition, $|x^*| < \delta$,
                    so it remains only to show
                    $\abs{\half - 1 + x^*} \geq \quarter$.
                    Here is our deduction.
                    \begin{align*}
                        0 < x^* < \quarter \\
                        -\quarter < x^* < 0 \\
                        1 - \quarter < 1 - x^* < 1 \\
                        -1 < x^* - 1 < \quarter - 1 \\
                        \half - 1 < \half + x^* - 1 < \half + \quarter - 1 \\
                        -\half < \half + x^* - 1 = \half - f(x^*) < -\quarter \\
                        \epsilon = \quarter < f(x^*) - \half \\
                        \epsilon = \quarter < \abs{\half - f(x^*)}
                    \end{align*}
                    So $\abs{\half - f(x^*)} > \epsilon$ as required.
            \end{description}

            This establishes that $f$ is not continuous at $x = 0$.

        \item[Case] $0 < x < \half$.

            Define the sequence $(x_n)_{n\in\N}$ by $x_1 = r$ where $r$ is
            an arbitrary rational strictly between $0$ and $x$ and
            $x_{n+1} = r_{n+1}$ where $r_n$ is an arbitrary rational strictly
            between $x_n$ and $x$. This sequence's supremum is obviously $x$.
            Notice that the sequence is bounded above and increasing, so it is
            convergent, specifically to $x$.

            Use a similar construction for $(y_n)_{n\in\N}$ consisting only of
            irrationals, but always \emph{decreasing} towards $x$. Notice that
            both $(y_n)$ and $(x_n)$ converge to $x$.

            Consider the sequence $(f(x_n))_{n\in\N}$. Its terms are
            applications of $f$ to rationals (other than $0$ and $\half$), so
            we can evaluate $f$ to obtain that $f(x_n) = x_n$, so
            \begin{equation*}
                \lim_{n\to\infty} f(x_n) = x
            \end{equation*}

            Consider the sequence $(f(y_n))_{n\in\N}$. Its terms are
            applications of $f$ to irrationals, so we can evaluate $f$ to
            obtain that $f(y_n) = 1 - y_n$. Since the constant sequence $1$
            converges, and the sequence $y_n$ converges, we obtain
            \begin{equation*}
                \lim_{n\to\infty} f(y_n) = 1 - x
            \end{equation*}

            However, this is a contradiction of the sequential criterion: the
            limit of $f$ approching $x$ therefore does not exist, since we
            found two sequences that when passed through the function produce
            sequences that do not converge to the same value.

        \item[Case] $\half < x < 1$.
            The same reasoning as in the previous case applies, so we will not
            repeat the argument. The reason we split the cases was to avoid the
            possibility of the terms in the sequences being $\half$, which
            trips up the very simple argument showing that the sequences
            converge.

        \item[Case] $x = \half$.
            We will show that the limit of $f$ at $x = \half$ exists, and is
            equal to $\half$.

            First, let us study the limit on the left. Suppose we have a
            convergent sequence $(x_n)$ such that
            $\lim_{n\to\infty} x_n = \half$ and every $x_n < \half$. Take some
            arbitrary term in the sequence $x_n$. It is either rational or
            irrational, so $x_n \leq f(x_n) \leq 1 - x_n$
            (since $x_n < \half$).
            By the squeeze theorem, $\lim_{n\to\infty} f(x_n) = \half$, since
            both $\lim_{n\to\infty} x_n = \half$
            and $\lim_{n\to\infty} (1 - x_n) = \half$.

            The same reasoning applies to the limit on the right, except that
            the bounds for $f(x_n)$ will be $1 - x_n < f(x_n) < x_n$ since
            $x_n > \half$.

            Hence, $\lim_{x \to \half} f(x) = \half$. However, $f(\half) = 0$,
            so $f$ is not continuous at $x = \half$.

        \item[Case] $x = 1$.
            We construct two convergent sequences $(x_n)$ and $(y_n)$, the
            former consisting only of rationals and the latter consisting only
            of irrationals. Both sequences are strictly increasing, and are
            bounded by $1$. (We use the same construction again using repeated
            uses of the density of both sets.)

            Passing the sequences through $f$ will produce different limits,
            namely $1$ and $0$, by the same argument again. This contradicts
            the sequential criterion, so $f$ is not continuous at $1$ either.
    \end{description}

    Hence, $\forall x \in [0, 1]$, $f$ is not continuous.
\end{proof}

\end{document}
