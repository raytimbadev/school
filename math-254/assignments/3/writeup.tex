\documentclass[letterpaper,11pt]{article}

\author{Jacob Thomas Errington (260636023)}
\title{Assignment \#3\\Honours analysis 1 -- MATH 254}
\date{18 October 2016}

\usepackage{amsmath,amssymb,amsthm}
\usepackage[margin=2.0cm]{geometry}
\usepackage{mathtools}

\DeclarePairedDelimiter\floor{\lfloor}{\rfloor}

\newtheorem{prop}{Proposition}

\newcommand{\D}{\mathbb{D}}
\newcommand{\K}{\mathbb{K}}
\newcommand{\N}{\mathbb{N}}
\newcommand{\R}{\mathbb{R}}
\newcommand{\Q}{\mathbb{Q}}
\newcommand{\Z}{\mathbb{Z}}

\begin{document}

\maketitle

\section*{\#3 -- The accumulation points of $\Q$}

\begin{prop}
    The set of accumulation points of $\Q$ is $\R$.
\end{prop}

\begin{proof}
    Take arbitrary $x \in \R$ and consider an arbitrary nonempty interval
    $(x - \epsilon, x + \epsilon)$ around $x$. By the density of $\Q$, there
    exists $q \in \Q$ such that $q \in (x - \epsilon, x + \epsilon)$.
    Some cases arise.
    \begin{description}
        \item[Case] $q = x$. Then, using the density of $\Q$, there exists
            $q^\prime \in \Q$ such that $q \in (x - \epsilon, x)$. Rename
            $q^\prime$ to $q$.
        \item[Case] $q \neq x$. Then there is nothing to do.
    \end{description}
    Hence, $q \in \Q \cap (x - \epsilon, x + \epsilon)$. Therefore,
    $\Q \cap (x - \epsilon, x + \epsilon) \neq \emptyset$. Since arbitrary
    $x \in R$ result in the satisfaction of the defining property of an
    accumulation point, we conclude that $\R$ is the set of accumulation points
    of $\Q$.
\end{proof}

For any bounded infinite set $S \subset \R$, let
\begin{equation*}
    A = \{ x \in \R : (x, \infty) \cap S \text{ infinite}\}
\end{equation*}

\begin{prop}
    The set $A$ is nonempty, and $\sup{S}$ is an upper bound of $A$.
\end{prop}

\begin{proof}
    First, we will show $\exists x \in \R$ such that $(x, \infty) \cap S$ is
    infinite. We know that
    \begin{equation*}
        S \subseteq [\inf{S}, \sup{S}] \subset [\inf{S}, \infty)
    \end{equation*}
    so taking $x = \inf{S}$, we get that $[x, \infty) \cap S = S$ is infinite.
    Consider then $(x, \infty) \cap S$. It differs from $[x, \infty) \cap S$ by
    a finite number (specifically one) of elements from $(x, \infty) \cap S$.
    Adding or removing a finite number of elements from an infinite set results
    in an infinite set, so $(x, \infty) \cap S$ is infinite.
    Hence, $\inf{S} \in A$, so $A$ is nonempty.

    Next, we want to show that $\sup{S}$ is an upper bound of $A$, i.e.
    \begin{equation*}
        \forall x \in A: x \leq \sup{S}
    \end{equation*}

    Take arbitrary $x \in A$. We want to show that $x \leq \sup{S}$. Suppose by
    contradiction that $x > \sup{S}$. We know that $(x, \infty) \cap S$ is
    infinite. In particular, it must be nonempty, so
    $\exists y \in (x, \infty) \cap S$. By definition of set intersection,
    $y \in (x, \infty)$ -- which means $y > x$ -- and $y \in S$. We know that
    $S \subset [\inf{S}, \sup{S}]$, so $\inf{S} \leq y \leq \sup{S}$. But
    $y \leq \sup{S} < x < y$ which is a contradiction. Hence $x \leq \sup{S}$,
    and since $x$ was arbitrary, we have by generalization that
    $\forall x \in A$, $x \leq \sup{S}$. This is the definition of an upper
    bound.
\end{proof}

\begin{prop}
    $\sup{A}$ is an accumulation point of the set $S$ that underlies its
    definition.
\end{prop}

\begin{proof}
    Take arbitrary $\epsilon > 0$. We want to show that $\sup{A}$ satisfies
    \begin{equation*}
        S \cap (\sup{A} - \epsilon, \sup{A} + \epsilon) \setminus \{\sup{A}\}
        \neq \emptyset
    \end{equation*}
    so we need to find $y \in \R$ such that
    \begin{itemize}
        \item $y \in S$
        \item $y \in (\sup{A} - \epsilon, \sup{A} + \epsilon)$
        \item $y \neq \sup{A}$
    \end{itemize}

    Suppose by contradiction that such a $y$ does not exist. Then in
    particular, $S \cap (\sup{A} - \epsilon, \sup{A}) = \emptyset$ and
    $S \cap (\sup{A}, \sup{A} + \epsilon) = \emptyset$.

    Now supose further that $\sup{A} \in A$.
    Then $(\sup{A}, \infty) \cap S$ is not finite.
    Since the region between $\sup{A}$ and $\sup{A} + \epsilon$ is empty (in
    the sense that it contains no elements of $S$), there are also infinitely
    many elements in $(\sup{A} + \epsilon, \infty) \cap S$ because all we did
    was ``cut off'' the empty region. But then this set satisfies the
    definition for elements of $A$, so $\sup{A} + \epsilon \in A$.
    But then
    $\sup{A} < \sup{A} + \epsilon \in A$ which
    contradicts the definition of $\sup{A}$. Hence, $\sup{A} \notin A$.
    But then, since the region $(\sup{A} - \epsilon, \sup{A})$ contains no
    elements of $S$, there can be no elements of $A$ in that region either, as
    they would fail to satisfy the defining property of $A$.
    But then
    $\sup{A} - \epsilon$ is greater than every element of $A$, which
    contradicts the definition of $\sup{A}$. Hence there must exist
    $y \in S
    \cap (\sup{A} - \epsilon, \sup{A} + \epsilon) \setminus \{\sup{A}\}$,
    so that set is nonempty, as required by the definition of an accumulation
    point.
\end{proof}

\begin{prop}[Variant of the Archimedean property]
    \begin{equation*}
        \forall x \in \R: \exists k \in \N: 10^k > x
    \end{equation*}
\end{prop}

\begin{proof}
    By contradiction. Suppose not. Then
    \begin{equation*}
        \exists x \in R: \forall k \in \N: 10^k \leq x
    \end{equation*}

    Hence, $x$ is an upper bound of the set $\K = \{ 10^k | k \in \N \}$.
    By the completeness of $\R$, $\exists u \in \R : u = \sup{K}$.
    Since $\forall p \in \K: p \leq u$, $\K \ni 10p \leq u$.
    Hence, $p \leq \frac{u}{10}$. But then $\frac{u}{10}$ is an upper bound of
    $\K$. Since $\frac{u}{10} < u$, this is a contradiction of $u = \sup{K}$.
\end{proof}

\begin{prop}
    The set
    \begin{equation*}
        \D = \left\{ \frac{a}{10^k} : a \in \Z,\, k \in \N \right\}
    \end{equation*}
    is dense in $\R$.
\end{prop}

\begin{proof}
    Take arbitrary $x, y \in \R$ and suppose $x < y$.
    Then $y - x > 0$.
    Hence $\frac{1}{y - x} > 0$.
    Using the variant of the Archimedean property shown above,
    \begin{equation*}
        \exists k \in \N: 10^n > \frac{1}{y - x}
    \end{equation*}
    Hence, $10^k y - 10^k x > 1$.
    Finally,
    \begin{equation*}
        10^k x < \floor*{10^k x + 1} \geq 10^k x + 1 < 10^k y
    \end{equation*}
    from which it follows that
    \begin{equation*}
        x < d = \frac{\floor*{10^k + 1}}{10^k} < y
    \end{equation*}

    Since $\floor*{10^k + 1} \in \Z$, $d \in \D$, which shows that between any
    two distinct reals, there exists an element of $\D$. Thus $\D$ is dense.
\end{proof}

\end{document}
