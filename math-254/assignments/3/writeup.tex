\documentclass[letterpaper,11pt]{article}

\author{Jacob Thomas Errington (260636023)}
\title{Assignment \#3\\Honours analysis 1 -- MATH 254}
\date{18 October 2016}

\usepackage{amsmath,amssymb,amsthm}
\usepackage[margin=2.0cm]{geometry}

\newtheorem{prop}{Proposition}

\newcommand{\R}{\mathbb{R}}
\newcommand{\Q}{\mathbb{Q}}

\begin{document}

\maketitle

\section*{\#3 -- The accumulation points of $\Q$}

\begin{prop}
    The set of accumulation points of $\Q$ is $\R$.
\end{prop}

\begin{proof}
    Take arbitrary $x \in \R$ and consider an arbitrary nonempty interval
    $(x - \epsilon, x + \epsilon)$ around $x$. By the density of $\Q$, there
    exists $q \in Q$ such that $q \in (x - \epsilon, x + \epsilon)$.
    Some cases arise.
    \begin{description}
        \item[Case] $q = x$. Then, using the density of $\Q$, there exists
            $q^\prime \in Q$ such that $q \in (x - \epsilon, x)$. Rename
            $q^\prime$ to $q$.
        \item[Case] $q \neq x$. Then there is nothing to do.
    \end{description}
    Hence, $q \in \Q \cap (x - \epsilon, x + \epsilon)$. Therefore,
    $\Q \cap (x - \epsilon, x + \epsilon) \neq \emptyset$. Since arbitrary
    $x \in R$ result in the satisfaction of the defining property of an
    accumulation point, we conclude that $\R$ is the set of accumulation points
    of $\Q$.
\end{proof}

\end{document}
