\documentclass[letterpaper,11pt]{article}

\usepackage{amsmath,amssymb,amsthm}
\usepackage[margin=2.0cm]{geometry}

\DeclareMathOperator{\perm}{perm}

\author{Jacob Thomas Errington (260626023)}
\title{Assignment \#5\\Theory of Computation -- COMP 330}
\date{19 November 2015}

\usepackage{amsmath,amssymb,amsthm}

\newcommand{\N}{{\ensuremath \mathbb{N}}}

\begin{document}

\maketitle

\begin{enumerate}
    \item
        Below is a context-free grammar for the language

        $$
        \{a^m b^n c^p d^q | m + n = p + q\}
        $$

        \begin{align*}
            S &\to A | A^\prime | B | B^\prime \\
            A &\to a (A | A^\prime | B) d \\
            A^\prime &\to a (A^\prime | B^\prime) c \\
            B &\to b (B | B^\prime) d | \epsilon \\
            B^\prime &\to b B^\prime c | \epsilon \\
        \end{align*}

    \item
        The language $\bar L$ contains all odd-length strings as well as all
        strings that can be split into $xy$ such that $|x| = |y|$ and there is
        at least one mismatched corresponding letter in $x$ and $y$.

        That is precisely the structure of our PDA: we nondeterministically
        choose between a machine that accepts all odd-length strings and a
        machine that will check for mismatches. Checking for odd length is
        simple, (we have seen a machine to check for even length in the context
        of a PDA to recognize $\{ ww^R | w \in \Sigma^*\}$; it's not much
        harder to build one for odd length) so we will focus on the machine to
        check for mismatches.

        The machine to find mismatches will guess which character is
        mismatched, and verify that guess. In particular, it will push $k$
        characters onto the stack and then transition to a state according to
        the present letter. That state serves as the memory to perform a
        comparison later. It will then pop those $k$ characters off the stack,
        and push $l$ characters onto the stack. It can then compare the present
        character with the one stored by the earlier choice of state. If there
        is a match, then we jam; else, we much check that we haven't a false
        positive by popping all characters from the stack and seeing whether
        we've reached the end of the input.

    \item
        Consider the regular language $L = (abc)^*$, which is of course
        context-free. The language $\perm{L}$ is the language of all strings
        having matching occurrences of $a$s, $b$s, and $c$s, which is not
        context-free as it requires a three-way matching.

    \item
        \begin{enumerate}
            \item
                Since $n$ is unbounded, it becomes impossible to decide in
                finite time whether a given element $m \in \bigcup_n C_n$; we
                can't just try all of them.

            \item
                This time, the statement is in fact true. We can use the same
                technique as dove-tailing for producing an enumerator given
                a decision procedure.
        \end{enumerate}

    \item
        \begin{enumerate}
            \item
                Suppose $K$ is computable. Then, there exists a total
                computable function $f : \Sigma^* \to 2$ that decides the
                membership of $K$. We can then take an arbitrary word
                $w \in \Sigma^*$, prepend $a$, and check whether it is in $K$
                by using $f$. If $aw \in K$, then then $aw$ is either in the
                first set of the union definition $K$ or it is in the second
                set. Of course, it can't be in the second set, since all those
                words start with $b$. Hence, we can use $f$ as a way to decide
                whether a given word is in $L$, which contradicts the
                undecidability of $L$.

            \item
                Suppose $K$ is CE. Then, there exists an algorithm $A$ that
                halts exactly on those elements $w \in \Sigma^*$ that are in
                $K$. Using $A$, we can build an algorithm that halts only on
                inputs that are in $\bar L$ as follows. We take an arbitrary
                word $w \in \Sigma^*$, and run $A$ with input $bw$. The
                resulting algorithm indeed halts only for those inputs that are
                in $\bar L$. Hence, $\bar L$ is CE. This contradicts the
                premise that $\bar L$ is not CE. Thus, $K$ is not CE.

            \item
                Suppose $K$ is co-CE. Then, $\bar K$ is CE and we have an
                algorithm $A$ that halts only on those inputs that are in
                $\bar K$. The language $\bar K$ consists of those strings
                $v \in \Sigma^*$ that are neither of the form $aw$ where
                $w \in L$ nor of the form $bw$ where $w \in \bar L$. We will
                show that $L$ is co-CE, i.e. that $\bar L$ is CE. To do so,
                we will build an algorithm that halts only on inputs that are
                in $\bar L$ as follows. We simply run $A(aw)$ where $w$ is the
                input to our new algorithm. Clearly, $aw$ is not of the form
                $bv$ where $v \in \bar L$, so this execution of $A$ will halt
                if and only if $aw$ is not of the form $av$ where $v \in L$.
                In other words, it will only halt if $w \notin L$. Hence, the
                new algorithm halts only for inputs that are in $\bar L$,
                implying that $L$ is co-CE. This is a contradiction of the
                premise that $\bar L$ is not CE, however. Thus, $K$ is not
                co-CE.
        \end{enumerate}
\end{enumerate}
\end{document}
