\documentclass[letterpaper,11pt]{article}

\author{Jacob Thomas Errington (260636023)}
\title{Assignment \#6\\Theory of Computing -- COMP 330}
\date{3 December 2015}

\usepackage{amsmath,amssymb,amsthm}

\newtheorem{proposition}{Proposition}
\newtheorem{claim}{Claim}

\begin{document}

\maketitle

\begin{enumerate}
    \item --

    \item --

    \item
        \begin{enumerate}
            \item
                Suppose $L$ is a regular language and $w \in \Sigma^*$. Let
                $L/w = \{x \in \Sigma^* | xw \in L\}$.

                \begin{proposition}
                    For all regular languages $L$ and any word $w \in
                    \Sigma^*$, $L/w$ is a regular language.
                \end{proposition}

                \begin{proof}
                    Consider the DFA for $L^R$ (the reverse language of $L$).
                    We run the word $w^R$ though it and keep track of which
                    transitions are taken. Let $q$ be the state the DFA is in
                    after reading $w^R$.  These transitions are those that
                    would be used to recognize $w$ in the original DFA. It
                    suffices to remove these transitions from the reverse DFA,
                    mark the start state as a normal state, mark $q$ as the
                    start state, determinize, and reverse the result.

                    The resulting DFA will accept any string such that
                    appending $w$ to it would result in a string accepted by
                    the initial DFA.  This is precisely the formulation of the
                    language $L/w$ in terms of DFAs.

                    Since we can build a DFA for $L/w$ for any choice of $L$
                    and $w$ using operations under which regularity is closed,
                    $L/w$ is regular.
                \end{proof}

            \item
                Suppose $G$ is a context-free grammar.

                \begin{proposition}
                    The regularity of the language generated by a context-free
                    grammar $G$ is undecidable.
                \end{proposition}

                \begin{proof}
                    Suppose it were possible to decide whether the language
                    generated by a CFG $G$ is regular. Then due to the
                    following claim, it would be possible to decide whether the
                    language generated by $G$ is $\Sigma^*$. This, however, was
                    proved undecidable in class by the method of valid
                    computations.
                \end{proof}

                \begin{claim}
                    The language
                    $$L = N\#\Sigma^* \cup \Sigma^*\# L(G)$$
                    is context-free, but regular if and only if
                    $L(G) = \Sigma^*$, where $N$ is context-free but not
                    regular.
                \end{claim}

                \begin{proof}
                    Since $G$ is a context-free grammar, $G$ is context-free.
                    Context-free grammars are closed under union, so $L$ is
                    context-free.

                    Suppose $L$ is regular. Then, there is a DFA for $L$.
                    Furthermore, it must be possible to build this DFA from
                    the PDA for $N\#\Sigma^* \cup \Sigma^*\# L(G)$. We can
                    imagine that PDA as consisting of two branches due to
                    the union. Since $N$ is not regular, in order for $L$
                    to be regular, it is necessary that for every input, it
                    be possible to avoid the branch involving $N$. Since
                    $N \subseteq \Sigma^*$, it is already possible to do so
                    up to the $\#$ letter, but after that, it will be
                    possible only if $\Sigma^* \subseteq L(G)$. Since
                    $\Sigma^*$ is the greatest language, this conclusion is
                    equivalent to the requirement $\Sigma^* = L(G)$.

                    Now suppose $L(G) = \Sigma^*$. Then $N\#\Sigma^*
                    \subseteq \Sigma^*\#\Sigma^*$, so
                    $L = \Sigma^*\#\Sigma^*$.
                    We can write a regular expression for $L$.
                    $$L = (a^* b^*)^* \# (a^* b^*)^*$$
                    Hence, $L$ is regular.
                \end{proof}
        \end{enumerate}
\end{enumerate}

\end{document}
