\documentclass[letterpaper,11pt]{article}

\author{Jacob Thomas Errington (260636023)}
\title{Assignment \#6\\Theory of Computing -- COMP 330}
\date{3 December 2015}

\usepackage{amsmath,amssymb,amsthm}
\usepackage[margin=2.0cm]{geometry}

\newtheorem{proposition}{Proposition}
\newtheorem{claim}{Claim}

\newcommand{\Z}{\mathbb{Z}}
\newcommand{\N}{\mathbb{N}}

\begin{document}

\maketitle

\begin{enumerate}
    \item
        Suppose $M$ is a Turing machine and $w$ is a word.

        \begin{proposition}
            Whether $M$ uses more than a given number $n$ of cells on its tape
            to process $w$ is decidable.
        \end{proposition}

        \begin{proof}
            If $M$ implements a computable function defined on $w$, then it
            suffices to see how many cells have been used when $M$ halts to
            decide. However, if the function implemented by $M$ is undefined on
            $w$, then the machine will diverge.

            Divergence can arise in two ways. Either the machine will diverge
            by returning to a prior configuration (the tape contents, the
            state, and the head position are exactly as they were some time
            before) or it will diverge by ``running off'' (there is a loop in
            the states, but the head and tape contents keep changing).

            In the presence of a space bound, the first type of divergence can
            be checked in finite time since there are finitely many possible
            configurations. It suffices to keep track of each
            configuration as the computation progresses. If the machine returns
            to a prior computation, then we know that it will loop forever
            within the space bound, and our decision process can output ``NO''.

            If the machine leaves the space bound, it will do so in finite time
            by the same argument.
        \end{proof}

    \item
        If the submarine is not moving, it clearly suffices to shoot in a
        spiral fashion starting at some arbitrary starting point to eventually
        hit the submarine. This can be generalized. Indeed, if
        $g : \N \to \Z \times \Z$ is a bijection, then shooting at tile $g(t)$
        at time $t$ will eventually hit an immobile submarine.

        Now, the submarine is not immobile. It has an unknown starting
        position, a speed, and a direction. We call this tuple a
        \emph{configuration}. The set of all such tuples
        $S = \Z \times \Z \times \N \times 4$ characterizes these
        configurations.

        There exists a bijection $f : \N \to S$, which identifies each time
        step with some configuration. Using it, we can check \emph{all} the
        configurations. To compute a position to shoot at, we construct a
        function $h : S \to \Z \times \Z$. This function takes the starting
        position and adds to it the direction multiplied by the velocity.

        Composing $f$ and $h$ gives us a function $g^\prime$ that tells us
        where to shoot at time $n$. Because $f$ is a bijection, we will
        eventually get to the configuration of the the submarine that we're
        playing against and win.

    \item
        \begin{enumerate}
            \item
                Suppose $L$ is a regular language and $w \in \Sigma^*$. Let
                $L/w = \{x \in \Sigma^* | xw \in L\}$.

                \begin{proposition}
                    For all regular languages $L$ and any word $w \in
                    \Sigma^*$, $L/w$ is a regular language.
                \end{proposition}

                \begin{proof}
                    Consider the DFA for $L^R$ (the reverse language of $L$).
                    We run the word $w^R$ though it and keep track of which
                    transitions are taken. Let $q$ be the state the DFA is in
                    after reading $w^R$.  These transitions are those that
                    would be used to recognize $w$ in the original DFA. It
                    suffices to remove these transitions from the reverse DFA,
                    mark the start state as a normal state, mark $q$ as the
                    start state, determinize, and reverse the result.

                    The resulting DFA will accept any string such that
                    appending $w$ to it would result in a string accepted by
                    the initial DFA.  This is precisely the formulation of the
                    language $L/w$ in terms of DFAs.

                    Since we can build a DFA for $L/w$ for any choice of $L$
                    and $w$ using operations under which regularity is closed,
                    $L/w$ is regular.
                \end{proof}

            \item
                Suppose $G$ is a context-free grammar.

                \begin{proposition}
                    The regularity of the language generated by a context-free
                    grammar $G$ is undecidable.
                \end{proposition}

                \begin{proof}
                    Suppose it were possible to decide whether the language
                    generated by a CFG $G$ is regular. Then due to the
                    following claim, it would be possible to decide whether the
                    language generated by $G$ is $\Sigma^*$. This, however, was
                    proved undecidable in class by the method of valid
                    computations.
                \end{proof}

                \begin{claim}
                    The language
                    $$L = N\#\Sigma^* \cup \Sigma^*\# L(G)$$
                    is context-free, but regular if and only if
                    $L(G) = \Sigma^*$, where $N$ is context-free but not
                    regular.
                \end{claim}

                \begin{proof}
                    Since $G$ is a context-free grammar, $L(G)$ is
                    context-free.
                    Context-free grammars are closed under union, so $L$ is
                    context-free.

                    Suppose $L$ is regular. Suppose $L(G) \neq \Sigma^*$. Then,
                    there exists $w \in \#L(G)$. The language $L/w$ is equal to
                    $N$, which is not regular. This is a contradiction, so
                    $L(G) = \Sigma^*$.

                    Now suppose $L(G) = \Sigma^*$. Then $N\#\Sigma^*
                    \subseteq \Sigma^*\#\Sigma^*$, so
                    $L = \Sigma^*\#\Sigma^*$.
                    We can write a regular expression for $L$.
                    $$L = (a^* b^*)^* \# (a^* b^*)^*$$
                    Hence, $L$ is regular.
                \end{proof}
        \end{enumerate}

    \item
        Suppose $L$ is a context-free language and $R$ is a regular language.

        \begin{proposition}
            It is undecidable whether $R \subseteq L$.
        \end{proposition}

        \begin{proof}
            If we take $R = \Sigma^*$, then deciding $R \subseteq L$ amounts to
            deciding whether $L = \Sigma^*$ since $\Sigma^*$ is the greatest
            language. It is known that checking whether a context-free language
            accepts all words is undecidable.
        \end{proof}

        \begin{proposition}
            It is decidable whether $L \subseteq R$.
        \end{proposition}
        
        \begin{proof}
            Consider $L \cap \bar R$. It is a context-free language since $\bar
            R$ is regular and the intersection of a regular language with a
            context-free language is context-free. We know then from set theory
            that
            $$L \cap \bar R = \emptyset \iff L \subseteq R$$
            Checking whether a context-free language is empty is decidable, and
            thus so is checking whether a context-free language is a subset of
            a regular language.
        \end{proof}
\end{enumerate}

\end{document}
