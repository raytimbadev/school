\documentclass[letterpaper,11pt]{article}

\author{Jacob Thomas Errington}
\date{5 November 2015}
\title{Assignment \#4\\Theory of Computing -- COMP 330}

\usepackage{amsmath,amssymb,amsthm}
\usepackage{qtree}

\begin{document}

\maketitle

\begin{enumerate}
    \item % TODO

    \item
        To show that the language for simple C-style expressions is ambiguous,
        we will present a string having multiple valid parses.

        We will consider the string $a+++b$, and show that there is more than
        one possible parse for it, as seen in the parse trees in figures
        \ref{fig:postincrement} and \ref{fig:preincrement}.

        \begin{figure}
            \Tree
            [.S
                [.V
                    [.Post
                        [.I
                            $a$
                        ]
                        $+$
                        $+$
                    ]
                ]
                $+$
                [.V
                    [.I
                        $b$
                    ]
                ]
            ]

            \caption{
                A parse tree for the postincrement interpretation of $a+++b$.
            }
            \label{fig:postincrement}
        \end{figure}

        \begin{figure}
            \Tree
            [.S
                [.V
                    [.I
                        $a$
                    ]
                ]
                $+$
                [.V
                    [.Pre
                        [.I
                            $b$
                        ]
                        $+$
                        $+$
                    ]
                ]
            ]

            \caption{
                A parse tree for the preincrement interpretation of $a+++b$.
            }
            \label{fig:preincrement}
        \end{figure}

        Since the given grammar is not recursive, the unary incrementing
        operators cannot be nested; the only ambiguous string is
        $a+++b$. We can write unambiguous grammars for each interpretation by
        exhausting all the possible parses, of which there are only finitely
        many due to the lack of recursion, and then eliminating common
        subexpressions of rules by introducing new rules for them. We have used
        that procedure in the following two unambiguous grammars.

        First, here is a grammar that will parse $a+++b$ as including a
        post-increment.
        \begin{align*}
            S &\to (I | \langle Pre \rangle) + (I | P ) \\
            S &\to \langle Post \rangle + P \\
            S &\to \langle Pre \rangle | \langle Post \rangle
        \end{align*}

        Next, we will write a grammar to parse $a+++b$ as including a
        pre-increment.
        \begin{align*}
            S &\to I + ( I | \langle Post \rangle ) \\
            S &\to P + (I | P) \\
            P &\to \langle Pre \rangle | \langle Post \rangle
        \end{align*}

        The $I$, $\langle Pre \rangle$, and $\langle Post \rangle$ productions
        are exactly as in the original grammar.

\end{enumerate}

\end{document}
