\documentclass{article}

\usepackage{float}
\usepackage[margin=1.5cm]{geometry}
\usepackage{amsmath,amssymb,amsthm}
\usepackage{algorithm,algorithmicx,algpseudocode}

\newtheorem{proposition}{Proposition}

\author{Jacob Errington (260636023)}
\date{12 February 2015}
\title{Assignment \#3\\COMP 252}

\begin{document}

\maketitle

\section{Tree conversions}

We assume that when the walk was constructed, $1$ was recorded every time the
iterate moved down, and $0$ was recorded each time it moved up.

\begin{algorithm}[H]
    \caption{Tree reconstruction from a Harris walk}
    \begin{algorithmic}
        \Require{A array $H$ representing the Harris walk}
        \Ensure{A tree $T$ whose shape is that described by $H$}
        \Function{RebuildHarris}{$A$}
            \State $T \gets $ an empty node

            \For{$i$ from $length[A]$ down to $2$}
                \If{$A[i] = 1$}
                    \If{A[i-1] = 0}
                        \State $S \gets$ a new node
                        \State $nextSibling[S] \gets T$
                        \State $T \gets S$
                        \State increment $i$
                    \Else
                        \If{$parent[T]$ is not null}
                            \State $T \gets parent[T]$
                        \Else
                            \State $S \gets$ a new node
                            \State $eldestChild[S] \gets T$
                            \State $T \gets S$
                        \EndIf
                    \EndIf
                \Else
                    \State $eldestChild[T] \gets$ a new node
                    \State $T \gets eldestChild[T]$
                \EndIf
            \EndFor

            \State $S \gets$ a new node
            \State $eldestChild[S] \gets T$
            \State \Return $S$
        \EndFunction
    \end{algorithmic}
\end{algorithm}

\section{Group testing}

\begin{algorithm}[H]
    \caption{Group testing}
    \begin{algorithmic}
        \Require{$n$ blood samples, of which $0 < k \leq n$ are contaminated}
        \Ensure{How many blood samples are contaminated}
        \Function{TestGroup}{$sampleGroup$}
            \If{$sampleGroup.size = 1$}
                \Return the unique contaminated sample
            \EndIf
            \State $lGroup, rGroup \gets$ split $sampleGroup$ into two equally-sized subgroups
            \State $l \gets$ \Call{Merge}{$subgroup1$}
            \State $r \gets$ \Call{Merge}{$subgroup2$}
            \State $s \gets \emptyset$
            \If{\Call{Test}{$l$}}
                \State $s \gets s \cup$ \Call{TestGroup}{lGroup}
            \EndIf
            \If{\Call{Test}{$r$}}
                \State $s \gets s \cup$ \Call{TestGroup}{rGroup}
            \EndIf
            \State \Return $s$
        \EndFunction
    \end{algorithmic}
\end{algorithm}

\begin{proposition}
    The \textsc{TestGroup} algorithm requires $O(k\log_2{n})$ uses of the \textsc{Test} oracle.
\end{proposition}

\begin{proof}
    In the worst case, contaminated samples are split roughly equally into both
    subgroups upon recursion. Therefore, the $k$ subgroups in which there is a
    single contaminated sample will be located around level $\log k$. From
    there, identifying that sample requires partitioning the samples by two
    repeatedly until it is found, which requires $O(\log n)$ time, $k$ times.

    Thus, overall, we get $O(k \log n)$.
\end{proof}

\section{Lower bounds}

\begin{description}
    \item[Binary oracle.] We consider three pools of elements, $O$, $W$, $D$.
        We start with all the elements in group $O$, and the algorithm will
        halt when group $O$ is depleted group $W$ contains only one element. We
        can establish the following flows:
        \begin{description}
            \item[O, O.] One element will move to $W$, one element will move to
                $D$.
            \item[W, W.] One element will remain in $W$, one element will move
                to $D$.
            \item[O, W.] The size of $W$ remains the same, and $D$ will
                increase by one.
            \item[D, *.] No change.
        \end{description}

        Due to the nature of the termination conditions, we can see we must
        remove every element from $O$, which can be done in two ways: comparing
        an element of $O$ with one of $W$ or comparing two elements of $O$. The
        former process removes one element from $O$, and the latter process
        removes two.

        We can construct equations representing these flows, considering the
        total count of elements to be $n$.
        \begin{equation*}
            n = f_{OO} + f_{OW}
        \end{equation*}

        And $n_W$ the number of elements in $W$, is however many have flowed in
        minus however many have flowed out.
        \begin{equation*}
            n_W = f_{OW} - f_{WW}
        \end{equation*}

        Ultimately, we need $n_W$ to be $1$, so we can rewrite these equations
        into one.
        \begin{equation*}
            n - 1 = f_{OO} + f_{WW}
        \end{equation*}

        That equation demonstrates that the lower bound on the flows,
        regardless of the chosen algorithm or inputs is exactly $n - 1$.

    \item[$k$-ary oracle.] Things are trickier because there are more cases to
        check. Again we consider three pools $O$, $W$, $D$ where all elements
        are initially in $O$.

        Let $S$ be a sample of $k$ elements used as input to the oracle. There
        are $w$ elements in $S$ from $W$, $o$ elements in $S$ from $O$, and $d$
        elements in $S$ from $D$.

        Since elements from $D$ do not affect the flow, we may consider without
        loss of generality that $d = 0$. Thus, $k = w + o$.

        Then, we must consider two cases.
        \begin{itemize}
            \item If $w = 0$, then $o = k$: one element will move to $W$, and
                the remaining $k-1$ will move to $D$.
            \item If $w > 0$, then $W$ will lose $w - 1$ elements, $O$ will
                lose $k - w$ elements, and $D$ will gain $k - w + 1$ elements.
        \end{itemize}

        The halting conditions are the same as before: $O$ must be emptied and
        $W$ must contain a single element, since each element must bear witness
        to the maximality.

        Let's write the number of elements removed from $O$ as a function of
        $w$.
        $$
        f(w) = \begin{cases}
            k & \text{if } w = 0 \\
            k - w & \text{if } w > 0
        \end{cases}
        $$

        To empty $O$, we must consider all possible ways in which the oracle
        can be used.
        \begin{align*}
            n &= a_0 f(0) + a_1 f(1) + a_2 f(2) + \cdots + a_k f(k) \\
            n &= a_0 k + a_1 (k - 1) + a_2 (k - 2) + \cdots + a_{k-1} (k - (k-1)) + a_k (k - k)
        \end{align*}
        where $a_i$ represents the number of times the oracle has been used in
        the $i$th fashion.

        Each time we use the oracle to remove some number of elements from $O$,
        we add some corresponding number of elements to $W$, so let's write the
        flow for $W$ as a function of $w$.
        $$
        g(w) = \begin{cases}
            1 & \text{if } w = 0 \\
            1 - w & \text{if } w > 0
        \end{cases}
        $$

        Then we get the following equation, reusing the constants $a_i$.
        \begin{align*}
            1 &= a_0 g(0) + a_1 g(1) + \cdots + a_k g(k) \\
            1 &= a_0 1 + a_1 (1 - 1) + \cdots + a_k (1 - k)
        \end{align*}

        Now let's subtract these equations.
        \begin{align*}
            n - 1 &= a_0 \left(f(0) - g(0)\right) + a_1 \left(f(1) - g(1)\right) + \cdots + a_k \left(f(k) - g(k)\right) \\
            n - 1 &= a_0 \left(k - 1\right) + \sum_{i=1}^k a_i \left(k - i - 1 + i\right) \\
            n - 1 &= a_0 \left(k - 1\right) + (k-1) \sum_{i=1}^k a_i \\
            n - 1 &= a_0 (k - 1) + \alpha (k - 1)
        \end{align*}
        where $\alpha$ is the total number of times the oracle is used with at
        least one element from $W$.
        Therefore, the lower bound for the number of uses of the oracle is
        $$ a_0 + \alpha = \frac{n - 1}{k - 1} $$
\end{description}

\end{document}
