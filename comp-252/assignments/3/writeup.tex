\documentclass{article}

\usepackage{amsmath,amssymb,amsthm}
\usepackage{algorithm,algorithmicx,algpseudocode}

\newtheorem{proposition}{Proposition}

\author{Jacob Errington (260636023)}
\date{12 February 2015}
\title{Assignment \#3\\COMP 252}

\begin{document}

\maketitle

\section{Group testing}

\begin{algorithm}
    \caption{Group testing}
    \begin{algorithmic}
        \Require{$n$ blood samples, of which $0 < k \leq n$ are contaminated}
        \Ensure{How many blood samples are contaminated}
        \Function{TestGroup}{$sampleGroup$}
            \If{$sampleGroup.size = 1$}
                \Return $1$
            \EndIf
            \State $lGroup, rGroup \gets$ split $sampleGroup$ into two equally-sized subgroups
            \State $l \gets$ \Call{Merge}{$subgroup1$}
            \State $r \gets$ \Call{Merge}{$subgroup2$}
            \State $s \gets 0$
            \If{\Call{Test}{$l$}}
                \State increase $s$ by \Call{TestGroup}{lGroup}
            \EndIf
            \If{\Call{Test}{$r$}}
                \State increase $s$ by \Call{TestGroup}{rGroup}
            \EndIf
            \State \Return $s$
        \EndFunction
    \end{algorithmic}
\end{algorithm}

\begin{proposition}
    The algorithm \textsc{TextGroup} requires $O(k\log_2{n})$ uses of the \textsc{Test} oracle.
\end{proposition}

\begin{proof}
    The number of times in which both the left and right subgroups can be
    contaminated can occur at most $k$ times: in the worst-case split, one
    subgroup contains $1$ contaminated sample, and the other contains the
    remaining $k-1$ samples.

    Now we look at the recurrence for the algorithm:
    \begin{align*}
        T_1   &= 0 \\
        T_{n} &= 2 + \alpha T_{n/2}
    \end{align*}
    where $\alpha$ is $2$ at most $k$ times.
\end{proof}

\end{document}
