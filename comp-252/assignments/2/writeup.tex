\documentclass{article}

\usepackage{amsmath,amssymb}
\usepackage[margin=2cm]{geometry}

\DeclareMathOperator{\Tri}{Tri}

\author{Jacob Errington(260636023)}
\title{Assignment \#4\\COMP 252}

\begin{document}

\maketitle

\section{Minimum edit distance}

We'll denote the empty string by epsilon $\epsilon$. The edit distance matric for changing ``apple'' into ``pear'' will look like the following.

$$
    \begin{array}{c | c | c | c | c | c}
        ``apple'' & 5 & & & & \\
        \hline
        ``appl'' & 4 & & & & \\
        \hline
        ``app'' & 3 & & & & \\
        \hline
        ``ap'' & 2 & & & & \\
        \hline
        ``a'' & 1 & & & & \\
        \hline
        \epsilon & 0 & 1 & 2 & 3 & 4 \\
        \hline
                   & \epsilon & ``p'' & ``pe'' & ``pea'' & ``pear''
    \end{array}
$$

Changing $\epsilon$ into a string of length $n$ requires $n$ edits, namely insertions. There's no way around that. It's the base case, and by virtue of being trivial, we can write it directly into the matrix.
From that, we get the following two equations:

\begin{align*}
    distance\, A(0)\, A(m) &= m \\
    distance\, A(n)\, B(0) &= n
\end{align*}

If both the source and destination strings are nonempty, then the string lengths are $n$ and $m$. We check the distances of the three substrings.

\begin{align*}
    distance\, A(n-1)\, B(m) &\\
    distance\, A(n)\, B(m-1) &\\
    distance\, A(n-1)\, B(m-1) &
\end{align*}

To find the value at position $n,m$, we take the minimum of these distances and
add one to it unless $a(n) = b(m)$. Of course, we don't bother computing the
edit distance of the substrings more than once, so we store the results into
the matrix as we uncover them.

That being said, filling in a cell in the matrix takes time $O(1)$ when we know the distances of the three substrings, and there are $nm$ cells in total. Therefore, this algorithm is $O(nm)$.


\section{Optimal triangulation}

The input is a convex polygon $P$ defined by clockwise vertices $v_1, \cdots, v_n$. We'll assume that the indices ``wrap around'' in the intuitive way when they go out of bounds.

We consider chains that form smaller polygons. $P(i,j)$ is the subpolygon having vertices $v_i, \cdots, v_j$. And we'll consider the base case to be that of the triangle, whose triangulation has value zero as there are no diagonals.

In the inductive case, the idea is to hold edge $v_i, v_j$ fixed and split the $n$-gon into two , and recursively find the triangulations of the resulting $(n-1)$-gons. To find an optimal solution where $v_i, v_j$ is fixed, we have to try all triangles $ijk$ where $i < k < j$. 

From that, we get the two following equations.

\begin{align*}
    \Tri{(P(i, i))} &= 0 \\
    \Tri{(P(i, i+1))} &= 0 \\
    \Tri{(P(i, i+2))} &= 0 \\
    \Tri{(P(i, j))} &= \min_{i < k < j} \{\Tri{(i, k)} + \Tri{(k, j)} + |v_i + v_j|\}
\end{align*}

Each $\min$ takes time $O(n)$, and we fill out a two dimensional array keyed on $i$ and $j$, so overall, the algorithm is $O(n^3)$.


\section{Linear-time selection}

We'll analyse each section and add up.

\begin{enumerate}
    \item Sorting a fixed number of elements is some constant amount of work, in the $n \leq 7$ case.
    \item Partitioning into groups of seven is free, as it requires no comparisons.
    \item Since $|S|$ and $|L|$ are $n/7$, this takes $2 T(n/7)$.

        $s$ is the $\frac{3|S|}{5}$-th smallest element of $S$, and $l$ is the $\frac{2|L|}{5}$-th smallest element of $L$.

        In other words, $s$ is the $\frac{2|S|}{5}$-th largest element of $S$.

        From that, we can conclude that $s$ is ranked $\frac{2}{7} - \frac{2}{7} \frac{3}{5} = \frac{6}{35}$ and $l$ is ranked $\frac{5}{7} + \frac{2}{7} \frac{2}{5} = \frac{29}{35}$.

        So $B$ will be the biggest class, and we'll interest ourselves in $B$ for the analysis.

        Overall, we get the following.

        \begin{equation*}
            T(n) =
            \begin{cases}
                k & \text{if } n \leq 7 \\
                2 T(n/7) + T(23n/35) + 2n & \text{otherwise}
            \end{cases}
        \end{equation*}
\end{enumerate}
\end{document}
