\documentclass{article}

\usepackage{amsmath, amssymb, amsthm}
\usepackage{algorithm, algorithmicx, algpseudocode}
\usepackage{float}
\usepackage[margin=2.5cm]{geometry}

\newtheorem{proposition}{Proposition}

\DeclareMathOperator{\Key}{\mathtt{key}}
\DeclareMathOperator{\Enqueue}{\mathtt{enqueue}}
\DeclareMathOperator{\Dequeue}{\mathtt{dequeue}}
\DeclareMathOperator{\Peek}{\mathtt{peek}}
\DeclareMathOperator{\Remove}{\mathtt{remove}}
\DeclareMathOperator{\Insert}{\mathtt{insert}}
\DeclareMathOperator{\Intersects}{\mathtt{intersects}}
\DeclareMathOperator{\print}{\mathtt{print}}
\DeclareMathOperator{\leftChild}{\mathtt{leftChild}}
\DeclareMathOperator{\rightChild}{\mathtt{rightChild}}
\DeclareMathOperator{\Parent}{\mathtt{parent}}
\DeclareMathOperator{\RemoveMin}{\mathtt{removeMin}}

\author{Jacob Thomas Errington (260636023)}
\title{Assignment \#4\\Honours Algorithms and Data Structures (COMP 252)}
\date{10 March 2015}

\begin{document}

\maketitle

\section{Browsing a red-black tree}

\begin{enumerate}
    \item \emph{Finding $k$-th smallest element.}

        \begin{proposition}
            Given pointers to both the root and the smallest element of a
            red-black tree, finding an element $u$ such that $\Key{[u]} = x$
            can be done in time $O(1 + \log k)$ if $x$ is the $k$-th smallest
            element stored in the tree.
        \end{proposition}

        \begin{proof}
            The sub-trees of a red-black tree are also red-black trees, and so
            are balanced. Between the smallest element and the $k$-th smallest
            element, there are a total of $k$ elements, so the minimal subtree
            that contains these two elements will have height $\log k$. Thus,
            it suffices to find the youngest common ancestor of the two
            elements, since performing the ordinary lookup procedure from there
            will take time $O(\log k)$.

            Starting from the smallest element, we keep going to its parent
            (then its parent's parent, etc.) until we find an element -- call
            it $p$ -- whose \emph{parent's} key is greater than $x$. Therefore,
            $p$ will be the youngest common ancestor of the smallest and the
            $k$-th smallest element. Finding it takes time $O(\log k)$, and
            performing the regular lookup from there takes time $O(\log k)$, so
            overall this algorithm takes time $O(\log k)$.
        \end{proof}

    \item \emph{Listing the $k$ smallest elements.}

        \begin{proposition}
            Given pointers to both the root and the smallest element of a
            red-black tree, the $k$ smallest elements of the tree can be listed
            in time $O(k)$.
        \end{proposition}

        \begin{proof}
            The beginning of this proof is similar. We use the fact that there
            are $k$ elements in the subtree of interest, and we use a modified
            in-order traversal (since it's \emph{bottom-up} rather than
            \emph{top-down} as usual) starting from the smallest element.

            Since the traversal is bottom-up, we split it into two parts: one
            part walks up left leftmost edge of the tree parent by parent, and
            the other part performs augmented in-order traversals of the right
            subtrees of those leftmost elements.

            The traversal is augmented by a counter representing how many
            elements are left to output.  Each time an element is yielded, the
            global counter is decremented, until a total of $k$ elements have
            been yielded.

            \begin{algorithm}[H]
                \caption{List the smallest $k$ elements in a red-black tree
                given the smallest element.}
                \begin{algorithmic}
                    \Require{The smallest element $t$ in a red-black tree, the
                    number of elements to list $k$.}
                    \Ensure{The $k$ smallest elements are printed.}

                    \Function{Traverse}{$t$, $k$}
                        \If{$t$ is \textbf{null} or $k = 0$}
                            \State \Return $k$
                        \Else
                            \State $k^\prime \gets$
                            \Call{Traverse}{$\leftChild{[t]}$, $k$}
                            \State $k^\prime \gets$
                            \Call{Traverse}{$\rightChild{[t]}$, $k^\prime - 1$}
                            \State \Return $k^\prime$
                        \EndIf
                    \EndFunction

                    \Function{ListSmallest}{$t$, $k$}
                        \State $c \gets k - 1$
                        \State $\print{t}$
                        \While{$c > 0$}
                            \State $t \gets \Parent{[t]}$
                            \State $\print{t}$
                            \State $c \gets c - 1$
                            \State $c \gets $
                            \Call{Traverse}{$\rightChild{[t]}$, $c$}
                        \EndWhile
                    \EndFunction
                \end{algorithmic}
            \end{algorithm}

        The algorithm given is $O(k)$ since we essentially perform an in-order
        traversal of the minimal subtree containing the smallest element and
        the $k$-the smallest element, in which there are $k$ elements, all of
        which are printed.
    \end{proof}

\end{enumerate}

\section{Binary heaps}

\begin{proposition}
    The $k$ smallest items of a binary min-heap of size $n$ can be found in
    time $O(k \log k)$.
\end{proposition}

\begin{proof}
    We will create a second min-heap -- call it $H^\prime$ -- of size $k$ in
    which to insert elements as we discover them in the primary heap -- call it
    $H$. We will then traverse $H$ in a fashion prioritized by $H^\prime$, in
    which operations take $O(\log{k})$.

    \begin{algorithm}
        \caption{List the $k$ smallest elements of a binary min-heap.}
        \begin{algorithmic}
            \Require{The number of elements to find $k$, the binary min-heap $H$}
            \Ensure{The $k$ smallest elements are printed.}
            \Function{ListSmallestHeap}{$k$, $H$}
                \State Initialize a min-heap of size $k$, $H^\prime \gets
                \emptyset$
                \State $\Insert{[H^\prime, \Peek{[H]}]}$
                \For{$i = k \to 1$} \Comment{Executes a total of $k$ times}
                    \State $p \gets \RemoveMin{[H]}$ \Comment{$O(\log k)$}
                    \State $\print{p}$
                    \State $\Insert{[H, \leftChild{[p]}]}$
                    \Comment{$O(\log k)$}
                    \State $\Insert{[H, \rightChild{[p]}]}$
                    \Comment{$O(\log k)$}
                \EndFor
            \EndFunction
        \end{algorithmic}
    \end{algorithm}

    Inside the \texttt{for} loop, which is executed $k$ times in total, three
    $O(\log k)$ operations take place, so the overall time of this algorithm is
    $O(3k\log k) = O(k\log k)$ as required.
\end{proof}

\section{Overlapping circles}

\begin{algorithm}
    \caption{Determine if two circles overlap by a sweepline-inspired method}
    \begin{algorithmic}
        \item[]
            The \textsc{CircleOverlap} algorithm uses an interval tree to keep
            track of the $x$-spans of the circles as it sweeps across the
            plane. When two circles are confirmed to overlap on the $x$-axis,
            then they are checked for full overlap.

        \Require{$n$ closed circles in the plane, each determined by a center
        $(x_i, y_i)$ and a radius $r_i$.}

        \Ensure{Whether any two of those circles overlap.}

        \Function{CircleOverlap}{positions $(x_i, y_i)$ and radii $r_i$, for $1 \leq i \leq n$}
            \State Sort the circles by the function $\lambda (x, \_), r \to x - r$.
            \Comment $O(n \log n)$
            %\State Initialize an interval tree $T_x \gets \emptyset$ storing
            %circles, keyed on $\lambda (x, \_), r \to x - r$, with \emph{high}
            %calculated by $\lambda (x, \_), r \to x + r$.
            %\State Initialize an interval tree $T_y \gets \emptyset$ storing
            %circles, keyed on $\lambda (\_, y), r \to y - r$, with \emph{high}
            %calculated by $\lambda (\_, y), r \to y + r$.
            %\State Initialize an empty priority queue $Q \gets \emptyset$ keyed
            %on $\lambda (x, \_), r \to x + r$.
            %\For{$i = 1 \to n$}
            %    \While{$(x_q, y_q), r_q \gets \Peek{[Q]}$} \Comment $O(n \log n)$
            %        \If{$x_q + r_q < x_i - r_i$}
            %            \State $\Remove{[T_x, x_q - r_q]}$
            %            \State $\Remove{[T_y, y_q - y_q]}$
            %            \State $\Dequeue{[Q]}$
            %        \Else
            %            \State \texttt{break}
            %        \EndIf
            %    \EndWhile
            %    \If{$\Intersects{[T_x, ((x_i, y_i), r_i)]}$} \Comment $O(\log n)$
            %        \If{$\Intersects{[T_y, ((x_i, y_i), r_i)]}$} \Comment $O(\log n)$
            %            \State \Return \texttt{True}
            %        \EndIf
            %    \EndIf
            %\EndFor
            \State Initialize an empty interval tree $T_x \gets \emptyset$
            storing circles keyed on $\lambda (x, \_), r \to x - r$, with
            \emph{high} calculated by $\lambda (x, \_), r \to x + r$.
            \State Initialize an empty priority queue $Q \gets \emptyset$
            storing circles keyed on $\lambda (x, \_), r \to x + r$.
            \State $i \gets 2$
            \State $x \gets x_1 - r_1$
            \State $x_{next} \gets x$
            \While{$i \leq n$}
                \State $x_{next} \gets x_i - r_i$
                \If{$x = x_{next}$}
                    \Comment{The sweepline is at the beginning of a new
                        circle.}
                    \If{$cs \gets \Intersects{[T_x, ((x_i, y_i), r_i)]}$}
                        \For{$(x_c, y_c), r_c$ in $cs$}
                            \If{$\sqrt{(x_i - x_c)^2 + (y_i - y_c)^2}
                                    \leq r_i + r_c$}
                                \State \Return \texttt{True}
                            \EndIf
                        \EndFor
                    \EndIf
                    \State $\Enqueue{[Q, ((x_i, y_i), r_i)]}$
                    \State $\Insert{[T_x, ((x_i, y_i), r_i)]}$
                    \State $i \gets i + 1$
                \ElsIf{$(x_q, y_q), r_q \gets \Peek{[Q]}$}
                    \Comment{The sweepline is crossing some circles}
                    \If{$x_q + r_q < x_{next}$} \Comment{The next circle begins
                            after a current one ends.}
                        \State $\Dequeue{[Q]}$
                        \Comment{Remove the circle whose endpoint is soonest.}
                        \State $\Remove{[T_x, ((x_q, y_q), r_q)]}$
                    \EndIf
                \Else
                    \State $x \gets x_{next}$
                    \Comment{Advance the sweepline to the start of the next
                        circle.}
                \EndIf
            \EndWhile
            \State \Return \texttt{False}
        \EndFunction
    \end{algorithmic}
\end{algorithm}

\end{document}
